\author{AA.VV}
\title{Ma per chi sarebbero i vantaggi?}
\label{cha:direttore290413}
% \epigraph{Tra le numerose voci dissenzienti ricevute dopo la pubblicazione nel nostro giornale dell’“Opinione” di Giuseppe Felaco, riguardante l’inclusione scolastica degli alunni con disabilità, il sostegno e il ruolo degli insegnanti curricolari, ne abbiamo scelte alcune, cui ben volentieri diamo spazio, provenienti sia dal mondo associativo, che da quello degli insegnanti}
\begin{abstract}
Tra le numerose voci dissenzienti ricevute dopo la pubblicazione nel nostro giornale dell’“Opinione” di Giuseppe Felaco, riguardante l’inclusione scolastica degli alunni con disabilità, il sostegno e il ruolo degli insegnanti curricolari, ne abbiamo scelte alcune, cui ben volentieri diamo spazio, provenienti sia dal mondo associativo, che da quello degli insegnanti
\end{abstract}
\maketitle
Sta facendo decisamente discutere – e molto – l'opinione di Giuseppe Felaco, da noi pubblicata qualche giorno fa, con il titolo Tutti avrebbero dei vantaggi\ref{felaco260413}, ove l'autore scriveva tra l'altro che «La chiave di volta [per realizzare al meglio l'integrazione scolastica, N.d.R.] sta nel convertire i fondi che servono per pagare gli stipendi agli insegnanti di sostegno in incentivi mensili da assegnare agli insegnanti curricolari, che hanno una classe dove sia presente un alunno con disabilità. Nel caso poi si dovessero presentare delle difficoltà o l'insegnante curricolare ritenesse di non farcela a gestire la classe, egli potrebbe richiedere l'aiuto dell'insegnante di sostegno, da remunerare attingendo dal surplus assegnatogli in precedenza. A questo punto entrerebbero in gioco gli unici strumenti idonei: i compagni, tanti compagni, che diventerebbero i mezzi con cui gli insegnanti potrebbero realizzare la vera integrazione».
In particolare – dopo quella del genitore Daniele Brogi, da noi già pubblicata\ref{brogi260413} – numerose sono state le voci dissenzienti giunteci in redazione, sia dal fronte associativo che da quello degli insegnanti, curricolari e di sostegno.
Ne abbiamo scelte alcune, che di seguito ben volentieri pubblichiamo, incominciando da quella espressa da Carlo Hanau, docente dell'Università di Modena e Reggio Emilia, a nome dell'ANGSA (Associazione Nazionale Genitori Soggetti Autistici), secondo il quale la posizione di Felaco \caporali{contraddice l’azione condotta insieme all’ANGSA per la specializzazione degli educatori e di tutti gli insegnanti, in particolare quelli di sostegno, la cui utilità per migliorare la condizione delle persone con autismo non può essere messa in discussione.} 

Queste le ulteriori opinioni che abbiamo deciso per ora di pubblicare. Altre ne seguiranno nei prossimi giorni.

\caporali{Sono insegnante curricolare e non posso pensare di fare a meno delle insegnanti di sostegno o delle educatrici (sottopagate e oberate di lavoro). Forse qualcuno dimentica che in una classe di scuola primaria vivono come minimo 25 bambini, dei quali circa la metà avrebbero bisogno di un aiuto in più rispetto a quello che l’insegnante riesce a dare e che quindi si fa una fatica enorme a procedere tutti insieme.
Mi sembra che chi pensa di poter accantonare le insegnanti di sostegno o gli educatori non abbia presente il lavoro dei bambini, degli insegnanti, della scuola in generale. Forse che i bambini cooperanti e tanto bravi, attenti e sensibili che sono nella classe in cui opero avrebbero potuto diventare così da soli o soltanto con le insegnanti curricolari?.}
Angela Macchi (insegnante curricolare)

\caporali{Credo che “la luce negli occhi e la risata gioiosa”, di cui scrive Giuseppe Felaco, siano da costruire nella quotidianità della classe con la mediazione di un adulto, che difficilmente può essere un insegnante curricolare con classi da 27, 28 alunni, sempre più problematici.
Parlo da insegnante di sostegno ed esprimo la mia amarezza nel constatare che si agita sempre la prospettiva di un miglioramento economico del nostro misero stipendio, per arrivare a risultati che sarebbero nel migliore dei casi fittizi (si vedano ad esempio i recenti interventi sulla dislessia, demandati all'insegnante curricolare, e nella quasi totalità dei casi ridotti a meri adempimenti burocratici).
Ciò ci riporta indietro di anni luce. L'insegnante di sostegno, quando opera in sinergia con l'insegnante curricolare, è una vera e propria ricchezza e nessuna stortura e deficienza del sistema (e ce ne sono, lo so bene), nessuna inadempienza o fannullaggine mi potrà mai convincere del contrario.}
Lucia Piccolo (insegnante di sostegno)

\caporali{È del tutto impossibile che i docenti di classe siano gli unici a gestire i processi di integrazione nel gruppo classe e gli alunni non possono essere visti come \cit{strumenti primi}, \cit{insegnanti}, \cit{facilitatori}, \cit{sostituti educatori}.
Per una seria inclusione è necessaria la formazione degli insegnanti di classe, ma altrettanto indispensabile è il lavoro di equipe tra più adulti che supportano il bimbo con modalità differenziate e specifiche, permettendogli di inserirsi nel gruppo, secondo la sua possibile misura.
Ho fatto per dieci anni l'insegnante di classe, pur essendo insegnante di sostegno da vent'anni, ed è stata un'esperienza molto interessante, ma ho deciso di concluderla, anche perché mi sono trovata molte volte accanto a persone poco coinvolte e/o poco competenti e dovevo sostanzialmente lavorare per due, con grande affanno mio e dei bimbi.
È stato durissimo frenare la classe per contenere tutti in un processo condiviso o forzare il bambino a stare al passo con gli altri, quando al mattino si era svegliato \cit{con un profondo dolore dentro}.
Lavorare in più persone esperte del tema è la cosa migliore, perché si possono integrare diversi linguaggi e diverse metodologie di lavoro, diversi ritmi e diverse abilità comunicative in un'unica direzione, che è l'integrazione delle possibilità e delle potenzialità.
Credo però che sia fondamentale la presenza di personale specializzato, che sappia costruire un progetto su misura per il bimbo, riconoscendo anche ai compagni la possibilità di sperimentarsi in contemporanea, a diversi livelli di abilità.
Il rischio grosso dell'eliminazione dell’insegnante di sostegno sarebbe il ritorno alla socializzazione forzata, alla negazione dell'intervento individuale, alla negazione della possibilità concreta di creare esperienze di gruppo utili per la crescita del bambino e della classe. Una prospettiva, questa, che ritengo un disastro e che va alla pari con l'eliminazione delle ore delle compresenze e dei progetti, introdotta ultimamente come \cit{nuova} proposta, solo per una logica di diminuzione della spesa.
Lavoriamo piuttosto per formare bene gli insegnanti di sostegno, per formare bene gli insegnanti di classe, per mettere nelle scuole personale psicopedagogico che si occupi di sostenere la progettazione riguardante la disabilità, i bisogni speciali e gli stranieri. La logica non dev'essere alternativa, ma integrativa e mirata all'eccellenza, non alla massificazione e alla negazione del bisogno individuale.}
Luisa Formenti (insegnante di sostegno e curricolare)

\caporali{Dopo trentadue anni dedicati all’insegnamento su sostegno – pur nelle difficoltà quotidiane che incontriamo per far convergere diverse modalità di lavoro e dove non sempre si trova reciprocità e condivisione – rincuora sentire ribadire che l’integrazione è tale solo quando vengono attivate sinergie di competenze e ruoli.
Occorre davvero ricreare una cultura dell'integrazione a largo spettro, affinché tutte le componenti (dirigenti, insegnanti curricolari, di sostegno e specialisti) vengano coinvolte, almeno nell'interesse e nel coinvolgimento. Si pensi che ancora in molte scuole le Circolari riguardanti corsi e seminari sulla disabilità, sono offerte in visione (e firmate) solo agli insegnanti di sostegno! Se nei Collegi Docenti si spendessero più parole a favore di una buona integrazione, forse noi insegnanti saremmo più motivate a continuare a lavorare in questo stupendo ruolo}.
Elisabetta Malaguti (insegnante di sostegno)

\caporali{Mi piacerebbe davvero molto che bastasse una grande motivazione all’inclusione (seppure economica) a risolvere tutti i problemi di un bambino con disabilità e dei suoi compagni… Sarebbe come sostenere che basta accogliere le persone disabili con buona volontà, per risolvere tutto\dots  Magari!
Tanti sbagli educativi e tanti danni sono stati perpetrati da persone che avevano le migliori motivazioni a far bene. Il fatto è che purtroppo essere una persona con disabilità comporta avere una parte del proprio funzionamento che differisce dalle altre persone e se non fosse così, non si parlerebbe di disabili, ma solo di \cit{socialmente disadattati}. Se invece un bambino, per imparare a scuola, deve fare i conti con la differenza che oggettivamente c'è tra lui e gli altri bambini, questo comporta una modificazione consapevole e competente delle modalità, dei tempi, del rapporto numerico tra docente e discente, delle cose che si fanno e specialmente come si fanno. Purtroppo spesso fare qualcosa in un modo anziché in un altro, fa la differenza tra perdere tempo e fare qualcosa di utile.
Magari bastasse solo lavorare su di un clima di accettazione delle persone per rispondere a tutti i bisogni educativi speciali! Gli unici che si potrebbero giovare di un simile quadro idilliaco sono proprio le persone che hanno bisogno di credere che la neuro diversità – e in generale la diversità – derivante da un deficit, sia tutto sommato una condizione come un'altra, un disagio che si può azzerare, solo modificando la prospettiva di chi guarda.}
Maria Luisa Gargiulo (psicologa, psicoterapeuta)\footcite{Editoriale2013}

29 aprile 2013

© Riproduzione riservata