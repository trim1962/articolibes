\author{Simona Lancioni}
\title{E intanto i Tribunali continuano il loro lavoro…}
\phantomsection
\label{cha:lancioni240114}
\begin{abstract}
Fa sempre particolarmente piacere, infatti, proprio nei giorni in cui siamo stati costretti a denunciare il grave atto di un Comune del nostro Paese, che intende sostanzialmente subordinare l'iscrizione a scuola dei bimbi con disabilità a esigenze di bilancio, vedere il ripetersi di Sentenze dei Tribunali Amministrativi Regionali -– in questo caso della Toscana –- che ristabiliscono il pieno diritto allo studio degli alunni con disabilità
\end{abstract}
\maketitle
\datapub{24 gennaio 2014}
«Ogni individuo ha diritto all'istruzione. L'istruzione deve essere gratuita almeno per quanto riguarda le classi elementari e fondamentali. L'istruzione elementare deve essere obbligatoria»: esordisce così il primo comma dell'articolo 26 dalla Dichiarazione Universale dei Diritti Umani, proclamata dall'Assemblea Generale delle Nazioni Unite nel 1948. Un principio fatto proprio da molte Costituzioni dei Paesi europei, tra cui quella italiana (articoli 34 e 38), dalla Legge quadro sull'handicap (Legge 104/92\footcite{Legge_104_92}) e dalla Convenzione ONU sui Diritti delle Persone con Disabilità, ratificata dal nostro Paese con la Legge 18/09\footcite{Legge18_2009}.

Pertanto, non può certo sorprendere che la Sezione Prima del Tribunale Amministrativo Regionale (TAR) della Toscana, con la Sentenza n. 54/14\footcite{TARToscana2014} prodotta il 13 gennaio scorso, abbia accolto il ricorso presentato da due genitori di un alunno minorenne con disabilità, contro la scuola frequentata dallo stesso, al fine di veder riconosciuto il diritto del proprio figlio ad avere un insegnante di sostegno per un numero di ore settimanali adeguato alla sua patologia.

Ebbene, grazie a tale Sentenza, il ragazzo potrà fruire «di un numero di ore di sostegno corrispondente all'intero orario di frequenza scolastica», vale a dire 40 ore in luogo delle 16 precedentemente assegnate dall'Amministrazione Scolastica.

Il pronunciamento si pone in linea con una collaudata giurisprudenza, supportata anche da pronunciamenti della Corte Costituzionale e del Consiglio di Stato, che portano a definire il diritto allo studio dell'alunno con disabilità come un «vero e proprio diritto soggettivo, incomprimibile in dipendenza di carenze organiche del personale scolastico, ovvero di esigenze di bilancio».

La Sentenza del TAR della Toscana stabilisce inoltre il diritto al risarcimento del danno non patrimoniale –- quantificato in 600 euro per ogni mese di mancata applicazione delle disposizioni contenute nella Sentenza stessa, a partire dalla data di ricorso --, e condanna l'Amministrazione Scolastica al pagamento delle spese processuali (3.000 euro).

Il provvedimento si applica all'anno scolastico in corso, ma non per gli anni a venire poiché, come spiega il TAR, la «necessità di procedere alla periodica verifica delle condizioni psicofisiche dell'interessato impedisce di proiettare nel futuro gli accertamenti relativi agli anni precedenti». Il rischio, quindi, è che i genitori, per veder riconosciuto l'effettivo diritto allo studio del figlio, possano ritrovarsi a dover presentare ricorso tutti gli anni\footcite{Lancioni2014}. 
