\chapter{Alunni con bisogni educativi speciali, sono circa 1 milione}
\label{cha:ciambrone110613}
L'analisi di Raffaele Ciambrone (Miur, Ufficio disabilità): "Maggior tutela a studenti che non rientravano nei casi previsti dalle leggi 104/92 e 170/2010". Tra questi circa 80 mila ragazzi con sindrome da iperattività

ROMA - Alunni con bisogni educativi speciali non riconosciuti né "codificati" da nessuna legge. Ragazzi con disturbi del comportamento o dell'attenzione, immigrati da poco in Italia, minori con ritardi linguistici, sindrome di Asperger o un quoziente intellettivo tra 70 e 85, cioè appena sopra il limite previsto dalla normativa per l'assegnazione del sostegno (legge 104), che non rientrano nemmeno nella legge 170 per i disturbi specifici dell'apprendimento. Ma anche ragazzi che si trovano in una condizione di difficoltà socio-economica, linguistica e culturale. Sono circa un milione in Italia e per garantire loro il diritto allo studio è stata emanata una direttiva ministeriale (Strumenti d'intervento per alunni con bisogni educativi speciali e organizzazione territoriale per l'inclusione scolastica) che introduce un modello di intervento, attraverso piani didattici personalizzati e una responsabilità congiunta di scuola e famiglia.

La stima del Miur si basa su dati noti e proiezioni. Sono circa 215 mila gli alunni con disabilità, spiega Raffaele Ciambrone, dirigente del ministero dell'Istruzione (Ufficio disabilità). Oltre 90 mila quelli con disturbi specifici dell'apprendimento, un dato in crescita se si pensa che solo tra gli anni scolastici 2010/2011 e 2011/2012 le certificazioni sono aumentate del 37 per cento. "Considerando una media del 4 per cento dell'incidenza dei Dsa sulla popolazione scolastica, - spiega Ciambrone - stimiamo che i ragazzi con questi disturbi potranno arrivare a 300 mila". Ci sono poi i ragazzi con Adhd (sindrome da deficit di attenzione e iperattività) che sono circa 80 mila e quelli con funzionamento intellettivo limite (quoziente intellettivo tra 71 e 84). Secondo le indagini scientifiche si stima che siano circa 400 mila, prendendo in considerazione, anche in questo caso, una media dell'incidenza del 5 per cento, tra le diverse valutazioni degli esperti più accreditati.

Più difficile fare una valutazione numerica degli alunni che rientrano nell'area dello svantaggio economico o culturale. "Nel caso degli alunni stranieri, si prendono in considerazione gli alunni di origine straniera di recente immigrazione iscritti alle medie o alle superiori, i cosiddetti neo arrivati in Italia (Nai), in particolare dai paesi asiatici e nordafricani, che hanno maggiori difficoltà linguistiche perché parlano una lingua non latina", spiega Ciambrone. Per questi alunni e per quelli che provengono da situazione in difficoltà economiche (per le quali, tuttavia, esistono anche altri strumenti di sostegno) gli interventi "devono essere individuati sulla base di elementi oggettivi, ad esempio la segnalazione degli operatori dei servizi sociali, e di fondate considerazioni psicopedagogiche e didattiche".

"Volevamo sciogliere il legame tra certificazione medica e intervento educativo: - sottolinea Ciambrone - è inaccettabile che un intervento educativo possa essere ritardato per problemi burocratici. E il senso della direttiva è proprio quello di spostare il punto di vista dal piano clinico a quello educativo". "Abbiamo voluto assicurare maggior tutela ad alunni e studenti che non rientravano nei casi previsti dalle leggi 104/92 e 170/2010, nella prospettiva di una scuola sempre più accogliente e inclusiva. Rimane confermato e rafforzato il nostro impegno per gli alunni con disabilità per i quali, quest'anno, sono stati assegnati ulteriori 6.000 posti in più per il sostegno." (cch)

(11 giugno 2013)
