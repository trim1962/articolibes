\author{Andrea Canevaro}
\title{Didattica speciale e specialismi. Un chiarimento}
\label{Canevaro020713}
%\epigraph{\hspace*{20pt}}{Andrea Canevaro}
\maketitle
Credo che la dinamica che era stata avviata nella scuola, e che per brevità ho definito degli specialismi, vada chiarita e compresa. Se un insegnante incontra un bambino cieco, e quello stesso insegnante è totalmente sprovveduto, ossia ignora tutto di tiflologia, di didattica specifica, è abbastanza evidente che il suo lavoro risulterà poco utile e poco efficace. In anni ormai lontani, con qualche amico dell'Istituto Cavazza di Bologna, Istituto dei Ciechi, cercammo, con successo, di fornire a insegnanti impreparati un appoggio competente, con la finalità di permettere all'insegnante sprovveduto di impadronirsi delle capacità per seguire quel bambino o quella bambina “speciale”. In un anno, l'obiettivo fu raggiunto: l'insegnante non aveva più bisogno di un affiancamento specialistico, perché aveva integrato le opportune competenze. Questo modo di procedere non significa negare le specifiche esigenze, quanto piuttosto attivarsi per integrarle in un profilo professionale più completo.

La deriva degli specialismi tenta di dare risposte seguendo un'altra logica. \'{E} la logica della separazione delle figure professionali, con da una parte insegnanti generalisti, con il rischio di essere insegnanti generici; e dall'altra insegnanti specializzati, secondo il criterio di collegare le specifiche specializzazioni ai profili diagnostici corrispondenti per specificità. Questa deriva è, a mio parere, irta di rischi.

Proviamo a indicarne uno:
\begin{enumerate}
	\item Non sono poche le situazioni che si presentano con problemi multifattoriali. Un soggetto disarmonico, potrebbe essere considerato con ritardo mentale, ma anche con problemi psichici. E potrebbe avere qualche disturbo del comportamento alimentare. E’ un esempio, ma non dovrebbe distogliere dal fatto, articolabile in un numero non piccolo di esempi, di situazioni che si collocano nel termine “multifattoriale”. Quale specializzazione mettiamo in gioco? Tante specializzazioni quante sono le sfaccettature del problema?
\end{enumerate}
Proprio i ciechi hanno aperto una strada che chiameremmo del \cit{sostegno evolutivo}, secondo la logica che un soggetto cieco deve poter crescere in un mondo organizzato da chi vede. L'insegnante specialista sa proporre una dinamica evolutiva? Questa è facilitata o ostacolata dalla Cm relativa ai BES? Se la stessa Circolare indica una linea di tendenza che corregge la direzione della deriva specialistica, vedremo in questa una possibilità che si attivi la dinamica evolutiva. Il seguito dovrà essere lo snellimento e la diminuzione delle scartoffie da stilare, delle riunioni sfiancanti, degli scaricabarile umilianti, eccetera.

Se leggiamo questa inversione di tendenza come una premessa e una promessa, consideriamo che accanto ad essa vi sia qualche altro impegno in prospettiva di:
\begin{itemize}
	\item avviare un programma di restaurazione e bonifica degli edifici scolastici, che tenga conto delle reali esigenze di chi vive bisogni educativi speciali.
	\item Contenere il numero di alunni per classe, perché il gruppo classe sia vivibile per chi ha bisogni educativi speciali.
	\item Valorizzare modalità valutative che tengano conto della sincronia dei risultati e aprano alla diacronia delle possibilità.
	\item  Riportare le dirigenze scolastiche alle dimensioni che ne permettano l'assunzione di responsabilità non solo burocratiche, e cioè riducendo drasticamente le reggenze.
\end{itemize}
Queste indicazioni potranno essere prospettiche, e cioè da sviluppare nel tempo. Ma vanno lette in questa possibilità di sviluppo. Sapendo che le nostre cattive abitudini possono rendere difficile la lettura in prospettiva\footcite{Canevaro2013}. 