\author{Salvatore Nocera}
\title{Questa inclusione. Risposta a Goussot}
\label{cha:nocera300713}
\phantomsection
%\epigraph{\hspace*{20pt}}{Salvatore Nocera}
\maketitle
\datapub{30 Luglio 2013}
Torno a parlare di \glslink{besa}{BES}, stimolato dall'intervento del prof. Goussot\pageref{Goussot190713}. Premetto che io non mi intendo di pedagogia (ma solo di normativa), mentre l'intervento di Goussot è giocato tutto sul versante della pedagogia. Però, siccome ho vissuto in prima persona l'integrazione scolastica negli anni Cinquanta, quando ancora essa era di là da essere conosciuta e siccome l'intervento fa alcune considerazioni proprio sulla storia dell'integrazione, mi permetto di intervenire, soprattutto come testimone e studioso della normativa che ha affiancato e sostenuto il processo inclusivo in Italia.

Nell'intervento si sostiene che la recente normativa ministeriale sui \glslink{besa}{BES}, lungi dall'essere l'ampliamento della cultura inclusiva generalizzata, rimarrebbe affetta dalla logica delle categorizzazioni ed, addirittura, delle \cit{classi speciali e differenziali} che la cultura e la prassi dell'integrazione scolastica avrebbe voluto superare. Infatti la normativa avrebbe dapprima regolato gli interventi a favore degli alunni con disabilità; avrebbe poi regolato gli interventi a favore degli alunni con \glslink{dsaa}{DSA} ed avrebbe infine esteso questi anche agli alunni con \glslink{besa}{BES}, termine assai generico in cui si accomunano numerose altre categorie, formandone una terza composita in aggiunta alle due precedenti. Questo processo normativo, lungi dall'aver realizzato l'inclusione generalizzata, avrebbe determinato un'esaltazione delle differenziazioni categoriali.

Questa ricostruzione storica mi lascia perplesso. Infatti io ho frequentato la scuola comune come minorato visivo e debbo ai miei docenti curricolari di allora (quelli per il sostegno non erano ancora neppure nella mente di Giove) che si sono sforzati di aiutarmi a comunicare, tenendo conto che ero un minorato visivo. Mi hanno molto aiutato i miei compagni che pure hanno tenuto conto che ero un minorato visivo, prendendo appunti per me, venendo a casa per leggere ad alta voce in modo che io potessi fare i compiti etc. Per le traduzioni in classe dal Latino e Greco, il docente delle discipline mi si sedeva accanto col vocabolario sulle ginocchia e mi diceva \cit{dimmi il nominativo o il paradigma che ti serve; se me li dici giusti, io ti leggo ciò che è scritto sul vocabolario; altrimenti resto muto}. C'è quindi stata una didattica adattata alla mia situazione personale, che mi ha permesso di superare ottimamente il corso dei miei studi, compresa l'università.

Mi si potrebbe obiettare che io ero solo un minorato della vista, mentre con gli alunni con disabilità intellettive è cosa diversa. E qui è proprio il punto. Allora in una piccola città siciliana, Gela, un minorato visivo in una scuola era un marziano. Eppure la scuola si è adattata a rispondere ai miei bisogni educativi speciali.

In Italia, dopo i primi momenti di \cit{ inserimento selvaggio} dei primi anni Settanta la scuola ha cominciato ad accogliere tutte le differenti tipologie di disabilità cercando di trovare delle didattiche idonee a far superare loro le difficoltà di apprendimento. Ed anche la normativa si è piegata a saper fornire procedure idonee a rispondere ai bisogni educativi speciali, o se vogliamo  \cit{specifici} di ciascuno di loro. Così l'art 16 comma 2 della l.n. 104/92\footcite{Legge_104_92} prevede che nella scuola del primo ciclo di istruzione (primaria e secondaria di primo grado) la valutazione è positiva se gli alunni con disabilità, specie intellettiva, dimostrano di aver acquisito apprendimenti superiori a quelli iniziali, realizzati sulla base di un percorso didattico personalizzato impostato esclusivamente tenendo conto delle loro capacità e potenzialità.Per dimostrare ciò il terzo comma dello stesso articolo 16 prevede l'utilizzo di prove equipollenti a quelle dei compagni, ma differenti nelle modalità e talora anche nei contenuti; inoltre per gli alunni con disabilità fisiche o sensoriali l'art 13 comma 3 della stessa l.n. 104/92\footcite{Legge_104_92} prevedono l'assegnazioni di assistenti per l'autonomia e la comunicazione.

Trattasi quindi di una serie diversificata di interventi a seconda della differente situazione di difficoltà apprenditiva, che però tenta di realizzare un unico obiettivo: quello di assicurare a tutti i  \cit{diversi} la coeducazione coi coetanei nelle classi comuni.

Quello che il MIUR ha fatto con i \glslink{dsaa}{DSA}, a seguito della l.n. 170/2010\footcite{legge170} ed ora con la Direttiva del 27 Dicembre 2012\footcite{dir27Dic2012} e con la C M n.8/2013\footcite{cm8_2013} per gli altri casi di \glslink{besa}{BES}, mi pare sia in linea con questa impostazione che, a sua volta, mi pare applicazione dell'art 3 comma 2 della Costituzione che vuole che la Repubblica superi gli ostacoli che impediscono a tutti l'eguaglianza di fatto.

Trovare didattiche idonee per ciascuno e norme giuridiche che introducano procedure diverse da quelle comuni per consentire a tutti il successo scolastico, secondo le possibilità di ciascuno, non mi pare sia esaltare o ritornare alle categorizzazioni; al contrario mi pare sia uno sforzo per garantire, come dicevano i giuristi romani \cit{a ciascuno il suo}\footcite{Nocera2013}.