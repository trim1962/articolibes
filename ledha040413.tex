Scuola: condividere i cambiamenti con chi li deve attuare
a cura del Gruppo LEDHA Scuola*	

Sta sostanzialmente in questo la critica principale rivolta dal Gruppo LEDHA Scuola alla recente Direttiva Ministeriale del 27 dicembre 2012, sui Bisogni Educativi Speciali (BES) e alla successiva Circolare 8/13, che ne ha approfondito vari aspetti. Il rischio, quindi, è che tali norme difficilmente possano essere applicate, considerando anche la carenza strutturale di risorse nella scuola

Bimbo alla lavagan con aria corrucciataLa Circolare Ministeriale 8/13, esplicativa della Direttiva Ministeriale sui BES (Bisogni Educativi Speciali) del 27 dicembre 2012 [“Strumenti d’intervento per alunni con Bisogni Educativi Speciali e organizzazione territoriale per l’inclusione scolastica”, N.d.R.] ha il notevole pregio di aver posto all’ordine del giorno la necessità della presa in carico collegiale dei BES (ivi compresi, quindi, gli alunni con disabilità) da parte di tutti i docenti, il diritto alla personalizzazione dell’apprendimento attraverso la realizzazione di un Piano Didattico Personalizzato e il diritto al successo formativo di tutti gli alunni con difficoltà. E tuttavia, essa non fuga le perplessità suscitate dalla Direttiva [esposte a suo tempo in un ampio documento, sempre elaborato dal Gruppo LEDHA Scuola, N.d.R.] e sull’attuazione pratica delle misure proposte. Vediamo perché, punto per punto.

1. Disturbi specifici non esplicitati nella Legge 170/10 (DSA – disturbi specifici di apprendimento) – Deficit da disturbo dell’attenzione e dell’iperattività – Funzionamento cognitivo limite
A quanti, come noi, avanzavano dubbi sull’estensione – senza il supporto di una diagnosi clinica certa – delle misure previste dalla Legge 170/10 (stesura del Piano Didattico Personalizzato per ogni allievo; adozione di misure dispensative e di strumenti compensativi), anche ai nuovi casi di BES descritti dettagliatamente nella prima parte della Direttiva, la Circolare 8/13 risponde affermando che «ove non sia presente certificazione clinica o diagnosi, il Consiglio di Classe o il team docenti motiveranno opportunamente, verbalizzandole, le decisioni assunte sulla base di considerazioni pedagogiche e didattiche, ciò al fine di evitare contenzioso».

Dubbi e interrogativi
a) Nel caso dei disturbi evolutivi specifici non esplicitati nella Legge 170/10, elencati nella Direttiva al punto 1.2, ovvero dei disturbi nell’area del linguaggio o, al contrario, nelle aree non verbali, come nel caso del disturbo della coordinazione motoria, della disprassia o del disturbo non-verbale, ma anche del disturbo dello spettro autistico lieve, oltreché nel caso dei disturbi dell’attenzione e dell’iperattività (punto 1.3 della Direttiva), del funzionamento cognitivo limite o borderline (punto 1.4), è credibile che un team docente, “storicamente” non formato a formulare ipotesi differenziali in merito a diverse condizioni cliniche, sia in grado di distinguere, solo basandosi sulla rilevazione dei bisogni e con un discreto margine di certezza tra una situazione certificabile e una non?
b) Si può davvero pensare che basti il dettato ministeriale affinché gli insegnanti sostituiscano una documentazione clinica certa, con la verbalizzazione delle loro «fondate argomentazioni pedagogico-didattiche»?
c) Il ricorso alla certificazione clinica è obbligatorio solo in funzione dell’impegno economico (assunzione dell’insegnante di sostegno o adozione di onerosi strumenti compensativi) e/o della concessione di particolari dispense (vedi esonero dallo scritto della lingua straniera)?
d) È plausibile che si possa, senza una formazione obbligatoria di tutti i docenti sull’utilizzo dell’ICF [la Classificazione Internazionale del Funzionamento, della Disabilità e della Salute, definita dall’Organizzazione Mondiale della Sanità, N.d.R.], superare l’inveterata “logica della certificazione e delle etichettature” (come afferma ad esempio il professor Dario Ianes) e affermare la “lettura dei bisogni per tutti i BES”? Non è proprio la Direttiva nella sua prima parte a fare ampio ricorso a riferimenti clinici, in contrasto con l’ICF, di cui vorrebbe promuovere l’adozione?

A questo punto, purtroppo, è facile immaginare che si otterrà l’effetto opposto a quello della semplificazione, auspicato dalla Direttiva: le famiglie di molti alunni saranno infatti indotte a ricorrere a specialisti, affinché questi riconoscano o escludano una loro “appartenenza” a una delle categorie in cui la Direttiva suddivide i BES (DSA e altri Disturbi specifici; Alunni con disabilità; Alunni con svantaggio socio-culturale e linguistico), con un ulteriore aggravio del lavoro delle UONPIA [Unità Operative di Neuropsichiatria dell’Infanzia e dell’Adolescenza, N.d.R.] e delle strutture private accreditate, la cui cronica mancanza di risorse è sotto gli occhi di tutti ed è causa di pesanti ritardi anche solo nella formulazione delle diagnosi cliniche e delle valutazioni funzionali necessarie, per poter accedere ai Collegi di accertamento degli alunni in situazione di disabilità.

2. Area dello svantaggio socio-economico, linguistico e culturale
Per i bambini in situazione di svantaggio socio-culturale e linguistico (quasi ignorati nella Direttiva e a cui la Circolare 8/13 dà giustamente maggiore rilievo), il problema è sicuramente diverso. «Tali tipologie di BES – si afferma – dovranno essere individuate sulla base di elementi oggettivi (come ad es. una segnalazione degli operatori dei servizi sociali) ovvero di ben fondate considerazioni pedagogiche e didattiche». Per loro, come per gli altri BES, andranno attivati percorsi individualizzati e adottati strumenti compensativi e misure dispensative «per il tempo strettamente necessario», previo monitoraggio dell’efficacia di tali interventi.
La Circolare ricorda quindi l’opportunità, sulla base dell’articolo 5 del Decreto del Presidente della Repubblica (DPR) 89/09, dell’utilizzo delle due ore di seconda lingua nella scuola secondaria di primo grado, per l’apprendimento della lingua italiana.

Dubbi e interrogativi
a) Con quali modalità operative non è però dato sapere: si organizzeranno sottogruppi disciplinari, ma con quali risorse, in assenza di compresenze?
b) E alla scuola primaria, dove manca l’opportunità di una seconda lingua?

Alunna con disabilità in aula affollataPienamente d’accordo, invece, siamo sull’invito della Circolare a privilegiare maggiormente le strategie educative e didattiche, attraverso percorsi personalizzati, anziché fare ricorso a misure dispensative e strumenti compensativi, anche perché i fatti ci dicono che occorrono mesi – quando va bene – per avere un personal computer, peggio ancora per avere software dedicati! Persino i genitori degli alunni con disabilità faticano a ottenerli, anche se previsti nel PEI [Piano Educativo Individualizzato, N.d.R.], perché il Nomenclatore non li contempla e anche quando vi sono Leggi Regionali che ne dispongono la concessione, le pratiche sono lunghe e difficili. Senza contare il fatto che quest’anno, ad esempio, la Regione Lombardia ha concesso i contributi della propria Legge 23/99 per gli strumenti tecnologicamente avanzati – computer e software di base – solo agli alunni con DSA.

Infine, l’attivazione di interventi “on-off” creerà inevitabilmente qualche disaccordo fra i docenti del Consiglio di Classe ed è prevedibile ipotizzare scontenti fra le altre famiglie che, in presenza di situazioni borderline non certificabili o di disagio sociale non tempestivamente supportate dai servizi sociali, potrebbero contestare alla scuola che per alcuni studenti siano stati attivati progetti individualizzati e strategie varie a carattere dispensativo o compensativo e per i propri figli no.

3. Azioni a livello della singola istituzione scolastica
Anche su questo argomento, la Circolare 8/13 introduce elementi decisamente nuovi rispetto alla Direttiva del 27 dicembre 2012, innanzitutto approfondendo argomenti solo inizialmente accennati e apportando aggiunte significative, al punto che vien fatto di chiedere perché il Ministero abbia avuto fretta di deliberare, quando avrebbe potuto stendere un testo organico, che non desse luogo a confusioni interpretative e applicative, dopo avere sentito con più calma l’Osservatorio sull’Integrazione.
Nuova – e molto significativa – è ad esempio tutta la parte della Circolare sui Gruppi di Lavoro scolastici per l’inclusione dei BES. D’ora in poi, quindi, i GLHI che – sulla base dell’articolo 15 della Legge 104/92 – si devono occupare dell’inclusione degli alunni con disabilità a livello di Istituzione Scolastica, tratteranno le questioni relative a tutti i BES, con un nuovo nome: GLI, ovvero Gruppi di Lavoro per l’Inclusione.
Ci soddisfa il superamento dell’identificazione della disabilità come situazione di “confino”, non solo in senso pedagogico-didattico, ma anche in senso lato sociale, giungendo finalmente a considerarla una delle tante condizioni di vita, con pari diritti e opportunità rispetto alle altre situazioni, di “normalità” o di difficoltà.
Positivo anche che la Circolare ricordi finalmente che nei Piani dell’Offerta Formativa (POF) delle scuole non vadano scritte solo generiche enunciazioni di principio, ma siano indicati un concreto impegno programmatico per l’inclusione, criteri e procedure di utilizzo “funzionale” delle risorse presenti nella scuola, «sulla base di un progetto di inclusione condiviso con famiglie e servizi sociosanitari» e l’impegno a partecipare ad azioni di formazione e prevenzione, concordate a livello territoriale.
È il punto sicuramente più alto della Circolare, come è altrettanto significativo che essa reintroduca il tema – ampiamente “dimenticato” – della necessità della rilevazione, del monitoraggio e della valutazione del grado di inclusività che le scuole stesse dovranno effettuare.
Temiamo però – come Associazioni di persone con disabilità – che anche questi documenti, come già le Linee Guida per l’Integrazione Scolastica degli Alunni con Disabilità del 2009, rimangano lettera morta per i troppi “se” e i troppi “ma” che lasciano aperti, perché rischiano di essere disposizioni normative subìte e non condivise, con quanti – in primis i docenti tutti – hanno il dovere e la responsabilità di applicarle, lasciando altresì insoluti vari problemi, a partire dalle risorse insufficienti e dalle scadenze inattuabili, ad anno scolastico ormai inoltrato.
In realtà la scuola “bella”, in cui crediamo, stenta a realizzarsi e si scontra con la dura realtà dei fatti!

Ragazzi di scuola media in classeTra le «risorse specifiche e di coordinamento presenti nella scuola», che dovrebbero integrare i GLHI, la Circolare cita componenti che avrebbero già dovuto far parte di tali organismi a livello scolastico, vale a dire «funzioni strumentali, insegnanti per il sostegno, AEC [assistenti educativi-culturali, N.d.R.], assistenti alla comunicazione, docenti “disciplinari” con esperienza e/o con formazione specifica o con compiti di coordinamento tra le classi, genitori ed esperti istituzionali o esterni in regime di convenzionamento con la scuola».
Ci siamo sempre battute – come Associazioni di Persone con Disabilità – affinché collaborassero alla redazione del PDF e del PEI [rispettivamente Profilo Dinamico Funzionale e Piano Educativo Individualizzato, N.d.R.] degli alunni con disabilità, non solo gli insegnanti di sostegno, ma anche gli insegnanti curricolari, gli specialisti, i genitori e gli educatori (AEC e assistenti alla comunicazione), e tuttavia, in molte, troppe scuole, partecipano alla programmazione solo gli insegnanti di sostegno, o in qualche caso fortunato anche i genitori e qualche insegnante curricolare. Gli educatori, invece, neppure vengono considerati e nonostante la LEDHA (Lega per i Diritti delle Persone con Disabilità) si sia ad esempio più volte pronunciata in tal senso, essi non hanno nemmeno potere di “firma”.
La presenza degli specialisti esterni è divenuta quindi merce rara: i rappresentanti delle Unità Operative di Neuropsichiatria dell’Infanzia e dell’Adolescenza (UONPIA) o degli Enti privati accreditati e gli esperti istituzionali (rappresentanti degli Enti Locali) rarissimamente partecipano ai GLHI e, presso le sedi scolastiche, ai GLHO [Gruppi di Lavoro Handicap Operativi, N.d.R.]. Inoltre, quasi sempre sono gli insegnanti di sostegno – e solo di rado vi si aggiunge un rappresentante degli insegnanti curricolari – a recarsi presso la sede delle UONPIA o delle strutture accreditate, per concordare la programmazione educativo-didattica per gli allievi con disabilità.
È apprezzabile, infine, che la Circolare 8/13 abbia asserito che il completamento del PEI vada effettuato nel mese di settembre, considerato che, solo in rari casi, la stesura del Piano viene conclusa entro la fine di novembre.

Dubbi e interrogativi
a) Riusciranno poi i “nuovi” GLI da subito – visto che la Direttiva è immediatamente operativa – non solo a raccogliere, monitorare e valutare i PEI degli alunni con disabilità, ma anche ad analizzare le criticità e i punti di forza degli interventi di inclusione scolastica operati nell’anno per i tutti i BES presenti nell’Istituzione Scolastica e a stendere entro il mese di giugno il Piano Annuale per l’Inclusività?
b) Buono lo stimolo a che si formino i GLI (visto che ancora in molte scuole non esistono neppure i GLHI), ma pensare che si riuniscano preferibilmente una volta al mese appare sinceramente irrealistico.
Alunni in classec) Perché la Circolare propone che il GLI si riunisca possibilmente con cadenza mensile solo per la parte Risorse (quindi senza rappresentanti dei genitori, degli studenti e delle associazioni)?
d) Quali sono gli «esperti istituzionali o esterni in regime di convenzionamento con la scuola»? E a quali finanziamenti si pensa, visto che in molte scuole scarseggiano le risorse anche per le più elementari esigenze? Infatti, gli «esperti esterni» a volte costano e vengono coinvolti, come fanno i CTI/CTRH [Centri Territoriali per l’Integrazione, N.d.R.], solo “se e quando” arrivano i fondi.
e) Oltre all’autovalutazione del grado di inclusività della scuola, quali strumenti di rilevazione e monitoraggio della qualità dell’inclusione “terzi” verranno impiegati per garantire obiettività di giudizio?
f) Qual è il ruolo dell’insegnante di sostegno? Alla luce della Direttiva e nella Circolare, quale scenario si prefigura? Quando un illustre pedagogista come Dario Ianes parla di «risorse abbondanti per i BES in generale» e le identifica con un investimento massiccio per pagare più di 100.000 insegnanti di sostegno solo nelle scuole statali, pensa solo alla loro compresenze in classe o al loro passaggio in prospettiva al lavoro curricolare (si veda ad esempio la “proposta TreeLLLe”*) e quindi al loro utilizzo per l’inclusione di tutti i BES?
g) Qual è la tempistica per la richiesta di risorse per il sostegno degli alunni con disabilità alla luce del disposto della Circolare 8/13? Se il Piano Annuale per l’Inclusività, redatto (si veda la Circolare) «entro il mese di giugno» e contenente anche le proposte formulate dai singoli GLHO in sede di definizione dei PEI, va successivamente «discusso e deliberato dal Collegio dei Docenti e inviato ai competenti Uffici UUSSRR [Uffici Scolastici Regionali, N.d.R.], nonché ai GLIP e ai GLIR [rispettivamente Gruppi di Lavoro per l’Inclusione Provinciali e Regionali, N.d.R.] per la richiesta dell’organico di sostegno e alle altre istituzioni territoriali come proposta di assegnazione delle risorse di competenza», le richieste relative perverranno agli uffici competenti solo intorno a luglio, ormai in fase di definizione dell’organico di fatto.

4. Azioni a livello territoriale
La Circolare, come già la Direttiva, affida un ruolo fondamentale solo ai CTS (Centri Territoriali di Supporto) per l’inclusione dei BES e nulla aggiunge sul tema delle risorse per il funzionamento dei CTI (Centri Territoriali per l’Integrazione): parla di istituirne di nuovi o di convalidarli se “funzionali”, ma sempre con un ruolo di second’ordine o ausiliario rispetto ai CTS, attribuendo – come già aveva fatto la Direttiva – un’eccessiva enfasi all’impiego delle tecnologie assistive e agli strumenti tecnologicamente avanzati per l’inclusione, mentre nell’ultimo decennio i CTI/CTRH hanno svolto in molte situazioni un ruolo vicariante nell’informazione, consulenza e formazione dei  docenti, delle famiglie e del personale non docente, per le tematiche relative all’inclusione non solo degli alunni con disabilità, ma anche dei DSA, quando le Università non organizzavano corsi di aggiornamento o lo facevano a costi inaccessibili.
Ci sembra quindi un errore non valorizzare debitamente a livello distrettuale – attribuendo anche ad essi autonomia economica e organizzativa, come ai CTS – il ruolo chiave dei CTI, che hanno messo in rete le scuole e valorizzato nel territorio i rapporti interistituzionali.
Non ci sentiamo di condividere neppure la genericità con cui viene valutata (da chi e sulla base di quali parametri?) l’«approfondita competenza» e l’esperienza nelle nuove tecnologie dell’inclusione per l’area della disabilità, la «documentata e comprovata esperienza nel campo» e la partecipazione a master e corsi di perfezionamento per l’area dei disturbi evolutivi specifici. Vi è infatti il fondato timore che la frequenza di uno o più corsi in Didattica e Psicopedagogia per i DSA venga valutata più di anni di attività formativa e progettuale e che i docenti operatori dei CTS e CTI non siano sottoposti anch’essi a una valutazione oggettiva di titoli, meriti ed esperienza.

In conclusione, date le difficoltà applicative della Direttiva, solo in parte risolte nella Circolare, si possono prefigurare due scenari di segno opposto, ma entrambi tali da inficiare o anche solo attenuare le innovazioni positive contenute in questi Documenti Ministeriali: molte Istituzioni Scolastiche, date le oggettive difficoltà applicative e la carenza strutturale di risorse, potranno soprassedere, in attesa che siano emanate disposizioni più precise; altre, invece, cercheranno di attuare – presumiamo con non poche difficoltà – le disposizioni in esse contenute perché “lo dice la legge”.
In realtà, i cambiamenti – soprattutto quelli significativi che incidono sulle relazioni – si affermano non già perché imposti, ma perché condivisi con chi li deve attuare. E questo richiede flessibilità, rispetto e collaborazione.

*Parlando di “proposta TreeLLLe”, ci si riferisce alla proposta presente nel rapporto intitolato Gli alunni con disabilità nella scuola italiana: bilancio e proposte (Erickson, 2011), elaborato dalla Fondazione Agnelli – insieme all’Associazione TreeLLLe e alla Caritas Italiana - ponderoso studio sull’opportunità di mandare la maggioranza degli attuali docenti per il sostegno a insegnare nelle discipline curricolari di rispettiva abilitazione, lasciando solo una percentuale di essi a comporre gruppi di esperti itineranti a livello provinciale o subprovinciale, come consulenti esterni alle singole scuole.

La LEDHA (Lega per i Diritti delle Persone con Disabilità) è la componente lombarda della FISH (Federazione Italiana per il Superamento dell’Handicap).

4 aprile 2013
Ultimo aggiornamento: 10 aprile 2013 10:11