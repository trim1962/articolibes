\author{Carlo Avossa}
\title{Tutti disabili?}
\phantomsection
\label{cha:carlovavossa141013}
\maketitle
\datapub{14 ottobre 2013}
Il MIUR ha inventato la categoria dei “Bisogni Educativi Speciali“. In una Circolare Ministeriale del 6 marzo 2013\footcite{cm8_2013} ha comunicato alle scuole che ogni alunno in difficoltà deve essere considerato alla stregua di un diversamente abile: per ciascuno, infatti, il MIUR impone un Piano Didattico Personalizzato, anche in mancanza di una diagnosi o di un accertamento.

Per chiunque sia in difficoltà: che sia non italofono, che versi in condizioni di disagio socioambientale, che abbia un disturbo specifico dell'apprendimento, un ritardo evolutivo, la sindrome di deficit dell'attenzione, o anche che versi in una generica condizione di difficoltà scolastica… Sono tutti alunni, a detta del MIUR, che hanno “Bisogni Educativi Speciali“.

L'aggettivo “speciale” viene usato, nella letteratura scientifica, per identificare le pratiche scolastiche pensate a favore dei soggetti con disabilità, o diversamente abili. La nuova categoria comprende potenzialmente tutti gli alunni, anche quelli che hanno difficoltà momentanee.

Il provvedimento si può giudicare aberrante per molti motivi:
\begin{description}
\item[--]  La “personalizzazione” è la versione, ideologicamente connotata, di quella che veniva chiamata prima “individualizzazione“. Mentre “individualizzare” vuol dire semplicemente adattare l'insegnamento alle caratteristiche dell'individuo che apprende, “personalizzare” rimanda ad una precisa teoria del filosofo cattolico Maritain, avversario di materialismo ed empirismo, che vuole basare l'insegnamento su dogmi religiosi. Il MIUR prende posizione ufficiale in questo senso: ma può una filosofia essere imposta a tutti?
\item[--] In un momento in cui le scuole vedono un aumento del carico di lavoro del personale, un taglio delle compresenze (nella primaria), un aumento del numero di alunni, una diminuzione delle ore di insegnamento (nella secondaria), il taglio delle risorse finanziarie, il sottodimensionamento degli organici di sostegno, pensare di risolvere i problemi di mancanza di individualizzazione con una Circolare è ipocrita ed irrealistico. I bambini e i ragazzi con difficoltà staranno meglio?

\item[--]La concezione per cui tutti sono diversi e nessuno è normale colloca tutti gli alunni/le alunne “fuori” da un percorso culturale condiviso. Ognuno avrà il suo, “personale“. Ma pensare a una scuola dove tutto è “personale” contraddice le teorie educative attualmente più accreditate, ossia quelle che emergono dal pensiero del grande psicologo e pedagogista Vygostkij. Questi sostiene che il linguaggio, ma anche l'intelligenza, non possono che svilupparsi entro uno spazio sociale. Una scuola in cui non esiste un percorso condiviso ignora questo principio scientifico e diventa un servizio a domanda individuale. In questo modo la scuola non deve “pensare” la differenza di approccio tra l'educazione di un alunno con disabilità comprovata e quella di un altro alunno, del quale non esiste una diagnosi di disabilità. Tutti disabili, allora?

\item[--]Il rapporto numerico tra alunni diversamente abili certificati ed insegnanti di sostegno negli ultimi anni è sempre andato peggiorando e le ventilate assunzioni di docenti di sostegno, se avverranno, consentiranno di fatto poco più di un decente turn-over. L'idea dei BES serve a ridurne o azzerarne il fabbisogno: tutti gli alunni hanno bisogni educativi speciali, perciò tutti i docenti devono rispondervi, divenendo così di sostegno. Tutti insegnanti di sostegno, nessun insegnante di sostegno?

\item[--]Nella direttiva “BES” è previsto che sarà la scuola a “diagnosticare“ difficoltà, disturbi specifici dell'apprendimento o sindromi e patologie di varia natura, sostituendosi ai servizi sanitari di neuropsichiatria dell'infanzia e dell'adolescenza. Quanto saranno precise ed attendibili le diagnosi che una volta facevano i neuropsichiatri ed ora dovranno fare gli insegnanti? Il compito dell'insegnante è quello di un diagnosta?

\end{description}
C'è da chiedersi quale sarà l'esito realistico di tutta l'operazione. Gli insegnanti di classe (posto “comune“) agiranno senza insegnanti di sostegno, saturati su più classi, con classi “pollaio” molto numerose: e in ciascuna di esse avranno uno o più alunni disabili, più quelli con Disturbi Specifici dell'Apprendimento.

Inoltre avranno il dovere di pensare ed attuare, per ciascuno degli alunni in cui sarà vista una difficoltà di qualsiasi tipo, anche temporanea, un Piano di insegnamento “Personalizzato“. Oltre, ovviamente, che seguire la classe.

La conseguenza sarà un inevitabile decadimento dell'azione didattica ed educativa sull'intera classe e l'impossibilità di agire adeguatamente su alunni, con disabilità cosiddetta “lieve“, che spesso sono in grado di lavorare solo se seguiti in maniera individuale.

Questi i prevedibili effetti della geniale pensata dei “BES“.
Che, si badi bene, non è una legge, ma “solo” una direttiva ministeriale.

Invece che di “Bisogni educativi” si dovrebbe parlare di diritti educativi. Ma questa parola è ormai desueta.

C'è da essere indignati.
Ma anche da chiedersi perché, a scuola, nessuno lo dimostra\footcite{Avossa2013}.