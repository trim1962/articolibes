\author{Giuseppe Adernò}
\title{\cit{Abbissiamo} i Bes}
\phantomsection
\label{cha:aderno0107013}
\begin{abstract}
L'inclusione è un traguardo e la scuola è la palestra in grado per farlo raggiungere.
\end{abstract}
%\epigraph{L'inclusione è un traguardo e la scuola è la palestra in grado per farlo raggiungere.}{Giuseppe Adernò}
\maketitle
\datapub{1 Luglio 2013}
La recente circolare ministeriale\footcite{Nota_1551_2013} del 27 giugno, sul Piano Annuale per inclusività, a firma del capo dipartimento del Ministero, Lucrezia Stellacci, fa chiarezza sulla controversa questione dei Bisogni educativi speciali (Bes) nuova sigla che si aggiunge alle molteplici formule che caratterizzano il linguaggio specifico del mondo scolastico.
Sembra proprio che il prossimo anno scolastico “dovrà essere utilizzato per sperimentare e monitorare procedure, metodologie e pratiche anche organizzative” si legge nella circolare e quindi occorre lavorare per \cit{Abbissare i Bes}, espressione che in lingua siciliana vuol dire: \cit{sistemare, dare ordine, organizzare}, e c'è un anno di tempo per fare chiarezza sulla didattica dell'inclusività, operando in convergenza con i docenti di sostegno.
Da altri fronti: sindacali e di contestazioni, viene pronunciata, invece, un'altra espressione: “Inabissiamo i Bes”, quasi per farli sprofondare in un profondo abisso, legandoli ad un grosso macigno per non farli riemergere.
Leggendo le note e i comunicati riguardanti le delibere dei collegi dei docenti di diverse scuole, si evince il disagio e la difficoltà di dare attuazione ad una coerenza azione d'inclusività che, di fatto, nella scuola c'è sempre stata, almeno come intenzionalità educativa. Quando i docenti organizzano la classe per fasce di livelli e quando pianificano degli interventi di recupero e di sostegno o quando pianificano interventi didattici personalizzati e individualizzati, non fanno altro che trovare una risposta ai bisogni educativi degli studenti che sono tutti “speciali” nella loro singolarità e unicità.
Nulla di nuovo o di trascendente, quindi, in una scuola che, attenta all'alunno-persona, risponde ai bisogni del singolo e nel gruppo classe, dove il docente-educatore è capace di “saper guardare tutti ed osservare ciascuno”.Tutti gli alunni, handicappati o non, certificati o non, sono “bisognosi di particolari attenzioni” e la scuola risponde ai bisogni di ciascuno aiutandoli a crescere, ad esercitare abilità, ad acquisire competenze per essere, durante e dopo la scuola veri “cittadini” attivi e responsabili.
La circolare del 6 marzo 2013, in attuazione della direttiva del 27 dicembre 2012\footcite{dir27Dic2012} ha suggerito alle scuole di costituire un gruppo di lavoro per l'inclusione (in sigla GLI) il quale svolge, sulla carta, le seguenti funzioni: rilevazione dei Bes presenti nella scuola; raccolta e documentazione degli interventi didattico-educativi posti in essere anche in funzione di azioni di apprendimento organizzativo in rete tra scuole e/o in rapporto con azioni strategiche dell'Amministrazione; focus/confronto sui casi, consulenza e supporto ai colleghi sulle strategie/metodologie di gestione delle classi; rilevazione, monitoraggio e valutazione del livello d'inclusività della scuola; raccolta e coordinamento delle proposte formulate dai singoli GLH Operativi sulla base delle effettive esigenze; elaborazione di una proposta di Piano Annuale per l'Inclusività riferito a tutti gli alunni con Bes, da redigere al termine di ogni anno scolastico (entro il mese di giugno).
Troppi sono i compiti previsti e ritenendo che bisognasse metterli in atto tutti ed in breve tempo sorge spontanea la domanda: Con quali soldi? Con quale compenso aggiuntivo, giacché il Fondo d'Istituto è stato notevolmente decurtato? Come programmare i bisogni per le classi prime se non si conoscono gli alunni?
Nella recente circolare del 27 giugno si chiarisce il significato ed il valore da assegnare al Piano annuale dell'inclusività” (in sigla:\glslink{paia}{PAI}) che fa parte integrante del Pof in quanto descrive e fotografa la realtà dei “bisogni” della comunità scolastica, ai quali il Piano dell'offerta formativa dovrà dare puntuale risposta.
Il Pai, si legge nella circolare, è un “atto interno della scuola autonoma” e “non va inteso come un ulteriore adempimento burocratico, bensì come strumento che possa contribuire ad accrescere la consapevolezza dell'intera comunità educante sulla centralità e la trasversalità dei processi inclusivi in relazione alla qualità dei risultati educativi”.Così inteso, il Piano annuale dell'inclusione non va confuso o interpretato come un esercizio compilativo, né come “documento” da produrre, né tanto meno diventa il “piano formativo per gli alunni con bisogni educativi speciali”, ma è lo “strumento” per una progettazione dell'offerta formativa in senso “inclusivo” e quindi un'opportunità, quasi una finestra aperta verso una didattica innovativa, attenta ai bisogni di ciascuno nel realizzare gli obiettivi comuni, sollecitando in tal modo una fattiva interazione tra il docente di sostegno e i docenti curricolari di classe in un'operativa azione convergente in vista dell'effettiva integrazione degli alunni disabili o con difficoltà nel gruppo classe, così da poter crescere e camminare insieme.
Qualche tempo fa in una scuola che vantava la fama di essere “di eccellenza” c'era una scritta (ideale): “In questa scuola non ci sono handicappati, né figli di portinaie”, quasi ad indicare la tipologia di utenza e la selezione che veniva operata per le iscrizioni. Quale cultura d'inclusione veniva operata con quella mentalità?
Oggi la scuola, che opera in una società multietnica e multiculturale, non può operare differenze e discriminazioni, ma nel dare a ciascuno secondo i propri bisogni, deve essere scuola aperta e pronta a saper gestire in maniera efficace e nell'ottica del miglioramento, attraverso l'insegnamento e la didattica curricolare anche i “bisogni speciali”.
Ecco quindi la possibilità e l'occasione per riflettere sulla gestione delle classi composte in \cit{maniera equieterogenea}, così da poter attivare la didattica cooperativa, per classi aperte, per moduli organizzativi che superano le barriere delle classi e avvantaggiano la didattica attraverso la formazione di piccoli gruppi anche omogenei, in relazione ai bisogni di ciascuno.
Il Ministero e la Direzione Generale per lo Studente, nella proposta di sollecitare le scuole a mettere in atto specifici e dettagliati \cit{piani annuali per l'inclusività} ha inteso inoltre favorire la socializzazione della best-pratics, quale modello e stimolo di progetti operativi che hanno prodotto efficaci miglioramenti nella didattica e nel rendimento scolastico degli alunni. L'invio della documentazione all'indirizzo: dgstudente.direttoregenerale@istruzione.it avrà, appunto, tale scopo e costituisce un servizio per la crescita della scuola italiana, puntando sulla sinergia di tutti.
La circolare precisa inoltre che la richiesta dell'organico dei docenti di sostegno non è direttamente collegata alla redazione del Pai, ma segue le procedure e le modalità definite dalle singole Direzioni scolastiche regionali.
In quest'operazione documentativa dei Bes, infatti, alcuni docenti, i più maliziosi e i più oppositivi, hanno letto una strategia ministeriale finalizzata alla riduzione dei docenti di sostegno e quindi rifiutano in toto le attenzioni verso gli alunni con “particolari bisogni”; altri auspicano che con tale dichiarazione di Bes possano aumentare i posti di lavoro, magari con personale specializzato per attività didattiche alternative; altri ancora evidenziano un sovraccarico di lavoro per i docenti curricolari, per i quali gli alunni “difficili” o “particolari” sono stati finora un peso da portare avanti ed ora che sono elencati o “schedati” assumono una connotazione specifica per i quali si attendono aiuti, sostegni e risorse aggiuntive.
Il docente, invece che osserva, studia e guida l'intero gruppo classe, in maniera educativa e quindi segue ed è attento ai ritmi di apprendimento di ciascuno, trova nei Bes il riconoscimento di una classificazione che non modifica il suo agire didattico, ma lo conforta nella strada della personalizzazione dell'insegnamento e quindi con accogliente positività per il bene dei suoi alunni ne carpisce tutti i vantaggi e i benefici. Con questi presupposti si può affermare che “il bicchiere è quasi pieno” ed occorre attivare strategie e percorsi alternativi alla didattica tradizionale, supportati, se possibile, da personale esterno aggiuntivo, e mediante l'attivazione di laboratori operativi e finalizzati ad un apprendimento efficace.
La presenza di una figura aggiuntiva di educatore, animatore, esperto di relazione e di comunicazione, potrebbe essere di grande aiuto nella vita della classe per la soluzione di alcuni problemi che rimangono spesso irrisolti e nel tempo, si atrofizzano nella negatività.
Se questi interventi ministeriali aiutano e sostengono la professionalità del docente proiettandola verso una migliore prestazione ed una più elevata qualità di insegnamento, certamente gli studenti delle scuole italiane ne avranno un grande beneficio e la qualità della didattica potrà competere e dialogare con le scuole d'Europa\footcite{Aderno2007}.