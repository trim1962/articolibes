\author{Giovanni Maffullo}
\title{Per non buttare a mare una lunga storia di inclusione }
\label{cha:maffullo100513}
\begin{abstract}
Lo sconforto e la preoccupazione da un insegnante specializzato di sostegno, per il futuro dell'inclusione scolastica, espressi con una serie di riflessioni che arricchiscono ulteriormente il dibattito a più voci, da noi lanciato in queste settimane, su tali fondamentali questioni. \caporali{Per non buttare a mare –- scrive Giovanni Maffullo –- quanto di buono è stato fatto, pur tra mille difficoltà}
\end{abstract}
%\epigraph{Lo sconforto e la preoccupazione da un insegnante specializzato di sostegno, per il futuro dell'inclusione scolastica, espressi con una serie di riflessioni che arricchiscono ulteriormente il dibattito a più voci, da noi lanciato in queste settimane, su tali fondamentali questioni. «Per non buttare a mare – scrive Giovanni Maffullo – quanto di buono è stato fatto, pur tra mille difficoltà»}{Giovanni Maffullo}
\maketitle
\datapub{15 Maggio 2013}
Dopo aver letto su queste stesse pagine l'articolo A proposito di quelle proposte sull'inclusione\pageref{nocera020513} di Salvatore Nocera, vicepresidente della FISH (Federazione Italiana per il Superamento dell'Handicap), che ho il piacere di conoscere e stimare, sento il “dovere deontologico” di inserirmi nel dibattito intrapreso in queste settimane da «Superando.it».

Non sarò breve, ma neppure analitico, cercherò comunque di rendere completa la mia idea con un approccio olistico, partendo dalla situazione attuale in cui operiamo e, in taluni passaggi, potrà anche trasparire una certa acredine, semplicemente dettata, però, dal desiderio di non voler buttare a mare il tutto: se infatti l'integrazione zoppica, in realtà essa cammina e ha conquistato la sua autonomia anche grazie all'azione sistemica posta in essere quotidianamente da tantissimi insegnanti specializzati alias di sostegno.

Gli obiettivi che mi pongo saranno connessi con domande “aperte” le cui risposte auspico vengano amplificate sino alla Direzione Generale per gli Studenti presso il Ministero.
Innanzitutto non mi dilungo su ciò che condivido e che la prassi quotidiana mi aiuta a valorizzare all'interno della scuola (anche se in sedi diverse: aula, palestra, laboratorio, lavoro ad personam…), ovvero collaborare con i colleghi curricolari e compagni dello studente con disabilità, tramite un'azione congiunta volta a sostenere l'integrazione scolastica (le modalità operative sono le più variegate, dal cooperative learning*, alla peer education*, al circle time*, lavori manuali finalizzati al sostegno del protagonismo apprendimentale e formativo degli alunni con disabilità, valorizzando l'autonomia e l'intraprendenza individuale…). In altri termini, condivido pienamente l'approccio epistemico connesso con l'assioma che «gli attori primi dell'inclusione scolastica debbano essere i docenti curricolari e i compagni di classe».

Condivido altresì che la «formazione dei docenti curricolari» e il «ridotto numero di alunni» siano essenziali per garantire il ben-essere di tutto il gruppo classe e valorizzare le dinamiche relazionali che si instaurano non solo fra i docenti (all'interno delle quali l'insegnante specializzato gioca un ruolo determinate, al fine di tutelare e garantire un “sano” percorso formativo, grazie alle chiavi di lettura che egli spesso dà al medesimo fenomeno osservato. Avete un'idea di quali tensioni si creino fra colleghi, specialmente quando si affronta l'argomento BES [Bisogni Educativi Speciali, N.d.R.], dietro al quale si celano esigenze diversissime e precise istanze di giovani persone?), ma anche fra i docenti e gli alunni e in seno agli alunni stessi (avete un'idea dell'immenso lavoro che bisogna oggi fare in ambito emotivo-relazionale con gli alunni?).

E ancora, condivido il fatto che si debba andare oltre il modello bio-medico individuale che ha determinato la “sanitarizzazione” delle situazioni degli alunni, ostacolando l'approccio olistico e l'azione sistemica, secondo il quale le varie risorse istituzionali e umane dovrebbero cooperare nella co-realizzazione del tanto agognato progetto di vita. Non è proprio tale approccio, volto ad abilitare il singolo individuo, che ha ostacolato e ostacola tutt'oggi l'affermarsi del modello bio-psichico-sociale dell'ICF [la Classificazione Internazionale del Funzionamento, della Disabilità e della Salute, definita dall'Organizzazione Mondiale della Sanità nel 2001, N.d.R.] Non è proprio la volontà di “ria-bilitare” il disabile che ha impedito di usare strumenti psico-pedagogici volti a promuovere il cambiamento sia dell'articolazione della classe sia del modo di lavorare con gli studenti? (Sapete quale fatica si deve fare per superare la barriera del lavorare con tutta la classe e solo in quell'ora, per poter integrare mediante azioni di sensibilizzazione, recupero e sostegno?…).

Ci sono delle rigidità strutturali che vengono superate con flessibilità grazie allo sforzo individuale di taluni insegnanti, ma l'organizzazione non aiuta (sapete la fatica che si fa per articolare un semplice quadro orario atto a permettere allo studente in situazione di gravità di partecipare alle varie attività programmabili e calendarizzabili nelle varie materie?).

E l'elenco sarebbe ancora lungo, nel testimoniare in vivo le energie profuse dal Gruppo di Lavoro Handicap Operativo (GLHO), per poter sostenere un'effettiva integrazione nell'incedere quotidiano.
\section*{Formazione dei docenti e ripartizione della retribuzione}

L'attuale formazione dei docenti curricolari nell'area della disabilità (Nota Ministeriale n. 174 del 2013) prevede un percorso di tirocinio attivo effettuato a scuola pari a 75 ore. Quest'anno, segnalo, mi trovo a seguire una giovane collega che conseguirà un'abilitazione presso il Politecnico di Milano e che farà, essendo iniziati tardivamente i TFA [Tirocini Formativi Attivi, N.d.R.], “ben” 25 ore di tirocinio sul campo, anziché le 75 previste… È questo il modo con cui il MIUR preparerà i futuri docenti delle varie materie nell'ambito dei BEI [Bisogni Educativi Individuali, N.d.R.] È questa la risposta alle istanze delle associazioni di avere insegnanti capaci e competenti nell'affrontare con il gruppo classe i vari bisogni soggettivi?

A proposito poi della questione di ripartire la retribuzione degli insegnanti di sostegno ai docenti curricolari, ricordo che per assumere tale incarico, ho dovuto conseguire una specializzazione che – ai tempi dei corsi biennali di specializzazione polivalente – era molto impegnativa (circa un terzo delle 1.300 ore di corso erano rappresentate da tirocinio diretto e indiretto), ma il Ministero, pur richiedendone il conseguimento, non l'ha mai riconosciuta in termini economici e mai è stata fatta oggetto di contrattazione fra Sindacati e ARAN [Agenzia per la Rappresentanza Negoziale delle Pubbliche Amministrazioni, N.d.R.]. In altri comparti pubblici – sanità in primis – ogni specializzazione, se richiesta, viene incentivata economicamente o determina un inquadramento economico diverso.

Quindi, negli anni, noi insegnanti specializzati abbiamo subito un danno economico, ma, in realtà, ora non solo siamo considerati come “insegnanti di serie C” (la “serie A” si identifica con i curricolari “teorici”, la “serie B” è rappresentata dagli ITP, la vituperata categoria di “insegnanti tecnico-pratici”, in via di estinzione grazie alla Riforma “Gelmini”, che in realtà cela ulteriori tagli), ma addirittura dobbiamo “rientrare nei ranghi”.

E l'agognata unicità della funzione docente dove è andata a finire? Il problema è “rientrare dalla spesa” che gli insegnanti specializzati rappresentano e che ad oggi è ancora l'unica voce economica che nel comparto della Pubblica Istruzione non si riesca a ridurre.

A mio parere, quindi, si sta consolidando un futuro “atipico” e che nel tempo di un lustro andrà in porto: gli insegnanti specializzati ridiventeranno curricolari, una “task force” di loro si trasformerà in specialisti e consulenti e verrà collocata presso gli avanzanti CTS [Centri Territoriali di Supporto, N.d.R.-]. Essi verranno quindi chiamati dai Consigli di Classe, all'occorrenza, in veste appunto di specialisti.

Inciso necessario: dalle nostre realtà scolastiche si evince che dopo l'intervento dello specialista neuropsichiatra infantile, psicologo ecc., gli insegnanti restano soli a operare nella quotidianità, senza alcun feedback né supervisione, ma, laddove è presente l'insegnante specializzato, egli comincia a tessere una rete di interazioni e interscambi che conducono a sintesi, ovvero a progettare un percorso che poi si fa vedere agli specialisti. Questi ultimi subito “capitalizzano”, ovvero se ne accaparrano, inserendolo nel fascicolo sanitario dell'alunno. Ma chi conosce lo stillicidio di energie profuse per poter far dialogare le varie Istituzioni? Se la scuola con i suoi operatori non si muove, gli altri lo fanno? No, attendono. Le altre Istituzioni – secondo un approccio psicologico chiaro – comunicano alla scuola: «Se tu hai un bisogno, vieni da me, altrimenti….»; di contro, l'approccio pedagogico è e deve essere diverso: «Se il bisogno non emerge spontaneamente, cerco di stimolare l'ambiente, vado incontro alla “domanda” di aiuto “incistata”» (cos'altro è il sempre più diffuso disagio giovanile, se non questo?).

In altre parole, i nostri specialisti socio-sanitari attendono nel loro setting (alias studio), ove tutt'al più ascoltano e forniscono miserrimi input-conoscitivi, ma lungi dal fornire indicazioni pragmatico-operative in chiave pedagogica e formativa (e d'altronde come potrebbero farlo, se gli esperti in termini metodologici e didattici siamo noi insegnanti? Ma molti curricolari attendono dallo specialista la panacea. Non succederà così anche per gli insegnanti specializzati che verranno allocati presso i CTS?).

E allora, chi si occuperà sul campo di ricercare neo-approcci strategici al servizio del successo formativo dell'alunno-persona? E a maggior ragione con gli alunni con BEI [Bisogni Educativi Individuali, N.d.R.]? Una buona preparazione e il desiderio di sperimentare anche empiricamente aiutano gli insegnanti a “osare” per raggiungere la “zona prossimale di sviluppo” e in questa azione formativa l'insegnante specializzato è di norma presente, espletando almeno quattro funzioni: monitoring, tutoring, coaching, evaluation [“monitoraggio”, “tutoraggio”, “addestramento” e “Valutazione”, N.d.R.].
\section*{Gli alunni nativi digitali e il futuro}

In realtà ci si occupa di scuola e di formazione solo allorquando a scuola ci sono i propri figli. Dopodiché, dell'Istituzione Scuola – che insieme alla famiglia concorre alla formazione dei futuri cittadini – ci si disinteressa. Alcune riflessioni in sequenza.

Le scuole possono e riescono ad ampliare la loro offerta formativa – POF [Piano dell'Offerta Formativa, N.d.R.] - solo ed esclusivamente grazie ai numerosissimi e ingenti contributi volontari che le famiglie elargiscono alla scuola stessa, altrimenti, se va bene, si limiterebbero a fare “semplice istruzione”, altro che formazione delle persone!

Gli studenti, nativi digitali, sono letteralmente “bombardati” da innumerevoli stimoli anche socio-relazionali, a tal punto che allorquando instaurano un contatto reale con gli amici (anziché virtuali tramite social network), tendono ad allontanarsi dal compito. Si ha un'idea della fatica quotidiana che occorre fare in veste di adulti nel richiamare i propri “pupilli”, sempre connessi, alla realtà reale? Si ha un'idea della fatica che bisogna fare per orientarli al compito? E si sa, grazie alla compresenza dell'insegnante specializzato che gira anche fra i banchi, quanti input propositivi vengono forniti al gruppo degli studenti?

Il riconoscimento sociale della funzione docente si è depauperato progressivamente e ciò è sotto gli occhi di tutti, tal ché i docenti sono “scesi in trincea”. Innanzitutto c'è la “burocrazia” – ovvero ottemperiamo a tutti gli aspetti procedurali e formali, ivi inclusi PDP e PEI [Piano Didattico Personalizzato per gli studenti con disturbi specifici dell'apprendimento-DSA e Piano Educativo Individualizzato per gli studenti con disabilità, N.d.R.], poi pensiamo – se c'è tempo – a parlare degli alunni in quanto persone. Si ha un'idea di quanto tempo necessiterebbe per poter approntare per ogni studente un piano di studi personalizzato (PSP), come quello previsto dalla Riforma Moratti e ribadito nella recente Circolare 8/13\footcite{cm8_2013} afferente ai Bisogni educativi Speciali, al fine di poter soddisfare sia le istanze di “eccellenza”, sia di sostenere le esigenze specifiche connesse con i BEI (Bisogni Educativi Individuali)? Quale professione di aiuto è così costantemente caricata di lavoro, molto sommerso, e contemporaneamente misconosciuta? (Come lavoratore dipendente, come faccio a tenere alta la motivazione se occorre far leva solo su quella intrinseca?).
Il mio fondato timore è che fra non molto il Ministero potrà dichiarare – come sostengono le associazioni – di avere formato tutti i neo-insegnanti curricolari, potendo quindi affermare, in occasione del prossimo rinnovo contrattuale (l'ultimo vero Contratto Collettivo Nazionale di Lavoro è stato posto in essere nel 1988!), di avere preparato con specifici master docenti che si occuperanno di disturbi specifici dell'apprendimento (DSA), presso i Centri Territoriali di Supporto, distaccando poi degli specialisti nell'area della disabilità (insegnanti specializzati “superpreparati” in ambiti specifici, quali i disturbi pervasivi dello sviluppo in primis), arrivando in tal modo a contenere finalmente le spese anche per gli insegnanti specializzati che rientreranno in “cattedra”.

Ebbene, se questa è la prospettiva – non si afferma che siamo tutti uguali nei diritti, ma tutti diversi nei bisogni? Che occorre dare di più e compensare a chi in termini neurobiologici ha ricevuto di meno? – ho delle perplessità sul come riusciremo nel quotidiano a fornire risposte concrete a chi ha bisogni speciali: persone con disabilità relazionale, persone che non hanno ancora l'uso della “matematica per la vita” (ovvero l'autonomia nell'uso del denaro), che non hanno congrue abilità in ambito emotivo-relazionale, che non sono autonome in termini socio-relazionali o nei “semplici” spostamenti (a proposito dell'orientamento spazio-temporale, se non effettuo uscite sul territorio, come lo acquisisco? Come potrò recarmi a scuola da solo usando i mezzi pubblici? I percorsi di alternanza scuola-lavoro preparati dall'insegnante specializzato chi li farà? Chi tesserà la rete dei sostegni sul territorio anche con le associazioni?).

Non mi dilungo oltre, ma sono sconfortato e preoccupato per il futuro dell'inclusione. E connessa a tale “idilliaca prospettiva”, avrei un'ultima osservazione: è risaputo che sia la nostra Normativa sia le nostre buone prassi – e in Italia, lungo tutto lo Stivale, anche se a macchia di leopardo, sappiamo benissimo che si attuano costantemente – vengono ammirate all'estero, in quanto, pur fra mille difficoltà, abbiamo saputo inserire e integrare gli alunni con disabilità nella scuola comune e di tutti. Ma a chi va ascritto in genere il merito di avere sostenuto il successo dell'integrazione? Innanzitutto allo studente stesso e subito dopo alla rete di sostegno che l'insegnante specializzato ha saputo creare, prendendosi a cuore il percorso formativo dell'alunno. Nei “casi” di gravità, che necessiteranno di un percorso differenziato e personalizzato, chi penserà – andando oltre il Piano Educativo Individualizzato – al Progetto di Vita, realizzando sperimentazioni nell'extrascuola?
\section*{Conclusione}

Non succederà che anziché potenziare e sostenere l'attuale processo di integrazione che zoppica – come potrebbe “correre”, invece, l'inclusione, con sani investimenti “sul capitale umano”! – dovrò osservare fra un decennio quanto ho dovuto di recente assistere in Germania?

Al giovane ragazzo disabile motorio in carrozzina (in Germania le altre “tipologie” di persone disabili frequentano le “scuole speciali”) è stato semplicemente detto – sia da parte dei compagni che del docente curricolare – di permanere da solo in aula, in quanto loro dovevano recarsi in laboratorio a lavorare e quindi si sarebbero rivisti a fine lezione: dopo un'ora…
*Per “cooperative learning” si intende in sostanza una metodologia di apprendimento cooperativo, per la gestione complessiva della classe; per “peer education” la partecipazione attiva degli studenti nei processi decisionali educativi; per “circle time” un metodo di lavoro ideato dalla Psicologia Umanistica, per aumentare la vicinanza emotiva e risolvere i conflitti.\footcite{maffullo1}

Insegnante specializzato e consigliere di orientamento. 
10 maggio 2013

Ultimo aggiornamento: 10 maggio 2013 15:20
© Riproduzione riservata
 
