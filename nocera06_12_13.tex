\author{Salvatore Nocera}
\title{Norme sui BES: meno chiacchiere e più applicazione}
\phantomsection
\label{cha:nocera061213}
\begin{abstract}
Ho forte il timore – scrive Salvatore Nocera – che mentre si discute sulla legittimità della recente normativa sui DSA (Disturbi Specifici di Apprendimento) e sugli altri casi di BES (Bisogni Educativi Speciali), come il disagio e lo svantaggio, si stia trascurando il problema assai più difficile e complicato dell'inclusione degli alunni con disabilità, sulla quale va registrato un indubbio calo di attenzione e di tensione
\end{abstract}
\maketitle
\datapub{6 dicembre 2013}
Il titolo di un mio recente articolo, Basta coi Bes\pageref{cha:nocera021213}. Pensiamo a una scuola inclusiva, pubblicato in un'altra testata [«La letteratura e noi.it», N.d.R.], mi ha attirato molte critiche, come se io volessi quasi cancellare la recente normativa sui BES (Bisogni Educativi Speciali).

Ci tengo allora a precisare che chi scrive è stato tra i primi a difendere tale normativa e a condividerne le ragioni di completamento della visione della scuola inclusiva, per la quale mi batto sin dagli Anni Sessanta sull'integrazione generalizzata degli alunni con disabilità.
Cosa intendevo dire allora con quell'articolo? Semplicemente ciò che voglio dire con questo che sto scrivendo: che si sta facendo un gran chiacchiericcio su un tema pedagogico importante, rischiando però di offuscare il problema di fondo e cioè come garantire in concreto – quotidianamente – il diritto all'educazione e all'istruzione di quegli alunni con BES particolarmente complessi, come sono gli alunni con disabilità. In altre parole, ho forte il timore che, mentre si discute sulla legittimità della recente normativa sui DSA (Disturbi Specifici di Apprendimento) e sugli altri casi di BES, come il disagio e lo svantaggio, si stia trascurando il problema assai più difficile e complicato dell'inclusione degli alunni con disabilità.

Molti, infatti, credono che, avendo ormai approvato in Italia una normativa ampia e articolata, gli alunni con disabilità abbiano finalmente raggiunto pienamente e in modo generalizzato la loro inclusione scolastica di qualità. E invece le numerose lagnanze che le famiglie manifestano e le troppe cause legali da esse proposte stanno purtroppo a dimostrare il contrario.

Certo, non è che l'inclusione degli alunni con disabilità sia un fallimento, ma si va notando negli ultimi anni un crescente divario tra ciò che la normativa afferma e la disapplicazione della stessa. Basti pensare al mancato rispetto del tetto massimo di venti alunni nelle classi frequentate da alunni con disabilità oppure alla mancata presa in carico del progetto inclusivo da parte di molti, troppi, docenti curricolari, specie di scuola secondaria, che lo delegano totalmente ai soli docenti per il sostegno, e ciò per la “legale” mancata formazione iniziale e obbligatoria in servizio dei docenti curricolari stessi, sulle didattiche inclusive. O ancora, si pensi alla mancata collaborazione delle ASL e degli Enti Locali, prevista per legge, ma sempre più generalizzata a causa dei tagli alla spesa pubblica.

Per questo sono assai preoccupato per il futuro dell'inclusione scolastica di qualità di quelli che sono stati i casi più gravi che l'Italia, da sola al mondo, ebbe il coraggio pedagogico e giuridico di affrontare con notevole successo.
A questo punto si potrebbe obiettare che, concentrando l'impegno su casi meno gravi, quali il disagio e lo svantaggio, si gioverebbe a una maggiore attenzione ai casi più gravi concernenti la disabilità. Io però, avendo vissuto tutta la vicenda storica dell'inclusione scolastica, ho esattamente l'impressione opposta e cioè che sia stata proprio l'attenzione alla qualità dell'inclusione scolastica degli alunni con disabilità che ha permesso alla scuola italiana di affrontare con maggiore impegno e formazione quella degli alunni stranieri, degli alunni con DSA e di quelli con altri BES, Bisogni Educativi Speciali o Specifici, che dir si voglia.

La mia impressione può essere confutata e tuttavia non mi sembra possa essere confutato il mio giudizio sul calo di attenzione e tensione sulla qualità dell'inclusione degli alunni con disabilità.

Gli stessi critici più accaniti della recente normativa sui BES hanno dovuto ammettere che, grazie alla Circolare Ministeriale 8/13\footcite{cm8_2013} del 6 marzo di quest'anno e alla Nota Ministeriale n. 2563\footcite{Nota_2563_2013} del 22 novembre scorso, sono stati chiariti molti punti oscuri della Direttiva sui BES del 27 dicembre 2012\footcite{dir27Dic2012} e notevolmente ridimensionato il problema dell'individuazione dei nuovi BES e dei PDP (Piani Didattici Personalizzati) che, inizialmente, sembrava avrebbero dovuto letteralmente “sommergere” la scuola italiana.

E allora , dal momento che è stato definitivamente chiarito con la recente Nota 2563/13 che ormai l'individuazione e la gestione dei nuovi casi è sostanzialmente rimessa ai soli docenti, indipendentemente da eventuali “diagnosi di BES”, occorre ridurre le polemiche su questa importante normativa e concentrarsi di più sulla sua attuazione in termini didattici e sulla ripresa dell'attuazione didattica della normativa sull'inclusione degli alunni con disabilità.

Faciliterebbe per altro questo nuovo slancio operativo l'approvazione della recente norma dell'articolo 16 della Legge 128/13\footcite{Legge_128_2013} sull'obbligo di formazione in servizio di tutti i docenti che abbiano in classe alunni con disabilità o altri BES. E gioverebbe pure l'attuazione del Decreto del Presidente della Repubblica (DPR) 80/13\footcite{DPR_80_2013} sull'individuazione di indicatori di qualità del sistema di istruzione che permettessero anche l'autovalutazione delle scuole sul livello da loro realizzato di didattiche inclusive.

E ancora, nel medio periodo gioverebbe l'attuazione del Decreto Ministeriale 249/10\footcite{DM_249_2010} sulla formazione iniziale dei futuri docenti, comprendente anche un certo numero di crediti universitari formativi sulle didattiche inclusive, un provvedimento che ancora non decolla e nel quale occorrerebbe aumentare il numero di crediti formativi per i futuri docenti delle scuole secondarie.

Senza dimenticare nemmeno l'attuazione di un ruolo a sé stante dei docenti per il sostegno, come da articolo 14 della Legge 104/92\footcite{Legge_104_92}, in modo da garantirne una scelta professionale definitiva, evitando l'attuale precarietà.

Occorrerebbe infine la realizzazione dell'organico funzionale di reti di scuole, in modo da garantire una seria continuità docente, la cui mancanza oggi disorienta gli alunni, e specie quelli più fragili. Si potranno finalmente vedere attuate queste importanti norme, comprese quelle più recenti sui BES?\footcite{Nocera2013d}
