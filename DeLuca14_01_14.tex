\author{Gaetano De Luca}
\title{Tutela antidiscriminatoria e inclusione scolastica}
\phantomsection
\label{cha:DeLuca140113}
\begin{abstract}
La tutela antidiscriminatoria – scrive Gaetano De Luca – costituisce l'unico strumento giuridico che consente di difendere e garantire un reale ed effettivo processo di inclusione e di partecipazione, proprio in quanto partendo dall'oggettiva diversità che contraddistingue la condizione di disabilità di un alunno, impone un trattamento disuguale per compensare la situazione di oggettivo svantaggio
\end{abstract}
\maketitle
\datapub{14 gennaio 2014}
Le recenti Ordinanze dei Tribunali che hanno riconosciuto una condotta discriminatoria nell'inadeguata e insufficiente assegnazione di ore di sostegno hanno stimolato un interessante dibattito e sollevato delle perplessità sul rischio di compromettere e pregiudicare i principi dell'inclusione scolastica [nell'elenco qui a fianco, alcuni testi da noi pubblicati, che hanno anch'essi dato vita a tale dibattito, N.d.R.].
Con questa riflessione, vorrei cercare di evidenziare come questi provvedimenti della Magistratura costituiscano in realtà degli importantissimi e preziosi precedenti in quanto danno concreta attuazione al nuovo approccio giuridico-culturale alla disabilità introdotto dalla Convenzione ONU sui Diritti delle Persone con Disabilità\footcite{onu2006}.

La prima perplessità che è stata sollevata riguarda il rischio che il ricorso alla Magistratura per ottenere maggiori ore di sostegno possa consolidare di fatto la tendenza in atto a delegare il processo di inclusione scolastica ai soli insegnanti di sostegno. Sulla base di questi rischi, si ritiene pertanto inopportuno puntare tutto sui ricorsi.
In relazione a questa prima critica, credo che alla luce dell'attuale situazione (docenti curricolari non formati sulla disabilità e classi numerose), la scelta dei genitori di intraprendere la via giudiziaria costituisca una via forzata e necessaria. Nel caso in cui rinunciassero infatti a rivolgersi alla Magistratura, i loro figli rischierebbero di frequentare la scuola con un orario ridotto oppure rimarrebbero in classe senza quella risorsa didattica specializzata e quindi sostanzialmente parcheggiati e isolati dal resto della classe.
È indiscutibile che un reale processo di inclusione si verifichi solo quando si effettuano esperienze e si attivano apprendimenti insieme agli altri, quando si condividono obiettivi e strategie di lavoro, sedendo gli uni accanto agli altri. Dall'altra parte, però, non si può dimenticare che alcuni bambini – per poter stare in classe insieme agli altri – hanno oggi bisogno in ogni caso di un supporto finalizzato a far loro superare la barriera costituita non tanto o non solo dalla propria condizione personale, ma prima di tutto dal contesto sociale, ambientale, culturale e relazionale, che fa fatica a capire e ad interagire con bambini e ragazzi che convivono con una menomazione.
Data per altro l'attuale impostazione e organizzazione della scuola italiana, credo probabilmente che nemmeno un corpo di insegnanti curricolari adeguatamente formati sulle questioni dell'inclusione scolastica e sulla didattica speciale sarebbe ugualmente sufficiente a garantire un reale ed effettivo processo di inclusione, che richiede non solo insegnanti competenti, ma anche vere e proprie figure di supporto personalizzate sulla base delle specifiche esigenze delle diverse disabilità.
Infatti, se da una parte è vero che è l'intera comunità scolastica a dover essere coinvolta nel processo in questione (e non solo una figura professionale specifica), non si può comunque disconoscere la specificità del ruolo e della funzione svolta dall'insegnante di sostegno, che è quella di supportare l'intero Consiglio di Classe nel prendersi in carico il progetto educativo individualizzato dell'alunno con disabilità.
L'insegnante di sostegno non è l'insegnante “personalizzato” dell'alunno con disabilità, a lui assegnato in via esclusiva. Si tratta invece di un insegnante assegnato all'intera classe. Ma non si deve dimenticare che si tratta di una figura strumentale alla realizzazione di un diritto di titolarità dello stesso alunno con disabilità, quello all'inclusione scolastica. L'insegnante di sostegno, infatti, deve aiutare i docenti della classe a individuare le modalità più adeguate per rendere più facile a questi alunni l'approccio allo studio delle diverse materie.
In altre parole, l'insegnante di sostegno è una figura professionale di tutta la classe, ma assegnato con lo specifico obiettivo di consentire a un alunno con disabilità la realizzazione del suo diritto allo studio. Non si può pertanto disconoscerne la funzione fondamentale, dal momento che egli costituisce spesso lo strumento necessario all'alunno per poter essere davvero incluso.

La seconda perplessità espressa sul ricorso alla Magistratura riguarda invece l'utilizzo della tutela antidiscriminatoria al posto del tradizionale ricorso al TAR [Tribunale Amministrativo Regionale, N.d.R.].
Su questo aspetto non riesco a comprendere quali possano essere le differenze in termini di temuti pregiudizi ai principi dell'inclusione. Entrambi i tipi di azione giudiziale tendono infatti a fare ottenere all'alunno con disabilità le ore corrispondenti alle sue effettive esigenze, senonché mentre il ricorso al TAR si focalizza sulla lesione di specifiche normative (quelle sul diritto all'inclusione scolastica e quella sull'assegnazione di ore di sostegno adeguate), chiedendo che il diritto allo studio leso venga non solo accertato ma attuato attraverso la condanna del Ministero ad assegnare maggiori ore, il ricorso al Tribunale Ordinario ai sensi della Legge 67/06\footcite{Legge_67_2006} [“Misure per la tutela giudiziaria delle persone con disabilità vittime di discriminazioni”, N.d.R.] si focalizza sull'effetto discriminatorio di tale situazione.

L'approccio antidiscriminatorio ha proprio il pregio di evidenziare l'effetto che l'inadeguata assegnazione di ore di sostegno può avere sul processo di inclusione scolastica. È evidente, infatti, che un'insufficiente assegnazione di ore di sostegno possa mettere l'alunno con disabilità in una condizione di maggiore svantaggio rispetto ai suoi compagni, proprio in virtù dei suoi bisogni speciali legati a un'oggettiva condizione di diversità. Tale approccio, quindi, ha il vantaggio di non limitarsi ad evidenziare la violazione di una normativa specifica – quella, lo ribadiamo, sul diritto all'integrazione e all'inclusione scolastica e quella sull'assegnazione di ore di sostegno specializzato – ma di dare rilievo a qualsiasi effetto discriminante.

Un altro punto di forza della tutela antidiscriminatoria è poi quello di poter essere utilizzata per dare rilievo a tutte quelle situazioni in cui la Pubblica Amministrazione non venga incontro alle esigenze degli alunni con disabilità, non predisponendo gli strumenti compensativi di cui questi hanno bisogno. Solo con il ricorso antidiscriminatorio, infatti, si può attuare il principio introdotto dalla Convenzione ONU, secondo cui costituisce discriminazione anche la mancata predisposizione di un «accomodamento ragionevole». Ed è proprio l'insegnante di sostegno a costituire l'accomodamento ragionevole volto ad alleviare – con una prestazione aggiuntiva – la situazione di svantaggio della persona con disabilità.

La tutela antidiscriminatoria non può pertanto mettere a rischio i principi dell'inclusione scolastica, ma, al contrario, ne rafforza il suo valore di principio fondamentale in quanto contribuisce a diffondere il nuovo approccio giuridico culturale della Convenzione ONU, secondo la quale la condizione di esclusione, emarginazione, non partecipazione e limitazione nell'esercizio dei propri diritti può essere contrastata e superata attraverso appunto la predisposizione di un «accomodamento ragionevole».

Auspicare un minor ricorso all'insegnante di sostegno, affidando il processo di inclusione ai docenti curricolari – anche se formati e competenti – pur se comprensibile, potrebbe per altro portare a un rischio opposto, quello cioè di abbracciare una logica e un approccio che tenda a considerare gli alunni con disabilità “solo formalmente” uguali agli altri. Un approccio di questo tipo avrebbe l'effetto di mettere l'alunno con disabilità in una condizione di sostanziale svantaggio e quindi di essere discriminato indirettamente.

Ritengo quindi, in conclusione, che la tutela antidiscriminatoria costituisca l'unico strumento giuridico che consente di difendere e garantire un reale ed effettivo processo di inclusione e di partecipazione, proprio in quanto partendo dall'oggettiva diversità che contraddistingue la condizione di disabilità di un alunno, impone un trattamento disuguale per compensare la situazione di oggettivo svantaggio.

Se l'obiettivo dell'inclusione scolastica è quello di portare gli alunni con disabilità in una condizione di piena parità ed uguaglianza rispetto al resto della classe, tale obiettivo non può che essere perseguito (laddove le condizioni di discriminazione fondata sulla disabilità lo rendano necessario) con degli strumenti di supporto, di favore, con azioni positive, in altre parole con degli accomodamenti ragionevoli\footcite{Luca2014}.

