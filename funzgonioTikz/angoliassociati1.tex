% Tikz File 'mytikz.tex'
\documentclass{standalone}
\input{../Mod_base/grafica}

%italianizziomo gli operatori

\DeclareMathOperator{\sen}{sen}
\DeclareMathOperator{\tg}{tg}
\DeclareMathOperator{\cotg}{cotg}
\DeclareMathOperator{\arcsen}{arcsen}
\DeclareMathOperator{\arctg}{arctg}
\DeclareMathOperator{\arccotg}{arccotg}
%\DeclareMathOperator{\sec}{sec}
\DeclareMathOperator{\cosec}{cosec}
\providecommand{\abs}[1]{\lvert#1\rvert}
\providecommand{\norm}[1]{\lVert#1\rVert} 
%\newcommand{\gradi}{\ensuremath{^\circ}}
%\newcommand{\gradi}{\ensuremath{\degree}}
%\newcommand{\minuti}{\ensuremath{\arcminute}}
\newcommand{\PA}[1]{\boldsymbol{\mathcal{#1}}}
\newcommand{\PDA}{\PA{P}}
\newcommand{\numberset}[1]{\boldsymbol{\mathbb{#1}}}
\newcommand{\N}{\numberset{N}}
\newcommand{\Nz}{\numberset{N}^{}_{0}}
\newcommand{\Z}{\numberset{Z}}
\newcommand{\Zn}{\numberset{Z^{-}}}
\newcommand{\Zp}{\numberset{Z^{+}}}
\newcommand{\Q}{\numberset{Q}}
\newcommand{\Qp}{\numberset{Q^{+}}}
\newcommand{\Qn}{\numberset{Q^{-}}}
\newcommand{\R}{\numberset{R}}
\newcommand{\Rp}{\numberset{R^{+}}}
\newcommand{\Rn}{\numberset{R^{-}}}
\newcommand{\nq}{\ensuremath{\mathbb{Q}}}
\newcommand{\bld}[1]{\mbox{\boldmath $#1$}}
\newcommand{\C}{\numberset{C}}
%spazi fantasma
%\newcommand{\spa}{\phantom{1}}
%\let\origcleardoublepage\cleardoublepage
%\newcommand{\clearemptydoublepage}{%
%\clearpage
%{\pagestyle{empty}\origcleardoublepage}%
%}
%\let\cleardoublepage\clearemptydoublepage
%spazio insecabile
%\newcommand{\nbs}{\nobreakspace}
%le costanti
\newcommand{\costante}{\textrm{costante}}
\DeclareMathOperator{\uimm}{j}
  \let\Re\undefined\DeclareMathOperator{\Re}{Re}
  \let\Im\undefined\DeclareMathOperator{\Im}{Im}
\DeclareMathOperator{\mcd}{mcd}
\DeclareMathOperator{\mcm}{mcm}
\input{../Mod_base/unita_misura}
%\usetikzlibrary{...}
\begin{document}
\begin{tikzpicture}[>=triangle 45]
% draw the coordinates

\pgfmathsetmacro{\raggio}{4};
\pgfmathsetmacro{\pangolo}{40};
\pgfmathsetmacro{\sangolo}{{180-\pangolo}};
\pgfmathsetmacro{\mraggio}{\raggio/3};
\pgfmathsetmacro{\sraggio}{1.9*\raggio};
% draw the unit circle
\draw[->] (0,-\raggio-\mraggio) -- (0,\raggio+\mraggio) node[above,fill=white] {$y$};
\draw[->] (-\raggio-\mraggio,0) -- (\raggio+\mraggio,0) node[right,fill=white] {$x$};
\draw[thick] (0,0) circle(\raggio);
\coordinate [label= below left:$O$] (OO)at(0,0);
\coordinate (P)  at ({\raggio*cos(\pangolo},{\raggio*sin(\pangolo});
\coordinate (PX)  at ({\raggio*cos(\pangolo},0);
\coordinate (PY)  at (0,{\raggio*sin(\pangolo});
\coordinate (Q)  at ({\raggio*cos(\sangolo},{\raggio*sin(\sangolo});
\coordinate (QX)  at ({\raggio*cos(\sangolo},0);
\coordinate (QY)  at (0,{\raggio*sin(\sangolo});
\coordinate (M)  at (0,{\raggio*sin(\sangolo});
\node at (P) [label=right:$P$]{};
\node at (Q) [label=left:$Q$]{};
\fill [color=black] (OO) circle (1.5pt);
\fill [color=black] (P) circle (1.5pt);
\fill [color=black] (PX) circle (1.5pt);
\fill [color=black] (PY) circle (1.5pt);
\fill [color=black] (M) circle (1.5pt);
\fill [color=black] (Q) circle (1.5pt);
\fill [color=black] (QX) circle (1.5pt);
\fill [color=black] (QY) circle (1.5pt);
%\draw[->] {(\sraggio/2.5},0 ) arc (0:\pangolo:\sraggio/2.5) node[above] {$\arco$};
\draw(P)--(PY) node[midway,above] {$ \cos(\alpha)$};
\draw(Q)--(QY) node[midway,above] {$\cos(\ang{180}-\alpha)$};
\draw(P)--(PX) node[midway,above,right] {$\sen(\alpha)$};
\draw(Q)--(QX)node[midway,below,left] {$\sen(\ang{180}-\alpha)$};
\draw [->] (0:\sraggio/3) arc (0:\sangolo:\sraggio/3) node[sloped,below right] {$\ang{180}-\alpha$};
\draw [->] (0:\sraggio/4) arc (0:\pangolo:\sraggio/4) node[above] {$\alpha$};
\draw(OO)--(P);
\draw(OO)--(Q);
\end{tikzpicture}
\end{document}