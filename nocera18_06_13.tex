\author{Salvatore Nocera}
\title{Perché non va bloccata quella Circolare sui BES}
\label{cha:nocera180613}
\begin{abstract}
Nell'ambito del nostro dibattito sull'inclusione scolastica – centrato con gli ultimi interventi specie sulla recente normativa riguardante i Bisogni Educativi Speciali (BES) – l'insegnante Carlo Scataglini aveva ribadito la necessità di bloccare l'applicazione della Circolare 8/13, \cit{linea-guida} di quei provvedimenti. Secondo Salvatore Nocera, invece, non è questa la strada giusta e serve una maggiore gradualità
\end{abstract}
%\epigraph{Nell’àmbito del nostro dibattito sull’inclusione scolastica – centrato con gli ultimi interventi specie sulla recente normativa riguardante i Bisogni Educativi Speciali (BES) – l’insegnante Carlo Scataglini aveva ribadito la necessità di bloccare l’applicazione della Circolare 8/13, “linea-guida” di quei provvedimenti. Secondo Salvatore Nocera, invece, non è questa la strada giusta e serve una maggiore gradualità}{Salvatore Nocera}
\maketitle
Facendo il punto, in queste stesse pagine, sull'ampio dibattito riguardante la recente normativa sui BES (Bisogni Educativi Speciali) [il riferimento è alla Direttiva Ministeriale del 27 dicembre 2012\footcite{dir27Dic2012} e alla Circolare Ministeriale 8/13\footcite{cm8_2013}, N.d.R.], Carlo Scataglini\ref{cha:scataglini170613} propone schematicamente e lucidamente tre aspetti positivi – che condivido – e altri tre negativi, oltre a una serie di proposte, molte delle quali mi lasciano assai perplesso.
Veniamo subito alle tre critiche:
\begin{enumerate}
	\item Scataglini si diffonde lungamente, per dimostrare come – a seguito della normativa citata -, verrebbe negato il sostegno agli alunni meno gravi ai quali non si riconoscerebbe più la qualifica di alunni con disabilità, ai sensi dell'articolo 3, comma 1 della Legge 104/92\footcite{Legge_104_92}, poiché il “limite interno” tra quelli certificabili e quelli non certificabili è talmente infinitesimale che quasi tutti verrebbero scartati, finendo così nel gruppo degli altri BES, per i quali non è prevista risorsa alcuna.
	Mi chiedo però perché mai questa separazione dovrebbe essere stata determinata dalla recente normativa sui BES: nessuno, infatti, ha impedito sino ad oggi – se era così facile separare – di effettuarla, tanto più che le ragioni finanziarie che oggi premono sono già presenti in Italia da molto tempo e in modo pressante da almeno tre anni. Eppure, questi tagli al sostegno non ci sono stati a livello nazionale, se è vero che le certificazioni sono aumentate sino a contare oltre 200.000 alunni, con posti di sostegno ad oltre la metà di loro.
	E in ogni caso ritengo che i professionisti delle ASL debbano adottare certi codici previsti dalle Classificazioni ICD10 dell'Organizzazione Mondiale della Sanità, non potendo quindi agevolmente infischiarsene, pena la delegittimazione della loro stessa professionalità e un incremento esponenziale dei ricorsi ai Tribunali Amministrativi Regionali (TAR), ai quali, del resto, le famiglie sono ormai abituate da anni, con costanti esiti positivi.
	\item Si insiste poi sul fatto che, mancando il numero necessario dei docenti di sostegno, a causa dei tagli della loro quantità, ci sarebbe solo una consulenza esterna, effettuata da gruppi di docenti specializzati “itineranti”.
	A me sembra che la recente normativa non dica ciò; questo, invece, è un aspetto significativo dell'ipotesi di lavoro avanzata dalla nota ricerca della Fondazione Agnelli [ci si riferisce alle ipotesi avanzate nel rapporto intitolato Gli alunni con disabilità nella scuola italiana: bilancio e proposte, Erickson, 2011, elaborato appunto dalla Fondazione Agnelli, insieme all'Associazione TreeLLLe\footcite{treellle2011alunni} e alla Caritas Italiana, N.d.R.], che però non è stata sposata dal Ministero. Se si immagina che ci sarà un taglio dei docenti per il sostegno per gli alunni con disabilità lieve, si può anche immaginare che il Ministero abbia preso la strada di quell'ipotesi della Fondazione Agnelli, ma ciò non mi pare corrisponda alla realtà attuale dei fatti.
	\item Scataglini sostiene quindi che l'introduzione del termine BES aumenti le etichette nelle quali “incasellare” e “stigmatizzare” gli alunni con difficoltà di apprendimento.
\end{enumerate}
Invero la Direttiva Ministeriale del 27 dicembre 2012 si dilunga a spiegare che BES «non è un'ulteriore etichetta, ma anzi è il termine riassuntivo di tutti i casi di alunni con difficoltà di apprendimento». Che poi succeda nella prassi – così come si sono volgarmente classificati gli alunni con disabilità come “prima categoria di BES”, quelli con DSA [disturbi specifici di apprendimento, N.d.R.] come “seconda categoria” e i nuovi BES (svantaggio, disagio e altri casi) come “terza categoria” –  non può addebitarsi a questa recente normativa, perché allora occorrerebbe risalire addirittura alla Legge Quadro 104/92 e forse ancor prima. Certo, il rischio c'è, ma non per causa della recente normativa, bensì di una certa tendenza a semplificare le cose complesse.

Per quanto poi riguarda le proposte avanzate da Scataglini:
\begin{description}
	\item[a]  bloccare l'applicazione immediata della Direttiva del 27 dicembre 2012 e della Circolare 8/13 sui BES, per aprire un ampio dibattito nelle scuole, con il concorso di tutti gli interessati operatori della scuola, degli Enti Locali e delle ASL, nonché delle famiglie.
	Ebbene, mentre si condivide la proposta di discuterne anche nelle scuole - ciò che invero sta avvenendo sin dal mese di marzo in moltissime scuole e associazioni – non si vede l'utilità di sospenderne l'applicazione da subito. Ciò, infatti, impedirebbe agli alunni con DSA e con altri BES, individuati dai Consigli di Classe, di avvalersi dei benefici che la recente normativa ha posto a loro disposizione, quali le diagnosi provvisorie di DSA in attesa di quella definitiva dell'ASL e l'utilizzo di strumenti compensativi e dispensativi riconosciuti dai Consigli di Classe – in casi particolari -, anche agli altri BES.
	Inoltre non verrebbe formulato il PAI (Piano delle Attività Inclusive), da attuarsi entro questo mese di giugno, che può costringere tutte le scuole a cominciare a ragionare sui punti di forza e di debolezza delle loro capacità inclusive, anche per richiedere al Governo una migliore distribuzione delle scarse risorse disponibili.
	\item [b]spalmare le attuali risorse disponibili, cioè il numero dei docenti per il sostegno, a favore di tutti i casi di BES (la qual cosa, comunque, richiederebbe pur sempre l'individuazione anche di questi nuovi BES ).
	Per questa decisione, però, occorrerebbe, a mio avviso, un'ampia discussione nel Paese che comporterebbe, come conclusione, la modifica della normativa vigente – confermata e rafforzata dalla giurisprudenza della Corte Costituzionale – secondo la quale il sostegno va assegnato esclusivamente agli alunni con disabilità certificata e a nessun altro. Non è quindi cosa che può farsi immediatamente.
	\item[c] realizzare immediatamente la formazione di tutti i docenti sulla didattica inclusiva.
	È questa una proposta che come FISH (Federazione Italiana per il Superamento dell'Handicap) già da tempo formuliamo, ma occorre una norma che avvii la formazione iniziale in tal senso per tutti i futuri docenti e una obbligatoria in servizio. Questo, però, non può essere fatto immediatamente, perché occorre una Proposta di Legge per la formazione iniziale e accordi sindacali per quella obbligatoria in servizio. Se c'è comunque la convergenza delle associazioni e dei docenti, la cosa ha molte più probabilità di essere realizzata nel prossimo futuro.
	\item [d]creare un nuovo profilo dei docenti per il sostegno che sappiano operare con tutti i casi di BES. Questa proposta necessita dello scioglimento di un nodo fondamentale, ovvero se i docenti specializzati debbano occuparsi di tutti o solo degli alunni con disabilità. E data la denuncia di Scataglini circa i tagli ai posti di sostegno, la proposta mi sembra contraddittoria, perché ridurrebbe invece che aumentare le ore di sostegno disponibili per gli alunni certificati con disabilità, come attualmente avviene.
	\item [e] creare un'apposita classe di concorso per il sostegno, in modo da realizzare una vera e stabile scelta professionale. Anche questa è una proposta della FISH che risale a molti anni addietro e che adesso stiamo concretizzando in una Proposta di Legge ad hoc.
\end{description}

In conclusione, alcune proposte, a mio avviso, sono condivisibili, ma altre – specie se supportate da ipotetiche denunce di tagli al sostegno – non mi sembrano tali. E in ogni caso la democrazia è bella perché nel dibattito di idee diverse, e talora contrapposte, si può pervenire a soluzioni maggiormente ragionate e non frettolose.
Per questo non condivido e non sottoscriverò né farò propaganda per quella Petizione Referendum BES. Fermiamo la CM 8 e costruiamo il cambiamento, ritenendo che la recente normativa abbia se non altro avuto il merito di avere riacceso il dibattito culturale sull'inclusione in Italia che langue da quasi tredici anni.
Certo, quei recenti provvedimenti non sono perfetti e anche noi della FISH ne chiediamo delle correzioni, specie per la parte concernente l'organizzazione e il raccordo dei nuovi organismi (CTI-Centri territoriali per l'Inclusione, CTS-Centri Territoriali di Supporto) con quelli precedenti (ad esempio i GLIP-Gruppi di Lavoro per l'Inclusione Scolastica degli Alunni con Disabilità Provinciali). Però l'analisi di Scatolini mi sembra troppo catastrofista e le soluzioni proposte di tipo “palingenetico”, mentre credo occorra gradualità, purché, ovviamente, non “gattopardesca”.\footcite{nocera6}

Vicepresidente nazionale della FISH (Federazione Italiana per il Superamento dell’Handicap). Il presente testo è già apparso in «www.La Letteratura e noi.it», con il titolo “Recente normativa sui BES: eccessiva visione catastrofica e soluzioni palingenetiche” e viene qui ripreso, con alcuni riadattamenti al diverso contenitore, per gentile concessione.

18 giugno 2013

© Riproduzione riservata