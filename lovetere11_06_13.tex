\author{Daniele Lo Vetere}
\title{Inclusione, gride manzoniane e burocrazia}
\label{cha:danilellovetere1}
\begin{abstract}
Solo \caporali{ridando fondi alla scuola e dignità agli insegnanti e attuando molte altre cose concrete}, scrive Daniele Lo Vetere, si otterrà davvero \caporali{qualità e inclusività nella scuola}, non certo con nuovi \caporali{castelli di burocrazia}, ai quali sembrano invece portare, ad esempio, i recenti provvedimenti ministeriali sui Bisogni Educativi Speciali
\end{abstract}
%\epigraph{Solo «ridando fondi alla scuola e dignità agli insegnanti e attuando molte altre cose concrete», scrive Daniele Lo Vetere, si otterrà davvero «qualità e inclusività nella scuola», non certo con nuovi «castelli di burocrazia», ai quali sembrano invece portare, ad esempio, i recenti provvedimenti ministeriali sui Bisogni Educativi Speciali}{Daniele Lo Vetere}
\maketitle
Visto che Salvatore Nocera, su queste stesse pagine\ref{cha:nocera030613} [“I Bisogni Educativi Speciali, i fatti e le paure”, N.d.R.] oppone fatti a paure, mi permetto anch'io di opporre fatti a quella che mi pare un'eccessiva fiducia nella capacità degli atti giuridici di farsi realtà.

Ricordate le gride manzoniane? Cosa voleva suggerirci quel profondo conoscitore del guazzabuglio del cuore umano e della storia in quelle pagine per i più noiose e inutilmente digressive rispetto alla trama? Più o meno che la realtà sta su un piano intangibile per leggi astratte e vaniloquenti e che, davanti a questo mancato incontro, è perfettamente inutile scriverne di nuove, più complicate e stringenti.

A me pare che al Ministero gli azzeccagarbugli che redigono gride pensino che enunciare qualcosa equivalga ad ottenerla.
Fuor di metafora, come si pensa di risolvere il problema dell'inclusione? Grosso modo così: 
\begin{enumerate}
	\item diminuendo il sostegno (limitandolo ai casi più gravi);
	\item  aumentando a dismisura la fattispecie “Bisogni Educativi Speciali” (BES) e mettendo un po' tutto insieme (dall'L2 [Didattica dell'Italiano a Stranieri, N.d.R.] ai comportamentali, dai deficit cognitivi alla dislessia e via classificando);
	\item  attribuendo per decreto ai docenti poteri taumaturgici (questa la spiego tra un attimo).
\end{enumerate}
 Insomma, un castello di burocrazia.

Concretamente:
\begin{enumerate}
	\item Io constato una diminuzione delle ore di sostegno (potrei citare casi a me noti, ma spero ci si fidi, per brevità). Tuttavia, ammettiamo che abbia ragione Nocera e che il sostegno non sia stato ridotto. Di fronte a una normativa che amplia i casi bisognosi di sostegno, ci sarebbe bisogno di più insegnanti, non della stessa quantità. O no?
	\item Chi stabilirà chi ha diritto al Piano Individualizzato e chi no, se viene a mancare il requisito (ristretto, se si vuole anche troppo, ma almeno chiaro e inequivocabile) della certificazione? Perché Andrea, ad esempio, la cui famiglia vive un evidente disagio culturale e una deprivazione linguistica (anche questo ormai rientra nei BES), ha il PDP [Piano Didattico Personalizzato, N.d.R.] e Mario, che è (o pensa di essere) nella stessa situazione no? Prevedo a questo punto una sfilza di ricorsi (vinti) dalle famiglie perché Mario e Tizio e Caio non hanno ottenuto la didattica individualizzata o, in alternativa, Consigli di Classe che fanno PDP arrotondando abbondantemente per eccesso, pur di evitare eventuali contenziosi.
	
	Se Nocera sa come funziona la scuola oggi, tutto questo dovrebbe suonargli familiare.
	
	Corollario di 1) e 2): se tutto è BES, nulla è BES; ovvero, più BES e meno insegnanti di sostegno (o lo stesso numero) = meno sostegno a chi ne ha davvero bisogno. Inoltre, scrivere PDP e fare riunioni per verificarne l'attuazione è qualcosa che costa tempo. Non sembra a Nocera che questo sia tempo burocratico sottratto al sostegno vero, dentro le classi? Dirà: ma gli insegnanti, se non li costringi con documenti e riunioni di verifica, lasciano il disabile da solo in fondo all'aula. Proprio qui l'esempio delle gride è calzante: la realtà non funziona? E ci faccio su una bella legge! E se invece ci dovessimo piuttosto interrogare su alcune condizioni materiali del lavoro degli insegnanti? (Ne dirò qualcosa alla fine).
	\item Gli insegnanti curricolari, molto ma molto banalmente, non sono in grado di gestire la didattica come i documenti ministeriali vanno avventatamente enunciando. Nessun docente può (magari contemporaneamente, e davanti a 25-30 allievi) insegnare la sua materia, alfabetizzare il ragazzo straniero, ricorrere a una didattica speciale per il ragazzo dislessico e a una per il ragazzo che ha un lieve ritardo cognitivo, considerando che magari, al banco di fianco, siede un comportamentale che non si riesce a contenere.
\end{enumerate}
A me sembra una considerazione di tale buon senso che sono costretto a dedurre che – per entrare al Ministero – i futuri funzionari vengano sottoposti a un esame scrupolosissimo, per verificare che di buon senso siano completamente privi.

Intendiamoci: gli insegnanti curricolari devono fare corsi di aggiornamento su ciascuno di questi àmbiti, per sapere di che si tratta, per collaborare con i docenti ad hoc, per non finire addirittura per ostacolarli. Far gruppo, far circolare le informazioni e le competenze, questa sì che è scuola inclusiva. Ma pensare che lo stesso insegnante possa saper fare tutto è ridicolo. Semplicemente, chi è biondo non può esser bruno.

Davvero Nocera crede che basti dare in mano a un insegnante curricolare un ragazzo straniero bisognoso di alfabetizzazione, dirgli «il corsicino di aggiornamento te l'ho fatto, poi la Circolare parla chiaro, prego, includilo» e dire che questa è scuola dell'inclusività? Io la chiamo frustrazione del docente (che potrà al più navigare a vista, sopravvivere, e con uno spaventoso senso di inadeguatezza) e fallimento didattico e pedagogico nell'apprendimento del ragazzo, che avrebbe bisogno di ben altro.

Mi si perdoni se insisto in particolare sulla L2, ma ciò è per due ragioni: il fatto che anche l'italiano agli stranieri – tutt'altra cosa rispetto alla disabilità e ai disturbi dell'apprendimento – sia finito insieme ai BES, la dice lunga sulla ratio di questi documenti, che fanno un bel minestrone di tutto; poi perché mi sto specializzando in questi mesi proprio in L2. E che cosa ho imparato? Che io non sono un insegnante di L2, anche con una specializzazione (altro che corsicino di aggiornamento!) in tasca: potrò infatti fregiarmi del titolo solo dopo molta esperienza, molto altro studio, insomma dopo una costruzione lenta di competenze professionali. E qui stiamo parlando di chiedere agli insegnanti di fare tutto! Suvvia, siamo seri. Ovviamente, sempre che si vogliano ancora avere insegnanti e non generici badanti. Vogliamo questo? Perché purtroppo le condizioni in cui lentamente la nostra scuola sta slittando sono proprio quelle di luoghi dove sopravvivere alla meno peggio, studenti e docenti.
Quando dunque al Ministero si parlerà di ridare fondi alla scuola e dignità agli insegnanti, quando si lavorerà a un sistema di formazione e reclutamento seri (non: «SSIS, ah no scusate, non va, ecco, TFA, ah no, scusate, non va, ecco, concorso, ah no scusate…» [con SSIS e TFA si fa  rispettivamente riferimento a Scuola di Specializzazione all'Insegnamento Secondario e a Tirocinio Formativo Attivo, N.d.R.]), quando si tornerà a parlare di aggiornamento (di ogni genere, ma anche e soprattutto disciplinare), quando si valorizzeranno gli insegnanti giovani e motivati, invece di dar loro calci nel sedere, tenendoli per decenni nel precariato, quando si dirà che un insegnante competente è un insegnante che sa fare alcune cose bene, non tante male, quando si penserà, magari, a classi “scomponibili” in certi momenti della mattinata, in cui a piccoli gruppi con esigenze diverse si offrirà una didattica diversificata (e quando si faranno molte altre cose concrete che rinuncio a enunciare per ragioni di spazio e di pazienza di chi mi legge), allora otterremo qualità e inclusività: anche sui BES.

Queste sono le condizioni materiali cui bisogna badare e che bisogna riformare: il resto sono Circolari Ministeriali.\footcite{LoVetere1}
11 giugno 2013
