\author{Salvatore Nocera}
\title{Sostegno: troppa grazia, Sant'Antonio!}
\phantomsection
\label{cha:nocera050214}
\begin{abstract}
Giova veramente alle famiglie che promuovono i ricorsi e a un'Amministrazione Scolastica perennemente condannata, l'inarrestabile progressione che vede tutti i TAR, il Consiglio di Stato e la stessa Corte Costituzionale ritenere le ore di sostegno come l'unica risorsa per l'inclusione scolastica degli alunni con disabilità? Non proprio, secondo Salvatore Nocera, che ne spiega approfonditamente le ragioni
\end{abstract}
\maketitle
\datapub{5 febbraio 2014}
Notiamo, negli ultimi anni, una crescita esponenziale del numero di Sentenze dei TAR (Tribunali Amministrativi Regionali) che assegnano il massimo di ore di sostegno, non solo imponendo all'Amministrazione l'assegnazione di una cattedra di sostegno per ogni alunno certificato con disabilità grave, ma anche secondo una nuova interpretazione del rapporto 1 a 1 e cioè un'ora di sostegno per ogni ora di insegnamento, per tutta la durata settimanale di frequenza scolastica dell'alunno.

In tal modo succede che i due concetti di rapporto 1 a 1 possano coincidere nel caso di scuola dell'infanzia o primaria solo mattutina, ma che invece – nel caso di tempo pieno nelle scuole dell'infanzia e primarie e di ordinaria frequenza nelle scuole secondarie (normalmente di almeno 24 ore, ma spesso assai di più) --, si leggano Sentenze in cui vengono assegnate addirittura anche 40 ore di sostegno. Ciò comporta la presenza di più di un'insegnante di sostegno con lo stesso alunno e talora anche di più di due: ad esempio nelle scuole secondarie una cattedra non può superare le 18 ore e quindi per coprire 40 ore di sostegno occorrono due cattedre e un quarto!

Le Sentenze, poi, non si limitano più a casi singoli, ma dovendo decidere su ricorsi collettivi, vengono pronunciate a favore di svariate unità e talora decine di alunni. Il solo TAR della Toscana, ad esempio, già in questo inizio di 2014 si è pronunciato più volte sia in favore dell'intera durata dell'orario di frequenza degli alunni, sia a favore di numerosi alunni insieme (Sentenze 32/14\footcite{TARToscana2014a}, 33/14\footcite{TARToscana2014b}, 54/14\footcite{TARToscana2014c} [della quale si legga già anche in altra parte del nostro giornale\pageref{cha:lancioni240114}, N.d.R.] e 151/14\footcite{TARToscana2014d}). Si ha inoltre notizia di una recentissima Sentenza a favore di undici alunni di un'unica scuola di Pisa, che hanno ottenuto il riconoscimento di 40 ore ciascuno tramite il medesimo provvedimento.
Ovviamente i TAR non si limitano solo alla condanna dell'Amministrazione Scolastica ad assegnare più ore di sostegno, ma anche alla rifusione delle spese di causa e al risarcimento dei danni patrimoniali e non patrimoniali. Quanto poi alla certezza che l'obbligo dell'Amministrazione venga eseguito, si nota nelle Sentenze degli ultimi mesi anche la nomina di Commissari ad Acta nella persona del Direttore Scolastico Regionale e del Capo Dipartimento della Programmazione del Ministero dell'Istruzione, con il compito di provvedere entro un termine ben preciso all'assegnazione delle ore di sostegno riconosciute in sentenza.
E sembra del tutto inutile che l'Amministrazione Scolastica ricorra a degli espedienti, come ad esempio ha fatto l'Ufficio Scolastico di Udine, che ha stipulato un'intesa con l'ASL locale, in forza della quale quest'ultima avrebbe dovuto riconoscere due tipologie di disabilità grave: una “eccezionale” per il rapporto 1 a 1 e una “straordinaria” con rapporto 1 a 2. Tale distinzione, infatti, operata con una semplice intesa amministrativa, contrasta con la disposizione dell'articolo 3, comma 3 della Legge 104/92\footcite{Legge_104_92}, che riconosce un solo caso di gravità, come ha del resto affermato la Corte Costituzionale con la Sentenza 80/10\footcite{SCC_80_2010}. E infatti il TAR del Friuli Venezia Giulia non ha avuto difficoltà ad annullare tale illegittimo provvedimento.
Né, infine, l'Amministrazione può trovare un'ultima trincea di difesa chiedendo al Parlamento di votare una legge che impedisca l'eccessivo ricorso alle ore di sostegno, com'è avvenuto con la Legge 244/07 (articolo 1, commi 413 e 414). Infatti, la Corte Costituzionale – tramite la citata Sentenza 80/10 – ha annullato i suddetti commi, motivando che il nucleo essenziale del diritto all'inclusione scolastica, costituito dalle ore di sostegno, non può essere compresso e violato per motivi di tagli alla spesa pubblica.

Di fronte pertanto a questa deriva, c'è da chiedersi se ciò giovi veramente alle famiglie che promuovono i ricorsi e all'Amministrazione che si vede perennemente condannare.
Infatti, per quanto riguarda le famiglie, esse – se veramente vogliono la coeducazione e l'istruzione dei propri figli con disabilità con i compagni nelle sezioni e nelle classi ordinarie delle scuole di ogni ordine e grado, come da Legge 104/92\footcite{Legge_104_92}, articolo 12, comma 2 – devono rendersi conto che avere un'insegnante per il sostegno per tutta la durata dell'orario scolastico spesso provoca l'esclusione del figlio proprio da quell'integrazione con i compagni che la legge ha voluto garantire. In altri termini, si potrebbe qui applicare il gustoso episodio del fedele che, avendo chiesto a Sant'Antonio la grazia di riuscire a saltare sul proprio somaro, aveva prese uno slancio tale da sorvolarlo e cascare dall'altra parte.
Quanto poi all'Amministrazione Scolastica, c'è da chiedersi se si stia rendendo conto che ormai tutti i TAR – ma anche il Consiglio di Stato e la Corte Costituzionale - ritengono come unica risorsa per l'inclusione scolastica soltanto le ore di sostegno.
Invero, la cultura e la prassi dell'inclusione scolastica, sin dalle origini (fine Anni Sessanta -- primi Anni Settanta) hanno puntato, come risorsa primaria, sulla presa in carico da parte dei docenti curricolari che venivano aggiornati continuamente sulle didattiche inclusive, “sostenuti” da insegnanti specializzati per il sostegno didattico. Ebbene, ad oggi l'Amministrazione Scolastica non sembra proprio rendersi conto di ciò, dal momento che finora non ha provveduto ad attuare percorsi di formazione iniziale sulle didattiche inclusive per i futuri docenti curricolari, né i ricorrenti corsi di aggiornamento obbligatori in servizio.
Per altro, grazie all'emanazione del Decreto Ministeriale 249/10\footcite{DM_249_2010}, in parte si sarebbe già potuto provvedere alla formazione iniziale, ma non risulta proprio che ciò sia avvenuto. Per quanto concerne invece l'aggiornamento obbligatorio in servizio, la recente Legge 128/13\footcite{Legge_128_2013} (articolo 16, comma 1, lettera b) lo ha previsto espressamente per le didattiche inclusive.
Il Ministero, quindi, dispone già di tutti gli strumenti utili a ridare ai docenti curricolari il ruolo primario nella presa in carico del progetto inclusivo, evitando così la disastrosa delega ai soli docenti per il sostegno. Ma vorrà realmente farlo\footcite{Nocera2014b}?
