\author{Daniele Brogi}
\title{L'integrazione, il sostegno e gli insegnanti curricolari}
\phantomsection
\label{brogi260413}
\begin{abstract}
Solo applicando correttamente l'apparato legislativo italiano sull'integrazione scolastica degli alunni con disabilità, ovvero con una coerente interazione tra l'insegnante curricolare e quello di sostegno, si potranno ottenere risultati in linea con norme avanzate come quelle del nostro Paese. Una riflessione che prende spunto da un'altra opinione pubblicata dal nostro stesso giornale
\end{abstract}
\maketitle
\datapub{26 Aprile 2013}
%\epigraph{Solo applicando correttamente l'apparato legislativo italiano sull'integrazione scolastica degli alunni con disabilità, ovvero con una coerente interazione tra l'insegnante curricolare e quello di sostegno, si potranno ottenere risultati in linea con norme avanzate come quelle del nostro Paese. Una riflessione che prende spunto da un'altra opinione pubblicata dal nostro stesso giornale}{Daniele Brogi}
Mi suscita molte perplessità l'opinione del genitore Giuseppe Felacopage\pageref{cha:felaco260413}, apparsa su queste pagine con il titolo Tutti avrebbero dei vantaggi, poiché la ritengo il frutto della rassegnazione a un sistema incapace di offrire quanto realmente necessario, per un apparato legislativo come quello italiano, reputato “fiore all'occhiello” d'Europa in materia di integrazione scolastica, che al contempo deve però fare i conti con gravi carenze sia di applicazione che di conoscenza.
Le norme citate, infatti, prevedono l'insegnante di sostegno come tale per la classe e questo è un passaggio fondamentale nell'interpretazione del ruolo, in quanto troppo spesso anche dove non ve ne sia l'effettiva necessità, tale figura è concepita come una sorta di “baby-sitter” dell'alunno con disabilità, condizione al tempo spesso creatasi per un'anomalia generale del corpo insegnante, in quanto la professionalità della suddetta figura era stata concepita allo scopo di interagire con l'insegnante curricolare, per portare avanti la classe – per quanto possibile – con le medesime tempistiche d'apprendimento, processo voluto anche per diminuire il divario con i compagni da una parte, tra gli insegnanti stessi dall'altra.

Credo dunque che non sia certo lasciando solo un insegnante curricolare in una classe magari di venticinque alunni, con un alunno con disabilità e con programmi ministeriali sempre più complessi, prove INVALSI [Istituto Nazionale per la Valutazione del Sistema Educativo di Istruzione e di Formazione, N.d.R.] ecc., che si possa favorire l'integrazione. Lo si può fare, invece, tramite l'applicazione di quanto concepito dalle leggi, a partire dalla coerente assegnazione di insegnanti, sulla base delle effettive esigenze rilevate nello svolgimento dei Gruppi Lavoro Handicap Operativi (GLHO)\glslink{glhoa}, dove a seguito del riconoscimento della necessità di sostegno didattico con rapporto di uno a uno, vengano erogati coerenti contratti a tempo indeterminato per gli insegnanti di sostegno, incentivandoli nell'intraprendere questa strada formativa e potendo poi assegnarli all'intero ciclo scolastico dell'alunno e conseguentemente della classe, al pari degli insegnanti curricolari.
Non credo pertanto che il problema dell'incoerente compenso economico degli insegnanti curricolari sia risolvibile, “mandando al macero” chi ha intrapreso una strada coscienziosamente più vicina alle ragioni del cuore che non della carriera… Un problema, questo, che sicuramente va risolto, ma non certo a scapito degli alunni con disabilità e degli insegnanti.

Riferendomi infine al titolo stesso dell'opinione da cui ho preso spunto (Tutti avrebbero dei vantaggi), personalmente mi sembra che nel contesto prospettato in quel testo di vantaggi non se ne colgano, nemmeno a livello di premessa.\footcite{Brogi2013a}

Genitore.

26 aprile 2013

© Riproduzione riservata
