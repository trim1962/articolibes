\author{Gruppo LEDHA Scuola}
\title{La  direttiva  del  MIUR  su i bisogni educativi speciali e i dubbi delle associazioni di persone con disabilità}
\label{cha:ledha110213}

%\epigraph{...}{a cura del Gruppo LEDHA Scuola}
\maketitle
La Direttiva del Ministero dell'Istruzione sui B.E.S. (traduzione italiana dell'inglese Special Needs
Education) del 27/12/2012 “Strumenti di intervento per alunni con bisogni educativi speciali e
organizzazione territoriale per l'inclusione scolastica”\footcite{dir27Dic2012}
 è un documento di notevole importanza che prende
atto della complessità di esigenze educative e didattiche di cui oggi il mondo della scuola deve
tener conto nella pratica quotidiana.
I numeri danno ragione dell'allarme, anche se si dispone solo di stime dell'Istituto Superiore della
Sanità e il MIUR fornisce tardi, in modo spesso incompleto e a volte contraddittorio, gli unici dati
certi relativi agli alunni con certificazione di situazione di disabilità (DPCM 185/2006)\footcite{DPCM22_02_06_N_185} e - solo
recentemente - agli alunni con certificazione di Disturbo specifico dell'Apprendimento (DSA, di
cui alla L. 170/2010\footcite{legge170}):
\begin{enumerate}
	\item  Allevi con disabilità: nelle scuole statali sono 202314 nel corrente anno scolastico, il 2,5\%
	della popolazione scolastica totale (7.858.077), come si desume dai dati \glslink{sidia}{SIDI} forniti il
	13/12/2012 dal MIUR alle OOSS, che però non includono i numeri degli alunni con
	disabilità che frequentano le scuole non statali (nell'ultimo dato circolante nell'anno
	scolastico 2009/2010 erano 16.217);
	\item Allievi con \glslink{dsaa}{DSA}: sono 70.000 quelli finora certificati ma, a detta del Ministero, le indagini
	scientifiche propendono per una stima che va dai 200.000 ai 350.000 alunni;
	\item Allievi con funzionamento cognitivo limite (o borderline) : i bambini o i ragazzi il cui
	Quoziente Intellettivo Globale risponde a una misura che va dai 70 agli 85 punti e non
	presenta specificità: sono circa il 2,5\% del totale degli alunni, circa 200.000;
	\item  Alunni con deficit da disturbo dell'attenzione e dell'iperattività, definiti con l’acronimo
	ADHD (dall'inglese Attention Deficit Hyperactivity Disorder): l'Istituto Superiore di Sanità
	stima che siano l’1\% della popolazione scolastica italiana, cioè all'incirca 80.000 alunni.
\end{enumerate}

Senza tener conto di altre importanti problematiche, connesse alla frequenza degli alunni in
situazione di svantaggio socio-culturale e degli alunni stranieri non nati in Italia, si parla di circa il
9-10\% della popolazione scolastica italiana.
La Direttiva Ministeriale nella prima Parte illustra con chiarezza l'entità e le caratteristiche dei
Bisogni Educativi Speciali che chiedono di essere soddisfatti in una scuola, qual è la nostra, che
da più di trentanni, a partire dalla L. 517/1977\footcite{Legge_517_77}, è stata assunta come “punto di riferimento per le
politiche di inclusione in Europa e non solo” e i cui principi “hanno contribuito a fare del sistema
di istruzione italiano un luogo di conoscenza, sviluppo e socializzazione per tutti, sottolineandone
gli aspetti inclusivi piuttosto che quelli selettivi”.
L'inclusione nella scuola italiana è in sofferenza, ma le soluzioni che la Direttiva propone nella
seconda parte del Documento ci sembrano decisamente poco convincenti, come ha già avuto
modo di rilevare in un'approfondita analisi Salvatore Nocera\ref{cha:nocera310113}.

CERTIFICAZIONI: la Direttiva non dice nulla sulle certificazioni dei bisogni educativi speciali, ad
esclusione degli alunni con disabilità e con DSA, per cui la normativa ha stabilito che
l'inquadramento diagnostico deve pervenire da centri pubblici o da centri privati accreditati,
specialisti nella branca della patologia evidenziata. Dal momento che in più passi della Direttiva si
sottolinea la necessità della formulazione, anche per gli allievi con funzionamento cognitivo
limite e disturbi dell'attenzione, del Piano Didattico Personalizzato, e del ricorso alle misure
compensative e dispensative previste della L. 170/2010\footcite{legge170}, i Consigli di Classe dovranno
inevitabilmente prendere visione di una documentazione clinica certa. Poco importa se nel
Documento si invitano gli insegnanti a ispirarsi alla Legge 53/2003\footcite{Legge_53_2003} e alla Legge 170/2010, come
norme primarie di riferimento “senza bisogno di ulteriori precisazioni di carattere normativo”.
\begin{itemize}
	\item  Cosa succederebbe se gli insegnanti, in forza di una valutazione di disturbo
	dell'attenzione, chiedessero per un allievo l'acquisto o la cessione in comodato
	gratuito di un pc o di altre tecnologie, e la diagnosi venisse di lì a poco sconfessata
	perché sono mutate le condizioni dell'alunno?
	\item \'{E} pensabile che i Docenti si predispongano a stendere Piani Didattici
	Personalizzati, a chiedere consulenze, istruzioni, supervisione e formazione per il
	reperimento e l'utilizzo di ausili senza l'avallo di un preciso dettato normativo?
	Sono in grado, spesso senza alcuna formazione né iniziale né in itinere, di stendere
	una programmazione individualizzata nella prospettiva della piena inclusione?
	\item I Consigli di Classe – sostiene S. Nocera - dovendo applicare a tutti i BES le misure
	compensative e dispensative previste dalla L. 170/2010. dovranno disporre
	inevitabilmente di una documentazione clinica certa e dovranno formulare
	“considerazioni di carattere psicopedagogico e didattico non discutibili al fine di
	evitare contenziosi con altri alunni ai quali tali benefici non vengano concessi”.
\end{itemize}

PRESA IN CARICO E FORMAZIONE DEL PERSONALE SCOLASTICO: nella prima parte della Direttiva
ci appare positiva la sottolineatura della necessità che la presa in carico degli alunni con Bisogni
Educativi Speciali sia collegiale. Riprendendo le Linee Guida per l'Integrazione degli Alunni con
Disabilità del 2009\footcite{LineGuida2009} e in attuazione dell'art. 19 della L. 111/2011 di Stabilizzazione Finanziaria\footcite{legge_111_2011}, il
Documento sottolinea l'importanza della formazione dei dirigenti e degli insegnanti curricolari
sull'inclusione scolastica degli alunni con disabilità. Degli alunni con BES deve occuparsi tutto il
personale scolastico, non solo le figure che attualmente e impropriamente vengono identificate
come le uniche risorse disponibili per realizzare l'inclusione scolastica, ovvero gli insegnanti di
sostegno e gli assistenti educatori assegnati dall'Ente Locale.

Una spinta propulsiva in tal senso è stata data dalla Legge 170/2010 e dalle Linee Guida del
12/7/2011\footcite{LineGuida2011} sui DSA entrate in vigore nell'a.s. 2011/2012: per gli alunni con Disturbi Specifici non è
prevista dalla legge l'assegnazione di alcun supporto specializzato (v. insegnante di sostegno), ma
sono obbligatorie sia la certificazione, sia la stesura di un Piano Didattico Personalizzato, con
l'indicazione delle misure compensative e dispensative da mettere in atto per garantire
l'inclusione. \'{E} per questo motivo che, pur non essendo più obbligatoria per contratto di lavoro la
formazione iniziale e in servizio sulla didattica speciale degli insegnanti curricolari, i 35
corsi/master attivati nel 2011/2012 sulla “Didattica e psicopedagogia dei disturbi specifici
dell'apprendimento” (definiti dal Documento impropriamente “specializzazione”) hanno visto
una massiccia adesione di insegnanti curricolari e registrato 12.000 domande di iscrizione a
fronte di 3.500/4.000 posti disponibili.

La Direttiva sui BES esporta il modello di presa in carico dei DSA a tutti i Bisogni Educativi Speciali,
alunni con disabilità inclusi: oltre a valorizzare la presa in carico da parte anche degli insegnanti
di classe, il Documento in più passi sottolinea inoltre la necessità di assumere un approccio più
“educativo” che clinico, che faccia superare l'identificazione degli alunni con disabilità con il loro
deficit e sulla base unicamente della certificazione, definita troppo stigmatizzante e angusta,
anche alla luce del modello diagnostico \glslink{icfa}{ICF} dell'OMS, che mette in evidenza la complessa
interazione esistente tra le condizioni di salute dell'individuo e i fattori ambientali e personali. La
medesima certificazione può infatti avere esiti completamente diversi se il contesto presenta
barriere o facilitatori allo sviluppo della persona.
\begin{itemize}
	\item Ma, ci chiediamo, quanti insegnanti conoscono l'ICF?
	\item Quanti tra loro conoscono gli ausili, sanno dove provarli e farli provare all'allievo,
	conoscono e suggeriscono alla scuola e all'allievo l'iter per acquisirli e le
	metodologie più idonee per il loro impiego?
	\item Ha mai fatto il Ministero un sondaggio tra i docenti, specialmente quelli della
	scuola secondaria di primo e secondo grado, raccogliendo le loro opinioni sul
	concetto di “presa in carico” da parte dei docenti curricolari?
	\item Ha mai fatto il Ministero un sondaggio tra i docenti, specialmente quelli della
	scuola secondaria di primo e secondo grado, raccogliendo le loro opinioni sul
	concetto di “presa in carico” da parte dei docenti curricolari?
	\item Come può un professore di una disciplina, che non ha mai trattato la didattica
	inclusiva, garantire allo studente con disabilità (intellettiva, ad es.) un successo
	formativo e un percorso scolastico pieno, in cui l'allievo possa giustamente godere
	di tutte le opportunità formative, al pari dei suoi compagni?
\end{itemize}
I CENTRI TERRITORIALI DI SUPPORTO (\glslink{ctsa}{CTS}) IN RETE PER REALIZZARE L'INCLUSIONE SCOLASTICA
DEI BES: Il Ministero attribuisce, nella seconda parte della Direttiva, soprattutto ai CTS, istituiti
presso scuole polo e presenti uno per provincia (tranne che nelle città metropolitane, dove
saranno in numero superiore), l'organizzazione territoriale dell'inclusione scolastica degli alunni
con Bisogni Educativi Speciali.

Attualmente sono 96 su tutto il territorio nazionale: quindi, se passasse il riordino previsto con il
decreto dalla spending review, attualmente congelato e da ridiscutere entro la fine del 2013,
andrebbero dimezzati. Sono stati creati nel 2006 nell'ambito del Progetto “Nuove Tecnologie e
Disabilità\footcite{MIUR2006}” (\glslink{ntda}{NTD}) e dovrebbero collaborare nella definizione di una rete di supporto al processo
di integrazione con altre risorse territoriali: i GLH delle singole scuole, non meglio identificati GLH
di rete o distrettuali, i Centri Territoriali per l'Integrazione (\glslink{ctia}{CTI}, CTRH in Lombardia) a livello di
distretto sociosanitario, i Glip provinciali e i Glir regionali, previsti dalle Linee Guida del 2009\footcite{LineGuida2009}.
La Direttiva attribuisce ad essi un ruolo chiave per l'inclusione dei BES, soprattutto in
conseguenza dell'importanza attribuita dalla normativa sui DSA alle misure compensative (mezzi
di apprendimento alternativi e tecnologie informatiche: pc, sintesi vocale, calcolatrice, dizionari
digitali) ma negli ultimi due anni gli stanziamenti dei fondi per l'integrazione scolastica degli
alunni con disabilità (ex L. 440/97\footcite{Legge_440_97}), che pure sono diminuiti, sono stati in parte destinati a
finanziare i corsi/master per i DSA e sono stati parallelamente finanziati dal MIUR i CTS,
identificati come le strutture necessarie al territorio per la proposta di adeguate tecnologie
assistive non solo per le disabilità, ma anche per i DSA.
Il Documento prevede che operi all'interno dei CTS un'equipe di tre insegnanti curricolari e di
sostegno, comandati (non si capisce se a orario pieno o parziale) per almeno un triennio,
affiancata al bisogno da esperti in grado di offrire “soluzioni rapide e concrete per determinate
problematiche funzionali”, e fissa stanziamenti del MIUR e degli Uffici Scolastici regionali per
coprirne le spese, di informazione/formazione, acquisto ausili e funzionamento.
I CTS dovrebbero fare di tutto: oltre ad occuparsi di informazione sugli ausili, sul loro acquisto,
sul loro utilizzo e sulle competenze e pratiche didattiche che ne rendano efficace l'uso, devono
offrire anche consulenza agli insegnanti, organizzare formazione, promuovere la valorizzazione e
la diffusione di buone prassi, la ricerca didattica e la sperimentazione di nuovi ausili.
Gradualmente i CTS estenderebbero il loro intervento non solo alle tematiche connesse alle
nuove tecnologie, ma a tutto l'ambito della disabilità, e dei disturbi evolutivi specifici.
\begin{enumerate}
	\item Puntando tutto sui CTS, organismi previsti da soli atti amministrativi, a discapito di
	altre realtà sorte dal basso e consolidate nel territorio come i CTI/CTRH, che sono
	coordinamenti territoriali per l'integrazione degli alunni con disabilità già
	funzionanti a livello di distretto socio-sanitario di scuole, che si sono associate con
	Accordi di rete formalizzati, dopo un lavoro ormai decennale, di intervento sulle
	scuole e sul personale scolastico, non si rischia di buttare il bambino con l'acqua
	sporca?
	\item La Direttiva sembra voler attribuire gradualmente il lavoro di consulenza,
	informazione e formazione, che molti CTI svolgono, anche in modo encomiabile e a
	dispetto di modesti finanziamenti, con gli insegnanti (soprattutto insegnanti di
	sostegno senza specializzazione, ma anche docenti specializzati, insegnanti
	curricolari, collaboratori scolastici ed educatori) e con le famiglie delle persone con
	disabilità. Come potranno tre persone in ogni Provincia svolgere tutto il lavoro fin
	qui svolto anche da altre realtà, CTI in testa, presenti in modo articolato e
	diversificato nel territorio?
	\item Che rapporti manterranno i CTS con i GLHI presenti nelle scuole e che stimolo
	daranno alla loro costituzione e al loro funzionamento?
	\item Quali rapporti con i Glip provinciali previsti dall'art. 15 della Legge 104/1992 e con
	i nuovi Glir Regionali (in Lombardia il Glir si è da poco insediato, nel febbraio del
	2011), costituiti sulla base delle Linee Guida Ministeriali per l'Integrazione del
	2009? Saranno anch'essi poco alla volta esautorati?
	\item Che ne sarà delle indicazioni contenute nelle Linee Guida regionali formulate dal
	Glir lombardo per i Centri Territoriali di risorse per la disabilità?
\end{enumerate}
Noi pensiamo che l'inclusione degli alunni con Bisogni Educativi Speciali non possa dirsi attuata
se confinata alla dotazione di ausili e alla consulenza di esperti su chiamata (i cosiddetti tecnici
“con la valigia”), soprattutto se si parla di problematiche complesse collegate alla disabilità, per
cui condizioni personali e situazione ambientale rifuggono da qualsiasi omologazione.
Non basta dotare un ragazzo con disabilità di un pc o di un software per poterlo dire “incluso”!
Lo sanno bene molti bravi insegnanti di sostegno, insegnanti curricolari ed educatori che
progettano e operano insieme da anni per l'inclusione, quella vera\footcite{GruppoLEDHA2013a}.

Donatella Morra e Maria Spallino
Gruppo LEDHAscuola
