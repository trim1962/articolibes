\chapter{Funzioni Sinusoidali}
\label{sec:FunzioniSinusoidali}
\minitoc
\mtcskip                                % put some skip here
\minilof                                % a minilof
\mtcskip                                % put some skip here
\minilot
\altapriorita{Inserire testo}
\begin{figure}
	\begin{subfigure}[b]{.5\linewidth}
		\centering\includestandalone[width=7.5cm]{funzgonioTikz/asinomegat}
		\caption{Grafico di $y=A\sen\omega t$}\label{fig:asinomegat}
	\end{subfigure}%
	\qquad\qquad
	\begin{subfigure}[b]{.5\linewidth}
		\centering\includestandalone[width=7.5cm]{funzgonioTikz/acosomegat}
		\caption{Grafico di $y=A\cos\omega t$}\label{fig:acosomegat}
	\end{subfigure}
	\caption{Funzioni sinusoidali}
	\label{fig:Funzionisinusoidali}
\end{figure}
\begin{figure}
	\begin{subfigure}[b]{.5\linewidth}
		\centering\includestandalone[width=7.5cm]{funzgonioTikz/asinomegadiversit}
		\caption{Funzioni di frequenze diverse}\label{fig:frequenzediverse}
	\end{subfigure}%
		\qquad\qquad
	\begin{subfigure}[b]{.5\linewidth}
		\centering\includestandalone[width=7.0cm]{funzgonioTikz/ampiezzediverse}
		\caption{Funzioni di ampiezze diverse}\label{fig:ampiezzediverse}
	\end{subfigure}
	\caption{Confronto fra funzioni di frequenza o ampiezza diverse}
	\label{fig:ampiezzediversefrequenzediverse}
\end{figure}
\begin{figure}
	\begin{subfigure}[b]{.5\linewidth}
		\centering\includestandalone[width=7.5cm]{funzgonioTikz/AsinomegaTSfasamentoAnticipato}
		\caption{Funzioni in anticipo di fase}\label{fig:AsinomegaTSfasamentoAnticipato}
	\end{subfigure}%
		\qquad\qquad
	\begin{subfigure}[b]{.5\linewidth}
		\centering\includestandalone[width=7.5cm]{funzgonioTikz/AsinomegaTSfasamentoRitardato}
		\caption{Funzioni in ritardo di fase}\label{fig:AsinomegaTSfasamentoRitardato}
	\end{subfigure}
	\caption{Funzioni che differiscono per la fase}%
	\label{fig:Funzionichedifferisconoperlafase}%
\end{figure}
\begin{figure}
	\begin{subfigure}[b]{0.5\linewidth}
		\centering\includestandalone[width=7.5cm]{funzgonioTikz/AsinAnticipoDiFase}
		\caption{Quadratura di fase in anticipo}\label{fig:QuadraturaFaseAnticipo}
	\end{subfigure}%
		\qquad\qquad
	\begin{subfigure}[b]{0.5\linewidth}
		\centering\includestandalone[width=7.5cm]{funzgonioTikz/AsinRitardoDiFase}
		\caption{Quadratura di fase in ritardo}\label{fig:QuadraturaFaseARitardo}
	\end{subfigure}
	\begin{subfigure}[b]{0.5\linewidth}
			\centering\includestandalone[width=7.5cm]{funzgonioTikz/opposizionedifase}
			\caption{Opposizione di fase}\label{fig:Opposizionedifase}
	\end{subfigure}
	\caption{Funzioni in quadratura e opposizione}%
	\label{fig:Funzioniinquadratura}%
\end{figure}
