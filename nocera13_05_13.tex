\author{Salvatore Nocera}
\title{Amici docenti per il sostegno, relax! }
\phantomsection
\label{cha:nocera130513}
\begin{abstract}
\caporali{Ci sarà sempre bisogno dei docenti specializzati per il sostegno –- scrive Salvatore Nocera –- come emerge anche dall'attuale normativa e da quella in gestazione}. Una serie di riflessioni che prendono spunto dall'intervento del docente specializzato Giovanni Maffullo, da noi pubblicato nei giorni scorsi, per dare ulteriore sostanza al dibattito sui temi dell'inclusione scolastica, avviato su queste pagine da qualche settimana
\end{abstract}
%\epigraph{«Ci sarà sempre bisogno dei docenti specializzati per il sostegno – scrive Salvatore Nocera – come emerge anche dall'attuale normativa e da quella in gestazione». Una serie di riflessioni che prendono spunto dall'intervento del docente specializzato Giovanni Maffullo, da noi pubblicato nei giorni scorsi, per dare ulteriore sostanza al dibattito sui temi dell'inclusione scolastica, avviato su queste pagine da qualche settimana}{Salvatore Nocera}
\maketitle
\datapub{13 Maggio 2013}
Ho letto con molto interesse l'articolo del professor Giovanni Maffullo\pageref{cha:maffullo100513}, intitolato Per non buttare a mare una lunga storia di inclusione, pubblicato da «Superando.it» il 10 maggio scorso, riscontrando in esso un grave stato di frustrazione e timore per le prospettive dell'inclusione in Italia.

Comprendo lo stato d'animo dello scrivente, perché egli è stato certamente frastornato da recenti pubblicazioni, come quella della Fondazione Agnelli sull'abolizione della figura dei docenti per il sostegno, l'invio dell'80\% di essi nelle classi (come docenti curricolari) e l'innalzamento del residuo 20\% in “iperuranei” gruppi di lavoro itineranti, esterni alle scuole, per far letteralmente “piovere dal cielo” i distillati suggerimenti delle loro consulenze, distantissime, però, dalla realtà quotidiana del lavoro in classe\footcite{treellle2011alunni} [ci si riferisce alle ipotesi avanzate nel rapporto intitolato Gli alunni con disabilità nella scuola italiana: bilancio e proposte, Erickson, 2011, elaborato appunto dalla Fondazione Agnelli, insieme all'Associazione TreeLLLe e alla Caritas Italiana, N.d.R.].
Ebbene, per la conoscenza che ho della scuola attiva e della realtà  istituzionale che si muove per produrre norme giuridiche, posso tuttavia rassicurare Giovanni Maffullo che tutto quello che è stato ipotizzato dalla Fondazione Agnelli è pura “fantadidattica”, anche se in Trentino si sta effettivamente avviando una sperimentazione in tal senso. Si parla però di una Provincia Autonoma – come quella di Trento, appunto – oggettivamente molto ricca, che può quindi permettersi anche di “giocare alla fantainclusione scolastica”!

Infatti, anche se e quando si riusciranno ad avere docenti curricolari molto più preparati di oggi sulle didattiche inclusive e classi meno numerose nel rispetto degli articoli 4 e 5, comma 2 del Decreto del Presidente della Repubblica (DPR) 81/09\footcite{DPR_81_2009} – condizioni che, come ho avuto modo di scrivere più volte, ritengo indispensabili per una buona qualità dell'inclusione scolastica – ci sarà sempre bisogno dei docenti specializzati per il sostegno. E questa non è solo una mia constatazione, ma è quanto emerge dall'attuale normativa e da quella in gestazione.

Ad esempio, l'articolo 50 della Legge 35/12\footcite{Legge_35_2012}, richiamato dalla Circolare Ministeriale 8/13\footcite{cm8_2013} sui BES (Bisogni Educativi Speciali) o BEI (Bisogni Educativi Individuali), prevede la costituzione di organici funzionali di sostegno a livello di singole scuole o meglio di reti di scuole.
Concordo altresì con il professor Maffullo sul fatto che senza docenti per il sostegno non sia possibile seguire casi delicatissimi che richiedono una personalizzazione estrema. Non concordo però con lui, quando si dimostra letteralmente “terrorizzato” dall'“alluvionale” numero di PDP (Piani Didattici Personalizzati), che i Consigli di Classe dovrebbero elaborare per gli alunni dichiarati con disturbi specifici dell'apprendimento (DSA) e con BES (o BEI), insieme ai PEI (Piani Educativi Individualizzati) per gli alunni con disabilità. Infatti, a differenza di questi ultimi – che richiedono una visione a tutto tondo degli interventi di ciascun operatore scolastico ed extrascolastico – i PDP dovrebbero, a mio avviso, essere composti solo da poche righe in cui si dica perché quel determinato alunno necessiti di strumenti compensativi o dispensativi in questa o quella disciplina.

Il PDP, infatti, è uno strumento che ha un duplice compito: individuare da una parte se e sino a quando un alunno necessiti di strumenti compensativi e dispensativi, per metterlo a suo agio nel raggiungimento del successo formativo; costituire, dall'altra, una prova giuridica che possa spuntare gli strali di favoritismo lanciati da parte degli altri alunni ai quali tali strumenti non siano stati concessi dal Consiglio di Classe. Quindi non mi sembra che si possa parlare di un enorme sovraccarico di lavoro per i docenti curricolari.

Quanto ai docenti per il sostegno, essi dovranno continuare ad occuparsi prioritariamente dell'inclusione degli alunni con disabilità, essendo espressamente vietata dalla Legge 170/10 l'assegnazione di tale figura agli alunni con DSA e quindi anche con BES (o BEI).
I docenti per il sostegno, perciò, data la loro specializzazione, dovranno “sostenere” i colleghi curricolari, i quali, a loro volta, dovranno essere sempre più formati – sia inizialmente, sia con aggiornamento obbligatorio in servizio – nell'effettuazione dei loro insegnamenti e nella valutazione dei risultati ottenuti dagli alunni con disabilità. E dovrà finalmente cessare l'aberrante prassi diffusasi negli ultimi anni della delega totale da parte di un numero purtroppo crescente di docenti curricolari dell'insegnamento disciplinare e della loro valutazione ai soli docenti per il sostegno. La normativa stessa è decisamente orientata in tal senso. Infatti, se si leggono con attenzione le Linee Guida per l'Integrazione Scolastica degli Alunni con Disabilità del 4 agosto 2009\footcite{LineGuida2009} e pure il DPR 122/09\footcite{DPR_122_2009} sulla valutazione degli alunni, si vedrà che sono chiaramente distinti i compiti dei docenti curricolari e di quelli per il sostegno.

Per i primi, si legge all'articolo 9 del DPR 122/09\footcite{DPR_122_2009}, riguardante gli alunni con disabilità, e all'articolo 10, riguardante quelli con DSA (e ora si dovranno aggiungere anche quelli con BES o BEI), che essi devono valutare gli «apprendimenti disciplinari rispettivi dei singoli alunni»; invece, per i docenti per il sostegno, nei primi articoli dello stesso DPR, si legge che essi dovranno valutare «se e sino a che punto siano stati raggiunti da ciascun alunno con e senza disabilità gli obiettivi dell'inclusione, indicati nell'articolo 12, comma 3 della Legge 104/92\footcite{Legge_104_92}», espressamente citato, ovvero: la crescita negli apprendimenti in generale, nella comunicazione, nella socializzazione e nelle relazioni.

È del resto assai strano che tale distinzione non sia stata operata dal Ministero nel predisporre i registri elettronici, laddove manca un'apposita colonna per i voti dei docenti per il sostegno. Ma si tratta di una lacuna che il Ministero stesso dovrà immediatamente colmare, pena l'illegittimità delle valutazioni, per carenza del «collegio perfetto», come previsto sempre dalla Legge 104/92\footcite{Legge_104_92}.
Mi sembra, quindi, che tutte le norme appena citate confermino che la figura del docente per il sostegno non potrà essere eliminata, né “sublimata” in gruppi itineranti, ma sempre più dovrà collaborare con i docenti curricolari. Dunque, collega Maffullo e quanti hanno i suoi stessi timori, credo che almeno su questo ci si possa rilassare!\footcite{nocera3}


Vicepresidente della FISH (Federazione Italiana per il Superamento dell'Handicap).
13 maggio 2013
© Riproduzione riservata
