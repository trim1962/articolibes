\author{Salvatore Nocera}
\title{Basta coi BES. Pensiamo a una scuola inclusiva}
\label{cha:nocera021213}
%\begin{abstract}
%	Infatti, una Nota Ministeriale prodotta nei giorni scorsi chiarisce ulteriormente alcuni punti controversi della recente normativa sui Bisogni Educativi Speciali (BES), fissata tra la fine del 2012 e l'inizio di quest'anno, che tanto ha già fatto discutere. Appare in particolare importante la sottolineatura data da un lato alla prevalenza delle valutazioni pedagogiche da parte dei docenti, dall'altro al rispetto dell'autonomia scolastica
%\end{abstract}
\maketitle
\datapub{2 dicembre 2013}
Ho letto con attenzione il dotto e lungo commento di Raffaele Iosa\ref{cha:Iosa281113} sulla Nota MIUR del 22 Novembre scorso sui bes. Dal momento che io mi intendo appena di normativa, non mi permetto di entrare nell’analisi delle argomentazioni pedagogiche da lui svolte e mi limito a cogliere ciò che mi sembra utile per placare questa polemica e prospettare un futuro di positivo lavoro per le scuole.

In ultima analisi, anche Iosa, come me, sembra dire che la Nota del 22 Novembre ridimensiona molto la Direttiva del 27 Dicembre 2012\footcite{dir27Dic2012} , già ridimensionata dalla c m n. 8/2013\footcite{cm8_2013}. Ecco alcune sue parole:

“Con questo confuso parterre di norme i nuovi Bes catalogati secondo la Direttiva si ridurranno a quattro gatti. Nel dare “potere” alle scuole in questo modo tutto romano avverrà sicuramente una forte riduzione quantitativa dei nuovi Bes, perché le scuole non sono così allupate di carte quando in cambio c’è il nulla (né soldi né personale), ma continueranno ad agire sul disagio individuale con l’attenzione di prima, bene o male secondo le scuole, evitando targhe inutili e lavorando sottovia, speriamo con buon senso.”

Egli dà atto alla nuova Nota di aver ridato piena autonomia alle scuole, quando precisa che, anche in presenza di diagnosi che non certifichino disabilità o DSA, i docenti sono liberi di valutare le situazioni ed adottare le decisioni didattiche che riterranno opportune come avevano sempre fatto e come continueranno a fare. Però il saggio di Iosa ha un’impostazione più ampia e profonda, attaccando frontalmente la “deriva iatrogena “ della scuola. E lì concordo con lui. Però ciò non si deve tanto alla normativa sui bes, come ridimensionata con l’ultima Nota, bensì alla legge 170/2010 sui DSA. Ma qui il Ministero non ha colpa, trattandosi di una legge che gli è stata imposta dal Parlamento, sotto la pressione di forti lobbies mediche. Bisogna quindi prendersela, non tanto col MIUR, quanto col Parlamento e con la cultura imperante nel nostro Paese.

Anzi ad essere più coerenti, anche la normativa sulla disabilità a partire dalla legg 104/1992 è pur essa pervasa da una logica iatrogena. Sarà pur vero che qui la sanità fa il suo compito nel certificare i casi di vera e propria disabilità. Però, ormai, alla luce dell’ICF, la scuola deve superare il momento sanitario certificativo, per effettuare una descrizione dei bisogni educativi degli alunni con disabilità, tenendo conto del contesto culturale ed ambientale nel quale essi vivono, per formulare progetti didattici inclusivi ponendo in prima linea i fattori “favorenti”, costituiti dalla formazione dei docenti curricolari, e dalle nuove tecnologie, per superare le barriere costituite dai deficit intellettivi, sensoriali e fisici degli alunni e dal contesto formato da classi troppo numerose e da una mancata formazione iniziale ed obbligatoria in servizio dei docenti curricolari.

Ancor oggi continuiamo a pretendere che gli alunni con disabilità si adattino ad una scuola rigida, mentre dovrebbe essere la scuola a doversi adattare con le sue strategie didattiche ai bisogni educativi di questi alunni per riuscire veramente ad includerli in un gioco di reciproci adattamenti, in cui però quelli della didattica dovrebbero prevalere sulla medicina.

Quindi, ridimensionata la portata qualitativa e quantitativa degli effetti della recente normativa sui BES, mi permetto di invitare quanti, come Iosa e me, credono in una scuola inclusiva, a riprendere in mano il faticoso lavoro dell’inclusione degli alunni con disabilità, per migliorarla grazie all’autonomia della scuola che dovrà essere sostenuta politicamente da maggiore attenzione e mezzi finanziari.

Negli ultimi tredici anni l’attenzione per l’inclusione scolastica degli alunni con disabilità è venuta scemando sia nella classe politica che nella società. La recente normativa sui BES ha avuto, se non altro, il merito indiretto di risuscitarla e riproporla al Paese come un problema che le sole norme non possono risolvere se non sono saggiamente e correttamente applicate da tutti i protagonisti che sono in primo luogo i docenti curricolari ed i compagni di classe.

E’ questa la sfida che il nostro Paese ha lanciato al mondo oltre quarant’anni fa e che ora rischia di perdere. Smettiamo di distrarci con altri problemi meno gravi e “riprendiamoci la pedagogia” per realizzare un’inclusione di qualità degli alunni con disabilità; se la scuola saprà essere veramente inclusiva con loro, lo saprà essere anche con gli altri casi.