\chapter{Logaritmi}
\label{sec:Logaritmi}
\minitoc
\mtcskip                                % put some skip here
\minilof                                % a minilof
\mtcskip                                % put some skip here
\minilot
\section{Logaritmo}
\label{sec:Lograritmo}
Un'equazione esponenziale non ha sempre una soluzione per esempio \[10^x=20\] In questo caso la soluzione esiste ma non è nota, chiamo logaritmo\index{Logaritmo} di venti in base dieci l'esponente che bisogna dare a dieci per ottenere venti e lo indico con $\log_{10}20$
\begin{definizione}
	Chiamo logaritmo in base $a$ di $b$ l'esponente che bisogna dare ad $a$ per ottenere $b$ \[x=\log_ba\Leftrightarrow b^{x}=a\quad a>0\quad a\neq  1\quad b>0 \]
\end{definizione} 
\section{Funzione Logaritmica}
\label{Funzione Logaritmica}
La funzione $y=log_ax$ si chiama funzione logaritmica\index{Funzione!Logaritmica} è fortemente collegata alla funzione esponenziale $y=a^x$. La figura\nobs\vref{fig:funzioniLogEsempio1} confronta la funzione esponenziale e la funzione logaritmica  quando la base è maggiore di uno $a>1$. La curva è simmetrica rispetto alla bisettrice del primo terzo quadrante. 
\begin{enumerate}
	\item Il grafico della funzione logaritmo occupa il semipiano positivo delle x.
	\item Tutti i grafici passano per il punto $(1,0)$.
	\item La funzione è crescente\index{Funzione!Crescente} all'aumentare dell'incognita. $x_1<x_2\quad f(x_1)<f(x_2)$ 
	\item L'asse delle $y$ è un asintoto verticale\index{Asintoto!verticale}.
\end{enumerate} 
\begin{figure}
	\centering
	\begin{subfigure}[b]{.4\linewidth}
		\centering
	\includestandalone[width=\textwidth]{FunzioniLog/esempio1}
		\caption{$a>1$}
		\label{fig:funzioniLogEsempio1}
	\end{subfigure}\qquad
	\centering
	\begin{subfigure}[b]{.4\linewidth}
		\centering
	\includestandalone[width=\textwidth]{FunzioniLog/esempio2}
		\caption{$0<a<1$}
		\label{fig:funzioniLogEsempio2}
	\end{subfigure}%
	\caption{Funzioni esponenziali e logaritmiche}
	\label{fig:funzExp1}
\end{figure}
La figura\nobs\vref{fig:funzioniLogEsempio2} confronta la funzione esponenziale e la funzione logaritmica  quando la base è compresa fra zero e uno $0<a<1$ . La curva è simmetrica rispetto alla bisettrice del primo terzo quadrante. 
\begin{enumerate}
	\item Il grafico della funzione logaritmo occupa il semipiano positivo delle x.
	\item Tutti i grafici passano per il punto $(1,0)$.
	\item La funzione è decrescente\index{Funzione!Decrescente} all'aumentare dell'incognita. $x_1<x_2\quad f(x_1)>f(x_2)$ 
	\item L'asse delle $y$ è un asintoto verticale\index{Asintoto!verticale}.
\end{enumerate}
La figura\nobs\vref{fig:FunzioniLogEsempio3} confranta fre di loro i due grafici.
\begin{figure}
\centering
\includestandalone[width=0.6\linewidth]{FunzioniLog/esempio3}
%\includegraphics[width=0.7\linewidth]{./}
\caption{Funzioni logaritmiche}
\label{fig:FunzioniLogEsempio3}
\end{figure}
\section{Proprietà del logaritmi}
Le proprietà del logaritmi sono legate alle proprietà delle potenze. $\log_ab=x$ è l'esponente $x$ che bisogna dare ad $a$ per ottenere $b$ quindi $x=\log_ab\Leftrightarrow a^{x}=b$. 
\label{sec:ProprietadelLogaritmi}
%\begin{table}
%	\centering
%	%\begin{center}
%	\begin{tikzpicture}[line cap=round,line join=round,>=triangle 45,x=1.0cm,y=1.0cm]
%	\draw[->,color=black] (-1,0) -- (6.0,0);
%	\foreach \x in {-1,1,2,3,4,5}
%	\draw[shift={(\x,0)},color=black] (0pt,2pt) -- (0pt,-2pt) node[below] {\footnotesize $\x$};
%	\draw[color=black] (5.86,0.02) node [anchor=south west] { x};
%	\draw[->,color=black] (0,-2.13) -- (0,2.18);
%	\foreach \y in {-2,-1,1,2}
%	\draw[shift={(0,\y)},color=black] (2pt,0pt) -- (-2pt,0pt) node[left] {\footnotesize $\y$};
%	\draw[color=black] (0pt,-10pt) node[right] {\footnotesize $0$};
%	\clip(-1,-2.13) rectangle (5.0,2.18);
%	\draw plot[raw gnuplot, id=func0] function{set samples 100; set xrange [0.1:6.0]; plot log(x)/log(2)};
%	\draw plot[raw gnuplot, id=func1] function{set samples 100; set xrange [0.1:6.0]; plot log(x)/log(0.5)};
%	\draw (4.0,1.5) node[anchor=north west] {$\mathbf{log_a x}$};
%	\draw (2.1,-0.78) node[anchor=north west] {$\mathbf{0<a<1}$};
%	\draw (2.1,1.06) node[anchor=north west] {$\mathbf{a>1}$};
%	\end{tikzpicture}
%	%\subfloat[][]{\pagestyle{empty}
\begin{tikzpicture}[line cap=round,line join=round,>=triangle 45,x=1.0cm,y=1.0cm]
\draw[->,color=black] (-1,0) -- (6.0,0);
\foreach \x in {-1,1,2,3,4,5}
\draw[shift={(\x,0)},color=black] (0pt,2pt) -- (0pt,-2pt) node[below] {\footnotesize $\x$};
\draw[color=black] (5.86,0.02) node [anchor=south west] { x};
\draw[->,color=black] (0,-2.13) -- (0,2.18);
\foreach \y in {-2,-1,1,2}
\draw[shift={(0,\y)},color=black] (2pt,0pt) -- (-2pt,0pt) node[left] {\footnotesize $\y$};
\draw[color=black] (0pt,-10pt) node[right] {\footnotesize $0$};
\clip(-1,-2.13) rectangle (5.0,2.18);
\draw plot[raw gnuplot, id=func0] function{set samples 100; set xrange [0.1:6.0]; plot log(x)/log(2)};
\draw plot[raw gnuplot, id=func1] function{set samples 100; set xrange [0.1:6.0]; plot log(x)/log(0.5)};
\draw (4.0,1.5) node[anchor=north west] {$\mathbf{log_a x}$};
\draw (2.1,-0.78) node[anchor=north west] {$\mathbf{0<a<1}$};
\draw (2.1,1.06) node[anchor=north west] {$\mathbf{a>1}$};
\end{tikzpicture}
}
%	\caption{Funzioni logaritmiche}
%	\label{tab:FunzioneLog}
%	%\end{center}
%\end{table}
\begin{table}
\centering
\begin{tabular}{ll}
\toprule
Proprietà&Formula\\
\midrule
definizione&$x=\log_ba\Leftrightarrow b^{x}=a$\\
&\\
logaritmo della base&$\log_bb=1$\\
&\\
logaritmo dell'unità&$\log_b1=0$\\
&\\
logaritmo del prodotto&$\log_bac=\log_ba+\log_bc$\\
&\\
logaritmo della divisione&$\log_b\dfrac{a}{c}=\log_ba-\log_bc$\\
&\\
logaritmo della potenza&$\log_ba^{n}=n\log_ba$\\
&\\
logaritmo della radice&$\log_b\sqrt[n]{a}=\dfrac{1}{n}log_ba$\\
&\\
cambio di base&$\log_cb=\dfrac{\log_ab}{\log_ac}$\\
&\\
potenza della base&$\log_ab=\log_{a^n}b^{n}$\\
&\\
&$\log_ab\cdot\log_ba=1$\\
\bottomrule
\end{tabular}
\caption{Logaritmi}
\label{tab:logaritmi}
\end{table}
