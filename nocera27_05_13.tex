\chapter{Non vogliamo “giocare alla fantainclusione”}
\label{cha:gavosto270513}
\epigraph{Prende spunto, Andrea Gavosto, direttore della Fondazione Agnelli, da un recente intervento di Salvatore Nocera, vicepresidente della FISH (Federazione Italiana per il Superamento dell’Handicap), per riprendere i contenuti di un Rapporto sugli alunni con disabilità, realizzato nel 2011 dalla Fondazione stessa, insieme alla Caritas Italiana e all’AssociazioneTreellle, e chiarirne ulteriormente i reali obiettivi}{Andrea Gavosto}

Si discute molto, in questi mesi, sul presente e il futuro dell'inclusione scolastica

Non ci appassiona la “fantadidattica”, né ci piace giocare alla “fantainclusione”; siamo, invece, interessati a sperimentazioni che mettano seriamente alla prova – prima di generalizzarle – soluzioni innovative, per superare i limiti che in Italia ostacolano la piena inclusione di uno spettro sempre più ampio di Bisogni Educativi Speciali [d'ora in poi BES, N.d.R.]: alunni con disabilità, difficoltà o svantaggi.
Non vogliamo abolire gli insegnanti di sostegno domani e neppure dopodomani; vogliamo, semmai, lavorare affinché in tempi ragionevoli tutti gli insegnanti italiani siano messi in condizione di sostenere e rendere sostenibile – anche in termini di risorse – una strategia d'inclusione scolastica dei BES più efficace: riteniamo che questo obiettivo implichi un tendenziale superamento della figura dell'insegnante di sostegno, così come la conosciamo oggi.
Per riprendere questi temi e senza intenzioni polemiche, chiediamo perciò ospitalità a «Superando.it», per commentare il recente intervento\vref*{cha:nocera130513} di Salvatore Nocera, vicepresidente della FISH (Federazione Italiana per il Superamento dell'Handicap), che ha criticato le proposte formulate nel Rapporto Gli alunni con disabilità nella scuola italiana: bilancio e proposte, pubblicato da Erickson\footcite{treellle2011alunni} nel 2011 e realizzato da Caritas Italiana, AssociazioneTreellle e Fondazione Giovanni Agnelli, con la partecipazione di esperti italiani di inclusione, fra cui Italo Fiorin e Dario Ianes.

Per il dettaglio delle analisi e delle linee di politica, rimandiamo al volume e alla sintesi disponibile nel nostro sito. In schematicissima sintesi, alla luce delle criticità dell'attuale modello italiano di integrazione della disabilità nella scuola e delle prospettive aperte dalla nuova legislazione per una più ampia inclusione scolastica dei BES, il Rapporto proponeva di:
\begin{description}
	\item[(i)] valorizzare l'autonomia delle scuole nella lettura dei bisogni e nella progettazione degli interventi, in coerenza con l'orientamento culturale condiviso a livello internazionale che suggerisce il passaggio da un approccio prevalentemente medico a un approccio prevalentemente pedagogico;
	\item[(ii)]creare nuovi centri territoriali, che dispongano di insegnanti e personale ad alta specializzazione, e di concerto con le scuole definiscano e coordinino le risorse finanziarie, professionali e tecnologiche per l’inclusione, svolgendo anche formazione e consulenza alle scuole, come pure una funzione di “sportello unico” per le famiglie;
	\item[(iii)] gradualmente superare l’“indissolubile binomio” alunno con disabilità certificato/insegnante di sostegno, come pure la distinzione fra insegnanti di sostegno e curricolari, mirando, da un lato, alla piena corresponsabilizzazione di tutti i docenti, attraverso una generalizzata formazione in didattica per i BES, dall’altro, al progressivo riassorbimento nell’organico curricolare di una parte consistente degli insegnanti di sostegno, assegnati sulla base della lettura dei bisogni delle scuole stesse, delle loro competenze specifiche (che non andrebbero disperse, ma valorizzate in relazione all’obiettivo di una maggiore inclusività) e della concertazione con i centri territoriali.
\end{description}

(ii) 
(iii)

Il vicepresidente della FISH sembra appiattire completamente sul presente il senso e l’orizzonte delle nostre proposte, manifestando, inoltre, aperto scetticismo su una sperimentazione che in Trentino sta partendo in questa direzione. Dovrebbe, tuttavia, sapere – dopo i tanti confronti avuti in questi mesi – che il nostro intento è di delineare un processo di cambiamento inevitabilmente di lungo respiro temporale e che, in questa prospettiva, sperimentazioni su piccola scala possono dare informazioni utili sulla reale efficacia di una strategia innovativa.
Vorremmo aggiungere che molte premesse da cui ha preso le mosse la nostra riflessione sono le medesime che da tempo la FISH propone all’attenzione dell’opinione pubblica. Fra queste, la necessità (a) di potenziare l’inclusione scolastica degli alunni con BES, prevedendo sistematicamente il coinvolgimento di tutti gli operatori scolastici; (b) di attivare a questo scopo reti di supporto, formazione e consulenza, valorizzando le professionalità disponibili, e di potenziare le reti territoriali per costruire strutture in grado di sostenere realmente le scuole; (c) di sperimentare nuove modalità organizzative in grado di intervenire in modo efficace ed economicamente sostenibile, ancora di più alla luce dell’estensione del numero dei BES dopo le recenti novità normative.

Per quanto poi riguarda il ruolo degli insegnanti di sostegno e di quelli curricolari, non possiamo che sottoscrivere totalmente quanto scriveva recentemente lo stesso Nocera, ricostruendo l’evoluzione del modello italiano di integrazione: «L’ipotesi innovativa da cui partì allora l’Italia era che i responsabili primari dell’inclusione fossero i docenti curricolari […]. Purtroppo tale disegno originario […] è stato profondamente offuscato e il ruolo di sostegno dei docenti specializzati è divenuto preminente e addirittura “assorbente”. In altre parole, il docente per il sostegno è divenuto quasi la “protesi didattica” dell’alunno con disabilità, favorito in questa deriva dalla delega dei docenti curricolari ai soli docenti di sostegno».
La nostra proposta mira proprio a dare seguito a quelle premesse e a recuperare per il futuro prossimo quanto di buono c’era in quella intuizione originaria, in particolare, la necessità del pieno coinvolgimento di tutti i docenti.

Quanto scrivevamo due anni fa nel Rapporto è la soluzione più appropriata, utile e sostenibile per una prospettiva d’inclusione dei BES sempre più ampia? Ovviamente, noi pensiamo di sì, consapevoli che in ogni caso richiederà tempi lunghi e aggiustamenti graduali. Nondimeno, come sempre avviene con le politiche fortemente innovative, anche in questo caso non possiamo conoscerne a priori l’efficacia.
Perciò riteniamo che essa vada preliminarmente verificata: la sperimentazione trentina, in cui la Fondazione Agnelli è impegnata insieme all’IPRASE [Istituto Provinciale per la Ricerca, la Sperimentazione e l’Aggiornamento Educativi, N.d.R.], è coerente con questo proposito di verifica sperimentale dell’efficacia e della sua sostenibilità, anche in termini organizzativi e sociali (pur consapevoli dei limiti di applicabilità del metodo sperimentale in contesti sociali reali). La sperimentazione che si avvierà punta, in particolare, alla formazione dei Consigli di Classe, quindi va nella direzione di rendere più competente l’intera scuola e questo grazie a un’ulteriore qualificazione degli insegnanti di sostegno.

In definitiva, non abbiamo la presunzione di applicare un modello aprioristicamente ritenuto migliore del presente, ma riteniamo opportuno tentare nuove strade se si intende proseguire nella direzione di una scuola che: (1) voglia essere sempre più inclusiva; (2) lo sia efficacemente (e non resti solo sulla carta con norme ambiziose e un organico complesso di leggi e azioni che vengono disattese); (3) lo sia senza che si sottraggano risorse all’inclusione, ma che queste vengano utilizzate in modo innovativo, flessibile e sinergico, massimizzando il potenziale formativo di tutti gli attori coinvolti in direzione di una maggiore capacità inclusiva.
Non ci pare che questo sia “giocare alla fantainclusione”.

Direttore della Fondazione Giovanni Agnelli.

27 maggio 2013

© Riproduzione riservata