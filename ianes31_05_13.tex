\author{Dario Ianes}
\title{BES e didattica inclusiva: alcune opportunità da cogliere}
\phantomsection
\label{cha:ianes1}
%\epigraph{Ci sono certamente alcuni punti deboli, sottolinea Dario Ianes, nelle recenti disposizioni ministeriali sui Bisogni Educativi Speciali (BES) e sulla gestione dell'inclusione, ma assai di più egli ritiene siano le buone opportunità da cogliere. Un ulteriore importante contributo al dibattito da noi avviato in queste settimane, sul presente e il futuro dell'inclusione scolastica}{Dario Ianes}
\begin{abstract}
Ci sono certamente alcuni punti deboli, sottolinea Dario Ianes, nelle recenti disposizioni ministeriali sui Bisogni Educativi Speciali (BES) e sulla gestione dell'inclusione, ma assai di più egli ritiene siano le buone opportunità da cogliere. Un ulteriore importante contributo al dibattito da noi avviato in queste settimane, sul presente e il futuro dell'inclusione scolastica
\end{abstract}
\maketitle
\datapub{31 Maggio 2013}
I recenti atti ministeriali sul tema degli alunni con  \glslink{besa}{BES} e sulla gestione dell'inclusione [il riferimento è alla Direttiva Ministeriale del 27 dicembre 2012\footcite{dir27Dic2012} e alla Circolare Ministeriale 8/13\footcite{cm8_2013}, N.d.R.] hanno prodotto un dibattito notevole nel nostro Paese, con posizioni molto diverse. In modo molto sintetico, ma spero chiaro, vorrei riassumere nei punti seguenti la mia posizione.

\begin{enumerate}
	\item Il concetto di “bisogno” ha anche delle connotazioni negative nella nostra lingua e credo che tale negatività condizioni troppo alcune posizioni critiche nei confronti del concetto di \glslink{besa}{BES} , ma questo improprio “effetto alone” va superato. Credo cioè si dovrebbe considerare il concetto di bisogno non tanto come una mancanza, una privazione o una deficienza, in sé negativa, ma come una situazione di dipendenza (interdipendenza) della persona dai suoi ecosistemi, relazione che – se tutto va sufficientemente bene – porta alla persona che cresce alimenti positivi per il suo sviluppo.
	
	In altre parole, la persona cresce bene in apprendimenti e partecipazione, se questa relazione porta risposte e alimenti adeguati al suo sviluppo.
	\item Il Bisogno Educativo Speciale non è diverso da uno normale, è divenuto tale quando la situazione di funzionamento biopsicosociale problematica della persona ha reso per lei difficile trovare una risposta adeguata ai suoi bisogni. Ad esempio, un bambino di 4 anni potrebbe trovare un carente alimento al suo bisogno di autonomia, vivendo in un contesto familiare deprivante e problematico.
	\item Quando si parla di “funzionamento della persona” in un'ottica biopsicosociale, ci si riferisce all'intreccio complesso e multidimensionale dell'ICF\footcite{icf} [ove ICF sta per Classificazione del Funzionamento, della Disabilità e della Salute, definita nel 2001 dall'Organizzazione Mondiale della Sanità, N.d.R.], dove giocano un ruolo fondamentale le interazioni tra condizioni fisiche, corpo, competenze personali, partecipazione sociale, contesti ambientali e contesti personali.
	\item Quando ci si riferisce alla “problematicità” del funzionamento, ritengo che la si debba valutare tale soltanto se in modo inter soggettivo possiamo definire che la persona – a causa di quel funzionamento particolare – subisce un danno, un ostacolo o viene stigmatizzata in modo da subire una perdita di opportunità e di libertà di sviluppo. Una particolarità della persona che porti disagio (certo non un danno o simili) soltanto a chi la circonda e non alla persona stessa, è una differenza che va tutelata e preservata e non va fatto alcun tentativo di cambiarla.
	\item Il concetto di \glslink{besa}{BES}  non è clinico, né tanto meno medico. Non lo si trova infatti in alcun sistema di classificazione delle patologie, tipo \glslink{icda}{ICD} 10 o \glslink{dsma}{DSM} V.
	\item Il concetto di \glslink{besa}{BES}  è politico, nella misura in cui stabilisce – come macro categoria – quali siano le situazioni che hanno diritto a forme di individualizzazione e personalizzazione nella scuola.
	\item L'estensione del diritto alla personalizzazione dei percorsi formativi e di valutazione anche ad alunni non compresi prima nelle Leggi 104/92\footcite{Legge_104_92} e 170/10\footcite{legge170} è un positivo passo in avanti verso politiche scolastiche più eque e inclusive. In questo modo molte situazioni di alunni che prima non erano riconosciuti e tutelati ora lo possono essere.
	\item Tale estensione del diritto alla personalizzazione è un altro passo avanti verso una scuola pienamente inclusiva (l'Inclusive Education), fatto nel solco della tradizione italiana dell'integrazione scolastica, che parte dalle situazioni di disabilità, poi estende le tutele agli alunni con \glslink{dsaa}{DSA}, e ora a quelli con altre condizioni di BES, oltre a quelle classiche delle due norme or ora citate.
	È la via italiana all'inclusione, quella che passa da difficoltà ad altra difficoltà, piuttosto che partire da un radicale cambiamento della scuola per tutti gli alunni con le loro varie differenze, come sostengono gli studiosi della corrente che va sotto il nome di Disabilities Studies. Credo che queste due vie stiano progressivamente convergendo, perché l'obiettivo è comune (una scuola inclusiva per il 100\% degli alunni) e molto simili sono le considerazioni critiche e le proposte innovative. In ogni caso la tradizione italiana è questa e vogliamo valorizzarla.
	\item Il rischio di fenomeni di labeling [“etichettatura”, N.d.R.] e di micro esclusione è ovviamente sempre presente, ma non dipende certo dall'introduzione del concetto di BES. La scuola esclude anche senza etichetta, dipende da quale orientamento prende. Fenomeni di micro esclusione sono all'ordine del giorno nelle nostre scuole e colpiscono ogni tipo di alunno, da quello con disabilità a quello straniero e gli insegnanti escludono per tanti e diversi motivi.
	
	Se un insegnante ha in classe alunni che gli creano qualche tipo di problema e non vuole – o non sa – attivare strategie efficaci per personalizzare la loro partecipazione e apprendimento, tenderà ad escluderli, etichetta o meno. Il fatto che alcuni alunni saranno riconosciuti come alunni con BES non sarà uno “scivolo” per mandarli fuori, perché la nostra scuola non prevede percorsi separati, fuori dalla classe, per gli alunni riconosciuti \glslink{besa}{BES}. Chi teme questo forse “sente” che nelle viscere delle nostre scuole cresce un desiderio di percorsi separati?
	\item Le recenti disposizioni ministeriali sostengono e valorizzano il ruolo pedagogico e didattico del Team Docenti e del Consiglio di Classe anche nel momento dell'individuazione dell'alunno come alunno con \glslink{besa}{BES}. Gli insegnanti – anche se non avranno in mano un pezzo di carta medico, o sociale – dovranno valutare pedagogicamente e didatticamente il funzionamento problematico dell'alunno, con la loro competenza professionale. Certo, non in modo autarchico, ma collaborando ove possibile.
	
	Se qualcuno teme l'invasione della scuola da parte di “orde” di medici o psicologi che offriranno “individuazioni” di alunni \glslink{besa}{BES}  e diffonderanno questa nuova “malattia” per un ovvio interesse di bottega, si tranquillizzi e cerchi invece di sviluppare la competenza valutativa pedagogica e didattica degli insegnanti, che in moltissimi casi c'è, ma è sepolta da consuetudini di delega ai servizi sanitari. E poi non si tratta di fare diagnosi, ovviamente, ma di riconoscere una situazione di problematicità.
	\item  Le recenti disposizioni ministeriali riconoscono agli insegnanti la possibilità di individuare l'alunno con BES sulla base di «ben fondate considerazioni pedagogiche e didattiche»: ottima cosa, da anni insistiamo sul fatto che la scuola deve riappropriarsi di un forte ruolo che ad essa è proprio, lo sostenemmo fin dalle critiche all'Atto di Indirizzo del 1994\footcite{DPR1994} [DPR del 24 febbraio 1994, N.d.R.], che tagliava (e taglia) fuori la scuola dalla Diagnosi Funzionale, che invece deve essere pedagogica e didattica. Per me ben fondate significa fondate su un'antropologia ICF-OMS e sul concetto di problematicità centrato sulla persona.
	\item Problema della scarsa formazione di moltissimi insegnanti curricolari su questi temi: bene, cosa aspettano i sindacati a lanciare una campagna contrattuale per una formazione continua obbligatoria e per riformare la scandalosa carenza di questi temi nella formazione universitaria Gelmini per la scuola secondaria?
	\item Problema del nuovo carico di lavoro richiesto dagli alunni con \glslink{besa}{BES}: certamente la professione di insegnante si è fatta sempre più complessa e perciò deve smettere di essere un lavoro “di ripiego” o “di comodo”, un lavoro per troppi anni bistrattato nel patto perverso del «lavori poco e ti pago poco»; deve diventare una vera e propria professione alta, con un percorso universitario che va dai cinque ai sei anni, più uno per il sostegno, con un impegno pieno e stipendi adeguati.
	
	Su questo tema ci vuole coraggio vero da parte di tutti e non “sortite alla Profumo” [ci si riferisce al precedente ministro dell'Istruzione Francesco Profumo e a una delle sue ultime proposte, N.d.R.], per un paio di ore in più
	Si discute molto, in questi mesi, sul presente e il futuro dell'inclusione scolastica
	\item Il Piano Didattico Personalizzato (PDP) sarà fatto da tutti i docenti e non delegato al sostegno: ottima cosa, perché la responsabilità didattica è di tutti.
	\item I vari PDP della classe, accanto ad eventuali PEI\glslink{peia}{PEI}   e altri \glslink{pdpa}{PDP}  per alunni con \glslink{dsaa}{DSA}, dovranno raccordarsi in una progettazione inclusiva della classe. In una didattica strutturalmente inclusiva: e questa è una sfida di altissimo livello, assolutamente strategica.
	
	Collegialmente gli insegnanti proveranno a definire alcuni elementi di Didattica Inclusiva che costruiranno la quotidianità delle attività formative, una quotidianità per tutti, fatta in modo da accogliere le attività personalizzate. A questo livello si dovrà pensare all'adattamento dei materiali e dei testi, all'attivazione della risorsa compagni di classe (apprendimento cooperativo e tutoring), a varie forme di differenziazione, alla didattica laboratoriale, all'uso inclusivo delle tecnologie. Questa progettazione di classe è un valore aggiunto fondamentale alle varie individualizzazioni-personalizzazioni.
	\item Il Gruppo di Lavoro per l'Inclusione (GLI) può aggiungere altro valore prezioso alle varie proposte di progettazione di classe con i vari PEI/PDP. E questa è la seconda sfida strategica da cogliere: il GLI si limiterà a raccogliere le varie progettazioni di classe, confezionarle con un bel fiocco descrittivo dei vari alunni, e inviarle all'approvazione del Collegio dei Docenti e all'iter di negoziazione delle risorse? Qui c'è invece l'opportunità di creare altro valore aggiunto, elaborando nel Piano Annuale dell'Inclusione quelle strategie funzionali a livello di istituzione scolastica che ottimizzino e massimizzino le risorse presenti, come ad esempio un uso intelligente dell'orario, della formazione delle classi, delle sinergie con altre realtà territoriali ecc.
	\item  A qualcuno, in queste settimane, è sorto il timore che gli insegnanti di sostegno vengano utilizzati, in questa logica “funzionale”, anche per tutti gli altri alunni con BES, rendendo ancora più drammatica la situazione della “coperta corta”. Ma questo non è né previsto né consentito. Si leggano a tal proposito i vari recenti commenti proposti su queste stesse pagine da Salvatore Nocera, vicepresidente della FISH (Federazione Italiana per il Superamento dell'Handicap).
	\item Qualcuno addirittura pensa che con queste recenti disposizioni sugli alunni con \glslink{besa}{BES} si daranno insegnanti di sostegno soltanto agli alunni con disabilità gravi; alcuni hanno addirittura letto l'acronimo BES come “bisogna eliminare il sostegno”… Ma anche qui rimando agli articoli di Nocera su queste stesse pagine, in cui egli nega decisamente questa interpretazione catastrofista.
	\item  Le recenti disposizioni insistono molto su un livello di intelligenza territoriale, il CTS\glslink{ctsa}{CTS} , dove si dovrebbero comporre, con ulteriore valore aggiunto, i vari \glslink{paia}{PAI}  delle scuole, in relazione alle varie fonti territoriali di risorse (Uffici Scolastici Provinciali, Comuni, Province, ASL ecc.).
	
	Questo è un punto ancora debole, per ovvi motivi strutturali, di possibilità di funzionamento, e di complessità del compito. Questo terzo livello di “intelligenza” auspicato, dopo quello del Consiglio di Classe e del \glslink{glia}{GLI} , chiede ulteriore elaborazione, ma ricordo che questa dimensione – inter istituzionale e territoriale – mostrava anche in altre proposte o disposizioni evidenti debolezze (si veda ad esempio la seconda parte dell'intesa Stato-Regioni del 20 marzo 2008\footcite{ra_39_2008} e la proposta dei CRI, i Centri Risorse per l'Integrazione Scolastica, del Rapporto Caritas, Treellle e Fondazione Agnelli del 2011).
	\item La numerosità delle classi, eccessiva spesso anche in presenza di uno o più alunni con disabilità, ostacolerà l'applicazione delle disposizioni sugli alunni con BES? Ma allora, cosa aspettano le associazioni dei familiari ad attivare una class action nei confronti del Ministero, per far rispettare il DPR 81/09\footcite{DPR_81_2009} ? Gli insegnanti le sosterranno?
	\item La microcategoria degli alunni con BES è già stata introdotta dalla legge di riforma della scuola del Trentino (Legge Provinciale 5/06), dove abbiamo, nella categoria degli alunni con \glslink{besa}{BES}, gli alunni con disabilità, quelli con DSA e quelli con altre e varie forme di svantaggio, problemi ecc. Dunque quasi sette anni fa e mi sembra – anche attraverso due ricerche fatte come componente del Comitato di Valutazione della Scuola Trentina, di cui si vedano i report nel portale dedicato – che non sia accaduto nulla di quello che gli avversari delle recenti disposizioni ministeriali temono: stress da superlavoro degli insegnanti, etichettatura iatrogena di massa, medicalizzazione delle situazioni degli alunni, licenziamenti di insegnanti di sostegno, anzi\footcite{ianes1}.
\end{enumerate}



 
31 maggio 2013

Ultimo aggiornamento: 31 maggio 2013 18:49
© Riproduzione riservata
 
 
