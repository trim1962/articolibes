\author{Tullio Carapella}
\title{L'iniziativa del MIUR sui BES: inaccettabile il metodo, pericolosa nel merito}
\label{cha:TullioCarapella14102013}
\maketitle
\epigraph{Tutti noi soffriamo di una malattia, di una malattia di base, per così dire, che è inseparabile da ciò che siamo e che, in un certo modo, fa ciò che siamo, se anzi non è più esatto dire che ciascuno di noi è la propria malattia, per causa sua siamo così poco, così come per causa sua riusciamo a essere tanto}{L'anno della morte di Ricardo Reis,José Saramago}
\section*{Questioni di metodo: mai mettere il carro avanti ai buoi.}
Lo scrivo in premessa a scanso di equivoci: quanto partorito dal ministero dell'Istruzione in merito ai \cit{bisogni educativi speciali}, in particolare la Direttiva del dicembre 2012\footcite{dir27Dic2012} e la Circolare 8 del marzo 2013\footcite{cm8_2013}, mi suona falso e inutile come i buoni propositi di un bambino in una letterina a babbo natale.

Mi riferisco, in particolare, non ai vistosi limiti, ma proprio a quanto di meglio contengono quei due documenti, cioè ai punti di forza che con tanto entusiasmo sottolineano a più riprese i difensori di questa nuova iniziativa ministeriale. Punti di forza facilmente sintetizzabili, perché risiedono nella volontà di coinvolgere tutto il corpo docente nel cercare risposte ai bisogni di ognuno dei nostri alunni, soprattutto di chi è maggiormente in difficoltà.

\section*{Bei propositi, nessuna sostanza.}
È un po' come se il ministero dell'agricoltura avesse emanato un decreto per invitare tutti gli interessati, vigili del fuoco, forestali e, perché no?, anche i contadini, a spegnere gli incendi estivi in meno di cinque minuti, prima che le fiamme si propaghino. Anche in questo caso dovremmo constatare che si tratta di una raccomandazione encomiabile, per poi magari aggiungere che un domani si dovrà pure pensare a comprare i Canadair, o almeno a fornire l'acqua alle pompe, o almeno a sostituire le pompe che nel tempo si sono usurate, o almeno ad evitare di dismetterle, le pompe, per ridurre le spese.

Chiedo venia per la metafora poco ortodossa, ma di questo stiamo parlando: una scuola che affronta ogni giorno nuove emergenze e lo fa con pochi mezzi, oggi molto meno di ieri, e spesso lo fa con scarsa preparazione, che non è un optional, se non vogliamo che ci mettiamo tutti quanti, certo con tanta buona volontà, a soffiare sulle fiamme nella speranza che sia così che si spegne un incendio.

Rispetto a questa obiezione è inaccettabile che si risponda \cit{intanto applica la legge, che domani vedremo di fornirti anche gli strumenti per farlo!}, perché:
\begin{description}
	\item[--]  I fini non si raggiungono senza i mezzi.
	
	 Nemmeno \cit{nel frattempo}. Spesso anzi ribaltare le sequenze di causa ed effetto provoca più danni che benefici. Ne è testimonianza questo grottesco inizio di anno scolastico, nel quale tra i docenti si sono accese discussioni e talvolta scontri per affermare ognuno la propria ardita interpretazione dell'ormai celebre perla ministeriale: le tipologie di Bes \cit{dovranno essere individuate sulla base di ben fondate considerazioni psicopedagogiche e didattiche}.
	 
	 Discussioni senza alcuna possibilità di sintesi, perché già solo il dire \cit{ben fondate} con tutta evidenza non vuol dire assolutamente nulla. E se oggi il Ministero intende correre ai ripari, lo farà inevitabilmente ingarbugliando ancor più la matassa e aggiungendo al danno la beffa.
	 
	 Circola in questi giorni la bozza della Circolare che il Miur sta per partorire, nella quale evidentemente si intende dettagliare meglio cosa sia un Bes e per limitarne l'uso, precisando che l'individuazione del \cit{bisognoso} va fatta necessariamente all'unanimità dal consiglio di classe, in presenza di una diagnosi clinica e di una richiesta da parte dei genitori. Una novità che evidentemente limiterebbe molto il ricorso a questo \cit{nuovo strumento}, se non fosse che negli stessi passaggi si dice pure che si può attivare un PDP anche in presenza di “difficoltà non meglio precisate”.
	 
	 È accettabile che in atti ufficiali ci si esprima con tanta superficialità? E qualcuno potrà spiegare al Miur che l'anno scolastico è cominciato da più di un mese? Avremo la forza di indignarci e alterarci almeno un po' perché prima ci hanno ordinato di “individuare i bes” e oggi, a cose fatte, ci spiegano pure meglio chi sono e come doveva avvenire l'individuazione? Possiamo prendere almeno questo piccolo dettaglio come testimonianza e misura del pressapochismo di chi ci vuole fermi e pronti a combattere al fronte senza schioppo e senza coordinate?
	 \item [--] È inaccettabile perché comunque non ci crediamo più.
	 
	  Non si può rimandare sempre a domani il momento in cui anche il governo sosterrà un impegno normativo (quella sui Bes non è nemmeno legge) e finanziario per dare sostanza ai suoi buoni propositi, perché sappiamo che dalle parti di Viale Trastevere questo domani non viene mai. Il recente passato è pieno di esempi di promesse ministeriali mai mantenute e basterebbe citare i piani di assunzione e i sontuosi stanziamenti per l'edilizia scolastica del ministro Profumo per capire che il difettuccio delle menzogne a buon mercato non dimora solo a destra.
	  
	  Certo qualcuno potrà notare che la Carrozza appare almeno meno antipatica e, trascurando il fatto che anche nell'ultimo documento di programmazione economica è previsto un taglio di fondi per l'istruzione, si può sottolineare che è già stato approvato un ottimo piano di assunzioni, in particolare per il personale di sostegno.
	  
	  A ben vedere, però, anche questo è meno confortante di quanto vogliano farci credere e, alla luce delle esperienze recenti, è tutto da dimostrare che venga attuato. Il DL 104 dello scorso settembre prevede infatti che tra il 2013 e il 2015 siano immessi in ruolo 26.684 docenti di sostegno (si aggiungerebbero ai 63.384 attuali). Si tratta quindi di circa 9.000 docenti all'anno, non pochi, ma nemmeno tantissimi, considerando che ai 63.000 andranno sottratti i tanti docenti di sostegno che decidono ogni anno di passare su posto comune, oppure vanno in pensione o, ahimè, a volte ci lasciano prima.
	  
	  Per questo primo anno, era prevista la tranche più piccola di immissioni: 4447 docenti da assumere entro il 7 ottobre che, a quanto pare, è già passato una settimana fa, ed invano…
	  
	  Seguendo il vecchio adagio di Totò, però, non si può non sottolineare che ancora una volta “è la somma che fa il totale”, perché le due cifre sopra richiamate danno in totale i 90.000 che dal 2006/2007 sono la cifra che il ministero dell'Istruzione (e quello delle Finanze) ritengono di non dover superare, malgrado negli ultimi sette anni sia cresciuto il numero di studenti con disabilità certificate. Per questa ragione, ogni anno di più, gli Uffici Scolastici hanno dovuto inseguire emergenze, tappare falle con coperte comunque corte, deliberare ad anno scolastico iniziato nuove assegnazioni di sostegno in deroga e questo ha fatto sì che i dati più recenti indichino che in Italia lavorino sul sostegno più di 101.000 docenti e che malgrado ciò è lontano l'obiettivo del rapporto medio di 1 insegnante ogni due studenti con disabilità.
	  
	  Allora quando la ministra Carrozza afferma che si vogliono stabilizzare “tutti” i docenti di sostegno arrivando a 90.000 (sempre che intanto il tempo si fermi per i “vecchi” 63.000), non può che suonare un campanello d'allarme. Che fine farebbero gli altri 11.000? Vuoi vedere che ha ragione Scataglini\footcite{Scataglini2013} quando afferma che la normativa sui Bes sarà anche uno strumento in grado di togliere, certo a piccoli passi, il sostegno ai ragazzi con “funzionamento intellettivo limite” /FUL), vale a dire a chi ha una disabilità non grave?
\end{description}
\section*{Addentrandoci appena un po' nel merito: BES, DA, DSA, FIL… chi parla male pensa male e può fare danni.}
Se pure per un attimo provassimo a far finta che le questioni di metodo non sono dirimenti e che vale la pena per un attimo addentrarsi nei meandri delle due letterine di natale, non potremmo che constatare che queste suscitano molti più dubbi che certezze. Le domande che vengono dalle scuole, infatti, sono tante e restano inevase.

L'Ufficio scolastico lombardo, ad esempio, ha confezionato un ottimo lavoro di sintesi\footcite{USRperlaLombardia2103} di quelle, spesso davvero interessanti, che vengono dagli insegnanti, chiaramente senza sentire il dovere di fornire alcuna risposta.

Io ne rilancio per brevità solo un paio, che non hanno certo la pretesa di essere espressione di tutto il vasto universo di dubbi che la normativa sui Bes ha suscitato, ma che sono un piccolo spaccato degli equivoci che questa questione ha generato. Nemmeno io darò risposte esaustive, non lo farei nemmeno se le avessi, perché dobbiamo pretendere che la matassa la sbrogli chi l'ha creata, se è in grado.

\section*{La vera categoria \cit{nuova} è quella dei BES?}
La stragrande maggioranza dei docenti crede che i Bes siano una categoria nuova, che si somma a quelle già \cit{normate} (alunni con disabilità con la 104 del '92, disturbi dell'apprendimento con la 170 del 2010), ma i “ben informati” sanno che non è esattamente così. La categoria dei Bes, infatti, non si somma a quelle DA (o DVA, vale a dire degli alunni con disabilità) e DSA, ma le “ingloba“. Come appare evidente dalla Circolare di marzo e dalla modulistica prodotta da diversi uffici scolastici a partire dal mese di giugno, sotto la sigla BES si includono tre grandi categorie: le due “già note” (disabilità e disturbi evolutivi specifici) più una terza, quella dello svantaggio (socio-economico e/o linguistico/culturale).

La vera novità è proprio relativa a quest'ultima categoria: per la prima volta si chiede ai consigli di classe di predisporre un piano personalizzato per chi vive in una situazione di generico “disagio“, tutto da interpretare. L'individuazione e la programmazione per i disagiati è dunque la vera novità. Bes è invece il contenitore che intende includere questi e “i bisogni” che la legislazione precedente ci chiedeva già di soddisfare.

Anche questa potrebbe rivelarsi una questione non di dettaglio. Se è vero infatti che “tutta la comunità educante” interviene ed è responsabile del soddisfacimento dei Bisogni Educativi Speciali (leggi: tutti i docenti sono di sostegno per i Bes) e se di pari passo è vero che anche gli alunni con disabilità sono Bes, allora non può che derivarne che anche per alunne e alunni con disabilità dovrà essere l'intero consiglio di classe a farsi carico della programmazione individualizzata.

In sé l'idea che tutti i colleghi si attivino pare molto bella finanche a me, se non fosse che il “tutti sono di sostegno” è stato usato per portare avanti l'idea che non occorre più il docente di sostegno specializzato, almeno da due o tre anni, cioè da quando un illuminante documento di Fondazione Agnelli, Caritas e Treellle (Comunione e Liberazione) l'ha proposto forse per la prima volta in modo esplicito.

Certo non assisteremo dall'oggi al domani alla cancellazione del sostegno, ma vale la pena sin da subito suonare un campanello d'allarme, soprattutto per il pericolo che alunne e alunni con disabilità non gravi vengano con poca grazia a partire da oggi calati nel calderone dei disagiati, o, come già accennato, si trasformino i FIL con sostegno nella categoria dei FIL senza sostegno (essendo previsti per i Funzionamenti Intellettivi Limite entrambe le opzioni).

\section*{Approccio medico o politico?}
Concludo con qualche considerazione su un quesito che mi affascina e forse mi preoccupa più di ogni altro: il concetto di BES è tratto dalla scienza medica? A chi ha criticato le misure ministeriali perché presuppongono, da parte dei docenti, l'attivazione di conoscenze in campo sanitario che non possediamo e non possiamo possedere, c'è stato chi, con ottime argomentazioni, ha risposto che il concetto di bisogno non è affatto medico.

Scrive, ad esempio, Dario Ianes:
\begin{quote}
	Il concetto di BES non è clinico, né tanto meno medico\mancatesto Il concetto di BES è politico, nella misura in cui stabilisce, come macro categoria, quali siano le situazioni che diano diritto a forme di personalizzazione nella scuola.
\end{quote}

La trovo una risposta convincente e vado avanti felice nella lettura dello stesso articolo del 30 maggio. Più avanti mi imbatto nella risposta ad un altro dubbio che, come già accennato, mi rode: quali sono le “ben fondate considerazioni pedagogiche” per l'individuazione di un BES? Risponde ancora Ianes:
\begin{quote}
Per me ben fondate significa fondate su un'antropologia ICF-OMS\mancatesto
\end{quote}

Resto interdetto, non tanto perché non ho capito la risposta, nel senso che non mi ha risolto il problema dell'individuazione del BES, anzi. Resto interdetto proprio per ciò che ho capito: so infatti che l'OMS è l'Organizzazione Mondiale della Sanità e che ICF sta per \cit{Classificazione Internazionale del Funzionamento, della Disabilità e della Salute}. Nasce forte il dubbio che la dimensione della scienza medica, che abbiamo appena deciso di far uscire dalla porta, stia facendo provocatoriamente capolino dalla finestra.

Ma forse è proprio qui il problema. A questo punto non provo più a cercare risposte preconfezionate e provo a costruirmele da solo, partendo dall'esperienza di vita reale, nelle nostre scuole e con i nostri ragazzi.

La scelta che facciamo, che abbiamo sempre fatto, molto prima che per noi inventassero i BES, è sempre stata politica e non medica. È la scelta di individuare con uno sguardo o con un breve scambio di battute il bisogno speciale che di giorno in giorno, di momento in momento, un soggetto tra i sei e i diciannove anni può esprimere e di regolare di conseguenza il livello di richieste.

È il riconoscimento del bisogno di ridere o di piangere, di starsene da solo con se stesso, perché oggi ho litigato con la tipa, perché ho le mie cose, perché questo ritardo comincia a preoccuparmi, perché voi italiani non vi capisco, perché non me ne frega niente, perché non ci capisco niente, perché mio padre picchia mia madre, perché mia madre ci ha lasciati proprio in una mattina di pioggia come questa, perché il ragazzo dei miei sogni mi ha finalmente sorriso e figuriamoci se posso pensare a Diocleziano!

È sempre stata una scelta politica, fatta di misure compensative o dispensative già molto prima che sapessimo che si chiamavano così, e osteggiata nei fatti da chi ha voluto imporci una scuola ogni anno più asettica e più “oggettiva” (sarebbe sin troppo facile dimostrare, ad esempio, che l'Invalsi è nemica dei Bisogni Speciali).

Qualcuno dirà che proprio per questo non c'è problema: la normativa attuale e futura non farà che regolamentare ciò che i docenti generosi hanno sempre fatto e costringere i meno generosi a farlo. Balle. Non sarà tanta carta da compilare a “creare equilibri” nella voglia di dare.

Qualcuno aggiunge pure, in modo più o meno esplicito, che una classe docente \cit{fondata su un'antropologia ICF} avrà finalmente diritto ad accedere a stipendi più decenti, perché non svolgerà più un lavoretto, ma un lavoro vero. Lavoriamo già oggi e lavoravamo ieri, magari dedicando meno tempo alle carte e più ai ragazzi. Mi rifiuto di pensare che rendendo più decenti i nostri stipendi (cosa che abbiamo il diritto di rivendicare sin da subito) lavoreremo meglio.

Quello che serve è formazione, ma non per imbrattare carte. Ciò che non serve è imporci di dare risposte mutuate dalla scienza medica a problemi individuati con criteri politici.

Ho letto una volta da qualche parte che il problema della medicina, almeno di quella che meglio conosciamo, è la pretesa di curare la malattia piuttosto che il malato. I progressi scientifici e tecnologici in campo sanitario hanno migliorato enormemente in pochi anni la capacità di produrre diagnosi precise. Individuato il male, la medicina spesso si limita ad avviare protocolli, fatti di interventi e trattamenti più o meno sempre uguali.

Allo stato attuale noi insegnanti potremo al più improvvisare diagnosi senza avere né le conoscenze, né le tecnologie proprie delle strutture sanitarie. Poi applicheremo protocolli fatti di crocette messe su modelli prestampati.

Potremo così rendere felice il nostro ministro. Daremo alla famiglia dell'alunno con ritardo lieve la scorciatoia per evitare il calvario della certificazione, ma gli negheremo il sostegno specializzato, concederemo l'uso della calcolatrice alla studentessa straniera e perdoneremo al nullatenente il non saper coniugare il verbo avere\footcite{Carapella2013}.