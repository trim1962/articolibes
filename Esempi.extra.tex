\section{Goniometria}
\label{sec:EsempiGoniometria}
\begin{table}[H]
	\caption{Trovare il seno e il coseno di un angolo come somma di angoli}
	\label{tab:Trovaresenocosenonotatangenteangolo}
	\begin{enumerate}
		\item Prerequisiti 
		\begin{itemize}
			\item I radicali
			\item Circonferenza goniometrica
			\item Seno, Coseno, Tangente, Cotangente
			\item Tabella angoli noti
			\begin {align*}
			%			\cos(\alpha-\beta)=	{}&\cos\alpha\cos\beta+\sen\alpha\sen\beta \\ 
			\cos(\alpha+\beta)=	{}&\cos\alpha\cos\beta-\sen\alpha\sen\beta \\ 
			%			\sen\left(\alpha-\beta\right)={} &\sen\alpha\cos\beta-\cos\alpha\sen\beta\\
			\sen\left(\alpha+\beta\right)={}&\sen\alpha\cos\beta+\cos\alpha\sen\beta
		\end{align*}
	\end{itemize}
	\item Scopo: Determinare le funzioni goniometriche di una angolo somma di angoli
	\item Testo: Determinare il valore di $\cos(\ang{75})$ e di $\sen(\ang{75})$
	\item Svolgimento: Si usano le formule di somma  del seno e del coseno.
	\begin{enumerate}
		\item $\ang{70}=\ang{45}+\ang{30}$
		\item coseno:
		\begin{enumerate}
			\item $\cos(\alpha+\beta)=\cos\alpha\cos\beta-\sen\alpha\sen\beta$
			\item $\cos(\ang{45}+\ang{30})=\cos\ang{45}\cos\ang{30}-\sen\ang{45}\sen\ang{30}$
			\item $=\dfrac{\sqrt{2}}{2}\dfrac{\sqrt{3}}{2}-\dfrac{\sqrt{2}}{2}\dfrac{1}{2}$
			\item $=\dfrac{\sqrt{6}}{4}-\dfrac{\sqrt{2}}{4}=\dfrac{\sqrt{6}-\sqrt{2}}{4}$
		\end{enumerate}
		\item seno:
		\begin{enumerate}
			\item $\sen\left(\alpha+\beta\right)=\sen\alpha\cos\beta+\cos\alpha\sen\beta$
			\item $\sen\left(\ang{45}+\ang{30}\right)=\sen\ang{45} \cos\ang{30}+\cos\ang{45}\sen\ang{30} $
			\item $\dfrac{\sqrt{2}}{2}\dfrac{\sqrt{3}}{2}+\dfrac{\sqrt{2}}{2}$
			\item $=\dfrac{\sqrt{6}}{4}+\dfrac{\sqrt{2}}{4}=\dfrac{\sqrt{6}+\sqrt{2}}{4}$
		\end{enumerate}
	\end{enumerate}
\end{enumerate}
\end{table}
\begin{table}[H]
	\caption{Trovare il seno e il coseno di un angolo come differenza di angoli}
	\label{tab:Trovarediffangoli1}
	\begin{enumerate}
		\item Prerequisiti 
		\begin{itemize}
			\item I radicali
			\item Circonferenza goniometrica
			\item Seno, Coseno, Tangente, Cotangente
			\item Tabella angoli noti
			\begin {align*}
			\cos(\alpha-\beta)=	{}&\cos\alpha\cos\beta+\sen\alpha\sen\beta \\ 
			%\cos(\alpha+\beta)=	{}&\cos\alpha\cos\beta-\sen\alpha\sen\beta \\ 
			\sen\left(\alpha-\beta\right)={} &\sen\alpha\cos\beta-\cos\alpha\sen\beta\\
			%\sen\left(\alpha+\beta\right)={}&\sen\alpha\cos\beta+\cos\alpha\sen\beta
		\end{align*}
	\end{itemize}
	\item Scopo: Determinare le funzioni goniometriche di una angolo somma di angoli
	\item Testo: Determinare il valore di $\cos(\ang{15})$ e di $\sen(\ang{15})$
	\item Svolgimento: Si usano le formule di somma  del seno e del coseno.
	\begin{enumerate}
		\item $\ang{10}=\ang{45}-\ang{30}$
		\item coseno:
		\begin{enumerate}
			\item $\cos(\alpha-\beta)=\cos\alpha\cos\beta+\sen\alpha\sen\beta$
			\item $\cos(\ang{45}-\ang{30})=\cos\ang{45}\cos\ang{30}+\sen\ang{45}\sen\ang{30}$
			\item $=\dfrac{\sqrt{2}}{2}\dfrac{\sqrt{3}}{2}+\dfrac{\sqrt{2}}{2}\dfrac{1}{2}$
			\item $=\dfrac{\sqrt{6}}{4}+\dfrac{\sqrt{2}}{4}=\dfrac{\sqrt{6}+\sqrt{2}}{4}$
		\end{enumerate}
		\item seno:
		\begin{enumerate}
			\item $\sen\left(\alpha-\beta\right)=\sen\alpha\cos\beta-\cos\alpha\sen\beta$
			\item $\sen\left(\ang{45}-\ang{30}\right)=\sen\ang{45} \cos\ang{30}-\cos\ang{45}\sen\ang{30} $
			\item $\dfrac{\sqrt{2}}{2}\dfrac{\sqrt{3}}{2}-\dfrac{\sqrt{2}}{2}$
			\item $=\dfrac{\sqrt{6}}{4}-\dfrac{\sqrt{2}}{4}=\dfrac{\sqrt{6}-\sqrt{2}}{4}$
		\end{enumerate}
	\end{enumerate}
\end{enumerate}
\end{table}