\author{Carlo Scataglini}
\title{Preferivo quando si parlava di Giulio, Lorenzo o Azjri…}
\label{cha:scataglini1}
\begin{abstract}
Perché, secondo Carlo Scataglini, la recente individuazione dei \glslink{besa}{BES}, da parte del Ministero dell'Istruzione, rischia di accentuare l'etichettatura e la medicalizzazione, svilendo la componente didattica e scolastica. \caporali{\'E questa –- scrive concludendo il suo intervento, che arricchisce ulteriormente il nostro dibattito sul presente e il futuro dell'inclusione scolastica –- non è certo una prospettiva inclusiva}
\end{abstract}
%\epigraph{Perché, secondo Carlo Scataglini, la recente individuazione dei \glslink{besa}{BES}, da parte del Ministero dell'Istruzione, rischia di accentuare l'etichettatura e la medicalizzazione, svilendo la componente didattica e scolastica. «E questa – scrive concludendo il suo intervento, che arricchisce ulteriormente il nostro dibattito sul presente e il futuro dell'inclusione scolastica – non è certo una prospettiva inclusiva»}{Carlo Scataglini}
\maketitle
\caporali{Senti, collega, io dovrei spiegare le equazioni. È un argomento un po' ostico ed è necessaria la massima attenzione. Tu e Marco usate forbici, colla e pennarelli e il rumore che fate crea distrazione per gli altri. Penso sia il caso\dots}. Penso sia il caso\dots che vi togliate dalle scatole! Questo era il senso. Questa era la richiesta: uscire dalla classe e andare in un altro luogo, l'aula di sostegno. Era il mese di maggio del 1990, una vita fa.

\'E una vita che faccio l'insegnante di sostegno alle medie e questa cosa può capitare. In verità prima, anni fa, capitava più spesso, ora quasi mai. I colleghi di matematica e anche quelli delle altre materie sono più sensibili, più preparati, più coinvolti. Dopo ventitré anni da quella volta mi è successo di nuovo, qualche giorno fa. Stavolta, però, non è stato un collega di matematica, né di un'altra materia, è stato l'\glslink{invalsia}{INVALSI}.

«I \glslink{dsaa}{DSA}, i disabili e tutti gli altri \glslink{besa}{BES}  possono anche svolgere le prove (tanto poi comunque non vengono conteggiate!), ma se hanno bisogno di lettura ad alta voce o dell'affiancamento di un docente di sostegno, devono svolgere la prova in un altro luogo!». Bene, le parole delle Note e delle Circolari \glslink{invalsia}{INVALSI} non erano propriamente queste, ma il senso concreto del messaggio e delle disposizioni era proprio quel «toglietevi dalle scatole!» di ventitré anni fa. Non è un bel segnale, no!
Già tutte le preoccupazioni da \glslink{besa}{BES} bastavano! A scuola, da tre mesi, non si parla d'altro, pure i collaboratori scolastici, il dirigente e il personale di segreteria ne parlano. Questo è positivo, certo. Si parla di \glslink{besa}{BES} e di inclusione, questo è un fatto nuovo. Chi conosce la scuola perché la frequenta e ci lavora ogni giorno riesce però a leggere tra le righe di una Direttiva o di una Circolare Ministeriale e a coglierne gli effetti concreti e nefasti.

Individuazione, progettazione e attuazione dell'intervento didattico personalizzato: questi sono i tre momenti del processo previsto per i \glslink{besa}{BES}. L'individuazione prevede una suddivisione dei \glslink{besa}{BES} in tre fasce, ben visibili nella bozza di Piano Annuale dell'Inclusività proposta dal Ministero:
\begin{description}
	\item[$-$] le disabilità certificate (Legge 104/92\footcite{Legge_104_92}, articolo 3, commi 1 e 3);
	\item i disturbi evolutivi specifici (\glslink{dsaa}{DSA}, \glslink{adhda}{ADHD}, \glslink{dopa}{DOP}, \glslink{fila}{FIL}, altro) [ove gli acronimi stanno rispettivamente per disturbi specifici dell'apprendimento, deficit di attenzione e iperattività, disturbi oppositivi provocatori e funzionamento intellettivo limite, N.d.R.];
	\item lo svantaggio (socio-economico, linguistico-culturale, comportamentale-relazionale).
\end{description}

In pratica, solo per la prima fascia, quella dei disabili in situazione di gravità, dovrebbero essere previsti il sostegno didattico e specializzato, con insegnante di sostegno, e il Piano Educativo Individualizzato (PEI). Per le altre due fasce non ci dovrebbe essere il sostegno di un docente specializzato, ma l'intervento degli insegnanti disciplinari che stilano e attuano un Piano Didattico Personalizzato (PDP).

Quegli alunni che ieri venivano chiamati “psicofisici” transiterebbero quasi in massa nei disturbi evolutivi specifici, senza sostegno specializzato. Non so dare percentuali precise a livello nazionale, posso parlare delle scuole della mia Provincia: le situazioni di gravità non superano il 20\%. Ciò vuol dire che c'è un 80\% di alunni che rischia di perdere il sostegno specializzato e, ovviamente, una percentuale corrispondente di docenti di sostegno che rischia di perdere il posto.
Ma torniamo all'individuazione. Si è detto in passato che andava superato il concetto di integrazione legata a un certificato medico. Bene, ora viene proposto un modello organizzativo che prevede l'inclusione legata a un numero imprecisato di certificati, riferiti a svariate sigle che corrispondono ad altrettanti \glslink{besa}{\glslink{besa}{BES}}. Prossimamente potremo ascoltare in Sala Professori discussioni preoccupate, del tipo: «Nella mia classe ho un \glslink{dsaa}{DSA}, un \glslink{adhda}{ADHD} e un \glslink{fila}{FIL}, è una situazione insostenibile!», «Vuoi scherzare? Faresti a cambio con me che ho due ITA \glslink{l2a}{L2} e tre “SVANT”, di cui uno socio-economico?».

Io resto un nostalgico e preferivo quando in Sala Professori non usavamo sigle, ma parlavamo dei tormenti, delle difficoltà, delle speranze e degli sforzi di Giulia, Jacopo, Pietro, Lorenzo e Azjri. L'individuazione dei \glslink{besa}{BES} rischia di portare all'etichettatura e alla medicalizzazione, con un massiccio e invasivo intervento sanitario nel momento dell'individuazione e anche in quelli successivi, che andrebbe ad erodere e a svilire la componente didattica e scolastica in genere. Non è una bella prospettiva. Non è decisamente una prospettiva inclusiva.
C'è poi la fase di progettazione, a livello di singolo istituto scolastico (Piano Annuale dell'Inclusività) e a livello di singolo alunno con \glslink{besa}{BES} (Piano Didattico Personalizzato). Qui gli insegnanti del Consiglio di Classe potranno contare su un massiccio numero di organismi e di gruppi di lavoro di consulenza.  \glslink{glha}{GLH} operativo, \glslink{glia}{GLI}, \glslink{ctia}{CTI}, \glslink{ctsa}{CST},\glslink{glipa}{GLIP}, \glslink{glhpa}{GLHP}, GLIR\glslink{glira}{GLIR} ecc. ecc.: sono queste le sigle dei gruppi di lavoro con cui i docenti dovranno familiarizzare per avere supporto e coordinamento interni ed esterni alla propria scuola. Chi sa di scuola e di risorse finanziarie a disposizione, sa comunque che questi gruppi di lavoro sono legati per lo più a competenze individuali e a spirito volontaristico.

Basti l'esempio del \glslink{glia}{GLI}. Si tratta del Gruppo di Lavoro interno a ogni singola scuola che prima si chiamava \glslink{glha}{GLH} (Gruppo di Lavoro per l'Handicap) e ora si chiama appunto \glslink{glia}{GLI} (Gruppo di Lavoro per l'Inclusione). Ne devono far parte alcuni insegnanti della scuola (di sostegno e curricolari) con particolari competenze, personale esperto esterno e genitori, oltre al dirigente scolastico. Il \glslink{glia}{GLI} – recita la Circolare 8/13\footcite{cm8_2013} sui \glslink{besa}{BES} del marzo scorso – si deve riunire almeno una volta al mese, ma, visto il numero molto corposo di compiti da svolgere, di fatto dovrà farlo molto più spesso. Il \glslink{glia}{GLI} può riunirsi in orario di servizio (come in orario di servizio? In orario di servizio gli insegnanti sono in classe!), oppure fuori dall'orario di servizio, utilizzando i fondi d'istituto (quelli, per intenderci, che non ci bastano per pagare le sostituzioni dei colleghi assenti e che ci creano problemi per l'acquisto di computer e Lavagne Interattive Multimediali, ma anche di carta per le fotocopie e perfino di carta igienica). Bene, quindi, numerosi gruppi di supporto all'inclusione.

Supporto a distanza! Già, perché questa è la grossa novità: bisogna passare da un sostegno in presenza, dall'insegnante di sostegno – segregante e ostacolo alla vera autonomia e all'inclusione – a un sostegno diversificato e di prossimità. «Dal sostegno ai sostegni!», questo è il nuovo slogan. Nella terza fase, infatti, quella dell'attuazione dell'intervento didattico con i \glslink{besa}{BES}, non ci sarà il sostegno tradizionale con insegnante specializzato, ma ci saranno «i sostegni di consulenza». In pratica, nella fase dell'attuazione didattica in classe, l'insegnante disciplinare, che verrà formato e aggiornato in maniera intensiva ed esaustiva, se la vedrà da solo. Da solo in una classe di ventinove alunni. Da solo con quattro o cinque Piani Didattici Personalizzati da attuare. Da solo con indispensabili metodologie inclusive da adottare, quali l'apprendimento cooperativo, il tutoring, l'apprendimento senza errori, l'approccio meta cognitivo che, da sempre, sono state utilizzate grazie alla contemporaneità e alla stretta collaborazione in classe col docente di sostegno specializzato. Da solo a utilizzare strategie didattiche “di cerniera”, quali l'adattamento dei libri di testo, la creazione di schede di aiuto disciplinare, la costruzione di mappe e schemi, la creazione di agende del compito, la gestione di laboratori inclusivi che, da sempre, hanno richiesto l'intervento di docenti realmente specializzati e competenti, come gli insegnanti di sostegno\dots
In realtà, metodologie inclusive e strategie didattiche “di cerniera” hanno bisogno di specializzazione e presenza operativa, non di consulenza a distanza. I nostri alunni con Bisogni Educativi Speciali e con disabilità hanno bisogno di sostegno reale e non di consulenza a distanza.

Quelli che ora chiamano \glslink{fila}{FIL} – funzionamento intellettivo limite – i bambini e i ragazzi che hanno difficoltà lievi, ma che rischiano ogni giorno di perdere contatto con l'attività della classe e di perdere motivazioni e amore per la scuola, hanno bisogno di presenza. Di una presenza specializzata che sappia semplificare e adattare un testo, creare uno schema, creare un'agenda del compito, organizzare un laboratorio inclusivo per la creazione condivisa di materiali didattici. Di una presenza specializzata che sappia ascoltare, sorridere, spronare, motivare, toccare, abbracciare e sudare. Che sappia poi, al momento giusto, staccarsi, allontanarsi, lasciar fare, rendere autonomi. C'è bisogno di una presenza specializzata, c'è bisogno di contatto operativo in classe, non da lontano. Altro che sostegno a distanza! Altro che sostegno di prossimità!
Sono convinto che una nuova cultura inclusiva sia necessaria. Sono convinto che l'attenzione verso tutti i Bisogni Educativi Speciali sia un'esigenza inderogabile della scuola pubblica e dei bambini e dei ragazzi che in essa vivono. Sono convinto che la ricerca e la letteratura sulle metodologie inclusive e sulle strategie didattiche “di cerniera” debbano moltiplicare i loro sforzi e produrre innovazioni sempre più funzionali. Ma non prendiamoci in giro!

Non proviamo a nascondere dietro a un impianto scientifico e valoriale – come quello dell'inclusione e dei \glslink{besa}{BES} – un tentativo di taglio delle risorse finanziarie alla scuola. Di taglio dei posti di sostegno. Non commettiamo, nello stesso tempo, l'errore di sprecare addirittura le risorse che ci sono. Le risorse professionali di chi, per venti o trent'anni, ha costruito una personale specializzazione fatta di studio e di esperienza diretta tra i banchi delle aule della scuola pubblica.

Innovare vuol dire far funzionare bene quello che c'è, lavorando sulle criticità, stimolando le idee, la partecipazione, la presa in carico. L'innovazione si produrrà rivedendo l'attuale e nefasto approccio ai \glslink{besa}{BES}, eliminando l'etichettatura sistematica e la medicalizzazione, coinvolgendo nell'intervento di sostegno chi è veramente specializzato, non lasciando soli i docenti di classe proprio nel momento decisivo della didattica personalizzata.

L'innovazione si attua facendo circolare veramente le risorse professionali che sono numerose all'interno della scuola pubblica, formando veramente tutti i docenti sulle metodologie e sulle strategie didattiche per l'inclusione. Valorizzare le professionalità e le specializzazioni. Questa è la vera innovazione! Questo è il vero sostegno alla scuola pubblica!\footcite{scataglini1}

Insegnante specializzato all'Aquila, formatore sulle metodologie di recupero e sostegno e docente a contratto di Didattica della Matematica per l'Integrazione alla Facoltà di Scienze della Formazione dell'Università dell'Aquila. Il presente testo è già apparso in «www.La Letteratura e noi.it», con il titolo “Le scatole e le etichette. Sull'\glslink{invalsia}{INVALSI} e i \glslink{besa}{BES} nella scuola pubblica” e viene qui ripreso, con minimi riadattamenti al diverso contenitore, per gentile concessione. 
30 maggio 2013
© Riproduzione riservata
