\author{Salvatore Nocera}
\title{Summum Ius Summa Iniuria}
\label{cha:nocera130613}
\begin{abstract}
Ovvero, letteralmente, “Somma giustizia, somma ingiustizia”, o meglio ancora, in un italiano più corrente, “il troppo diritto stroppia”. Usa una nota formula latina, Salvatore Nocera, nel replicare a chi, da queste stesse pagine, lo aveva criticato per avere “invaso il campo” della didattica con la normativa. Continua il nostro dibattito sull'inclusione scolastica e sulla recente Circolare Ministeriale riguardante i Bisogni Educativi Speciali
\end{abstract}
%\epigraph{Ovvero, letteralmente, “Somma giustizia, somma ingiustizia”, o meglio ancora, in un italiano più corrente, “il troppo diritto stroppia”. Usa una nota formula latina, Salvatore Nocera, nel replicare a chi, da queste stesse pagine, lo aveva criticato per avere “invaso il campo” della didattica con la normativa. Continua il nostro dibattito sull'inclusione scolastica e sulla recente Circolare Ministeriale riguardante i Bisogni Educativi Speciali}{di Salvatore Nocera}
\maketitle
La nota espressione latina del titolo –- Summum ius, summa iniuria, appunto –- che letteralmente significa “Somma giustizia, somma ingiustizia”, potrebbe anche essere tradotta in italiano corrente come “il troppo diritto stroppia”.

L'ho pensato dopo avere letto, su queste stesse pagine, la vibrante e appassionata replica del professor Daniele Lo Vetere (Inclusione, gride manzoniane e burocrazia~\ref{cha:danilellovetere1}), alla mia precedente risposta (I Bisogni Educativi Speciali, i fatti e le paure)~\ref{cha:nocera030613} all'articolo di Carlo Scataglini (Preferivo quando si parlava di Giulio, Lorenzo o Azjri…)~\ref{cha:scataglini1}, sentendomi toccato sul vivo con la critica per un'“invasione di campo” nella didattica da parte della normativa. Anch'io ritengo che il rischio ci sia, ma occorre guardare da vicino le cose, come fa il professor Lo Vetere, ma pure come faccio io, con l'orecchio ai richiami ineliminabili dello Stato di diritto che regola la scuola pubblica e che rilascia titoli di studio con un valore legale.
Voglio dire innanzitutto che le classi scolastiche – così come descritte nell'articolo rispetto al quale intendo replicare – esistevano già prima della recente Circolare Ministeriale 8/13\footcite{cm8_2013} sui Bisogni Educativi Speciali (BES). Non è stata quella Circolare a crearle, anzi essa ne ha preso giuridicamente coscienza, prevedendo per i casi di svantaggio e disagio non certificabili né come disabilità né come DSA [disturbi specifici dell'apprendimento, N.d.R.], la possibilità che su richiesta della famiglia, lo stesso Consiglio di Classe provvedesse a trovare delle soluzioni didattiche, tra le quali anche gli strumenti compensativi e dispensativi già previsti per legge per gli alunni con DSA, pretendendo che tali decisioni venissero verbalizzate, motivate e sottoscritte. E ciò per giustificare agevolazioni che – senza quelle motivazioni – potrebbero risultare discriminatorie ai danni di altri alunni che non godono degli stessi interventi, con conseguenti ricorsi ai Tribunali Amministrativi Regionali (TAR).

Quindi, a mio avviso, la Circolare 8/13\footcite{cm8_2013} potrà forse provocare ricorsi, ma questi saranno sempre inferiori di numero rispetto a quelli che si potrebbero determinare se si consentissero agevolazioni senza darne una motivazione.

Ma veniamo ai tre punti di critica sollevati da Lo Vetere, secondo il quale il Ministero risolverebbe i problemi riducendo i docenti per il sostegno, aumentando i casi di BES, scaricando sui docenti curricolari tutti questi problemi, senza uno straccio di formazione che non potrebbe per altro renderli “tuttologi”, cosa che comunque non avrebbe senso.

Rispetto al primo punto, va detto che il numero globale dei docenti per il sostegno negli ultimi anni è andato sempre crescendo, avendo superato la soglia dei 100.000, rispetto ai circa 200.000 alunni certificati con disabilità. Pertanto il rapporto medio nazionale si aggira intorno a un docente ogni due alunni. Ovviamente questa è una media e Trilussa ci ha inequivocabilmente spiegato “l'inganno”, con la storiella del “pollo pro capite”. Occorre perciò redistribuire meglio tale risorsa.

Il problema starebbe dunque nell'aumento dei BES, e tuttavia – lo ripeto – esso non è stato creato dalla Circolare 8/13 la quale si è limitata a prenderne atto. Il vero problema, quindi, sarebbe costituito dai docenti non sufficientemente preparati e dal sovraffollamento delle classi.

Sulla formazione dei docenti dobbiamo un po' chiarirci. Infatti la Circolare 8/13~\footcite{cm8_2013} non pretende che gli insegnanti curricolari divengano – seguendo l'esempio indicato dall'articolo di Lo Vetere – esperti di lingua straniera, per sapere insegnare l'italiano agli alunni stranieri. Assolutamente no, per questo ci sono i mediatori culturali. Stando alla Circolare, invece, i docenti del Consiglio di Classe devono concordare le modalità per facilitare a questi alunni l'apprendimento della lingua italiana tramite i mediatori culturali, la presenza dei compagni e la propria mediazione didattica, utilizzando eventualmente qualche parola di lingua degli alunni, come ci capita quando vogliamo parlare con persone straniere incontrate nelle nostre strade o all'estero.

Né è previsto dalla Circolare che i docenti curricolari debbano essere esperti in tutti i BES, comprese anche le molteplici forme di disabilità e i DSA. È sufficiente che essi abbiano una formazione iniziale sulle didattiche inclusive e una ricorrente formazione in servizio sui casi che di anno in anno si presenteranno nelle loro classi.

E ancora una volta voglio ribadire che questa realtà di classi composite non l'ha creata la Circolare 8/13 sui BES. Essa, anzi, cerca di fornire alcune indicazioni di soluzioni, che possono ovviamente essere migliorate.
Il vero problema di difficile soluzione – ma nemmeno questo è stato creato dalla Circolare di cui si parla – è costituito invece dal sovraffollamento delle classi. Per gli alunni con disabilità una soluzione è stata trovata con gli articoli 4 e 5 comma 2 del Decreto del Presidente della Repubblica (DPR) 81/09\footcite{DPR_81_2009} che ha fissato a venti (massimo ventidue) il tetto del numero di alunni per le prime classi e quindi a scorrimento in tutte quelle successive, a partire dall'anno scolastico 2009-2010. Se poi in concreto per molte classi queste norme non vengono rispettate neppure dagli Uffici Scolastici o dallo stesso Ministero, occorre che da cittadini, amanti dello Stato di diritto, ci si adoperi per farle rispettare anche, ove necessario, con il ricorso alla Magistratura.

Per gli altri casi di BES, purtroppo, la “riforma Moratti” prima e la “riforma Gelmini” poi hanno creato un problema che appare realmente come insuperabile, senza un intervento normativo correttivo che fissi dei limiti numerici anche per le classi con questi alunni.

So bene che questa scelta di politica legislativa comporta una spesa pubblica, ma il problema, come più volte detto, non lo ha creato la Circolare 8/13: essa lo ha trovato, cercando – a legislazione vigente – di rintracciare qualche soluzione. Spetta dunque proprio a noi – come cittadini che crediamo in una scuola pubblica, ove si realizzi un'inclusione di qualità - adoperarci perché la normativa possa orientarsi in tal senso.

Invito pertanto quanti criticano la Circolare 8/13\footcite{cm8_2013} – se condividono questa mia analisi – ad unirsi a noi della FISH (Federazione Italiana per il Superamento dell'Handicap), per trovare le soluzioni politiche possibili\footcite{nocera1}.


12 giugno 2013

Ultimo aggiornamento: 12 giugno 2013 19:13
© Riproduzione riservata
