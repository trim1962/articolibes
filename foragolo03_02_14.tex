\author{Flavio Fogarolo}
\title{Personalizzare, individualizzare e altri bizantinismi}
\phantomsection
\label{cha:fogarolo030214}
\begin{abstract}
«Secondo il dizionario – scrive Flavio Fogarolo – il termine “bizantinismo” è l'atteggiamento di chi eccede in sottigliezze o complica inutilmente i problemi. Proprio quello di cui la scuola ha oggi meno bisogno. E invece è proprio quello che si fa, ad esempio con “bizantinismi accademici” come la distinzione tra personalizzare e individualizzare, che non ha alcuna ricaduta operativa e non serve a niente nella scuola reale»
\end{abstract}
\maketitle
\datapub{3 febbraio 2014}

Che differenza c'è tra personalizzazione e individualizzazione? Se qualcuno me lo chiede, rispondo di solito con un'altra domanda: perché lo vuoi sapere? Se è per un esame o per un concorso, solidarizzo, per forza, e in qualche modo cerco di dare una risposta. Necessariamente nozionistica, purtroppo. Ma se il dubbio viene da insegnanti semplicemente curiosi di sapere, dico: «Lascia perdere, pensa a cose più serie e non complicarti la vita per nulla».
Secondo me, infatti, la distinzione tra personalizzare e individualizzare è un puro bizantinismo accademico che non ha alcuna ricaduta operativa e non serve a niente nella  scuola reale. Capisco che il linguaggio tecnico debba essere più preciso di quello comune, ma la specificità dev'essere razionale e avere una funzione. Il giurista distingue tra frode e dolo, mentre l'uomo della strada dirà semplicemente che c'è stato un imbroglio, ma le distinzioni legali hanno una logica e delle conseguenze ben precise e a nessuno verrebbe in mente di classificare minuziosamente le varie tipologie di reato, se poi le conseguenze – penali o amministrative – fossero in ogni caso sempre le stesse.

Parlando di didattica individualizzata e personalizzata, da un po' di anni a questa parte – non molti in verità – i “sacri testi” partono dall'assioma che i due termini non siano sinonimi, ma all'atto pratico, quando è il momento di dire cosa si fa in un caso e cosa nell'altro, le distinzioni scompaiono e tutti parlano di “didattica individualizzata-personalizzata”.

Si vedano ad esempio le Linee Guida per i DSA\footcite{LineGuida2011} (Disturbi Specifici dell'Apprendimento, N.d.R.) del 2011 e i modelli di PDP (Piani Didattici Personalizzati) proposti dal Ministero. È come quando si fa fare ai cittadini la raccolta differenziata, ma poi i rifiuti finiscono tutti nella stessa discarica! E d'altra parte, sarebbe ben duro fare distinzioni categoriche, considerando la nebulosità e le evidenti, e pesanti, contraddizioni tra le due pseudo definizioni, nonché le varianti e difformità.
Mi attengo quindi al “minimo sindacale” e riprendo ciò che dicono le citate Linee Guida: «L'azione formativa individualizzata pone obiettivi comuni per tutti i componenti del gruppo-classe ma è concepita adattando le metodologie in funzione delle caratteristiche individuali dei discenti\mancatesto. L'azione formativa personalizzata \mancatesto può  porsi  obiettivi  diversi per  ciascun discente».
Ma se didattica individualizzata significa differenziare la metodologia didattica e non gli obiettivi, perché per gli alunni con disabilità, con i quali si adatta praticamente tutto, si predispone un PEI (Piano Educativo Individualizzato)? Viceversa, è proprio per gli alunni con DSA che si dovrebbe intervenire sull'individualizzazione, considerando che – sempre secondo quelle Linee Guida -, per loro la progettazione educativa deve intervenire ma «non differenziare, in ordine agli obiettivi, il percorso di apprendimento dell'alunno o dello studente in questione». Se è così, quindi, perché il documento di programmazione dei DSA si chiama PDP, Piano Didattico Personalizzato, e non Individualizzato come sarebbe logico in base alla definizione?
Tra parentesi, consiglio sempre la “regola dei contrari” a quei “poveretti” di prima, quelli cioè che devono preparare un concorso o un esame e sono obbligati a distinguere tra due termini del tutto arbitrari e scollegati rispetto al linguaggio comune e che quindi tendono a confonderli: hai presente il PEI? Si chiama individualizzato, ma si fa personalizzazione. E il PDP si chiama personalizzato, ma riguarda l'individualizzazione. Basta capovolgere tutto, facile. O no?

In sostanza, la distinzione avrebbe senso se si dicesse, ad esempio, che con DSA e BES (Bisogni Educativi Speciali) si può fare solo individualizzazione, come sarebbe ovvio, visto che non si possono ridurre gli obiettivi, mentre con gli alunni con disabilità si può intervenire anche sulla personalizzazione. Ma la Legge 170/10\footcite{legge170} sui DSA dice chiaramente (articolo 5, comma 2 a) che questi alunni hanno diritto a una didattica «individualizzata e personalizzata» e quindi la distinzione non si può fare.
Chiariamo meglio: di sicuro è utile riflettere su come, caso per caso, vada differenziato l'insegnamento relativamente a metodi, tempi, obiettivi, con riferimento in particolare ai livelli culturali minimi di competenza e di cittadinanza che vanno garantiti a tutti. Contesto invece la pretesa di calare dall'alto una distinzione lessicale senza nessun riferimento all'uso reale, anche da parte degli addetti ai lavori (vedi PEI e PDP), e soprattutto senza nessuna ricaduta operativa.
Secondo il dizionario, tra l'altro, individualizzare e personalizzare, checché se ne dica, sono sinonimi e anche nella scuola erano considerati così fino a poco tempo fa. Ricordo ad esempio che negli Anni Novanta, prima dell'autonomia, le scuole dovevano stendere il PEI, inteso come Piano Educativo di Istituto (precursore del POF-Piano dell'Offerta Formativa), e in molte zone d'Italia, per evitare confusione di sigle, si è cambiato il nome dell'altro PEI, quello cioè di programmazione della disabilità, trasformandolo in PEP, Piano Educativo Personalizzato. Perché ovviamente, allora, i due termini erano tranquillamente considerati sinonimi.
E a proposito di dizionario: bizantinismo, ovvero atteggiamento di chi eccede in sottigliezze o complica inutilmente i problemi. Proprio quello di cui la scuola ha oggi meno bisogno\footcite{Fogarolo2014}.

 