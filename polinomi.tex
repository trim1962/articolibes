\chapter{Polinomi}
\label{cha:polinomi}
\minitoc
\mtcskip                                % put some skip here
\minilof                                % a minilof
\mtcskip                                % put some skip here
\minilot
\section{Somme}
\label{sec:somme}
La somma fra polinomi\index{Polinomi!somma} si ottiene sommando, se vi sono, i monomi simili che li compongono. La somma cambia solo la parte numerica di un monomio mai la sua parte letterale. Un esempio di somma è\nobs\vref{Fig:esempiosommamomomisimili1}
\begin{figure}
	
	 \begin{NodesList} %[margin=-3cm]
	 	\begin{align*}
	 		3a+2b^2+4a-6b^2  +2b&                           \AddNode\\
	 		(3+4)a+(2-6)b^2+2b&          \AddNode\\                                       		
	 		7a-4b^2+2b&   \AddNode\\
	 		\AddNode
	 	\end{align*}
	 	\LinkNodes{individuo i monomi simili}%
	 	%\LinkNodes{sommo i monomi simili}%
	 	\LinkNodes{$3+4$ e $2-6$}%
	 	
	 \end{NodesList}
	\caption{Esempio somma monomi simili}
	\label{Fig:esempiosommamomomisimili1}
\end{figure}
\section{Prodotti}
Il prodotto fra due polinomi\index{Polinomi!prodotto} si ottiene moltiplicando tutti i termini di un polinomio per tutti i termini dell'altro. 
\subsection{Monomio per un polinomio}
Il caso più semplice è il prodotto di un monomio per un binomio. Il monomio fuori della parentesi moltiplica il binomio all'interno come nella figura\nobs\vref{fig:prodottomonomiopolinomio}.
\begin{figure}
	\centering
	\begin{tikzpicture} [baseline]
	\node (a) {$A$};
	\node  (b)[right of=a, node distance=15]{$(B$};
	\node (p2)[right of=b, node distance=15]{$+$};
	\node (c)[right of=p2, node distance=15]{$C)$};
	\node (U)[right of=c, node distance=15]{$=$};
	\node (R)[right of=U, node distance=25]{$AB+AC$};
	\path (a.north) edge [bend left=45,-triangle 90](b.north);
	\path(a.south)edge [bend right=45,-triangle 90](c.south);
	\end{tikzpicture}
	\caption{Prodotto monomio polinomio}
	\label{fig:prodottomonomiopolinomio}
\end{figure}

Supponiamo di avere \[3(2a-5b)-7a(2a+3b)+5(a^2+3b)\]
si  procede come nella figura\nobs\vref{fig:monomiperpolinomi1}. In questo esempio abbiamo tre moltiplicazioni di un monomio per un binomio. A destra si vedono i risultati parziali e che poi sommati danno il risultato.

Supponiamo di avere \[2a(3a-6)-(6a^2-2b)-3a(a-2b)\]
si  procede come nella figura\nobs\vref{fig:monomiperpolinomi2}.Anche in questo esempio abbiamo tre moltiplicazioni di un monomio per un binomio nel secondo prodotto notare il $-1$ fuori della parentesi tonda che, pur non essendo indicato, in pratica cambia di segno i termini all'interno della parentesi. Anche qui a destra abbiamo  i risultati parziali delle tre moltiplicazioni e per finire la somma dei termini che da il risultato.
\begin{figure}
%\begin{NodesList}
%	\begin{displaymath}
%	\begin{aligned}
%	3(2a&-5b)&-7a(2a&+3b)&+5(a^2&+3b)&\AddNode[1]\AddNode[2]\AddNode[3]\AddNode[4]\AddNode[5]\AddNode[6]\AddNode[7]\AddNode[8]\\
%	6a+&\AddNode[1]\\ 
%	& -15b\AddNode[2]&\\
%	&& -14a^2\AddNode[3]&\\    
%	&& &-21ab\AddNode[4]&\\
%	&&& &+5a^2\AddNode[5]&\\
%	&&&& &+15b\AddNode[6]&\\
%	6a&-15b&-14a^2&-21ab&+5a^2&+15b\AddNode[7]&\\   
%	6a&-9a^2&-21ab&\AddNode[8]&\\   
%	\end{aligned}
%	\end{displaymath}
%	\tikzset{LabelStyle/.style = {left=0.1cm,pos=.5,text=red,fill=white}}
%	\LinkNodes[margin=3cm]{$3\cdot 2a$}%    
%	\LinkNodes[margin=2cm]{$3\cdot(-5b)$}%
%	\LinkNodes[margin=1cm]{$-7a\cdot(2a)$}%
%	\LinkNodes[margin=0cm]{$-7a\cdot(3b)$}%
%	\LinkNodes[margin=-1cm]{$5\cdot(a^2)$}%
%	\LinkNodes[margin=-2cm]{$5\cdot(3b)$}%  
%	\LinkNodes[margin=-3cm]{otteniamo}% 
%	\LinkNodes[margin=-3cm]{sommando}% 
%\end{NodesList}
\begin{NodesList}
	\begin{align*}
		\overbrace{3(2a-5b)}^{1}-\overbrace{7a(2a+3b)}^{2}+\overbrace{5(a^2+3b)}^{3}&\AddNode[1]\AddNode[2]\\
		6a+&\AddNode[1]&\tag{1}\\ 
		-15b&\AddNode[2]&\\
		6a-15b-7a(2a+3b)+5(a^2+3b)&\AddNode[3]\AddNode[4]\\
		-14a^2&\AddNode[3]&\tag{2}\\    
		-21ab&\AddNode[4]&\\
		6a-15b-14a-21ab+5(a^2+3b)&\nonumber\AddNode[5]\AddNode[6]\\
		+5a^2&\AddNode[5]&\tag{3}\\
		+15b&\AddNode[6]&\\
		6a-15b-14a^2-21ab+5a^2+15b&\nonumber\AddNode[7]&\\   
		6a-9a^2-21ab&\nonumber\AddNode[7] 
	\end{align*}
	\tikzset{LabelStyle/.style = {left=0.1cm,pos=0.5,text=red,fill=white}}
	\LinkNodes[margin=2cm]{$3\cdot 2a$}%    
	\LinkNodes[margin=2cm]{$3\cdot(-5b)$}%
	\LinkNodes[margin=2cm]{$-7a\cdot(2a)$}%
	\LinkNodes[margin=2cm]{$-7a\cdot(3b)$}%
	\LinkNodes[margin=2cm]{$5\cdot(a^2)$}%
	\LinkNodes[margin=2cm]{$5\cdot(3b)$}%  
	\LinkNodes[margin=2cm]{otteniamo}% 
	\LinkNodes[margin=2cm]{sommando}% 
\end{NodesList}
	\caption[]{Monomi per polinomi 1}
	\label{fig:monomiperpolinomi1}
\end{figure}

\begin{figure}
\begin{NodesList}
	\begin{align*}
		\overbrace{2a(3a-6)}^{1}-\overbrace{(6a^2-2b)}^{2}-\overbrace{3a(a-2b)}^{3}&\AddNode[1]\AddNode[2]\\
		6a^2+&\AddNode[1]&\tag{1}\\ 
		-12a&\AddNode[2]&\\
		6a^2-12a-(6a^2-2b)-3a(a-2b)&\AddNode[3]\AddNode[4]\\
		-6a^2&\AddNode[3]&\tag{2}\\    
		+2b&\AddNode[4]&\\
		6a^2-12a-6a^2+2b-3a(a-2b)&\AddNode[5]\AddNode[6]\\
		-6a&\AddNode[5]&\tag{3}\\
		+6ab&\AddNode[6]&\\
		6a^2-12a-6a^2+2b-6a+6ab&\AddNode[7]\\   
		-18a+2b+6ab&\AddNode[7]   
	\end{align*}
	\tikzset{LabelStyle/.style ={left=0.1cm,pos=0.5,text=red,fill=white}}
	\LinkNodes[margin=2cm]{$2a\cdot 3a$}%    
	\LinkNodes[margin=2cm]{$3\cdot(-5b)$}%
	\LinkNodes[margin=2cm]{$-1\cdot(6a^2)$}%
	\LinkNodes[margin=2cm]{$-1\cdot(-2b)$}%
	\LinkNodes[margin=2cm]{$-3a\cdot(a)$}%
	\LinkNodes[margin=2cm]{$-3a\cdot(-2b)$}%  
	\LinkNodes[margin=2cm]{otteniamo}% 
	\LinkNodes[margin=2cm]{sommando}% 
\end{NodesList}
	\caption[]{Monomi per polinomi 2}
	\label{fig:monomiperpolinomi2}
\end{figure}

\subsection{Polinomio per polinomio}
In questo caso il polinomio nella prima parentesi moltiplica il polinomio della seconda parentesi. In pratica ogni monomio della prima parentesi moltiplica ogni monomio della seconda come nella figura\nobs\vref{fig:prodottopolinomiopolinomio}

Supponiamo di avere \[(3a-2b)(2a-b)+(2a^2-2)(2-a)\]
si  procede come nella figura\nobs\vref{fig:polinomioperpolinomio1}. In questo esempio abbiamo due moltiplicazioni di un binomio per un binomio. A destra i passaggi parziali. Infine sommiamo  gli elementi simili e otteniamo la soluzione.


Supponiamo di avere \[(xy-2)[(xy-2)xy+4+2xy]-(xy-2)(x^2y^2+2xy+4)\]
si  procede come nella figura\nobs\vref{fig:polinomioperpolinomio2}. In questo esempio abbiamo quattro moltiplicazioni  fra vari polinomi. A complicare le cose vi sono le regole di precedenza. A destra i vari risultati parziali. Si procede seguendo l'ordine indicato sopra l'espressione. 
\begin{figure}
	\centering
	\begin{tikzpicture}
	\node (a) {$(A$};
	\node (p1)[right of=a, node distance=15]{$+$};
	\node (b)[right of=p1, node distance=15]{$B)$};
	\node (c)[right of=b, node distance=15]{$(C$};
	\node (p2)[right of=c, node distance=15]{$+$};
	\node (d)[right of=p2, , node distance=15]{$D)$};
	\node (U)[right of=d, node distance=15]{$=$};
	\node (R)[right of=U, node distance=50]{$AC+AD+BC+BD$};
	%\draw[->](a.north)to [bend left=45](c.north);
	\path (a.north) edge [bend left=45,-triangle 90](c.north);
	\path(a.north)edge [bend left=45,-triangle 90](d.north);
	\path(b.south)edge [bend right=45,-triangle 90](c.south);
	\path(b.south)edge [bend right=45,-triangle 90](d.south);
	
	\end{tikzpicture}
	\caption{Prodotto polinomio polinomio}
	\label{fig:prodottopolinomiopolinomio}
\end{figure}
\begin{figure}
\begin{NodesList}
	\begin{align*}
		\overbrace{(3a-2b)(2a-b)}^{1}+\overbrace{(2a^2-2)(2-a)}^{2}&\nonumber\AddNode[1]\AddNode[2]\AddNode[3]\AddNode[4]\\
		6a^2&\AddNode[1]&\tag{1}\\ 
		-3ab&\AddNode[2]&\\
		-4ab&\AddNode[3]&\\    
		+2b^2&\AddNode[4]&\\
		6a^2-7ab+2b^2+(2a^2-2)(2-a)&\nonumber\AddNode[5]\AddNode[6]\AddNode[7]\AddNode[8]\\
		4a^2&\AddNode[5]&\tag{2}\\
		-2a^3&\AddNode[6]&\\
		-4 &\AddNode[7]&\\   
		2a&\AddNode[8]&\\   
		6a^2-7ab+2b^2+4a^2-2a^3-4+2a&\nonumber\AddNode[9]\\
		10a^2+2b^2-7ab-2a^3-4+2a&\nonumber\AddNode[9]
	\end{align*}
	\tikzset{LabelStyle/.style = {left=.5cm,pos=.5,text=red,fill=white}}
	\LinkNodes[margin=2cm]{$3a\cdot 2a$}%    
	\LinkNodes[margin=2cm]{$3a\cdot(-b)$}%
	\LinkNodes[margin=2cm]{$-2b\cdot(2a)$}%
	\LinkNodes[margin=2cm]{$-2b\cdot(-b)$}%
	\LinkNodes[margin=2cm]{$2a^2\cdot(2)$}%
	\LinkNodes[margin=2cm]{$2a^2\cdot(-a)$}%  
	\LinkNodes[margin=2cm]{$-2\cdot 2$}% 
	\LinkNodes[margin=2cm]{$-2\cdot -a$}% 
	\LinkNodes[margin=2cm]{Sommando}%
\end{NodesList}

	\caption[]{Polinomio per polinomio 1}
	\label{fig:polinomioperpolinomio1}
\end{figure}

\begin{figure}
	\begin{NodesList}
		
		\begin{align*}
			\overbrace{(xy-2)\overbrace{[\underbrace{(xy-2)xy}_{1}+4+2xy]}^{2}}^{3}-\overbrace{(xy-2)(x^2y^2+2xy+4)}^{4} &\AddNode[1]\AddNode[2]\\
			x^2y^2&\AddNode[1]\\ 
			-2xy&\AddNode[2] \\
			\overbrace{(xy-2)\overbrace{[x^2y^2-2xy+4+2xy]}^{2}}^{3}-\overbrace{(xy-2)(x^2y^2+2xy+4)}^{4} &\AddNode[3]\\
			\overbrace{(xy-2)[x^2y^2+4]}^{3}-\overbrace{(xy-2)(x^2y^2+2xy+4)}^{4} &\AddNode[3]\\
			\overbrace{(xy-2)[x^2y^2+4]}^{3}-\overbrace{(xy-2)(x^2y^2+2xy+4)}^{4} &\AddNode[4]\AddNode[5]\AddNode[6]\AddNode[7]\\
			x^3y^3&\AddNode[4]\\    
			4xy&\AddNode[5]\\
			-2x^2y^2&\AddNode[6]\\
			-8&\AddNode[7]\\
			x^3y^3+4xy-2x^2y^2-8-\overbrace{(xy-2)(x^2y^2+2xy+4)}^{4} &\AddNode[8]\AddNode[9]\AddNode[10]\AddNode[11]\AddNode[12]\AddNode[13]\\
			-x^3y^3&\AddNode[8]\\
			-2x^2y^2&\AddNode[9]\\
			-4xy&\AddNode[10]\\   
			2x^2y^2&\AddNode[11] \\ 
			4xy&\AddNode[12]\\     
			8&\AddNode[13]\\   
			x^3y^3+4xy-2x^2y^2-8-x^3y^3-2x^2y^2-4xy+2x^2y^2+4xy+8 &\AddNode[14]\\
			4xy-2x^2y^2 &\AddNode[14]
		\end{align*}
		\tikzset{LabelStyle/.style = {left=0.2cm,pos=.5,text=red,fill=white}}
		\LinkNodes[margin=0cm]{$xy\cdot xy$}%         
		\LinkNodes[margin=0cm]{$-2\cdot xy$}%
		\LinkNodes[margin=0cm]{Sommando}%
		\LinkNodes[margin=0cm]{$xy\cdot(x^2y^2)$}%
		\LinkNodes[margin=0cm]{$4\cdot xy$}%
		\LinkNodes[margin=0cm]{$-2\cdot x^2y^2$}%
		\LinkNodes[margin=0cm]{$-2\cdot +4$}%
		\LinkNodes[margin=0cm]{$(-1)\cdot xy\cdot x^2y^2$}%
		\LinkNodes[margin=0cm]{$(-1)\cdot xy\cdot 2xy$}%
		\LinkNodes[margin=0cm]{$(-1)\cdot xy\cdot 4$}%
		\LinkNodes[margin=0cm]{$(-1)\cdot (-2)\cdot x^2y^2$}%
		\LinkNodes[margin=0cm]{$(-1)\cdot (-2)\cdot 2xy$}%
		\LinkNodes[margin=0cm]{$(-1)\cdot (-2)\cdot 4$}%
		\LinkNodes[margin=0cm]{Sommando}%
	\end{NodesList}
\caption[]{Polinomio per polinomio 2}
\label{fig:polinomioperpolinomio2}
\end{figure}
\subsection{Quadrato del binomio}
Il quadrato di un binomio è un particolare prodotto di un binomio per se stesso. Si calcola utilizzando la regola\[(A+B)^2=A^2+B^2+2AB\] che va letto:<< Il quadrato di in binomio è uguale al quadrato del primo termine più il quadrato del secondo termine più il doppio del prodotto del primo termine per il secondo termine>>. La figura\nobs\vref{fig:prodottoquadrattobinomio} spiega quanto detto prima.
\begin{figure}
	\centering
\begin{tikzpicture}
\node (a) {$(A$};
\node (p1)[right of=a, node distance=15]{$+$};
\node (b)[right of=p1, node distance=15]{$B)^2$};
\node (R)[right of=b, node distance=15]{$=$};
\node (as)[right of=R, node distance=15]{$A^2$};
\node (s1)[right of=as, node distance=15]{$+$};
\node (bs)[right of=s1, node distance=15]{$B^2$};
\node (s2)[right of=bs, node distance=15]{$+$};
\node (dp)[right of=s2, node distance=15]{$2AB$};
\path (a.north) edge [bend left=45,-triangle 90](as.north);
\path (a.north) edge [bend left=45,-triangle 90](dp.north);
\path(b.south)edge [bend right=45,-triangle 90](bs.south);
\path(b.south)edge [bend right=45,-triangle 90](dp.south);
\end{tikzpicture}
	\caption{Quadrato del binomio}
	\label{fig:prodottoquadrattobinomio}
\end{figure}
\begin{figure}
\begin{NodesList}
	\begin{align*}
		\left(a+2b\right)^2&\AddNode[1]\AddNode[2]\AddNode[3]\AddNode[4]\\
		+a^2&\AddNode[1]&\\ 
		+4b^2&\AddNode[2]&\\
		+4ab&\AddNode[3]\\
		\left(a+2b\right)^2=a^2+4b^2+4ab&\AddNode[4]
	\end{align*}
	\tikzset{LabelStyle/.style = {left=0.1cm,pos=0.5,text=red,fill=white}}
	\LinkNodes[margin=2cm]{$a\cdot a$}%    
	\LinkNodes[margin=2cm]{$2b\cdot 2b$}%
	\LinkNodes[margin=2cm]{$2\cdot a \cdot 2b$}%
	\LinkNodes[margin=2cm]{ottengo}% 
\end{NodesList}
	\caption[]{Quadrato binomio 1}
	\label{fig:polinomiquadratobinomio1}
\end{figure}
\begin{figure}
\begin{NodesList}
	\begin{align*}
		\left(2x-3y\right)^2&\AddNode[1]\AddNode[2]\AddNode[3]\AddNode[4]\\
		+4x^2&\AddNode[1]&\\ 
		+9y^2&\AddNode[2]&\\
		-12xy&\AddNode[3]\\
		\left(2x-3y\right)^2=4x^2+9y^2-12xy&\AddNode[4]
	\end{align*}
	\tikzset{LabelStyle/.style = {left=0.1cm,pos=0.5,text=red,fill=white}}
	\LinkNodes[margin=2cm]{$2x\cdot 2x$}%    
	\LinkNodes[margin=2cm]{$(-3y)\cdot (-3y)$}%
	\LinkNodes[margin=2cm]{$2\cdot (2x) \cdot(-3y)$}%
	\LinkNodes[margin=2cm]{ottengo}% 
\end{NodesList}
	\caption[]{Quadrato binomio 2}
	\label{fig:polinomiquadratobinomio2}
\end{figure}
\begin{figure}
\begin{NodesList}
	\begin{align*}
		\left(2-z\right)^2&\AddNode[1]\AddNode[2]\AddNode[3]\AddNode[4]\\
		+4&\AddNode[1]&\\ 
		+z^2&\AddNode[2]&\\
		-4z&\AddNode[3]\\
		\left(2-z\right)^2=4+z^2-4z&\AddNode[4]
	\end{align*}
	\tikzset{LabelStyle/.style = {left=0.1cm,pos=0.5,text=red,fill=white}}
	\LinkNodes[margin=2cm]{$2\cdot 2$}%    
	\LinkNodes[margin=2cm]{$(-z)\cdot (-z)$}%
	\LinkNodes[margin=2cm]{$2\cdot (2) \cdot(-z)$}%
	\LinkNodes[margin=2cm]{ottengo}% 
\end{NodesList}
	\caption[]{Quadrato binomio 3}
	\label{fig:polinomiquadratobinomio3}
\end{figure}
\subsection{Differenza di quadrati}
In questo caso il prodotto è fra due binomi in cui un termine mantiene il suo segno mentre l'altro lo cambia. Si calcola utilizzando la regola \[(A+B)(A-B)=A^2-B^2 \] che va letto:<< Al prodotto fra la somma di due termini con la loro differenza è uguale al quadrato del primo termine meno il quadrato del secondo termine>>. La figura\nobs\vref{fig:prodottodifferenzaquadrati} mostra come procedere.
\begin{figure}
	\centering
\begin{tikzpicture}
\node (ap) {$(A$};	
\node (P)[right of=ap, node distance=15]{$+$};
\node (bp)[right of=P, node distance=15]{$B)$};
\node (am)[right of=bp, node distance=15]{$(A$};
\node (M)[right of=am, node distance=15]{$-$};
\node (bm)[right of=M, node distance=15] {$B)$};
\node (S)[right of=bm, node distance=15] {$=$};
\node (aq)[right of=S, node distance=15] {$A^2$};
\node (M2)[right of=aq, node distance=15] {$-$};
\node (bq)[right of=M2, node distance=15] {$B^2$};
\path (ap.north) edge [bend left=45,-triangle 90](aq.north);
\path (bp.north) edge [bend left=45,-triangle 90](bq.north);
\path(am.south)edge [bend right=45,-triangle 90](aq.south);
\path(bm.south)edge [bend right=45,-triangle 90](bq.south);
\end{tikzpicture}
	\caption{Differenza di quadrati}
	\label{fig:prodottodifferenzaquadrati}
\end{figure}

