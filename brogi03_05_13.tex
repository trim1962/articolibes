\author{Daniele Brogi}
\title{Prima applicare le leggi, poi pensarne una modifica}
\phantomsection
\label{brogi030513}
 %\epigraph{Prende spunto, Daniele Brogi, da un precedente intervento da noi pubblicato, sostenendo che «in buona parte del Paese l'applicazione legislativa in materia di integrazione scolastica non è nemmeno mai stata presa in considerazione». «Prima dunque di reputare inadeguato quanto finora è stato in buona parte sperimentato solo a livello teorico – conclude -, dovremmo tutti insieme esigerne l'applicazione»}{Daniele Brogi}
\begin{abstract}
Prende spunto, Daniele Brogi, da un precedente intervento da noi pubblicato, sostenendo che \caporali{in buona parte del Paese l'applicazione legislativa in materia di integrazione scolastica non è nemmeno mai stata presa in considerazione». «Prima dunque di reputare inadeguato quanto finora è stato in buona parte sperimentato solo a livello teorico –- conclude --, dovremmo tutti insieme esigerne l'applicazione}
\end{abstract}
\maketitle
\datapub{3 Maggio 2013}
Alla luce dell'interessante dibattito che sta avendo luogo in «Superando.it» in materia di integrazione scolastica, vorrei intervenire ancora, rispetto al punto di vista dell'insegnante di sostegno Giulia Giani, autrice del pezzo intitolato “Scambio di ruoli” per un'efficace inclusione scolastica\pageref{cha:giani300413}, che reputo estremamente coerente con quanto necessario rispetto alle competenze degli insegnanti. Ne condivido anche il concetto di “cattedra-mista”, e tuttavia ritengo importante far presente che l'affermazione «se in decenni di tentativi di applicazione della legge non ci siamo riusciti [ad applicare le leggi vigenti per realizzare una buona inclusione scolastica, N.d.R.] e siamo giunti a un punto di crisi\dots», non corrisponde alla reale situazione in atto, in quanto in buona parte del Paese l'applicazione legislativa in materia d'integrazione scolastica non è nemmeno mai stata presa in considerazione, sin dagli albori della sua dichiarata valenza legislativa.

E questo è un concetto che differisce di molto da quanto sostenuto, poiché di fronte all'evidenza della totale disapplicazione, credo non vi siano nemmeno elementi e dati utili a mettere in discussione il sistema.

Inoltre, reputo fondamentale specificare che un tale investimento nella formazione del personale docente – per quanto di piena condivisione – dovrebbe essere di fatto coerentemente motivato da oggettive esigenze rilevabili nel corretto svolgimento dei Gruppi di Lavoro Handicap Operativi (GLHO), venendo meno alla diffusa pratica di assegnazione degli insegnanti di sostegno con rapporto nel migliori casi di 1/2 e in totale omissione delle richiesta di deroga, ovvero in totale disapplicazione di quanto doverosamente sostenuto ad esempio dalla Legge 296/2006\footcite{Legge_296_2006} (articolo 1, comma 605, lettera b), ove si parla di assegnazione «dopo un'attenta valutazione delle effettive esigenze rilevate».
Riterrei pertanto coerente che il primo passo verso un effettivo cambiamento per arrivare a un'applicazione qualitativa dell'inclusione scolastica, partisse da un'adeguata conoscenza della materia legislativa in vigore.

Una dichiarazione, quest'ultima, che nasce da una serie di dati raccolti oggettivamente in un'intera Provincia della Lombardia, all'interno della quale il corpo docenti di sostegno è quasi totalmente disinformato sulle norme che regolano lo svolgimento dei GLHO, dove la totalità dei genitori non viene convocata, dove i GLH d'Istituto si svolgono all'insegna di in processi interpretativi (senza l'effettiva convocazione delle parti in causa), e ancora dove il personale sanitario è sotto strutturato, per poter dare un coerente supporto allo svolgimento dei parametri sopracitati, e dove persino Circolari dell'Ufficio Scolastico Regionale – come ad esempio una Dichiarazione d'Intenti prodotta il 9 maggio 2011 – non raggiungono le scrivanie dei Dirigenti Scolastici di buona parte del territorio al quale sono destinate.

In conclusione credo che probabilmente, prima di reputare inadeguato quanto finora è stato in buona parte sperimentato solo a livello teorico, dovremmo tutti insieme esigerne l'applicazione, dopodiché avremo anche gli elementi necessari a richiedere eventuali modifiche\footcite{Brogi2013}.