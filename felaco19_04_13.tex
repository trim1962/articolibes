\author{Giuseppe Felaco}
\title{Non più solo i genitori!}
\label{cha:felaco190413}
\begin{abstract}
\cit{Amici} veri, scelti e proposti dalle famiglie, un approccio ludico alle richieste educative, l'importanza dell'acqua e soprattutto la frequentazione e l'inserimento delle persone con disabilità nei luoghi e nelle situazioni abituali dei coetanei normotipi: una riflessione su alcune strategie possibili, per alleggerire il peso delle famiglie, da sostenere tramite assegni di cura, erogati dai Comuni
\end{abstract}
\maketitle
\datapub{19 Aprile 2013}
%\epigraph{
%“Amici” veri, scelti e proposti dalle famiglie, un approccio ludico alle richieste educative, l'importanza dell'acqua e soprattutto la frequentazione e l'inserimento delle persone con disabilità nei luoghi e nelle situazioni abituali dei coetanei normotipi: una riflessione su alcune strategie possibili, per alleggerire il peso delle famiglie, da sostenere tramite assegni di cura, erogati dai Comuni}{Giuseppe Felaco}
Vien da dire che <<ci vorrebbe un amico/a>>, non necessariamente specializzato – perché non c'è una scuola, per imparare a fare l'amico – quanto più possibile “coetaneo”, pagato a ore, ma scelto e proposto dalla famiglia.
La storia, infatti, ci insegna che “l'osservazione e l'imitazione dell'altro” ha consentito ai nostri progenitori, via via fino a noi, di migliorare le proprie condizioni sociali ed economiche e che quando si ha come riferimento un modello positivo da imitare, più facilmente si può cambiare e migliorare. Se poi all'osservazione e all'imitazione, associamo l'approccio ludico, si possono ottenere risultati ancora più importanti. Perché le richieste educative non saranno più di peso, perché agli occhi dei bambini/ragazzi appariranno come un gioco. E sarà così, quindi, che il gioco – da illusione creativa – diventerà “antidoto” all'isolamento e capacità di interagire con il mondo.

Altro aspetto propositivo è l'acqua, la piscina, perché facilita e aumenta i tempi di attenzione, favorisce la gestione degli aspetti emotivi, promuove l'equilibrio motorio e riduce drasticamente i comportamenti aggressivi e le stereotipie. Resta comunque preminente la frequentazione e l'inserimento delle persone con disabilità nei luoghi e nelle situazioni abituali dei coetanei normotipi, per favorirne la conoscenza attraverso il meccanismo dell'imitazione.
Insomma, se vogliamo che raggiungano la normalità facciamogliela vivere!

E per sostenere le spese di tutto ciò? Una soluzione può essere l'assegno di cura – chiamato anche voucher o assegno terapeutico – contributo economico che i Comuni possono erogare alle famiglie che si impegnano ad assistere a casa, affrontandone anche i costi, persone non autosufficienti che altrimenti dovrebbero affidarsi a strutture di ricovero.
L'obiettivo di questa forma di assistenza è dunque quello di promuovere la domiciliarità, per ridurre il ricorso ai ricoveri in strutture residenziali che costerebbero allo Stato e alle tasche del Cittadino dieci volte tanto. È tempo di crisi, no, e quindi facciamo un po' di economia!\footcite{Felaco2013a}

Genitore.

19 aprile 2013

© Riproduzione riservata