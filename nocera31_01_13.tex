\author{Salvatore Nocera}
\title{La Direttiva Ministeriale sui Bisogni Educativi Speciali}
\phantomsection
\label{cha:nocera310113}
\begin{abstract}
È un documento importante, la recente Direttiva Ministeriale sui Bisogni Educativi Speciali (BES), che sostanzialmente completa il quadro italiano dell’inclusione scolastica. Ma proprio per tale importanza, i vari passaggi non del tutto chiari di esso rendono necessaria una riunione dell’attuale Osservatorio Ministeriale sulla Disabilità, come già richiesto anche dalle Federazioni FISH e FAND
\end{abstract}
\maketitle
\datapub{31 Gennaio 2013}
%\epigraph{È un documento importante, la recente Direttiva Ministeriale sui Bisogni Educativi Speciali (BES), che sostanzialmente completa il quadro italiano dell’inclusione scolastica. Ma proprio per tale importanza, i vari passaggi non del tutto chiari di esso rendono necessaria una riunione dell’attuale Osservatorio Ministeriale sulla Disabilità, come già richiesto anche dalle Federazioni FISH e FAND}{Salvatore Nocera}

È certamente un documento di notevole importanza la Direttiva\footcite{dir27Dic2012} del 27 dicembre 2012, intitolata Strumenti d'intervento per alunni con Bisogni Educativi Speciali e organizzazione territoriale per l'inclusione scolastica, pubblicata dal Ministero dell'Istruzione in questo mese di gennaio. È importante perché accoglie una serie di orientamenti da tempo presenti nei Paesi dell'Unione Europea, completando, in sostanza, il quadro italiano dell'inclusione scolastica.
Com'è noto, infatti, il nostro sistema è stato il primo in Europa a introdurre l'inclusione scolastica generalizzata degli alunni con disabilità e ha di recente riordinato i principi della stessa, con le Linee Guida\footcite{LineGuida2009} del 4 agosto 2009. A seguito poi della Legge 170/10\footcite{legge170}, ha emanato le Linee Guida\footcite{LineGuida2011} del 12 luglio 2011, relative all'inclusione scolastica degli alunni con DSA (disturbi specifici d'apprendimento, ovvero dislessia, disgrafia, dicalculia e disortografia). Ora, con questa nuova Direttiva, il Ministero fornisce indicazioni organizzative anche sull'inclusione di quegli alunni che non siano certificabili né con disabilità, né con DSA, ma che abbiano difficoltà di apprendimento dovute a svantaggio personale, familiare e socio-ambientale.

Con i termini Bisogni Educativi Speciali (d'ora in poi BES), si intendono esattamente:
\begin{description}
	\item[$-$] alunni con disabilità;
	\item[$-$] alunni con DSA;
	\item[$-$] alunni con svantaggio socio-economico, linguistico, culturale.
\end{description}

A tutte queste tipologie, la Direttiva del 27 dicembre scorso estende i benefici della citata Legge 170/10\footcite{legge170}, vale a dire le misure compensative e dispensative. Ma esaminiamo il testo del documento, prendendo in considerazione vari paragrafi specifici.
\begin{description}
	\item[Il paragrafo 1.3]  è dedicato agli alunni con deficit da disturbo dell'attenzione e dell'iperattività (ADHD) il cui numero viene stimato intorno agli 80.000. Per questi studenti, se vi è anche la certificazione di disabilità, scatta il diritto al sostegno; se invece tale certificazione manca, hanno comunque diritto ad avere le garanzie derivanti dalla Legge 170/10.
	\item [Il paragrafo 1.4] parla degli alunni con funzionamento cognitivo limite (borderline), stimati intorno ai 200.000.
	\item [Il paragrafo 1.5] fornisce alcuni orientamenti didattici a favore degli alunni con \glslink{besa}{BES}. Dal momento, quindi, che già la normativa precedente aveva fornito indicazioni per gli alunni con disabilità e per quelli con DSA, il paragrafo così recita, anche per gli altri casi di BES: «Le scuole – con determinazioni assunte dai Consigli di classe, risultanti dall'esame della documentazione clinica presentata dalle famiglie e sulla base di considerazioni di carattere psicopedagogico e didattico – possono avvalersi per tutti gli alunni con Bisogni Educativi Speciali degli strumenti compensativi e delle misure dispensative previste dalle disposizioni attuative della Legge 170/2010\footcite{legge170} (DM 5669/2011\footcite{decreto5669_2011}), meglio descritte nelle allegate Linee guida».
	È da osservare, tuttavia, che mentre per gli alunni con disabilità e con \glslink{dsaa}{DSA} la normativa ha stabilito che le certificazioni cliniche debbano pervenire esclusivamente dalle ASL o da centri convenzionati o accreditati con esse, qui la Direttiva nulla dice per gli altri casi di BES relativi allo svantaggio. Ed è questo un punto assai importante che il Ministero dovrà chiarire, in quanto anche per questi alunni viene resa obbligatoria la formulazione di un Piano Didattico Personalizzato in forza della Legge 53/03\footcite{Legge_53_2003}.
	Inoltre, dovendosi applicare anche a questi casi le misure compensative e dispensative previste dalla Legge 170/10, i Consigli di Classe dovranno disporre di una documentazione clinica certa e formulare «considerazioni di carattere psicopedagogico e didattico» non discutibili, al fine di evitare contenziosi con altri alunni ai quali tali benefici non vengano concessi.
	\item [Il paragrafo 1.6] riguarda l'impegno del Ministero  ad organizzare corsi di formazione per dirigenti e docenti curricolari sulla didattica inclusiva, a favore anche dei casi non certificabili come disabilità o come DSA
\end{description}
Per quanto poi riguarda il secondo paragrafo, esso è totalmente dedicato all'organizzazione territoriale per un'ottimale realizzazione dell'inclusione scolastica.
\begin{description}
	\item[Il paragrafo 2.1]  si concentra sui Centri Territoriali di Supporto (d'ora in poi CTS), istituiti presso Scuole Polo. La Direttiva in esame propone che ve ne sia uno per Provincia, collegati con altri \glslink{ctsa}{CTS} a livello di ambito di Distretto Socio-Sanitario di Base, a loro volta collegati con le singole scuole. Viene però tenuto presente che questi strumenti organizzativi – riguardanti tutti i BES – non possono ignorare l'esistenza dei GLIR (i Gruppi di Lavoro per l'Inclusione Scolastica degli Alunni con Disabilità Regionali, introdotti dalle già citate Linee Guida del 4 agosto 2009), dei GLIP (Gruppi a livello Provinciale) e dei GLHI (i Gruppi di Lavoro Handicap d'Istituto, introdotti dall'articolo 15 della Legge 104/92). Questo, infatti, si legge nella Direttiva: «Sarà cura degli Uffici Scolastici Regionali operare il raccordo tra i CTS e i GLIR, oltre che accordare i GLIP con i nuovi organismi previsti nella presente Direttiva».
	Questi «raccordi», pertanto, saranno determinanti per dare coerenza a tutto il sistema organizzativo. Infatti, mentre attualmente i CTS non sono sostenuti da finanziamenti certi, i GLIP (e quindi anche i GLIR) lo sono. Ci si chiede, quindi, se il Ministero intenderà riassorbire nei GLIR, nei GLIP e nei GLHI anche i compiti che la Direttiva prevede per i CTS, oppure lasciare in vita due linee organizzative parallele, come attualmente fa la Direttiva stessa.
	\item [Il paragrafo 2.1.2] prevede che presso i CTS provinciali operi un'equipe di docenti curricolari e di sostegno «specializzati» sui BES, tramite master universitari organizzati sulla base di un'intesa già esistente con il Ministero.
	Qui riteniamo singolare che si usi il termine «specializzati» per quanti conseguano il titolo del master universitario; la “specializzazione”, infatti, è un termine tecnico ben preciso che vale solo per i docenti per il sostegno e per le scuole di specializzazione post lauream. Nella Direttiva, invece, esso sembra usato in modo “atecnico” ed è necessario che il Ministero chiarisca questo punto.
	\item [Il paragrafo 2.2]riguarda le funzioni dei CTS che sono le seguenti:
	\begin{description}
		\item[2.2.1] Informazione e formazione
		\item[2.2.2]Consulenza
		\item[2.2.3]Gestione degli ausili e comodato d’uso
		\item[2.2.4]Buone pratiche e attività di ricerca e sperimentazione
		\item[2.2.5]Piano annuale d’intervento
		\item[2.2.6]Risorse economiche
		\item[2.2.7]Promozione di intese territoriali per l’inclusione
	\end{description}
\item [Il paragrafo 2.3] dice che ogni CTS Provinciale deve darsi un proprio regolamento interno.
\item [Il paragrafo 2.4] riguarda l'organizzazione dei CTS che fa perno sul Dirigente Scolastico della Scuola Polo presso cui essi sono istituiti. Si prevede in tal senso la presenza di almeno tre docenti «specializzati sui BES» che, secondo la normativa dei comandi, dovrebbero garantire per almeno un triennio la loro presenza di consulenza alle scuole della Provincia, anche tramite i CTS di àmbito distrettuale e i Gruppi di Lavoro per l'Inclusione delle singole scuole che si dovrebbero affiancare ai GLHI per i disabili.
Qui ci si chiede come il Ministero – dati gli attuali tagli al numero dei “comandati” – possa garantire la presenza di questo personale in aggiunta a quello comandato per i GLIP per gli alunni con disabilità. Non sarebbe più facile attribuire anche al personale comandato nei GLIP le competenze sull'inclusione, relative a tutti i casi di BES? Si attendono quindi chiarimenti.

Presso il CTS, poi, viene è istituito un Comitato Tecnico Scientifico, con il compito di formulare il piano annuale degli interventi, composto da: «il Dirigente Scolastico, un rappresentante degli operatori del CTS, un rappresentante designato dall'\glslink{usra}{USR}, e, ove possibile, un rappresentante dei Servizi Sanitari».
E ancora, si prevede la nomina di un referente regionale dei CTS e anche di un Coordinamento Nazionale, istituito presso la Direzione Generale per lo Studente del Ministero e composto da: «- Un rappresentante del MIUR – I referenti per la Disabilità/DSA degli Uffici Scolastici Regionali – I referenti regionali CTS – Un rappresentante del Ministero della Salute – Un rappresentante del Ministero delle politiche sociali e del lavoro – Eventuali rappresentanti della \glslink{fisha}{FISH}  e della \glslink{fanda}{FAND}  – Docenti universitari o esperti nelle tecnologie per l'integrazione. Il Coordinamento nazionale si rinnova ogni due anni. Il Comitato tecnico è costituito dal rappresentante del \glslink{miura}{MIUR}, che lo presiede, e da una rappresentanza di 4 referenti CTS e 4 referenti per la disabilità/DSA degli Uffici Scolastici Regionali».
Ebbene, questo disegno organizzativo – per altro molto logico – sembra proprio intersecarsi con quello parallelo già esistente per la disabilità. Infatti, nel Coordinamento Nazionale dei CTS, viene individuata anche la presenza di rappresentanti delle Federazioni FISH e FAND, che sono i membri naturali dell'attuale Osservatorio Ministeriale sull'Inclusione degli alunni con disabilità.
Se quindi il Ministero intende impostare un'organizzazione per tutti i BES, sarebbe forse più logico attribuire al già esistente Osservatorio sulla Disabilità – che ha pure un finanziamento organizzativo – i compiti che la Direttiva attribuisce istituendo Coordinamento dei CTS.
Strana è poi la dicitura di «referenti regionali per la disabilità/DSA»; infatti, disabilità e DSA – come ha ben precisato la Legge 170/10\footcite{legge170} – sono situazioni cliniche e giuridiche differenti e pure diversi devono essere i loro referenti, a meno di non voler attribuire a quelli per la disabilità anche i compiti riguardanti i DSA e in generale tutti i BES. In tal caso, però, sarebbe quanto meno necessario adeguare a queste nuove esigenze i programmi dei corsi di specializzazione dei futuri insegnanti di sostegno, di cui all'articolo 13 del Decreto Ministeriale 249/10\footcite{DM_249_2010}, ma ciò contrasterebbe con la normativa anche della presente Direttiva, che vieta l'assegnazione di docenti per il sostegno ai casi di BES che non siano certificati come disabilità.
Tutte queste perplessità, del resto, potrebbero essere fugate se ai CTS, ai loro referenti regionali, loro Coordinamento Nazionale e al rispettivo Comitato Tecnico, venissero esclusivamente attribuiti compiti relativi all'utilizzo delle nuove tecnologie ai fini dell'inclusione scolastica di tutti gli alunni con BES, ferme restando le competenze degli organismi previsti dalla precedente normativa, relativa sia alla disabilità che ai DSA. E indicazioni in tal senso sembrano trarsi da più parti della Direttiva.
\end{description}
Il documento si chiude con la previsione dell'istituzione di un portale internet, con articolazioni anche a livello locale, relativo a tutti i BES e che sia accessibile ai sensi della Legge 4/04\footcite{Legge_04_04}.

In conclusione, come detto inizialmente, questa Direttiva è certamente molto interessante, ma necessita di numerosi chiarimenti ministeriali, e forse anche dei suggerimenti da parte dell'attuale Osservatorio Ministeriale sulla Disabilità. Qualche spunto potrebbe venire ad esempio dal Regolamento Applicativo della Legge 5/06 della Provincia Autonoma di Trento sui Bisogni Educativi Speciali, approvato dalla Giunta Provinciale di Trento con la Delibera n. 1073 del 29 aprile 2008.
L'occasione è comunque propizia per ribadire – come chiesto con forza nelle scorse settimane dai Presidenti di FISH e FAND – l'opportunità di un'apposita riunione dell'attuale Osservatorio Ministeriale sulla Disabilità\footcite{nocera8}.

31 gennaio 2013
Ultimo aggiornamento: 31 gennaio 2013 11:53