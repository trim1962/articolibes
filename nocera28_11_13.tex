\author{Salvatore Nocera}
\title{Inclusione: un diritto acciaccato, ma non affievolito}
\phantomsection
\label{cha:Nocera281112}
\begin{abstract}
Proponiamo una vera e propria “agenda completa” dell'attuale “stato dell'arte” dell'inclusione scolastica nel nostro Paese, curata da chi, come Salvatore Nocera, ne segue l'evoluzione sin dagli esordi, ovvero dalla fine degli Anni Sessanta. «Pur non sottovalutando i rischi attuali per la qualità dell'inclusione -- scrive Nocera --, possiamo dire che tale diritto sia molto acciaccato, ma non ancora affievolito»

Impossibilitato a partecipare all'iniziativa in programma venerdì 29 novembre all'Istituto Comprensivo di Via G. Messina a Roma, intitolata “Disabilità a scuola: ancora un diritto?”, Salvatore Nocera, vicepresidente nazionale della FISH (Federazione Italiana per il Superamento dell'handicap), affida alle nostre pagine il testo dell'intervento elaborato per l'occasione, che si caratterizza come una vera e propria “agenda completa” dell'attuale “stato dell'arte” dell'inclusione scolastica nel nostro Paese.
\end{abstract}
\maketitle
\datapub{28 novembre 2013}

In questi ultimi anni l'attenzione della classe politica e della società sulle problematiche dell'inclusione scolastica in Italia è certamente scemata. Varie le cause, dalla presenza di Ministri più orientati alla meritocrazia (Letizia Moratti, Mariastella Gelmini), a una minore tensione morale e culturale nella società, da una maggiore rilevanza politica data all'ingresso nella scuola di studenti stranieri, sino alle forti riduzioni della spesa destinata alla scuola pubblica.

In generale, poi, si è creduto che con la copiosa legislazione inclusiva prodotta negli Anni Settanta, Ottanta e Novanta, il problema delle pari opportunità e del diritto allo studio degli alunni con disabilità fosse ormai definitivamente risolto. Non si è però tenuto conto di un aspetto assai importante, relativo a tutte le riforme e cioè che non basta varare delle leggi perché quelle riforme vengano realizzate; poi occorre infatti attuarle con atti amministrativi nelle singole realtà, le scuole, gli Enti Locali, le ASL, dove cioè si svolge in concreto la realizzazione dei diritti ed è necessario avviare delle buone prassi generalizzate di naturale attuazione delle norme, che così divengono vita quotidiana della società inclusiva.

E invece, nella quotidianità –- sia delle Amministrazioni, che delle singole scuole e delle singole classi –- non sempre, negli ultimi anni, la prassi inclusiva si è affermata in modo naturale. Certo, nella stragrande maggioranza dei casi, le norme sono state applicate, diventando buone prassi inclusive, testimoniate da numerosissimi convegni e incontri in tutta Italia, da seminari organizzati dalle singole scuole, da associazioni e organizzazioni di volontariato che fiancheggiano spesso gli sforzi delle istituzioni e delle famiglie. La stessa FISH (Federazione Italiana per il Superamento dell'Handicap) ha raccolto dati significativi con il concorso Le chiavi di scuola, che ha premiato per alcuni anni le migliori esperienze di inclusione e al quale hanno partecipato numerose scuole di tutto il Paese.

Però, soprattutto nei tempi più recenti, si sono riscontrati fenomeni come l'affollamento delle classi, la riduzione del numero dei docenti, la contrazione nel numero degli assistenti per l'autonomia e la comunicazione forniti dagli Enti Locali, la denuncia del taglio delle ore di sostegno, l'insufficiente formazione dei docenti curricolari sulle didattiche inclusive e la conseguente delega ai soli docenti per il sostegno dei progetti di inclusione, specie nelle scuole secondarie. Fatti, questi, che tutti insieme hanno certamente reso difficoltoso il processo inclusivo, dando anche la sensazione di un regresso sulla strada del consolidamento dei diritti.

Malgrado ciò – o forse proprio a causa di ciò – le famiglie e le associazioni hanno reagito su due diversi livelli, il primo dei quali è stato quello dei ricorsi alla Magistratura.

Basti pensare, in tal senso, alle migliaia di ricorsi ai Tribunali Amministrativi Regionali (TAR), per ottenere il ripristino del numero di ore di sostegno, e poi, negli anni più recenti, anche del numero di ore degli assistenti per l'autonomia e la comunicazione, nonché per il rispetto del tetto massimo di venti alunni per classe, fissato dall'articolo 5, comma 2 del Decreto del Presidente della Repubblica (DPR) 81/09\footcite{DPR_81_2009}.

E non ci si è limitati solo ai TAR, ma si è spesso arrivati anche al Consiglio di Stato e addirittura alla Corte Costituzionale. Si pensi, ad esempio, alla fondamentale Sentenza 80/10\footcite{SCC_80_2010} della Consulta, che ha ribadito il diritto ad ottenere la cattedra intera di sostegno nei casi certificati di gravi disabilità, specie intellettive e sensoriali, chiarendo inoltre che il diritto all'inclusione scolastica non può essere affievolito per motivi di tagli alla spesa pubblica.

Una seconda direzione d'azione è stata quella di contrastare, per quanto possibile, la normativa “controriformista” e, dove possibile, di migliorare le conquiste ottenute con la normativa inclusiva.

Qui si pensi, per il primo caso, alla già citata norma sul tetto massimo di alunni per classe, al ripristino dell'Osservatorio Ministeriale sull'Inclusione Scolastica, al DPR 122/09\footcite{DPR_122_2009} sulla valutazione degli alunni -– che chiarisce bene il ruolo e le finalità dell'inclusione scolastica --, fino alle Linee Guida Ministeriali del 4 agosto 2009\footcite{LineGuida2009}, che fanno il punto sullo stato della normativa inclusiva italiana, chiarendone gli ambiti di intervento sia da parte degli Enti Locali che dentro le singole scuole, con la distinzione dei compiti dei dirigenti, dei docenti curricolari, di quelli per il sostegno, dei collaboratori scolastici e delle stesse famiglie.

Per il secondo caso, invece, quello riguardante il miglioramento delle conquiste ottenute, si pensi ad esempio all'ampliamento della normativa inclusiva, estesa – sia pure con diritti assai diversi – agli alunni con DSA (Disturbi Specifici di Apprendimento, ovvero dislessia, disgrafia, discalculia, disortografia), tramite la Legge 170/10\footcite{legge170} prima, e le Linee Guida\footcite{LineGuida2011} applicative della stessa, poi, prodotte il 12 luglio 2011 (Decreto Ministeriale n. 5669/11\footcite{decreto5669_2011}). O si pensi ancora alla Direttiva Ministeriale del 27 dicembre 2012\footcite{dir27Dic2012} sui BES (Bisogni Educativi Speciali), alla Circolare che ne ha fissato l'applicazione (8/13\footcite{cm8_2013}) e alla recente Nota Ministeriale n. 2563/13\footcite{Nota_2563_2013} che ne ha chiarito alcuni passaggi fondamentali.

E da ultimo, ma non certo ultimo, va ricordato l'indubbio rilancio culturale e normativo dell'inclusione degli alunni con disabilità, coinciso con l'approvazione della Convenzione ONU  sui Diritti delle Persone con Disabilità ratificata dall'Italia con la Legge 18/09\footcite{Legge18_2009}.

Al tempo stesso, però, si deve rilevare un'accentuazione della presa di coscienza dei diritti piuttosto individualistica (il mio docente per il sostegno, il mio assistente educativo, la mia sentenza ecc.), più che una visione corale della volontà di cambiare la società, come fu agli inizi, a partire dalla fine degli Anni Sessanta.

Un ritorno a quei valori, per altro, si ha in un Progetto di Legge della FISH per il rilancio della qualità dell'inclusione scolastica, molte delle cui proposte sono state recepite nella mozione finale\pageref{cha:aavv101113} del nono Convegno Nazionale La Qualità dell'integrazione scolastica e sociale, svoltosi a Rimini, dall'On al 10 novembre scorsi, a cura del Centro Studi Erickson: dalla formazione iniziale dei futuri docenti sulle didattiche inclusive, alla formazione obbligatoria in servizio, specie dei docenti curricolari, che sono, assieme ai compagni, i veri protagonisti dell'inclusione; e ancora, maggiore collaborazione tra tutti i docenti e con gli Enti Locali, rispetto del tetto massimo di venti alunni per classe, maggiore continuità didattica non solo dei docenti per il sostegno.

In linea con tutto ciò marcia ad esempio la recente Legge 128/13, che abolendo all'articolo 15 (comma 3 bis) le aree disciplinari per il sostegno nelle scuole superiori, evita la delega ai docenti di sostegno, da parte di quelli curricolari, che dovranno quindi prendersi in carico il progetto inclusivo e formarsi per questo. L'articolo 16 (comma 1) della medesima norma, poi, ha introdotto il principio dell'aggiornamento obbligatorio in servizio di tutti i docenti sulle didattiche inclusive e non solo degli alunni con disabilità, ma anche di quelli con altri Bisogni Educativi Speciali.

Anche per questi motivi, quindi, pur non sottovalutando i rischi attuali per la qualità dell'inclusione, possiamo dire che tale diritto sia molto acciaccato, ma non ancora affievolito, specie se le nuove generazioni di studenti e docenti riacquisteranno la cultura corale della trasformazione della scuola, grazie proprio alla presenza attiva di alunni con difficoltà, che stimolano a far crescere tutti nella ricerca e nella prassi quotidiana.\footcite{Nocera2013c}