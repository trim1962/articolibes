\author{Giulia Giani}
\title{\cit{Cultura dell'inclusione} e leggi da applicare }
\phantomsection
\label{cha:giani140513}
\begin{abstract}
\caporali{Prima applicare le leggi sull'inclusione scolastica degli alunni con disabilità -– aveva scritto su queste pagine il genitore Daniele Brogi -– solo dopo pensarne una modifica}. \caporali{Ma se una legge dagli obiettivi così alti e giusti -– risponde Giulia Giani, insegnante di sostegno -– non viene applicata in modo così diffuso, non è che la sua applicazione sia troppo difficile, complessa e onerosa?}
\end{abstract}
%\epigraph{«Prima applicare le leggi sull'inclusione scolastica degli alunni con disabilità – aveva scritto su queste pagine il genitore Daniele Brogi – solo dopo pensarne una modifica». «Ma se una legge dagli obiettivi così alti e giusti – risponde Giulia Giani, insegnante di sostegno – non viene applicata in modo così diffuso, non è che la sua applicazione sia troppo difficile, complessa e onerosa?»}{Giulia Giani}
\maketitle
\datapub{14 Maggio 2013}
Con piacere ho deciso di intervenire ancora con brevi riflessioni a integrare ulteriormente il dibattito sull'inclusione scolastica iniziato su queste pagine, in particolare dialogando con il genitore Daniele Brogi, che nel suo ultimo articolo\pageref{brogi030513} – rispondendo a un mio precedente intervento\pageref{cha:giani300413} – ha manifestato una posizione differente da quella da me sostenuta rispetto all'applicazione delle leggi sull'inclusione scolastica.
Se avevo affermato che «in decenni di tentativi di applicazione della legge sull'inclusione scolastica non siamo riusciti a farlo», e che perciò è forse necessario ripensare ruoli e funzioni del personale in materia di inclusione, Brogi risponde che «in buona parte del Paese l'applicazione legislativa in materia d'integrazione scolastica non è nemmeno mai stata presa in considerazione» e che «questo è un concetto che differisce di molto da quanto sostenuto, poiché di fronte all'evidenza della totale disapplicazione, credo non vi siano nemmeno elementi e dati utili a mettere in discussione il sistema».

Se posso condividere il fatto che in buona parte del Paese la legge sull'inclusione non viene neanche presa in considerazione e anche il fatto che il concetto espresso è diverso dal ritenere che non ci si è riusciti (perché l'uno implica uno sforzo, l'altro è semplicemente disinteresse), non riesco tuttavia a condividere l'esito del ragionamento del genitore: mi pare che dal suo punto di vista si debba esigere “prima” l'applicazione della legge in tutte le realtà, per “poi” valutare l'efficacia del nostro sistema di inclusione.
La mia proposta di riflessione, invece, parte da un altro presupposto: se è vero infatti, a livello teorico, che bisogna «esigere l'applicazione di una legge», di fronte a una sua «totale disapplicazione» a una «non presa in considerazione» diffusa sul territorio, credo ci si debba interrogare profondamente.
Se una legge – che ha pure un obiettivo alto e giusto – non viene applicata in modo così diffuso, forse vi sono delle ragioni che rendono difficile, complessa, onerosa, non appetibile la sua applicazione. Forse vi è un disinteresse nelle scuole sull'inclusione e dobbiamo “prima” capirne i motivi, per “poi” cercare delle soluzioni efficaci.

È quanto ho cercato di mettere in luce in alcuni miei precedenti interventi, parlando delle intrinseche contraddizioni legate alla figura del docente di sostegno, alla sua formazione, ai suoi compiti. Credo che vada rintracciata in questo àmbito la ragione profonda di una scarsa applicazione della legge sull'inclusione.
Il punto di partenza è sempre riconducibile all'organizzazione dell'attività didattica in modo squilibrato tra gli attori coinvolti: ritengo infatti che la presenza del docente di sostegno, per come la conosciamo oggi, e la sua assegnazione “sulla carta” alle classi in cui è inserito l'alunno disabile – ben più frequentemente a svolgere assistenza ad personam all'alunno disabile stesso – abbia fatto in modo che il “sistema scuola” potesse delegare a questa figura le questioni in materia di disabilità, sentendosi in tal modo “tutelato” dalla presenza di personale specializzato e quindi “poco coinvolto” nel mettere in atto strategie di sistema.
Da qui, secondo me, hanno origine le disapplicazioni della legge che segnalava Daniele Brogi sul funzionamento del GLHO [Gruppo di Lavoro Handicap Operativo, N.d.R.], sull'assegnazione arbitraria dei docenti agli alunni senza considerarne le effettive esigenze, sullo scarso coinvolgimento dei genitori, sul personale sanitario sottostrutturato. «Tanto c'è qualcuno che se ne occupa…», oserei leggere tra le righe di quei pensieri poco interessati. Peccato che questo è il contrario dell'inclusione, ove tutti dovrebbero sentirsi coinvolti.
È decisamente più complesso – rispetto a quanto si fa oggi in molte realtà – mettere in atto tutte le richieste della legge, perché ciò richiede molte energie di coordinamento, di formazione, di tempo, nonché di spese economiche…  Senz'altro questa riflessione non vuole essere una giustificazione al fatto che la legge sia disapplicata, ma forse capirne le ragioni ci può aiutare a trovare soluzioni che portino proprio a quanto desidera Daniele Brogi, cioè «esigerne l'applicazione».
D'altra parte, il fatto di aver lavorato e di lavorare tuttora in una realtà ove invece la legge sull'inclusione scolastica viene applicata con coscienza e rigore, con un alto dispendio di energie, non mi ha impedito di cogliere le fragilità e le contraddizioni legate al ruolo che ricopro e quindi mi pare di poter affermare che non è tanto l'applicazione o meno della legge ad avermi fornito stimoli per una riflessione sulla distribuzione delle risorse, quanto la personale e quotidiana esperienza didattica in un contesto che funziona.

Credo che “pretendere” senza mettere in atto strategie che facilitino e rendano appetibile l'applicazione di una legge sia destinato al fallimento, a rimanere “lettera morta”, perché l'imposizione di qualcosa solo perché è “scritto”, ma non è “sentito”, senza un cambio di mentalità e un approccio culturale diverso, sarebbe vissuta come un ulteriore carico di lavoro dalle singole scuole e, allo stato attuale, molte realtà non investono energie e tempo nell'inclusione scolastica.
Per questi motivi credo che si possa già riflettere su quanto di inadeguato esiste, per poter proporre azioni che facilitino, incoraggino, attivino risorse nuove per un cambio di prospettiva, per fare in modo che l'inclusione scolastica “interessi” alla vita della scuola e non sia solo compito dei docenti di sostegno.

Sono però d'accordo con Daniele Brogi che prima di modificare la legge sia necessario applicarla: la nostra proposta di “scambio di ruoli” e di “cattedra mista”, ad esempio – che lo stesso genitore ha dichiarato di condividere – non mi pare abbia alcun ostacolo né esplicito né implicito nella legge a cui si fa riferimento. Non è scritto in alcuna parte di quella legge, infatti, che «l'insegnante specializzato debba svolgere il suo servizio interamente su sostegno»; la norma prescrive invece che «in presenza di alunno con disabilità vi debbano essere insegnanti specializzati»: appare evidente, quindi, che la nostra proposta di ripensare ruoli e funzioni non è né in contrasto né vuole una modifica della legge esistente.
La modifica reale consisterebbe solo nell'attribuzione di incarichi in modo più equo, paritario, sostenibile e coinvolgente nella vita della scuola, senza venir meno alla necessità di personale specializzato né sottraendo le risorse attualmente disponibili.
Quante più persone, però, sarebbero coinvolte nell'inclusione, senza essere marginalizzate? Quale cambio di mentalità si avrebbe? Forse proprio quello necessario per “esigere l'applicazione della legge”…\footcite{Giani2013}

Insegnante di sostegno.

14 maggio 2013

© Riproduzione riservata
