\author{Salvatore Nocera}
\title{I Bisogni Educativi Speciali, i fatti e le paure }
\phantomsection
\label{cha:nocera030613}
\begin{abstract}
In tema di inclusione degli alunni con disabilità, scrive Salvatore Nocera – prendendo spunto da un recente intervento apparso su queste pagine e ravvivando ulteriormente il dibattito su questi temi, in corso sul nostro giornale – i veri obiettivi devono essere quelli di far rispettare il numero massimo degli alunni per classe e di fare avviare concretamente la formazione di tutti i docenti curricolari sull'inclusione stessa
\end{abstract}
%\epigraph{In tema di inclusione degli alunni con disabilità, scrive Salvatore Nocera – prendendo spunto da un recente intervento apparso su queste pagine e ravvivando ulteriormente il dibattito su questi temi, in corso sul nostro giornale – i veri obiettivi devono essere quelli di far rispettare il numero massimo degli alunni per classe e di fare avviare concretamente la formazione di tutti i docenti curricolari sull'inclusione stessa}{Salvatore Nocera}
\maketitle
\datapub{3 Giugno 2013}
Mi è molto piaciuto l'articolo di Carlo Scataglini sui BES (Bisogni Educativi Speciali), pubblicato\pageref{cha:scataglini1} su queste stesse pagine, poiché frutto della sua grande esperienza di scuola attiva. Anch'io, che su tale questione ho difeso sia la Direttiva Ministeriale del 27 dicembre 2012\footcite{dir27Dic2012} sia la Circolare Ministeriale 8/13\footcite{cm8_2013}, ho manifestato perplessità circa l'attuazione pratica delle indicazioni operative. Ho contestato ad esempio l'idea di fare svolgere i GLI (Gruppi di Lavoro per l'Inclusione) durante l'orario di servizio e con tanta frequenza. Ho poi contestato anche l'idea di sostituire – come sembrava voler fare la Direttiva citata – i GLIP (Gruppi di Lavoro per l'Inclusione Provinciali) con i CTS (Centri Territoriali di Supporto): la Circolare 8/13, almeno, li ha messi assieme. A mio avviso, quindi, il Ministero dovrà tornare sugli aspetti operativi di queste norme.

Però nella vibrata difesa di Scataglini del lavoro in classe, colgo una proposta che – pur teoricamente legittima – mi sembra, almeno per oggi, difficilmente praticabile e cioè che i docenti per il sostegno debbano occuparsi di tutti i casi difficili certificati e non certificati. Ritengo ciò per ora impossibile, perché è ben noto che quando manca il docente per il sostegno, in troppi casi, specie nelle scuole secondarie, i docenti curricolari fanno uscire dalla classe l'alunno certificato o lo lasciano inattivo in fondo alla classe stessa.

Occorre – come giustamente scrive Scataglini – ridurre il numero degli alunni per classe e formare tutti i docenti curricolari. Queste richieste, formulate da anni dalla FISH (Federazione Italiana per il Superamento dell'Handicap) sembrano essere state accettate dal Ministero, che però ancora non ha posto in essere fatti normativi concreti, tranne che l'articolo 5, comma 2 del Decreto del Presidente della Repubblica (DPR) 81/09\footcite{DPR_81_2009}, concernente il tetto massimo di venti alunni nelle prime classi frequentate da alunni con disabilità, normalmente violato dagli stessi Uffici Scolastici Regionali, senza che il Ministero si muova, se non dopo le Sentenze dei Tribunali Amministrativi Regionali (TAR), che cominciano a fioccare anche in questo campo.
Dove però non concordo con l'articolo di Scataglini è con l'affermazione – data per certa – che agli alunni con disabilità non grave non verrà dato il prossimo anno il sostegno, che verrà riservato – e con ore ridotte – ai soli alunni certificati con disabilità grave.

A me ciò non risulta da nessun documento ufficiale, né da dichiarazioni dei Dirigenti Generali o dei politici al vertice del Ministero. Se poi qualche Ufficio Scolastico Regionale l'ha “messa in giro” in modo ufficioso, per giustificare i tagli che si vogliono fare alle ore di sostegno, e “per farsi belli” con il Ministero, sarà bene conoscere gli uffici di provenienza, per costringerli a smentire.

Le ore di sostegno vengono assicurate attualmente dalla Sentenza 80/10\footcite{SCC_80_2010} della Corte Costituzionale, che ha affermato il diritto incomprimibile di tutti gli alunni con disabilità certificata ai sensi dell'articolo 3, commi 1 o 3 della Legge 104/92; anzi, quella Sentenza ha stabilito che agli alunni certificati con gravità (ai sensi dell'articolo 3, comma 3 della medesima Legge 104/92\footcite{Legge_104_92}) spetti l'intera cattedra di sostegno, con riguardo alla specificità della disabilità (ad esempio studenti con disabilità intellettive o relazionali o pluriminorazioni).

Questi orientamenti vincolanti della Corte sono stati normati dalla Legge 122/10\footcite{Legge_122_2010} che all'articolo 9, comma 15 li ha ufficializzati legislativamente, mentre all'articolo 10, comma 5, ha stabilito che le richieste di tutte le ore per il sostegno, per disabilità lievi o gravi, vengano indicate in un PEI (Piano Educativo Individualizzato) che la scuola deve sinteticamente inviare agli Uffici Scolastici, entro il mese di maggio o di giugno – all'epoca cioè della formazione dell'organico di fatto -, per ottenere entro i primi di settembre le ore richieste. La risposta dovrà arrivare, come stabilisce l'articolo 19, comma 11 della Legge 111/11\footcite{legge_111_2011}, tramite l'invio alle singole scuole di un pacchetto di ore assegnate.
In linea generale, io ho sostenuto e continuo a sostenere che, qualora le ore assegnate siano inferiori a quelle richieste e documentate, e che quindi non corrispondano alle «effettive esigenze» dei singoli alunni – come stabilito dall'articolo 1, comma 605, lettera b della Legge 296/06\footcite{Legge_296_2006} – le famiglie possano fare immediatamente ricorso al TAR, anche in modo collettivo, in modo da contenere i costi. Davanti al TAR, infatti, l'Amministrazione non può difendersi adducendo i tagli alla spesa scolastica e il Patto di Stabilità, poiché la citata Sentenza 80/10 della Corte Costituzionale ha precisato appunto che il diritto all'integrazione scolastica non può essere compresso da vincoli di bilancio.

Quindi, se dovesse verificarsi ciò che teme e che anzi dà per certo Scataglini, è altrettanto certo – ma sulla base di testi giuridici incontestabili – che le famiglie ricorrenti vinceranno le cause e che l'Amministrazione dovrà non solo pagare le spese, ma, trattandosi della violazione di un diritto costituzionalmente protetto, dovrà pure risarcire i danni anche non patrimoniali, che le più recenti Sentenze, anche del Consiglio di Stato, fissano in circa 1.000 euro per ogni mese di ritardo della nomina del docente per il sostegno.

Quindi, credo non sia il caso di incrementare dicerie prive di fondamento, bensì di impegnarsi sempre di più a pretendere che si avvii la formazione iniziale e obbligatoria in servizio di tutti i docenti curricolari, in modo che essi possano realmente prendersi in carico, in prima persona, del progetto di inclusione scolastica, ciò che succedeva quando venne avviato il processo inclusivo alla fine degli Anni Sessanta, con il sostegno dei colleghi specializzati per il sostegno.
Se infine una lettura affrettata della ricerca del 2011 della Fondazione Agnelli [ci si riferisce alle ipotesi avanzate nel rapporto intitolato Gli alunni con disabilità nella scuola italiana: bilancio e proposte, Erickson, 2011, elaborato appunto dalla Fondazione Agnelli, insieme all'Associazione TreeLLLe e alla Caritas Italiana, N.d.R.], può dare l'impressione che questa della riduzione dei posti di sostegno sia la linea ministeriale, ci si sbaglia ad affermare che questo sia anche il progetto ministeriale.

Personalmente sono abituato a ragionare solo su prove ufficiali e fino a quando non vedo uno straccio di Circolare che dichiari ciò che Scataglini paventa, debbo sostenere trattarsi di una pura invenzione, frutto delle “paure vaganti” nel nostro mondo.\footcite{nocera2}



Vicepresidente nazionale della FISH (Federazione Italiana per il Superamento dell'Handicap). Il presente testo è già apparso in «www.La Letteratura e noi.it», con il titolo “I BES, i fatti e le paure. Risposta a Carlo Scataglini” e viene qui ripreso, con minimi riadattamenti al diverso contenitore, per gentile concessione.
3 giugno 2013
© Riproduzione riservata
 
 
