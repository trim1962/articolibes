\author{Carlo Scataglini}
\title{Perché andrebbe bloccata quella Circolare sui BES}
\phantomsection
\label{cha:scataglini170613}
\begin{abstract}
Continua il nostro dibattito sull'inclusione scolastica, centrato questa volta in particolare sulla recente Circolare Ministeriale riguardante i Bisogni Educativi Speciali (\glslink{besa}{BES}). In un nuovo intervento, l'insegnante Carlo Scataglini amplia ulteriormente la propria riflessione, ribadendo la necessità, a suo dire, di bloccare l'applicazione di quella Circolare, «dando voce a tutti coloro che, a qualsiasi titolo, hanno a cuore l'integrazione scolastica»
\end{abstract}
%\epigraph{Continua il nostro dibattito sull’inclusione scolastica, centrato questa volta in particolare sulla recente Circolare Ministeriale riguardante i Bisogni Educativi Speciali (\glslink{besa}{BES}). In un nuovo intervento, l'insegnante Carlo Scataglini amplia ulteriormente la propria riflessione, ribadendo la necessità, a suo dire, di bloccare l'applicazione di quella Circolare, «dando voce a tutti coloro che, a qualsiasi titolo, hanno a cuore l'integrazione scolastica»}{Carlo Scataglini}
\maketitle
\datapub{17 Giugno 2013}
Innanzitutto vorrei subito precisare una cosa: la mia non vuole essere e non è una rivendicazione sindacale, né una difesa a oltranza della mia categoria, quella degli insegnanti di sostegno. Sorrido, però, di fronte alla tenacia con la quale si cerca di negare l'evidenza, affermando che non ci saranno tagli di posti di sostegno e che a nessun alunno sarà tolto il diritto a ricevere un sostegno specializzato. Sorrido perché a chi lavora a scuola da qualche decennio, e a chi da una decina danni è nel Gruppo Tecnico Provinciale che studia le certificazioni e le singole situazioni degli istituti scolastici, per stilare gli organici provinciali del sostegno, non può sfuggire un particolare che in maniera maldestra si cerca di tenere celato. I cosiddetti \glslink{fila}{FIL}, gli alunni con funzionamento intellettivo limite o borderline cognitivo, sono la chiave di volta, il punto di snodo, il vero gioco di prestigio di tutta la strategia dei tagli al sostegno.

Mi spiego, spero in maniera chiara, e spero per l'ultima volta. L’8 maggio scorso, presso il Ministero dell'Istruzione, si è svolto un seminario-conferenza di servizio sui \glslink{besa}{BES} (Bisogni Educativi Speciali). In quell'occasione è stato consegnato agli Uffici Scolastici Regionali un modello di Piano Annuale inclusività nel quale vengono chiaramente definite tre fasce diverse di \glslink{besa}{\glslink{besa}{BES}}:
\begin{description}
	\item[$-$]  le disabilità certificate (Legge 104/92, articolo 3, commi 1 e 3);
	\item [$-$]i disturbi evolutivi specifici (\glslink{dsaa}{\glslink{dsaa}{DSA}}, \glslink{adhda}{ADHD} ,\glslink{dopa}{DOP}, \glslink{fila}{FIL}, altro); %[ove gli acronimi stanno rispettivamente per disturbi specifici dell'apprendimento, deficit di attenzione e iperattività, disturbi oppositivi provocatori e funzionamento intellettivo limite, N.d.R.]);
	\item [$-$]lo svantaggio (socio-economico, linguistico-culturale, comportamentale-relazionale).
\end{description}

Nella prima fascia, quindi, ci sono i disabili (gravi e non gravi) che hanno un certificato, hanno il sostegno specializzato e beneficiano di un Piano Educativo Individualizzato (\glslink{peia}{PEI}). Nella seconda ci sono coloro che – pur avendo un certificato – non avranno alcun sostegno specializzato e saranno seguiti dagli insegnanti di classe secondo un Piano Didattico Personalizzato (PDP), stilato e attuato dagli stessi insegnanti di classe. Nella terza fascia, infine, sono compresi gli svantaggiati, che non hanno certificato, non beneficeranno di alcun sostegno specializzato, verranno individuati dagli insegnanti di classe in base a svantaggi socio-economici (!), linguistico-culturali e comportamentale-relazionali e beneficeranno di un Piano Didattico Personalizzato (PDP), stilato e attuato dagli stessi insegnanti di classe.
Ha ragione chi dice che da nessuna parte e in nessuna norma sta scritto che il sostegno specializzato sarà assegnato solo nei casi di disabilità grave. Il punto è un altro. Chi sono gli alunni di prima fascia, cioè i disabili, che non hanno la situazione di gravità? Da molto tempo – come dicevo – collaboro alla realizzazione degli organici di sostegno della mia Provincia, come docente esperto e posso rispondere al volo: la maggior parte delle certificazioni di disabilità (senza situazione di gravità) riguarda gli alunni \glslink{fila}{FIL} (funzionamento intellettivo limite). Praticamente, ora, i \glslink{fila}{FIL} vengono sistemati contemporaneamente in due fasce diverse, una che prevede il sostegno e una che invece lo esclude.
\section*{Il dibattito e la verità}
Quindi, mi chiedo:
\begin{description}
	\item [$-$]I \glslink{fila}{FIL} saranno “da sostegno” o “semplici” Bisogni Educativi Speciali “da PDP”?
	\item [$-$]Chi deciderà l'una o l'altra possibilità?
	\item [$-$]È possibile prevedere che chi aveva già il sostegno continuerà ad averlo, mentre le nuove certificazioni di \glslink{fila}{FIL} no?
	\item  [$-$]Oppure saranno rivisitate immediatamente le vecchie certificazioni (che per lo più recitano testualmente: «Alunno con funzionamento intellettivo limite, necessita di intervento di sostegno scolastico»), per far transitare subito questi alunni dalla prima alla seconda fascia?
	\item  [$-$]Sarà stabilito un limite di \glslink{fila}{FIL} (praticamente un funzionamento intellettivo sopra il limite o sotto il limite!) al di là del quale si darà il sostegno e al di sotto del quale basterà l'insegnante di classe con il \glslink{pdpa}{PDP}? E chi effettuerà tali “millimetriche” misurazioni di limite?
	\item  [$-$]Verrà prospettata ai genitori una scelta discrezionale del tipo: «Più intervento individualizzato o meno impatto sociale?», «Sostegno o PDP?».
	\item  [$-$] Sarà lasciata totale discrezionalità all'equipe multidisciplinare dell'ASL?
	\item  [$-$]Conteranno qualcosa le valutazioni sulle reali esigenze didattiche degli alunni fatte dai consigli di classe, dalle scuole e dai \glslink{glia}{GLI} ?
\end{description}
Allo stato attuale non è possibile rispondere, perché queste domande vengono evase. Nelle more qualcuno sfida i timorosi: «Vedremo!», dice, e alla luce dei fatti agiremo. Intanto, però, è inutile negare che vi saranno tagli ai posti di sostegno, poiché si potranno fare “scivolare” i \glslink{fila}{FIL} da un piano ad un altro, da una condizione a un'altra. I tagli ci saranno, quindi, forse non domani mattina o dal prossimo anno scolastico, ma certamente nel giro di due o tre anni
\section*{Il dibattito e la realtà}
I nostri discorsi pieni di strane sigle, di percentuali, di stime occupazionali, di valutazioni entusiastiche o apocalittiche, di considerazioni di “pancia” o “politiche”, di punti di vista interni o molto lontani dalla quotidiana vita delle classi, spesso dimenticano proprio i due attori principali della questione, vale a dire l'alunno con funzionamento intellettivo limite (insieme, ovviamente, alla sua famiglia) e l'insegnante disciplinare.
L'alunno \glslink{fila}{FIL} e il suo insegnante di inglese, ad esempio. Quale sarà uno scenario plausibile in una futura ora di inglese? Classe di ventinove, tra cui, poniamo, il \glslink{fila}{FIL} di cui sopra, un \glslink{dsaa}{DSA}, due stranieri e uno svantaggio socio-economico. Bene, dice qualcuno, questi alunni c'erano anche prima, non è colpa della Circolare sui \glslink{besa}{BES} [Circolare 8/13, N.d.R.]. Certo, ma prima in classe i ventinove alunni e il loro prof. d'inglese potevano contare sull'intervento (specializzato) di un insegnante (specializzato) nominato e in servizio per un alunno (il \glslink{fila}{FIL}), ma che, per norma e per operatività, era un insegnante di tutta la classe, di sostegno all'alunno \glslink{fila}{FIL}, di sostegno a tutta la classe, di sostegno al prof. d'inglese.
In che modo di sostegno? Attraverso metodologie inclusive e strategie didattiche “di cerniera”, attraverso un sostegno diretto in classe, attraverso una gestione in presenza e collaborativa della didattica.
A tal proposito, va pure detto che non fa onore – a chi argomenta contro l'attuale assetto didattico – fare leva sul luogo comune dell'insegnante di sostegno che si isola o viene isolato dell'insegnante di classe che delega ed esclude. Poiché di luogo comune davvero si tratta. Qui sì che dobbiamo veramente applicare le norme che già ci sono. L'insegnante disciplinare delega? Bene, intervenga il dirigente scolastico. L'insegnante di sostegno non si prepara, non collabora e “si imbosca”? Basta l'intervento del dirigente scolastico. Non serve smantellare la scuola per riequilibrarne l'eventuale cattivo funzionamento.
\section*{Aspetti positivi e aspetti negativi}
Chiarite le posizioni e le diverse ragioni che le ispirano, a mio avviso, è necessario trovare una soluzione operativa al caos che l'imposizione della Circolare 8/13 sui \glslink{besa}{BES} rischia di originare.
Mi spiego. Non è completamente negativo l'approccio ai \glslink{besa}{BES} previsto da quella Circolare. Ci sono degli aspetti positivi, accanto ad altri che invece destano più di una perplessità. Proviamo dunque a conciliare critiche e proposte.
Volendo schematizzare, a mio parere, è possibile individuare tre punti di forza e tre evidenti criticità della nuova normativa sui \glslink{besa}{BES}. Li indico di seguito.
\section*{Aspetti positivi}
\begin{enumerate}
	\item I Bisogni Educativi Speciali esistono per davvero! Non sempre le nostre scuole hanno dedicato e dedicano la giusta attenzione a tali bisogni. Il dibattito, in alcuni casi mosso anche da perplessità, paure e senso di inadeguatezza, che si è scatenato nelle nostre scuole, nelle sale professori, nelle dirigenze scolastiche e nelle segreterie studenti, è un'importante novità. È un bene che ci siano tale attenzione e tale dibattito. Occorre però che le domande operative che ne scaturiscono non ricevano risposte ambigue o lontane dalla realtà.
	Ai docenti disciplinari, giustamente preoccupati, occorrerà che qualcuno spieghi ufficialmente che riceveranno consulenza a distanza da parte di una decina di gruppi di lavoro, ma che in classe poi dovranno vedersela da soli.
	\item  La Circolare 8/13 prevede che le scuole stilino un Piano Annuale dell'inclusivo e che sia il Collegio dei Docenti a prenderlo in carico. Finalmente si pensa di organizzare in modo funzionale tutte le risorse della scuola. Per anni ho provato a insistere personalmente su questo punto, scrivendo o intervenendo nei gruppi istituzionali di cui ho fatto parte. Mi pareva assurdo il fatto che il Ministero e gli Enti Locali, che forniscono rispettivamente insegnanti e assistenti, non si parlassero per assegnare le risorse, per capire veramente di cosa c'era bisogno. Il paradosso, però, è che oggi che si richiede alle scuole di stilare un Piano Annuale che comprenda tutte le risorse necessarie, tali risorse vadano incontro a tagli sempre più pesanti.
	Il rischio è che si vada verso una richiesta finalmente centrata sull'organizzazione e su quello che serve per farla funzionare proprio nel momento in cui quello che serve non sarà più possibile ottenerlo.
	\item  L'attuale approccio ai Bisogni Educativi Speciali prevede una formazione specifica di tutti gli insegnanti disciplinari e di quelli di sostegno. È un'esigenza inderogabile nella nostra scuola pubblica. Formazione specifica, prima di tutto sulle situazioni di gravità, che richiedono competenze, professionalità e tecniche che non si possono improvvisare e sulle quali anche noi insegnanti di sostegno dobbiamo prepararci in maniera sicuramente più adeguata. Formazione sulle metodologie inclusive e sulle strategie didattiche “di cerniera”. Formazione sui disturbi specifici di apprendimento, sulle modalità di conduzione della classe, sull'attivazione di attività laboratoriali e cooperative. Formazione sulla gestione collaborativa della lezione tra più docenti, sulle modalità di programmazione personalizzata e sulla verifica e valutazione dei risultati.
	Occorre però che ai buoni propositi (formazione per tutti!) faccia seguito la volontà di dedicare a questo progetto risorse finanziarie importanti e di individuare con molta attenzione le agenzie che saranno chiamate a formare i docenti. Occorre che vengano utilizzate in maniera massiccia le risorse di formazione interne alle scuole, che venga finalmente attivata una vera circolarità di risorse, anche attraverso la diffusione di buone prassi di integrazione. Occorre, insomma, che alle parole e agli intenti seguano gli strumenti necessari per dare una risposta concreta all'urgente necessità di formazione della nostra scuola.
\end{enumerate}
\section*{Aspetti negativi}
\begin{enumerate}
	\item Si è deciso di circoscrivere una competenza specifica e specializzata, come quella dei docenti di sostegno, all'interno di un ambito preciso, quello della disabilità grave, rinunciando a tale competenza per tutti gli altri Bisogni Educativi Speciali.
	\item  Si intende spostare il concetto stesso di sostegno dalla didattica operativa – fatta di strategie specifiche in presenza – a una sorta di \cit{consulenza a distanza}, che non può incidere operativamente sull'inclusione di tutti gli alunni nei percorsi comuni. Tale spostamento ha queste conseguenze: a) finisce col rendere impossibile il compito degli insegnanti disciplinari, di fatto unici responsabili e attuatori della didattica personalizzata; b) penalizza fortemente gli alunni con difficoltà (in particolare quelli con funzionamento intellettivo limite), che vengono privati di un Piano Educativo Individualizzato e della possibilità di ricevere un sostegno didattico specializzato.
	\item Si genera una molteplicità di categorie e sotto categorie di alunni con Bisogni Educativi Speciali, assegnando etichette, con o senza certificato, che producono frammentazione, divisione ed esclusione, più che favorire una vera inclusione.
\end{enumerate}
\section*{Il dibattito e una proposta}
E allora? Quale può essere la soluzione? Cosa bisogna fare per evitare il rischio che un tema così importante, come quello dei Bisogni Educativi Speciali, possa mandare in tilt il sistema organizzativo scolastico a danno principalmente degli stessi alunni?
Secondo me è indispensabile:
\begin{description}
	\item[$-$] bloccare immediatamente gli effetti organizzativi sul prossimo anno scolastico della Direttiva del 27 dicembre 2012 [“Strumenti d'intervento per alunni con Bisogni Educativi Speciali e organizzazione territoriale per l'inclusione scolastica”, N.d.R.] e della Circolare 8/13 sui \glslink{besa}{BES};
	\item[$-$] realizzare, nel prossimo anno scolastico, un dibattito costruttivo e condiviso, che parta dalla base (dalle scuole – dagli insegnanti – dai genitori – dalle associazioni – dal mondo della sanità) e che costruisca un modello organizzativo funzionale per la personalizzazione dei percorsi degli alunni con Bisogni Educativi Speciali nelle scuole di ogni ordine e grado;
\item[$-$]  prevedere e destinare le risorse di personale specializzato e le risorse finanziarie necessarie per un modello organizzativo che preveda un intervento personalizzato per tutti gli alunni con Bisogni Educativi Speciali, uscendo dalla consueta logica del «Voi intanto partite con le innovazioni che per le risorse vi faremo sapere poi…»;
	\item[$-$]attivare immediatamente la formazione specifica sui \glslink{besa}{BES} per i docenti disciplinari, per quelli di sostegno e anche per i dirigenti scolastici;
	\item[$-$]creare un nuovo pro\glslink{fila}{FIL}o degli insegnanti specializzati per il sostegno, ampliandone i compiti e le competenze, sia nell'ottica dei bisogni educativi speciali e del Piano Didattico Personalizzato, sia in quella delle situazioni di gravità che vanno affrontate con una competenza tecnica e specifica reale e adeguata;
	\item[$-$]creare una classe di concorso specifica per gli insegnanti di sostegno specializzati, in modo da richiedere una scelta professionale che in nessun modo possa essere dettata da esigenze di organici (sistemazione di docenti soprannumerari) o di opportunismo personale.
\end{description}
In definitiva, ritengo sia indispensabile prendere una decisione coraggiosa, adesso e prima di settembre. La decisione di bloccare gli effetti di un cambiamento di cui probabilmente sono stati sottovalutati gli effetti negativi. Credo valga la pena discuterne ancora, coinvolgendo le comunità scolastiche nella discussione, prima di mettere a rischio ciò che è stato costruito in oltre trentanni di integrazione nella nostra scuola pubblica.
E chi è d'accordo con le posizioni espresse in questo mio intervento, può anche firmare la petizione Referendum \glslink{besa}{BES}. Fermiamo la CM 8 e costruiamo il cambiamento, che punta a raggiungere cinquantamila e uno firmatari, numero simbolicamente corrispondente alla metà più uno dei docenti specializzati che insegnano nelle nostre scuole.
Obiettivo concreto di tale iniziativa è di dare voce a tutti gli insegnanti, agli stessi studenti e alle loro famiglie, alle associazioni e a tutti coloro che a qualsiasi titolo hanno a cuore l'integrazione scolastica.\footcite{Scataglini2013}

Insegnante specializzato all'Aquila, formatore sulle metodologie di recupero e sostegno e docente a contratto di Didattica della Matematica per l'Integrazione alla Facoltà di Scienze della Formazione dell'Università dell'Aquila. Il presente testo è già apparso in «www.La Letteratura e noi.it», con il titolo “Funzionamento Inclusivo Limite. Una proposta” e viene qui ripreso, con minimi riadattamenti al diverso contenitore, per gentile concessione.

17 giugno 2013
Ultimo aggiornamento: 18 giugno 2013 11:36