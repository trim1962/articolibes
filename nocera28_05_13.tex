\author{Salvatore Nocera}
\title{L'impatto con la realtà è l'unica prova attendibile }
\phantomsection
\label{cha:nocera280513}
%\epigraph{Come realizzare un'inclusione di qualità nell'attuale scuola italiana? Salvatore Nocera ravviva il dibattito in corso sulle pagine del nostro giornale e rispondendo a un intervento di Andrea Gavosto, presidente della Fondazione Agnelli, afferma che «l'unica ad essere incontestabile è la prova dei fatti». E aggiunge: «Vedremo quindi, tra le varie ipotesi che circolano, quale reggerà nell'impatto con la realtà»}{Salvatore Nocera}
\begin{abstract}
Come realizzare un'inclusione di qualità nell'attuale scuola italiana? Salvatore Nocera ravviva il dibattito in corso sulle pagine del nostro giornale e rispondendo a un intervento di Andrea Gavosto, presidente della Fondazione Agnelli, afferma che \caporali{l'unica ad essere incontestabile è la prova dei fatti}. E aggiunge: \caporali{Vedremo quindi, tra le varie ipotesi che circolano, quale reggerà nell'impatto con la realtà}
\end{abstract}
\maketitle
\datapub{28 maggio 2013}
In un mio articolo\pageref{cha:nocera130513}, pubblicato su queste pagine il 13 maggio scorso, cercavo, anche con un pizzico di ironia, di rassicurare gli insegnanti per il sostegno, che temevano di essere “cancellati” dalla scuola italiana. Nella discussione avevo sostenuto che attualmente essi non corrono tale rischio, che avrebbe potuto verificarsi solo con l'attuazione di un'ipotesi avanzata dalla Fondazione Agnelli [ci si riferisce alle ipotesi avanzate nel rapporto intitolato Gli alunni con disabilità nella scuola italiana: bilancio e proposte, Erickson, 2011, elaborato appunto dalla Fondazione Agnelli, insieme all'Associazione TreeLLLe e alla Caritas Italiana, N.d.R.], che per troppo impeto polemico, avevo definito di “fantadidattica” e “fantainclusione”. Ebbene, data la notoria serietà dei lavori scientifici della Fondazione Agnelli, mi rendo conto di avere ecceduto e pubblicamente me ne scuso.
Ma veniamo al punto del contendere: come cioè realizzare, nel prossimo futuro, un'inclusione di qualità nell'attuale situazione della scuola italiana.

La citata ipotesi di ricerca promossa dalla Fondazione Agnelli è di rimandare nelle classi, come insegnanti curricolari, l'80\% degli attuali centomila (e oltre) docenti per il sostegno, facendo sì che ad assumersi la responsabilità dell'inclusione degli alunni con disabilità siano gli ottocentomila docenti curricolari, formando in parallelo – con i restanti ventimila docenti per il sostegno – dei nuclei di consulenza itineranti per le scuole. Ciò a costi invariati e senza problemi di licenziamenti.

Un'ipotesi che in astratto mi è parsa subito interessante. Mi lascia invece del tutto perplesso la sua concreta attuazione a livello nazionale. Infatti:
\begin{description}
	\item[-] in un periodo storico in cui, con le Riforme Moratti e Gelmini, il numero dei docenti curricolari è stato drasticamente ridotto, come sarà possibile aumentarlo di circa un decimo del numero globale?
	\item[-] gli attuali docenti curricolari sono  di norma sprovvisti di formazione sulle didattiche inclusive e non è attualmente previsto alcun obbligo di formazione iniziale o in servizio in tal senso.
\end{description}

L'ipotesi della Fondazione Agnelli, del resto, prevede una formazione iniziale che sta avviandosi in Trentino, dove la ricerca-azione sta per svolgersi; ma si tratta di poche decine di docenti e la Provincia Autonoma di Trento dispone di abbondanti mezzi finanziari. Di contro, in Italia, quest'anno sono stati ulteriormente ridotti gli scarsi finanziamenti per la formazione, che comunque rimane contrattualmente volontaria.
Un'altra condizione indispensabile di realizzazione è costituita poi dal numero di alunni per classe. Oggi, infatti, malgrado l'articolo 5, comma 2 del Decreto del Presidente della Repubblica (DPR) 81/09\footcite{DPR_81_2009} – che fissa a venti il tetto massimo di alunni nelle classi frequentate da alunni con disabilità – abbiamo classi con almeno venticinque alunni, molte delle quali raggiungono le trenta unità, talora addirittura superandole. Con classi così numerose, come possono i docenti curricolari, anche se formati adeguatamente, prendersi cura da soli degli alunni con disabilità, che talora sono più di uno per classe?

E ancora, quei nuclei di consulenza itineranti, dei quali si parla nella proposta, attualmente sarebbero costituiti da docenti che vengono dal lavoro quotidiano in classe, ma a regime sarebbero composti da esperti formati e viventi fuori della scuola, che difficilmente potrebbero dare suggerimenti ai colleghi che lavorano “in trincea” e che anzi li vedrebbero come degli “intrusi saccenti”, ma ben poco operativi.

Io non mi intendo di didattica, mi occupo solo di norme e tuttavia, da un punto di vista organizzativo, mi sembra assai difficile che questa ipotesi di lavoro possa sostituirsi a quella attuale, che va per altro certamente – e notevolmente – riveduta. Per questo ho nutrito fin dall'inizio – e continuo a nutrire – molte riserve su tale ipotesi, la quale, anche se riuscisse ad essere sperimentata positivamente in una piccola e ricca Provincia Autonoma, difficilmente si potrebbe estendere a tutto il territorio nazionale.
L'ipotesi che come FISH (Federazione Italiana per il Superamento dell'Handicap) pensiamo invece di sottoporre al Ministero, ai Sindacati e al Parlamento, vede anch'essa il ritorno alla presa in carico del progetto inclusivo da parte dei docenti curricolari, per i quali, però, dovrà essere prevista normativamente una dignitosa formazione iniziale e obbligatoria in servizio sulle didattiche inclusive. Tali docenti dovrebbero essere “sostenuti” da insegnanti specializzati, formati in un'apposita classe di concorso, con un ruolo autonomo, in modo da avere una vera scelta professionale, mentre attualmente essa è di solito – tranne naturalmente le debite eccezioni – una sorta di ripiego, per trovare un posto di insegnamento dal quale fuggire appena terminato il quinquennio di “ferma”.

Fondamentale, inoltre, dovrà essere la riduzione del numero di alunni per classe, sì da facilitare un dialogo educativo personalizzato.

In queste condizioni di lavoro si potranno anche ridurre le ore di sostegno assegnate a ciascun alunno, a seconda delle sue difficoltà e del progresso qualitativo dell'inclusione, eliminando così le migliaia di ricorsi ai Tribunali Amministrativi Regionali (TAR), per avere il massimo delle ore di sostegno, dal momento che oggi – quando manca il docente specializzato – l'alunno è abbandonato in fondo alla classe o mandato fuori con un assistente o con un collaboratore scolastico, oppure ancora, insieme ad altri alunni con disabilità, nella cosiddetta “aula di sostegno”, ciò che è espressamente vietato dalle Linee Guida Ministeriali del 4 agosto 2009, ma che purtroppo continua a persistere in molte scuole.
Anche l'ipotesi della FISH può essere – ed è – contestata da talune parti, tutto è discutibile. Ma ritengo che la prova dei fatti sia l'unica incontestabile e avremo quindi modo di vedere quale, tra queste e altre ipotesi che circolano per l'Italia, reggerà nell'impatto con la realtà\footcite{nocera4}.

Vicepresidente della FISH (Federazione Italiana per il Superamento dell'Handicap).
28 maggio 2013
© Riproduzione riservata
