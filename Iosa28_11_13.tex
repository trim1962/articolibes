\author{Raffaele Iosa}
\title{La fine dei BES nel Paese dei cachi}
\phantomsection
\label{cha:Iosa281113}
%\begin{abstract}
%	Infatti, una Nota Ministeriale prodotta nei giorni scorsi chiarisce ulteriormente alcuni punti controversi della recente normativa sui Bisogni Educativi Speciali (BES), fissata tra la fine del 2012 e l'inizio di quest'anno, che tanto ha già fatto discutere. Appare in particolare importante la sottolineatura data da un lato alla prevalenza delle valutazioni pedagogiche da parte dei docenti, dall'altro al rispetto dell'autonomia scolastica
%\end{abstract}
\maketitle
\datapub{28 novembre 2013}

L'attesa CM di “chiarimento” del MIUR sulla questione BES, firma  Chiappetta, è dunque arrivata il 22 novembre\footcite{Nota_2563_2013}. Esito sconcertante, ancora una volta non definitiva e  un ossimoro: passi avanti per tornare indietro. Alcune parti della CM smontano durezze della Direttiva 2012\footcite{dir27Dic2012}; si cita finalmente il DPR 275/99\footcite{DPR_275_1999} autonomia, ma aumentano  anche contraddizioni e confusione,  a metà di un guado  interminabile. Guado bagnato  da questa  “sperimentazione”  in attesa di non si sa cosa, impantanati  su decisioni di dubbia legittimità da parte del MIUR. Ad esempio, tutti sanno che qui la parola “sperimentazione” è solo diplomatica, effetto bizzarro della trattativa sindacale di giugno che ha fatto ritirare gli ukaze su PAI e dintorni a generico laissez faire. Una sperimentazione che non ha vincolo giuridico e aumenta la vacuità. A chi ha scordato il Regolamento autonomia ricordo che l'art. 17  DPR 275/99\footcite{DPR_275_1999} ha abrogato il potere del MIUR di gestire sperimentazioni. Ricordo bene quando al MPI si discusse sulla fine del Ministero padre-padrone della sperimentazione e del dibattito parlamentare che confermò  all'unanimità.  Ma tanto vale, l'intera attuale azione sui BES si basa dunque su un flatus vocis senza coerenza giuridica.

Ma c'è di più: la retorica frase per cui la sperimentazione servirebbe a “monitorare procedure, metodologie e pratiche anche organizzative”  è di pessimo gusto normativo, visto che nulla è stato reso pubblico sul come si monitora, chi lo fa (gli autori autoreferenziali dei testi o soggetti terzi?), quanto costa. Quindi se la fanno e se la dicono?

Eppure ammetto che la CM Chiappetta su alcuni aspetti sembra fare qualche passo avanti di buon senso rispetto alla Direttiva.  Vi è  un tono più sobrio dall'enfasi del modello BES-PDP-PAI precotti (con i moduli mandati dal MIUR), soprattutto la (coraggiosa) conferma della competenza dei docenti sul considerare  la condizione individuale degli alunni a prescindere dalle carte mediche. Si vede il tentativo di riparare. Infatti la CM su vari aspetti semplifica,  smonta  rischi iatrogeni (es. gli stranieri), riduce l'alluvione obbligatoria di PDP, come segnala Salvatore Nocera. Ma basta?  Per me no, anche se l'oggi è meglio di ieri, perché restano troppe contraddizioni, figlie di una Direttiva con basi giuridica, scientifica pedagogica ingarbugliate nei suoi epistemi.  Come spesso accade, la CM ridimensiona, semplifica, dimentica. Forse di più non si poteva, nel paese dei cachi.
\section*{Torna una strana autonomia?}
C'è un clamoroso mini-lapsus nella prima pagina, dove (finalmente) in tema di individualizzazione si cita l'art. 4 del DPR 275/99\footcite{DPR_275_1999} Regolamento autonomia (e il suo rango costituzionale) con un incipit che dice “Giova forse ricordare che…”. Insomma quel forse a chi pensa: all'amnesia delle scuole o all'amnesia degli autori precedenti, che del Regolamento Autonomia si erano scordati?
Ma la CM Chiappetta doveva finire qui. Toccava al Ministero segnalare il problema, indicare gli obiettivi per migliorare l'inclusione, al resto ci avrebbero pensato le scuole. E poi (come non si fa mai) controllare, altra cosa del monitorare. Quindi niente calembour sui PDP, PAI, ecc..
La chiave dell’“autonomia” però si ferma all'affido alle scuole dell'autonoma decisione sui casi.

Per troppi anni l'autonomia è stata mortificata. Questo è solo un segnetto incompleto. \'{E} però una leva che le scuole devono utilizzare con forza, evitando che l'autonomia diventi irresponsabilità ma anche di farsi servilmente più realisti del re. Mi permetto un suggerimento: in fatto di inclusione si parta da quello che si fa già, si rimettano in ordine le sbavature. Poi poche sigle!

E a questo punto la CM Chiappetta poteva dire: “\'{E} potere delle scuole l'individualizzazione  didattica, ma anche se utilizzare o no  “scatole formali” dette PDP oppure scegliere “scatole Qui Quo Qua”, o altre. \'{E} rimandato alle scuole il come definire nel POF le azioni inclusive, se fare un capitolino detto PAI, tra  quello delle gite e le assenze, o un testo detto “Paperino” sparso tra varie parti del POF, ricordando però che non è “potere” ma “dovere” progettare inclusione (art. 3 DPR 275/99\footcite{DPR_275_1999} e art. 3 comma 2 Cost.).  

E invece no, il resto della CM si arzigogola su PDP e PAI stile  vecchio Ministero, più confusi di prima, con il paradosso di ben 4 testi normativi in 11 mesi!
Sottolineo che la questione autonomia è strategica, non è accademia, è l'idea se la scuola sia vista come leva attiva di per sé o invece una  massa di ignoranti da  ammaestrare. 

Noi siamo per il di per sé, concetto questo della medesima filosofia che vogliamo si usi verso quegli alunni che per svariati motivi faticano a scuola, e cioè non trattarli come malati di qualcosa,  cui offrire la nostra narcisa  riparazione (la filosofia iatrogena della Direttiva 2012\footcite{dir27Dic2012}, ma soggetti  di per sé dotati anche di doti, soprattutto di resilienza da promuovere con empatia per il loro successo, e che certificati e sigle potrebbero invece danneggiare.  

Ed è qui, nella stretta relazione tra “autonomia delle scuole” e “autonomia delle persone” che il vulnus perfino filosofico è aspro in questa vicenda normativa. A mio parere, nella Direttiva  la “non autonomia delle persone”  è circoscritta in una logica iatrogena, cui si vuol dare risposte per i loro presunti “bisogni educativi speciali”, e questa “non autonomia dei ragazzi” è non a caso connessa alla “non autonomia degli insegnanti” ridotti a seguire il castello formale dell'operare sui sintomi, del costruire tanti PDP, del trovare agevolazioni varie abbassando l'autonomia professionale assieme alle attese sui ragazzi, con difficili mediazioni e tanta confusione.

Un atteggiamento di natura illuministica che risente di una certa pedagogia direttiva dell’”elite dei puri” di cui temo risenta  a volte, conoscendolo un po',  anche Marco Rossi Doria. 

Nel  nuovo Decreto Scuola le azioni per contrastare la dispersione scolastica hanno una visione sistemica ben diversa dal modello BES, e spero che  Rossi Doria  questa volta sia più prudente nel normarne  la gestione, partendo dalla fiducia dei docenti, partendo da ciò che già si fa, favorendo il positivo che c'è, evitando di sembrare (in buona fede?) promotori di Verità da pedagogia di Stato.
\section*{I cachi del PDP}
La parte della CM sul PDP è, purtroppo, deludente.  Ricordo che il PDP, per la CM,  non è obbligatorio, non è dovuto da diagnosi clinica, è  azione pedagogica discrezionale. Ma ci può essere individualizzazione  dispensativa/compensativa senza un PDP e la codifica BES? La CM qui balbetta, si confonde, dice cose opposte tra loro. Ma questo è il cuore di tutta la questione!
Cerco di capire l'origine di questi balbettii. Nella CM si  rivela che la Direttiva aveva come priorità  l’ “inserimento” dei borderline e degli ADHD nelle categorie da “curare in modo speciale” perché affetti da “disturbi” clinicamente accertati e di “sicura origine neurobiologica”. Dunque:  alcune lobbies hanno spinto per avere “garanzie” su alcuni “stati personali” per avere una 170 bis senza fare una nuova legge. Parlo di diagnosi su cui gli studiosi litigano da anni sulla scientificità, comunque centrati sul sintomo (QI o ADHD) e non sulla persona. Solo dopo (si dice “inoltre, infine”) si pensa di aggiungere ai malati delle lobby  anche gli “altri inadeguati” tipo disagio sociale, disattenzione, migrazione... dentro un'unica confusa macro-categoria cui porre un “segno di separazione” con la sigla BES. A tutti, come aiuto,  si “concede” l'allargamento delle prebende della Legge 170\footcite{legge170} (dispensa  compensa,  il vero tormentone). Legge, questa,  cui vanno molte critiche per l'idolatria di un problema reinventato “disturbo” con varie teorie eziologiche e test discutibili,  un modello iatrogeno  che privatizza in rapporti di forza famiglie iperprotettive e scuola, isolaziona i ragazzi abbassandone le attese, medicalizza le didattiche. La 170 favorisce  fughe di altre lobbie per trattamenti simili di isolazione. Fine inclusione sociale. 

Ma qui  Chiappetta rischia: come potranno accettare adesso le lobbies dell'ADHD che i loro protetti non siano garantiti di dispensa e compensa perché è la scuola e non i dottori a decidere? Sento già aria di una CM Chiappetta bis o di una 170 bis. E come faranno gli iatrogenisti  allupati a dominare sugli 80.000 ADHD e i 200.000 border? Su questo vedo l'origine delle confuse versioni presenti in poche righe nel capitolo del PDP.

Prima di analizzare le confuse versioni sul PDP, però, notiamo una simpatica stranezza perché la categoria degli “Inoltre infine gli altri” riceve un nome. I classificatori della CM li battezzano “i difficoltati”, da non confondere con i “disturbati”. Fa un po’ sorridere leggere  la parola “difficoltà”, con   la scansione tra “ordinarie e gravi”,  per descrivere i ragazzi non H,  non DSA,  non Q.I. e  non ADHD che si vogliono infilare nel recinto BES. Servirebbe un giovane Basaglia a sgridarci con la sua “Maggioranza deviante”. Questa parola sembra banale, ma è parola complessa: persona vista considerando sia le difficoltà “dentro di sé” ma anche quelle date dal ”mondo intorno” o nuova categoria para-scientifica con una sua graduatoria di dolore?  Chissà!

\'{E} curioso si dica che queste difficoltà non sono di per sè  categoria da BES, anzi si affermi  “…la scuola può intervenire nella personalizzazione in tanti modi diversi, informali o strutturati, secondo i bisogni e la convenienza..”.  Ma perché solo i difficoltati sarebbero  non da BES? Tutti gli ADHD hanno bisogno di dispensa compensa o anche di altri diversi modi, informali e strutturati?
E’ evidente qui il nodo: la questione Bes si/no è solo questione di “potere” per le lobbies di famiglie e medici contro la scuola. Altro che dialogo: imporre a insegnanti lavativi il Metodo Terapeutico.

Ricapitoliamo le confuse versioni: in una parte della CM si dice che anche se i “difficoltati” non sono i “disturbati”, anche per loro si può concedere dispense e compense,  ma…necessariamente per tutti con un PDP e una sigla BES,  partendo anche dal grido di dolore pervenuto al MIUR (lo usava Stalin per giustificare gli arresti) con “richieste di docenti (genitori ADHD?)” per allargare  gli strumenti individualizzati.  Da un'altra parte, invece, la CM sostiene che  sia per i difficoltati che per i disturbati ADHD e Q.I. non è obbligatorio che la scuola li qualifichi come BES; la definizione è lasciata alla discrezionalità della scuola, sia sulla “persona” (Bes o no?) che sul progetto (PDP o no, dispensa/compensa o no).  Infatti, dice la CM,  sia ragazzi con “disturbi” sia  con “difficoltà”,  non compete alla scuola fare diagnosi, ma le compete decidere se ai ragazzi vanno date o no le attività individualizzate, a prescindere dalla diagnosi clinica, ma si sfuma sul tormentone: anche per la dispensa/compensa? Si aprono squarci non di autonomia nelle scuole ma di ansia: se abbiamo una diagnosi medica li chiamiamo  BES e facciamo un PDP o no?  Ci possono essere ritorsioni? Se ad Antonio un titolo di BES crea depressione, posso lo stesso dispensare e compensare?

Dunque, la nota Chiappetta avrà un effetto paradossale ma altamente salutare. Quello che dà il titolo a questo articolo: “la fine dei BES”. Con questo confuso parterre di norme  i nuovi Bes catalogati secondo la Direttiva si ridurranno a quattro gatti. Nel dare “potere” alle scuole in questo modo tutto romano avverrà sicuramente una forte riduzione quantitativa dei nuovi Bes, perché le scuole non sono così allupate di carte quando in cambio  c'è il nulla (né soldi né personale), ma continueranno ad agire sul disagio individuale con l'attenzione di prima, bene o male secondo le scuole, evitando targhe inutili e lavorando sottovia,  speriamo con buon senso. E perché la scuola deve rischiare quando le norme sono confuse ed è solo sperimentazione?  Prendete gli stranieri: prima dovevamo inserirli se arrivati da poco, adesso stiamo attenti che non ci siano discriminazioni! Bel cambio!

Dunque, la Direttiva BES viene nei fatti ridimensionata proprio nel suo cuore ideologico, quello del  classificare e separare. E il basso numero di bessati sarà il segno della fine. Perché la iatrogenesi punta ai numeri, che siano alti e ben protetti.  

Sorridiamo amaramente anche per il PAI, sfumato da Chiappetta  che riprende il linguaggio barocco della CM Stellacci  e addolcisce il militarismo della Direttiva che voleva PAI a giugno ed elenchi BES in bell'ordine! Adesso il  PAI è uno   “sfondo e fondamento sul quale sviluppare una didattica attenta ai bisogni di ciascuno nel realizzare gli obiettivi comuni”. Sfondo e fondamento: uno svarione linguistico da giuristi prestati al pedagogese del genericismo parolaio.

Anche la continuazione della sperimentazione porterà al grande risultato dei quattro gatti BES..

Ma scrivo questo con marezzare, perché le difficoltà del vivere e dell'essere sono grande tema educativo, trasversale a tutti, cui le classificazioni se mal poste producono ulteriori dolori. Un'occasione perduta a parlare seriamente di dolore umano per superficialità e ideologismo, ma almeno una sconfitta della furia iatrogena imperante.
A me pare, però, doveroso far comunque riflettere gli insegnanti sulle loro modalità osservative per “comprendere” le difficoltà dei ragazzi. Il rischio di para-clinicismo è forte.  Se si evita la discussione catalogante BES si/no, resta la necessità (eterna) di una conoscenza più profonda di tutti i nostri ragazzi, evitando di fermarsi a fredde prove  strutturate ma con un ascolto asimmetrico dell'altro che ne rilevi il sé unitario e complesso, in cui la scienza è servizio e non dogma.  Circa l'individualizzazione dell'insegnamento, è ora di parlarne non solo per chi fa fatica,  ma strutturale alla vita delle classi intere, cosa  difficile da fare con gli attuali organici, le classi numerose, le rigidità organizzative del centralismo gelminiano, l'eccesso di insegnamenti lineari e dissociati tra loro con le competenze ridotte a patetico nuovo libro Cuore.
A proposito di dispensa e compensa una soluzione c'era, ma troppo scandalosa per i ministeriali: riconoscere che dispensa e compensa sono già dentro l'articolo 4 del DPR 275/99 come azione “naturale” della scuola per tutti a prescindere da qualsiasi etichetta, quando serve e non come “concessione” di una Legge e “recinto protetto” di qualcuno stabilito dal dottore.

A proposito di questo, merita ascoltare gli ammonimenti di Salvatore Nocera sul caos che dispensa e compensa creano in ordine non al valutare quotidiano ma agli esami di Stato sul tema dell'equità, tema che ha fatto litigare questa primavera le Direzioni dello Studente e quella degli Ordinamenti.
\section*{Tra famiglie frastornate e insegnanti in difficoltà}
In questi mesi tra insegnanti, famiglie e dottori, se ne sono viste di tutti i colori. Ma dei genitori   le note dicono poco,  se non l'utilità di evitare  “contenziosi” come la peste.

Vi sono genitori che vengono con una diagnosi e chiedono-pretendono la 170 bis. Ci sono genitori che a sentire gli insegnanti proporre loro il BES reagiscono male. Con questi due tipi di famiglie ci sarebbe la soluzione della 104/92\footcite{Legge_104_92}: fare Bes solo con genitori che lo chiedono, ma è giusto? Ma volete confondere i disabili con questi altri difficoltati e disturbati? Rischiamo una “trattativa” per cui alla fine si certificherebbero per quieto vivere solo quei Bes di genitori affamati di protezione?
La confusione ha aumentato i problemi: in alcune famiglie è scattata la “fantasia DSA”, cioè quel confuso atteggiamento difensivo dei figli per cui ogni loro difficoltà, se certificata, trova un potere contrattuale nuovo con gli insegnanti cui imporre di essere “buoni” con i figli. Naturalmente  trovo anche consigli di classe poco aperti all'individualizzazione e all'ascolto degli alunni. Ma questa dialettica non va risolta da certificati, quanto da una relazione dialogante scuola-famiglia cui affidare   la responsabilità comune sul cuore della scuola, che è l'educazione. Questa relazione si costruisce con l'ascolto reciproco, non con certificati, né forme neo-contrattualistiche.
Infine, accade che arrivino alle scuole certificati clinici che “ai sensi della Direttiva MIUR del 27 dicembre 2012\footcite{dir27Dic2012}” dichiarano che l'alunno X  va considerato “militarmente” BES e quindi avente diritto a dispense e compense. Non basta dire che questo comportamento è illegittimo, come si deduce dalla nota Chiappetta: non tocca ai  medici  dirci cosa fare! Ma come si fa a litigare sempre?
\section*{Tornano i GLIP?}
Trovo, invece, assolutamente simpatica la frenata sull'architettura dei CTS e strutture locali viste da Roma in modo illuministico. La CM Chiappetta è chiara. Intanto ci sono per legge  i GLIP come struttura di integrazione orizzontale tra i soggetti istituzionali (il vero problema dell'integrazione oggi: la difficile integrazione tra istituzioni!), poi c'è l'autonomia regionale, la responsabilità delle USR, ed anche leggi che prevedono scuole-polo con una logica che va chiarita non in una Direttiva centralista. Piace quindi leggere che l'intera materia sarà ri-trattata con “successiva nota”.
\section*{Dulcis in fundo}
Dunque, pur essendo convinto che dopo la CM Chiappetta la questione BES inizia a finire nei fatti, penso sia opportuno avere maggior coraggio di visione sul futuro.
Temo, naturalmente, i fautori dei Bes che hanno denaro e potere, convinti della causa del nuovo recinto. Dobbiamo serenamente contrastare gli eccessi e gli abusi dei predicatori. Sarebbe, per esempio, opportuno ridimensionare i master universitari su DSA e adesso BES che stanno creando nuovi centri di potere e di predicazione, favorendo invece  pratiche  di ricerca-azione e di auto-aiuto che certo l'accademia  non conosce, favorendo la ricerca solidale su temi centrali per  l'istruzione

Penso quindi sia doveroso, per evitare una quinta CM Chiappetta e poi una sesta, suggerire tre cose:
\begin{enumerate}
	\item Sarebbe opportuno che la ministra Carrozza prendesse in mano l'intera faccenda, comprendendo come si sia ormai in un vicolo cieco che rischia di buttare i ragazzi con l'acqua sporca. Avochi a sé la questione, sospenda le quattro note ormai affastellate l'un l'altra, dia una moratoria per aprire una discussione sull'inclusione con uno sguardo sociale e pedagogico che eviti medicalizzazioni. Sarebbe anche utile che la ministra buttasse un occhio a chi materialmente scrive e decide al MIUR su questi temi,  per evitare a sé  e alle scuole confusione e conflitti inutili.
	\item Sarebbe buona cosa se più scuole possibili, con serena onestà intellettuale, decidessero in coscienza di rifiutare la Besissazione dei loro alunni in queste condizioni normative e di risorse (è sperimentazione e c'è l'autonomia), non per apatia ma per occuparsi meglio dei loro ragazzi e dei progetti inclusivi, esigendo chiarezza che non violi, con norme illegittime, l'autonomia didattica.
	\item Penso, infine, che molti  insegnanti, studiosi della materia, esperti debbano far sentire di più la loro voce perché si è oltre limite. Parli chi sa di inclusione per portare dialettica viva e non questo enfatico caos, che rischia di peggiorare e non migliorare l'inclusione. Vorrei parlassero anche i molti operatori che trovano la Legge 170 inguardabile, per la quale meriterebbe una seria inchiesta scientifica neutrale, per ricomporre le azioni inclusive della scuola entro ambiti di vera efficacia.
\end{enumerate}
Sullo sfondo  restano le questioni di natura iatrogena che stanno pervadendo la società e l'educazione, con un potere medicale della diagnosi che sta destrutturando la vita umana e non solo la scuola. E’ questo sfondo politico e culturale più profondo l'oggetto dei miei attuali studi, che vanno ben oltre la piccola questione BES, che da oggi si ridimensiona, per un recupero di un nuovo umanesimo che sappia cogliere la scienza non come potere del dogma, ma metta al centro l'umano come soggetto responsabile del mondo, sia di quello che c'è sia di quello che si vuol fare. L'Italia può anche non essere il paese dei cachi\footcite{Iosa2013}.