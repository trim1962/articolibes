\author{Salvatore Nocera}
\title{Un po' di serenità in più sui Bisogni Educativi Speciali}
\label{cha:nocera261113}
\begin{abstract}
	Infatti, una Nota Ministeriale prodotta nei giorni scorsi chiarisce ulteriormente alcuni punti controversi della recente normativa sui Bisogni Educativi Speciali (BES), fissata tra la fine del 2012 e l'inizio di quest'anno, che tanto ha già fatto discutere. Appare in particolare importante la sottolineatura data da un lato alla prevalenza delle valutazioni pedagogiche da parte dei docenti, dall'altro al rispetto dell'autonomia scolastica
\end{abstract}
\maketitle
\datapub{26 novembre 2013}
Con la recente Nota n. 2563\footcite{Nota_2563_2013}, prodotta il 22 novembre scorso, il Ministero dell'Istruzione, Università e Ricerca ha fornito una serie di ulteriori chiarimenti relativi alla normativa sui Bisogni Educativi Speciali (BES), fissata dalla Direttiva Ministeriale\footcite{dir27Dic2012} del 27 dicembre 2012 e dalla successiva Circolare 8/13\footcite{cm8_2013} del 6 marzo di quest'anno, che tanto hanno fatto discutere nei mesi scorsi.
Vediamo dunque quali sono i  punti più interessanti di questa nuova Nota Ministeriale.
\section*{PDP (Progetto Didattico Personalizzato)}
Molti sindacati e docenti avevano avanzato forti critiche al proliferare di PDP (Progetti Didattici Personalizzati), ciò che a loro avviso era stato indotto proprio dalla recente normativa. In tal senso, il Ministero fornisce chiarimenti e rassicurazioni in proposito, così come segue:
\caporali{In ultima analisi, al di là delle distinzioni sopra esposte, nel caso di difficoltà non meglio specificate, soltanto qualora nell'ambito del Consiglio di classe (nelle scuole secondarie) o del team docenti (nelle scuole primarie) si concordi di valutare l'efficacia di strumenti specifici questo potrà comportare l'adozione e quindi la compilazione di un Piano Didattico Personalizzato, con eventuali strumenti compensativi e/o misure dispensative. Non è compito della scuola certificare gli alunni con bisogni educativi speciali, ma individuare quelli per i quali è opportuna e necessaria l'adozione di particolari strategie didattiche}.
Solo nei casi, quindi, in cui si ritenga di consentire strumenti dispensativi e compensativi, ha senso formulare un PDP e la Nota del 22 novembre è ancora più esplicita nel lasciare la massima autonomia di giudizio ai docenti di fronte a diagnosi che non portino a certificazioni di disabilità e DSA (disturbi specifici dell'apprendimento): \caporali{Si ribadisce che, anche in presenza di richieste dei genitori accompagnate da diagnosi che però non hanno dato diritto alla certificazione di disabilità o di DSA, il Consiglio di classe è autonomo nel decidere se formulare o non formulare un Piano Didattico Personalizzato, avendo cura di verbalizzare le motivazioni della decisione}.
\section*{Alunni di cittadinanza non italiana}
Anche nei confronti di questi alunni si era lamentato il rischio di un eccesso di PDP. Qui la Nota Ministeriale n. 2563\footcite{Nota_2563_2013} ulteriormente chiarisce che \caporali{\mancatesto In particolare, per quanto concerne gli alunni con cittadinanza non italiana, è stato già chiarito nella C.M. n. 8/2013\footcite{cm8_2013} che essi necessitano anzitutto di interventi didattici relativi all'apprendimento della lingua e solo in via eccezionale della formalizzazione tramite un Piano Didattico Personalizzato. Si tratta soprattutto -- ma non solo -- di quegli alunni neo arrivati in Italia, ultra tredicenni, provenienti da Paesi di lingua non latina (stimati nel numero di circa 5.000, a fronte di oltre 750.000 alunni di cittadinanza non italiana) ovvero ove siano chiamate in causa altre problematiche. Non deve tuttavia costituire elemento discriminante (o addirittura discriminatorio) la provenienza da altro Paese e la mancanza della cittadinanza italiana. Come detto, tali interventi dovrebbero avere comunque natura transitoria}.
\section*{PAI (Piano Annuale per l'Inclusività)}
Anche questo punto era stato ritenuto da molti come un inutile aggravio burocratico. Il Ministero ora precisa che \caporali{\mancatesto II Piano annuale per l'inclusività deve essere inteso come un momento di riflessione di tutta la comunità educante per realizzare la cultura dell'inclusione, lo sfondo ed il fondamento sul quale sviluppare una didattica attenta ai bisogni di ciascuno nel realizzare gli obiettivi comuni, non dunque come un ulteriore adempimento burocratico, ma quale integrazione del Piano dell'Offerta Formativa, di cui è parte sostanziale (nota prot. n. 1551\footcite{Nota_1551_2013} del 27 giugno 2013). Scopo del piano è anche quello di far emergere criticità e punti di forza, rilevando le tipologie dei diversi bisogni educativi speciali e le risorse impiegabili, l'insieme delle difficoltà e dei disturbi riscontrati, dando consapevolezza alla comunità scolastica – in forma di quadro sintetico – di quanto sia consistente e variegato lo spettro delle criticità all'interno della scuola. Tale rilevazione sarà utile per orientare l'azione dell'Amministrazione a favore delle scuole che presentino particolari situazioni di complessità e difficoltà.}
Ciò, per altro, ribadisce quanto emergeva con chiarezza già dalla precedente normativa, e cioè che il POF (Piano dell'Offerta Formativa) deve avere come sua caratteristica la logica inclusiva verso gli alunni più deboli e che l'attenzione dell'Amministrazione Scolastica deve rivolgersi alle scuole maggiormente in difficoltà, per sostenerle in questo delicatissimo compito.
\section*{GLI (Gruppo di Lavoro per l'Inclusività)}
Molte critiche si erano poi appuntate sull'ampliamento delle funzioni e della composizione del GLI (Gruppo di Lavoro per l'Inclusività), che adesso dovrà occuparsi, oltre che della disabilità, pure degli altri casi di BES.
Su tale punto il Ministero chiarisce che «in relazione ai compiti del Gruppo di Lavoro per l'Inclusività, che assume, secondo quanto indicato nella C.M. n. 8/2013\footcite{cm8_2013}, funzioni di raccordo di tutte le risorse specifiche e di coordinamento presenti nella scuola, si rammenta il rispetto delle norme che tutelano la privacy nei confronti di tutti gli alunni con bisogni educativi speciali. In particolare, si precisa che nulla è innovato per quanto concerne il Gruppo di lavoro previsto all'art. 12, co. 5 della Legge 104/92\footcite{Legge_104_92} (GLH Operativo), in quanto lo stesso riguarda il singolo alunno con certificazione di disabilità ai fini dell'integrazione scolastica. A livello di Istituto, si precisa inoltre che le riunioni del Gruppo di lavoro per l'inclusività possono tenersi anche per articolazioni funzionali ossia per gruppi convocati su tematiche specifiche».
Riteniamo che soprattutto questo chiarimento finale sia particolarmente importante, prevedendo che il GLI possa riunirsi anche per Sezioni distinte, a seconda che si tratti, ad esempio, di alunni con disabilità – quando cioè dovranno intervenire tutti i soggetti già previsti dalla normativa – oppure di alunni con DSA o altri BES, in cui non necessita la presenza né degli operatori sanitari né degli insegnanti per il sostegno.
\section*{Precisazioni sui GLIP (Gruppi di Lavoro per l'Inclusione Scolastica degli Alunni con Disabilità Provinciali)}
Infine, erano rimasti dei coni d'ombra sui rapporti tra i nuovi organismi (CTS-Centri Territoriali di Supporto, CTI-Centri Territoriali per l'Inclusione) e quelli vecchi, come i GLIP (Gruppi di Lavoro per l'Inclusione Scolastica degli Alunni con Disabilità Provinciali), che alcuni ritenevano addirittura abrogati, anche per la mancata nomina di ispettori come loro coordinatori e data anche la loro progressiva riduzione numerica.
A tal proposito il Ministero precisa che \caporali{nulla è innovato per quanto riguarda i Gruppi di lavoro interistituzionali (GLIP), i cui compiti e la cui composizione sono previsti da una norma primaria (art. 15 Legge 104/92)\footcite{Legge_104_92}. Con successiva nota – nell'ottica dell'ottimizzazione e della funzionalità delle specifiche competenze – saranno ulteriormente definiti i compiti dei CTS e dei CTI, fermo restando quanto disposto nel D.M. del 12 luglio 2011\footcite{decreto5669_2011} [Decreto Ministeriale n. 5669/11, N.d.R.] e nelle Linee guida per il diritto alla studio di alunni e studenti con DSA}.

Questa Nota Ministeriale, in conclusione, ci sembra rivestire una notevole importanza per riportare un po' di serenità nelle scuole. Essa sottolinea infatti la prevalenza delle valutazioni pedagogiche da parte dei docenti, nell'individuare casi di svantaggio e disagio, rispetto al rischio di deriva sanitaria in campi ad essa sostanzialmente estranei. Inoltre, vi si ribadisce fortemente il rispetto dell'autonomia scolastica, della quale da più parti si denunciava il soffocamento da parte di un supposto centralismo ministeriale.
Si ritiene quindi che con questi ulteriori chiarimenti la sperimentazione della normativa sui BES, prevista per il corrente anno, possa dare i suoi frutti, con vantaggio sia per la serenità degli alunni che la professionalità dei docenti\footcite{Nocera2013a}.