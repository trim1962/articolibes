\author{Giulia Giani}
\title{“Scambio di ruoli” per un'efficace inclusione scolastica}
\label{cha:giani300413}
\begin{abstract}
Il dibattito sul significato dell'inclusione scolastica oggi, sul ruolo degli insegnanti di sostegno e di quelli curricolari –- avviato nei giorni scorsi dall'opinione di un genitore, che già tanto ha fatto discutere –- si arricchisce oggi del contributo di un'insegnante di sostegno già presente in passato su queste stesse pagine, con la proposta di uno \cit{scambio di ruoli} tra docenti. Vediamo come
\end{abstract}
%\epigraph{Il dibattito sul significato dell'inclusione scolastica oggi, sul ruolo degli insegnanti di sostegno e di quelli curricolari – avviato nei giorni scorsi dall'opinione di un genitore, che già tanto ha fatto discutere – si arricchisce oggi del contributo di un'insegnante di sostegno già presente in passato su queste stesse pagine, con la proposta di uno “scambio di ruoli” tra docenti. Vediamo come}{Giulia Giani}
\maketitle
Grazie agli interventi in «Superando.it» dei genitori Giuseppe Felaco e Daniele Brogi, in materia di inclusione scolastica, noi insegnanti di sostegno ci sentiamo ancora una volta chiamati a partecipare al dibattito iniziato sulle pagine di questo giornale nell'estate del 2012, per poter offrire ulteriori spunti di riflessione al dibattito. È infatti con nostra grande soddisfazione che il tema dell'inclusione scolastica e della ricerca di una sua reale efficacia sta ora diventando oggetto di dibattito non più solo tra “addetti ai lavori”, ma anche tra insegnanti e, soprattutto, tra genitori.
Negli ultimi anni scolastici ci sono stati molti articoli di giornale – sulle maggiori testate nazionali – che hanno messo in evidenza, in modo fuorviante, come i genitori abbiano condotto battaglie legali per ottenere sempre maggiori risorse di insegnanti di sostegno. Ritengo fuorvianti tali articoli non certo perché non sia vero che i genitori si battono per il diritto all'integrazione scolastica dei loro figli, quanto perché essi spostano l'oggetto di interesse da ciò che è davvero importante: i genitori, infatti, non sono interessati “al massimo numero di ore di sostegno possibili”, sono invece interessati alla qualità dell'inclusione scolastica; molti genitori, tuttavia, hanno espresso opinioni sul tema, che ci dimostrano come il loro grado di consapevolezza della complessità del sistema scolastico sia buono e, soprattutto, noi docenti abbiamo compreso che molti di loro pensano alla necessità di attivare altre risorse, che non siano riconducibili al solo “docente di sostegno”.

Le opinioni espresse da parte dei due genitori Giuseppe Felaco\ref{cha:felaco260413} e Daniele Brogi\ref{brogi260413} [rispettivamente intitolate “Tutti avrebbero dei vantaggi” e “L'integrazione, il sostegno e gli insegnanti curricolari”, N.d.R.] sono, a mio parere, l'espressione evidente di come, per la ricerca di un fine comune, si possano perseguire strade molto diverse, ma da docente di sostegno ritengo che il dibattito debba rimanere ancora aperto, perché in entrambe le visioni esistono punti di forza, ma anche di debolezza, che forse possono essere superati grazie a nuove proposte di articolazione della vita scolastica che superino l'attuale divisione dei ruoli, senza far venir meno le risorse necessarie al raggiungimento dello scopo.

Nell'articolo di Giuseppe Felaco, ritengo che il punto di forza sia individuabile nel desiderio di mettere al centro del processo di inclusione scolastica il ruolo dei docenti curricolari. Sicuramente negli ultimi anni la delega all'insegnante di sostegno dell'alunno disabile ha distorto la funzione che tale figura avrebbe dovuto assumere negli intenti del Legislatore e gli effetti di una delega costante sono evidenti nelle nostre scuole: accanto ad alcune significative esperienze di “buone prassi”, infatti, molte di più sono le realtà in cui la presenza del docente di sostegno funge più da elemento “di difesa” e “di conservazione” del sistema scuola dal cambiamento, piuttosto che come risorsa aggiuntiva per favorire l'integrazione.
Aula scolastica con alunno in carrozzina

Sono circa 200.000 gli studenti con disabilità che frequentano oggi le scuole italiane

Altro elemento di forza di quell'articolo è a mio parere la ricerca dei “vantaggi” che tutti potrebbero avere dal mettere al centro del processo di inclusione gli insegnanti curricolari: condivido molto la prospettiva di valorizzare la competenza dei docenti disciplinari nella didattica speciale, ma ancor più ho avuto esperienza diretta di come si possano attivare risorse fondamentali nei compagni di classe e di come attraverso “l'aiuto fra pari”, gli alunni con disabilità – ma anche più genericamente gli alunni con bisogni educativi speciali – possano sentirsi maggiormente coinvolti nel lavoro della classe e accrescere la loro autostima e la ricerca di un'autonomia personale.
Dietro le parole del signor Felaco, si può individuare la posizione di quei genitori che intendono l'inclusione dal punto di vista di chi non vuole essere classificato come “alunno speciale”, cioè che necessita dell'insegnante-di-sostegno, ma che vuole avere risposte dalla scuola nell'ottica di una “speciale normalità”, ove cioè i bisogni di tutti vengano compresi e ottengano risposta da parte degli insegnanti “normali”.

Il punto di debolezza dell'approccio alla questione del signor Felaco è a mio parere nell'idealistica possibilità che il docente curricolare si possa occupare di tutto, al di là delle riflessioni sull'aumento di retribuzione, che al momento non sono oggetto del mio interesse. Da docente ritengo infatti che siamo ben lontani dal poter pensare che il solo docente (anche con l'aiuto dei compagni di classe degli alunni), possa gestire efficacemente tutto ciò, per due ragioni: l'una è legata alla formazione dei docenti, l'altra all'organizzazione didattica delle attività.
Oggi, molti insegnanti, anche i più motivati, non sono adeguatamente formati per riuscire a gestire situazioni complesse e che richiedono sugli stessi contenuti disciplinari la capacità di organizzare attività a diversi livelli. Ritengo, tuttavia, che anche nell'ipotesi ottimistica che tutti lo fossero, senza l'adeguato supporto di un altro docente, di un “adulto”, di un “esperto”, la realizzazione pratica delle attività sarebbe carente. Una sola persona non è in grado di potersi occupare di così tante differenze all'interno di una classe nella didattica ordinaria, nella quotidianità, ma ha bisogno dell'aiuto e del supporto dei colleghi.

Questo è quanto, in modo molto concreto, ha individuato l'articolo di Daniele Brogi, che ha a mio parere il merito di aver messo in luce il fatto che l'insegnante di sostegno è stato pensato all'origine come un “mediatore”, che deve lavorare con l'intera classe e che è impensabile lasciare solo il docente curricolare a gestire situazioni così complesse.
Anche questo genitore, però, verso la fine del suo articolo, dice che non bisogna «mandare al macero chi ha intrapreso una strada coscienziosamente più vicina alle ragioni del cuore che non della carriera», cadendo in tal modo in un altro luogo comune, cioè quello che gli insegnanti di sostegno siano dei “missionari”, dei “sensibili”, che antepongono le ragioni del cuore a quelle della carriera… Perché? Gli insegnanti di sostegno non sono forse dei docenti come gli altri? Non lo fanno, anche loro, per professione? Non dovrebbero aspirare anche loro a una “carriera”?
Il punto di vista di Daniele Brogi rappresenterebbe quindi la posizione di quei genitori che intendono l'inclusione scolastica come il riconoscimento di una “diversità”, di cui bisogna prendersi “cura” con risorse specificamente dedicate e formate, coinvolte affettivamente e che privilegiano la sensibilità umana alla didattica, e che quindi concepiscono l'insegnante di sostegno come risorsa necessaria per realizzare risposte individualizzate.

Vengo ora alla riflessione da docente di sostegno: se è vero che bisogna puntare a una presa in carico dei docenti curricolari, perché è l'unico modo per non emarginare gli alunni in un “mondo a parte”, con un “insegnante a parte”, è anche vero che il curricolare non può fare tutto da solo, ma che ha bisogno dell'aiuto di qualcuno. Il problema è: chi deve essere questo “qualcuno”? Siamo sicuri che debba essere l'insegnante-di-sostegno che conosciamo oggi?
Dal mio punto di vista, ciò su cui i genitori non hanno riflettuto abbastanza è legato a quale formazione pensiamo debbano avere quelli che oggi chiamiamo “docenti di sostegno”.
Il punto di grave debolezza nell'organizzazione scolastica attuale è riconducibile a come noi concepiamo gli “insegnanti specializzati”: per essere insegnanti di sostegno oggi si richiede una “specializzazione” che ha come oggetto di studio prevalentemente materie pedagogiche, ma che presuppone una formazione disciplinare precedentemente acquisita, nella quale siamo abilitati. Quando entriamo in classe, tuttavia, siamo in compresenza con i colleghi curricolari in discipline che spesso non sono oggetto del nostro percorso di studi e  per le quali non abbiamo competenze. Come dire, si privilegia negli insegnanti di sostegno la “competenza pedagogica” rispetto quella disciplinare. Il problema, però, è che senza adeguate competenze disciplinari il nostro ruolo all'interno della classe viene privato di valore, perché non possiamo offrire adeguato supporto didattico né al collega curricolare, né all'alunno con disabilità, né ai suoi compagni di classe. L'insegnante di sostegno perde così la sua funzione, oltre a perdere autorevolezza agli occhi degli altri attori in gioco e a sviluppare spesso un senso di frustrazione personale, legato al doversi occupare di contenuti per i quali non ha competenze.

Daniele Brogi, nel suo articolo, afferma che bisognerebbe applicare le leggi vigenti per realizzare una buona inclusione scolastica: personalmente non lo credo possibile, non certo perché gli intenti del Legislatore non siano “buoni”, quanto perché alla base di tutto vi è una contraddizione forte tra come formiamo i docenti di sostegno e i compiti che si trovano a dovere svolgere nelle classi. Se in decenni di tentativi di applicazione della legge non ci siamo riusciti e siamo giunti a un punto di crisi, è bene ripensare i ruoli e le funzioni di tutti i docenti e l'organizzazione delle attività didattiche; bisogna affrontare la sfida di mantenere inalterato lo spirito della legge che tutto il mondo ci invidia, ma ripensare le modalità di intervento e puntare a una formazione ad ampio raggio di tutti i docenti in materia di BES [bisogni educativi speciali, N.d.R.].

La proposta, allora – che già qualche mese fa è partita dal gruppo di docenti di sostegno autodefinitisi come “bis-abili” – diventa sempre più attuale, anche alla luce del dibattito di questi genitori.
Provo a riformularla in questo modo: bisogna organizzare la vita di una classe, in cui sono presenti alunni con bisogni educativi speciali, in modo tale che sia valorizzata l'importanza dei docenti curricolari, attraverso la “compresenza” in classe non del binomio docente di materia-docente di sostegno, ma con una compresenza effettiva di due docenti con adeguate competenze disciplinari della materia oggetto di lezione, nonché con una competenza pedagogica solida anche in materia di bisogni educativi speciali.
Gli attuali docenti-di-sostegno – che svolgono il loro servizio interamente su sostegno – dovrebbero quindi essere diversamente utilizzati valorizzando anche le loro competenze disciplinari: in una classe farebbero “da sostegno” al collega nell'area disciplinare che conoscono e per cui hanno reali competenze di mediazione, nella classe a fianco sarebbe il collega a fare “da sostegno” a loro.
L'abbiamo chiamata “cattedra mista”: non più rigida separazione dei ruoli, ma ogni docente dovrebbe svolgere alcune ore da docente disciplinare e alcune ore “di sostegno”, attività che diverrebbe in tal modo un'occasione reale e concreta di arricchimento e valorizzazione professionale e non un'attività di generica assistenza, come oggi troppo spesso è concepita.

Le posizioni dei due genitori, quindi, si possono davvero integrare tra loro se si cambia il punto di partenza: un sostegno è necessario, ma non deve essere l'insegnante-di-sostegno; l'unica risposta concretamente praticabile nella scuola di oggi e con le risorse economiche che abbiamo a disposizione è un'effettiva compresenza e contitolarità per alcune ore alla settimana, valorizzando maggiormente il ruolo attivo di tutti gli attori in gioco nel processo di inclusione scolastica.“Scambio di ruoli” per un'efficace inclusione scolastica\footcite{Giani2013a}

Insegnante di sostegno.

30 aprile 2013
Ultimo aggiornamento: 30 aprile 2013 11:0