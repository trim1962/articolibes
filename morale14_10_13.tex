\author{Giorgio Morale}
\title{Quali risorse per i BES?\xheadbreak
Intervista di Giorgio Morale a Luisa Formenti}
\label{cha:morale141013}
\maketitle

Come sono cambiate in Italia nel corso degli anni le politiche scolastiche per l'inclusione di allievi con disabilità?

Negli ultimi dieci anni non si è vista una vera e propria politica scolastica dedicata all'inclusione, ma un insieme di azioni, non propriamente coordinate sul territorio nazionale e a volte anche in piena contraddizione tra loro.

Mentre in fasi precedenti l'innovazione era accompagnata da indicazioni precise e da strumenti economici diffusi e improntati a una logica di diritto generalizzato (Legge 104\footcite{Legge_104_92}, Legge per il Diritto allo studio, Legge 285\footcite{Legge_285_1997}) che le scuole potevano accogliere attraverso una progettazione diffusamente applicata, negli ultimi anni la scuola ha vissuto una politica dei tagli e degli interventi sporadici. I fondi sono stati progressivamente ridotti e direzionati principalmente in certe aree geografiche o su certi specifici progetti, creando un'integrazione di serie A, B, C…

Solo chi era in grado di arrivare in tempo, aveva competenze di alto livello o le giuste conoscenze, poteva garantire ai propri alunni, alla propria scuola o al proprio territorio, i servizi necessari per una buona integrazione.

La carente stabilizzazione del personale di sostegno, in minima parte realmente specializzato e ogni anno collocato su nuovi alunni, ha inoltre creato un alto livello di sofferenza nelle famiglie, che si sono trovate a dover combattere personalmente per il diritto all'inclusione del proprio figlio, che dovrebbe essere garantito di per sé dalle scuole e da personale competente, in grado di accompagnare l'alunno con continuità, per un tempo certamente più lungo di un anno scolastico.

La cartina di tornasole sono i siti e le newsletters, che raccolgono ogni giorno le voci dei genitori, e che rilevano alti livelli di diversificazione dei servizi e delle risorse, all'interno del territorio Nazionale.

Da quest'anno la politica per l'inclusione si arricchisce di un nuovo acronimo: BES (Bisogni Educativi Speciali). Cosa si intende per BES?

Per rispondere a questa domanda credo sia utile citare direttamente alcuni passaggi dei documenti che il Ministero ha prodotto in quest'ultimo anno, in modo di fare chiarezza prima di tutto sull'area d'intervento che ci troviamo a considerare, individuando di conseguenza quali possano essere i bisogni e le strategie più utili:
\begin{quote}
	“L'area dello svantaggio scolastico è molto più ampia di quella riferibile esplicitamente alla presenza di deficit. In ogni classe ci sono alunni che presentano una richiesta di speciale attenzione per una varietà di ragioni: svantaggio sociale e culturale, disturbi specifici di apprendimento e/o disturbi evolutivi specifici, difficoltà derivanti dalla non conoscenza della cultura e della lingua italiana perché appartenenti a culture diverse. Nel variegato panorama delle nostre scuole la complessità delle classi diviene sempre più evidente. Quest'area dello svantaggio scolastico, che ricomprende problematiche diverse, viene indicata come area dei Bisogni Educativi Speciali (in altri paesi europei: Special Educational Needs). Vi sono comprese tre grandi sotto-categorie: quella della disabilità; quella dei disturbi evolutivi specifici e quella dello svantaggio socio-economico, linguistico, culturale”. (Decreto CTS 27 dicembre 2012\footcite{dir27Dic2012})
\end{quote}
Secondo la nota del Ministero del 27 giugno 2013\footcite{Nota_1551_2013}
\begin{quote}
“ogni alunno, con continuità o per determinati periodi, può manifestare Bisogni Educativi Speciali: o per motivi fisici, biologici, fisiologici o anche per motivi psicologici, sociali, rispetto ai quali è necessario che le scuole offrano adeguata e personalizzata risposta”.
\end{quote}

L'introduzione, a partire dal presente anno scolastico, della categoria dei BES, è da considerare un passo avanti nell'inclusione oppure è un ennesimo modo per fare cassa a tutto danno della scuola?

Il fatto che si vada finalmente a considerare la necessità di impegnarsi per un'inclusione più ampia e coerente, prendendo in considerazione non solo le disabilità, ma anche le differenti situazioni di disagio e difficoltà, è una necessità già da molti anni.

La scuola, oltre a soffrire rispetto a situazioni divenute in alcuni casi del tutto ingestibili, non ha ricevuto in questi anni sufficienti strumenti per affrontarne i diversi aspetti, che non coinvolgono solo l'area degli apprendimenti, ma anche aspetti più propriamente psicologici e in alcuni casi anche psicopatologici, considerato l'aumento dei livelli di disagio psicologico e sociale, presenti nella realtà scolastica italiana.

La rilevazione dei bisogni e l'individuazione di risposte proprie di un ambito formativo, è diventato un impegno urgente, richiedendo una seria e coerente risposta istituzionale, che con il Decreto sui BES il Ministero cerca di prendere in considerazione, anche se non è ancora in grado di prospettare linee operative precise, ma si ferma alla delineazione di una serie di principi, in cui non possiamo che riconoscerci, e sui quali il Ministero chiede ora di lavorare, presentando proposte costruttive, basate sulle esperienze positive di questi anni.
\begin{quote}
“È inoltre intenzione della scrivente procedere a una raccolta delle migliori pratiche in ordine alla definizione dei Piani in parola. A tal fine si richiede la collaborazione delle SS.LL. affinché censiscano le proposte di P.A.I. realizzate nel loro territorio e trasmettendo copia delle rilevazioni, unitamente ad una selezione delle buone pratiche” (nota Ministero 27 giugno 2013\footcite{Nota_1551_2013}).
\end{quote}
Nelle scuole regna la confusione rispetto a cosa fare. Da una parte, adottando una considerazione allargata di allievi con bisogni speciali, in molte scuole tecniche o professionali o periferiche finisce per rientrare in questa categoria la gran parte degli studenti, con la necessità di moltiplicare piani personalizzati che rischiano di restare solo sulla carta. Da un'altra parte, infatti, se è facile moltiplicare la quantità di carta prodotta, non è altrettanto facile dare alle scuole le risorse finanziarie, la formazione, gli strumenti indispensabili per un intervento efficace.

Vi sono certamente scuole che hanno già affrontato queste tematiche in termini progettuali ed altre che sono molto più spaesate, ma va considerato che in questi anni erano già arrivate sia indicazioni che finanziamenti rispetto alle diverse problematiche che ora vengono considerate congiuntamente sotto la comune voce inclusione. E' anche vero che questi provvedimenti hanno responsabilizzato in particolare chi già era maggiormente coinvolto e preparato su questi problemi, dando spazio alla scelta autonoma delle scuole, ma non sempre finanziamenti continuativi e sufficienti per tutti. Così mentre alcune realtà sono rimaste nel disorientamento più totale, altre si sono attivate, assumendosi il carico e la responsabilità della sperimentazione, ricercando in vari modi i finanziamenti necessari, facendosi aiutare anche dalle famiglie, economicamente sempre più responsabilizzate.

A questo proposito va necessariamente considerato il tema della formazione diffusa per i docenti, perché l'abolizione dell'aggiornamento obbligatorio per gli insegnanti ha rallentato notevolmente in questi anni la crescita pedagogica e didattica delle realtà scolastiche, che era necessario continuassero questo cammino, per poter affrontare coerentemente le nuove sfide che l'ingresso dei bambini di diverse nazionalità, le nuove problematiche emergenti e la diffusione delle nuove tecnologie hanno cominciato ben presto a richiedere.

Sicuramente la valorizzazione della figura sociale e professionale dei docenti è una delle chiavi del miglioramento della qualità del sistema scolastico, forse la più importante, per questo mi pare centrale la proposta di introdurre un sistema di crediti formativi e professionali a cui ancorare una vera progressione di carriera per gli insegnanti, con figure e retribuzioni diversificate, in base ai percorsi formativi personali.

Nella sua formulazione iniziale la circolare sui BES sembrava porre la conoscenza diffusa sulla tematica dei bisogni speciali come alternativa all'intervento dell'insegnante di sostegno e dello specialista. Questa è davvero un'alternativa? Le scuole possono fare a meno dell'insegnante di sostegno o dello specialista?

Domandarci questo è come chiederci se sia più necessario il pane o l'acqua, per il benessere quotidiano del nostro corpo. Non possiamo porre in alternativa una buona capacità di gestione educativa e didattica diffusa all'interno della scuola, elemento basilare per rispondere ai diritti formativi di tutti gli alunni, con la necessità di fornire personale specializzato, in grado di accompagnare la crescita di quegli alunni che necessariamente richiedono specifiche tecniche di insegnamento, interventi individualizzati e tempi particolari di attenzione e cura.

In ogni realtà lavorativa noi incontriamo personale variamente formato e specializzato, che il servizio è tenuto a fornire, per rispondere alla domanda dell'organizzazione e dei propri utenti. Ci chiediamo per quale motivo nella scuola l'organizzazione del personale debba restare ferma a un'articolazione di tipo basilare: dirigente scolastico, personale di segreteria, personale docente (di differenti discipline, oppure di sostegno, ma non sempre specializzato), personale assistenziale o educativo, collaboratori.

Nel momento in cui la scuola italiana sceglie di percorrere la strada dell'inclusività, deve pensare a quale articolazione sia necessaria per permettere a ogni alunno e a ogni classe, di avere una più completa risposta al proprio diritto formativo.

Se alla scuola viene richiesta competenza, continuità, responsabilità e capacità di integrazione, è necessario che venga garantita anche una maggiore specializzazione degli insegnanti, che devono essere in grado di progettare percorsi didattici differenziati, ma anche gestire le diverse problematiche proposte da un gruppo classe.
\begin{quote}
“Tali complessi e delicati passaggi – proprio affinché l'elaborazione del P.A.I. non si risolva in un processo compilativo, di natura meramente burocratica anziché pedagogica – richiedono un percorso partecipato e condiviso da parte di tutte le componenti della comunità educante, facilitando processi di riflessione e approfondimento, dando modo e tempo per approfondire i temi delle didattiche inclusive, della gestione della classe, dei percorsi individualizzati, nella prospettiva di un miglioramento della qualità dell'integrazione scolastica, il cui modello – è bene ricordarlo – è assunto a punto di riferimento per le politiche inclusive in Europa e non solo.” (nota Ministero 27 giugno 2013)\footcite{Nota_1551_2013}
\end{quote}

Per fare questo si rende indispensabile la costituzione di nuove figure di sistema, in grado di dedicarsi con competenza all'analisi dei bisogni e alla gestione delle risorse, non solo economiche ma anche umane, nonché all'individuazione delle strategie gestionali necessarie per una reale innovazione; la scuola ha bisogno di personale in grado di favorire, avendone il ruolo e la competenza, uno sviluppo progressivo, articolato e armonico, dei diverse proposte formative, uscendo dalla dinamica del volontariato e dell'alternanza dei ruoli\footcite{Morale2013}.

*Insegnante specializzata, Psicopedagogista e Psicomotricista