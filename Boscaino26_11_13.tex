\author{Marina Boscaino}
\title{Bisogni Educativi Speciali, così cresce la disuguaglianza. Intervista a Alain Goussot}
\label{cha:Boscaiono261113}
%\begin{abstract}
%	Infatti, una Nota Ministeriale prodotta nei giorni scorsi chiarisce ulteriormente alcuni punti controversi della recente normativa sui Bisogni Educativi Speciali (BES), fissata tra la fine del 2012 e l'inizio di quest'anno, che tanto ha già fatto discutere. Appare in particolare importante la sottolineatura data da un lato alla prevalenza delle valutazioni pedagogiche da parte dei docenti, dall'altro al rispetto dell'autonomia scolastica
%\end{abstract}
\maketitle
\datapub{26 novembre 2013}

La legge italiana prevede, con una normativa all'avanguardia rispetto a tutti gli altri sistemi scolastici europei, il sostegno per gli studenti che si trovino in situazioni di disabilità. Da qualche tempo, però, tra i vari acronimi che sono piovuti sul mondo della scuola, si è fatto strada BES: Bisogni Educativi Speciali.

Rientrerebbero tra questi gli alunni con difficoltà nel processo di apprendimento dovute a svantaggio personale, familiare e socio-ambientale ai quali si assegnerebbe il diritto alla personalizzazione dell'apprendimento attraverso la redazione di un Piano Didattico Personalizzato da parte dei docenti. I BES possono presentarsi con continuità oppure per periodi circoscritti della vita dell'alunno, in quanto le cause che li generano possono anche avere origine fisica, biologica, fisiologica, psicologica o sociale.
\begin{quote}
“Gli alunni con Bisogni Educativi Speciali vivono una situazione particolare, che li ostacola nell'apprendimento e nello sviluppo: questa situazione negativa può essere a livello organico, biologico, oppure familiare, sociale, ambientale, contestuale o in combinazioni di queste. \mancatesto

Queste difficoltà possono essere globali e pervasive (si pensi all'autismo) oppure più specifiche (ad esempio nella dislessia), settoriali (disturbi del linguaggio, disturbi psicologici d'ansia, ad esempio); gravi o leggere, permanenti o (speriamo) transitorie.

In questi casi i normali bisogni educativi che tutti gli alunni hanno (bisogno di sviluppare competenze, bisogno di appartenenza, di identità, di valorizzazione, di accettazione, solo per citarne alcuni) si «arricchiscono» di qualcosa di particolare, di «speciale». Il loro bisogno normale di sviluppare competenze di autonomia, ad esempio, è complicato dal fatto che possono esserci deficit motori, cognitivi, oppure difficoltà familiari nel vivere positivamente l'autonomia e la crescita, e così via. In questo senso il Bisogno Educativo diventa «Speciale». Per lavorarci adeguatamente avremo dunque bisogno di competenze e risorse «speciali», migliori, più efficaci. \mancatesto“. (da Dario Ianes, “I Bisogni Educativi Speciali”).
\end{quote}

Per molti i BES rappresentano l'ultimo elemento di forzatura e di confusione professionale originata dal Miur: il paradosso è che si agisce in nome dell'inclusione. Normati da una direttiva\footcite{dir27Dic2012}, lo scorso dicembre, e poi da una circolare in marzo\footcite{cm8_2013}, l'accoglimento di quanto previsto da parte dei docenti non costituisce un obbligo, come ribadito in un convegno a Rimini qualche settimana fa dal dott. Ciambrone, dirigente ministeriale responsabile per l'integrazione scolastica, che li ha semmai enfaticamente definiti una “splendida opportunità per la scuola”. Al di là degli entusiasmi di Viale Traterevere, i problemi sono tanti, come peraltro si evince dalla definizione offerta dal prof. Ianes che cita una casistica di potenziali BES veramente eterogenea e sconfinata.

Il mondo della scuola ha espresso valutazioni piuttosto divergenti rispetto a questa problematica (all'interno della quale andrebbero a ricadere anche gli studenti con DSA, Disturbo Specifico dell'Apprendimento, come dislessia, discalculia, disgrafia, già previsti dalla l. 170/2010\footcite{legge170}): ci sono coloro che ritengono che sia necessario un intervento “speciale”, forse dimenticando che nei consigli di classe da sempre chi sa svolgere correttamente il proprio lavoro ha tenuto in conto le condizioni personali particolari degli studenti; coloro che eseguono pedissequamente ciò che uffici periferici del MIUR e dirigenti scolastici zelanti fanno piovere loro addosso (modulistica, piani personalizzati, colloqui con le famiglie); e coloro che sollevano problemi di merito e di metodo. Il dibattito è stato talmente vibrante che il Miur è stato costretto, il 23 novembre\footcite{Nota_2563_2013}, a pubblicare una nota del capo dipartimento Chiappetta. C'è chi ritiene che si tratti di una “scorciatoia” per inserire nel novero dei BES studenti che necessiterebbero del sostegno, con il conseguente taglio di posti di lavoro e, soprattutto, di diritto all'inclusione reale. C'è chi lamenta l'inserimento, all'interno della cosiddetta “funzione docente”, di competenze e carichi di lavoro che gli insegnanti non possono/devono/vogliono sopportare. Perché è chiaro che il riconoscimento dello studente bisognoso di azioni educative speciali e le conseguenti azioni sarebbero completamente a carico dei docenti. C'è chi contesta, infine, la costante tendenza alla “medicalizzazione” della divergenza rispetto a profili di “normalità” che le azioni del Miur stanno configurando da molto tempo.

Per provare a indicare come tutta la questione possa essere pensata in modo diverso dalla prospettiva (e dalla confusione di prospettive) individuate dal Miur, ho lasciato la parola ad Alain Goussot , ricercatore di Didattica e Pedagogia Speciale all'Università di Bologna , che fa parte del comitato scientifico dell'associazione “Giù le mani dai bambini”, della società Italiana di Pedagogia Speciale (SIPES) ed è membro onorario della Associazione dei Pedagogisti Italiani.
\section*{Qual è la sua opinione sui cosiddetti BES?}
Siamo di fronte una nuova categorizzazione della popolazione scolastica, che comprende alunni disabili certificati e alunni con disturbi specifici dell'apprendimento, difficoltà di apprendimento, difficoltà linguistico-culturali, disturbi del comportamento, disagio sociale e funzione intellettiva limite. Così si sposta pericolosamente una varietà ampia di alunni nella sfera dell’”anormalità”, della “devianza” in nome, paradossalmente, dell'inclusione. I bisogni educativi sono universali: ogni alunno – compreso il diversabile – ha proprie caratteristiche e particolarità. ma anche diritto all'eguaglianza delle opportunità: ambiente, contesto-scuola, insegnanti devono creare le condizioni per favorire l'espressione delle differenze e adattare gli approcci per rispondere alla pluralità di linguaggi e modi di essere degli alunni. Non a caso per capire l'Emilio di Rousseau occorre leggere anche il Contratto sociale, dove libertà, eguaglianza e responsabilità sociale sono strettamente collegate. Bisogni e diritti vanno insieme e la scuola è un passaggio fondamentale nella formazione dei cittadini; è anche quello che ci ha insegnato Dewey in “Democrazia e educazione”.
\section*{Quali sono le posizioni di massima del mondo accademico e della comunità scientifica sui Bes?}
Il mondo accademico è diviso: c'è chi sostiene la necessità di adeguare strumenti e metodi di analisi e classificazione per progettare a scuola e vede come un passo avanti l'attenzione verso i “Bisogni educativi speciali”; c'è invece chi, per ragioni di ordine pedagogico, scientifico e culturale scorge un rischio di medicalizzazione della sfera scolastica; c'è infine chi vede questa proposta come una nuova forma di stigmatizzazione sofisticata, che finirà per accentuare le diseguaglianze. A livello internazionale molti, come il gruppo dei Disability studies, ma anche diversi studiosi e ricercatori dell'area francofona, criticano la categorizzazione in sé. Va anche detto che quest'ultima ha un imprinting di natura psicologica e clinica: la pedagogia sembra a tenuta fuori dalla porta, forse anche per colpa degli stessi pedagogisti.
\section*{Diverse tipologie di individui vengono in qualche modo isolati dalla cosiddetta presunta “normalità”, in nome dell'inclusione. Qual è la ratio?}
Sono ormai due decenni che il mondo della scuola e dell'educazione è colonizzato dallo sguardo clinico-terapeutico. Questo approccio osserva per definire e classificare facendo leva sui sintomi; quello pedagogico, invece, osserva per comprendere facendo leva sulle potenzialità e le capacità. L'alunno con difficoltà di apprendimento non è più considerato come soggetto significante di una condizione sociale, culturale e familiare, ma come un soggetto portatore di problemi e come destinatario di interventi ‘curativi’ che lo devono riportare alla normalità. Nei documenti del ministero l'alunno non è mai visto come soggetto protagonista e attore/autore del proprio percorso. Da un riconoscimento delle differenze che si basa sul principio di eguaglianza, si passa ad una logica differenzialistica, che stigmatizza in modo sofisticato e accentua le diseguaglianze. Si sposta l'accento dall'interazione tra soggetto e ambiente sociale al singolo, visto come organo malato o disfunzionale da curare e riparare. La didattica viva viene trasformata in pura procedura tecnica e si fa dell'insegnante un consumatore di ricette standardizzate, da applicare in tutte le situazioni, prodotte dal business editoriale. Si perde di vista che l'insegnamento/apprendimento è anzitutto relazione, un processo complesso che fa dello spazio classe un laboratorio interattivo permanente. Si perde anche di vista che la stessa pedagogia e didattica speciale è per tutti: quello che viene inventato e sperimentato nell'esperienza con alunni disabili può funzionare con alunni senza disabilità. La dimensione pedagogica del lavoro dell'insegnante sembra invece venuta meno: nelle formazioni proposte ultimamente agli insegnanti sui DSA sembrava che si dovessero formare degli operatori della diagnosi o della neuropsichiatria e non degli operatori pedagogici. Il concetto centrale dovrebbe essere garantire l'accessibilità e non pretendere l'adattamento. E invece la stessa legge del 2010 sui DSA è un segnale molto chiaro: i disturbi specifici dell'apprendimento, che esistono, sono soprattutto visti dal punto di vista clinico; si arriva rischiare di identificare difficoltà e disturbi, con quello che questa logica comporta sul piano della predeterminazione del percorso di diversi alunni. L'ultima direttiva è in qualche modo l'epilogo di questa logica clinico-terapeutica e differenzialistica. Le ultime ricerche serie sulla scuola italiana hanno dimostrato che essa continua, contrariamente a quello che si pensa, a essere selettiva sul piano sociale e quindi ad accentuare le diseguaglianze. Si va sempre più nettamente verso una scuola a due velocità: quella per l'élite che ha i soldi nei ‘quartieri alti’ e quella per i figli del nuovo proletariato nelle periferie della società. Con la direttiva sui BES vi è anche il rischio molto concreto di dare un avvallo pseudo-scientifico ad un processo preoccupante in atto in molte scuole: le aule di sostegno che diventano sempre di più classi ghetto, le sezioni di serie A e di serie B negli istituti scolastici, le scuole ‘bene’ e quelle degradate, perché collocate in territori sociali e quartieri periferici. La logica burocratica-tecnocratica, che cala dall'alto delle proposte pasticciate e anche spesso inapplicabili, tende poi a considerare gli insegnanti come degli incompetenti, destinatari d'interventi ‘esperti’ e non degli attori delle trasformazioni. Sappiamo tutti che esistono tante criticità che vanno affrontate, che vi sono anche molti insegnanti poco preparati sul piano pedagogico e altri che dovrebbero cambiare mestiere. Ma esiste una grande massa d'insegnanti che lotta ogni giorno, che fa bene il proprio lavoro, che s'impegna spesso in modo disinteressato e con il senso della propria responsabilità nei confronti delle future generazioni.
\section*{Qual è l'apporto che la comunità scientifica intende concretamente dare a un mondo della scuola sfiancato da continui interventi normativi o pseudo-normativi, che affianca a quello del docente profili professionali incongrui e impropri?}
Compito della comunità scientifica è intanto diffondere tra gli insegnanti e gli operatori dell'educazione i risultati della ricerca e anche il confronto tra i diversi orientamenti. Penso che sarebbe quindi utile fornire una formazione plurale e completa agli insegnanti: è la base per fare delle scelte consapevoli e non farsi ‘colonizzare’ dall'ultima moda, spacciata come unica verità ’scientifica’. Penso anche che le società di pedagogia dovrebbero fare un lavoro di recupero del patrimonio pedagogico ricco e vario del passato, metterlo a disposizione del mondo della scuola: sono i fondamentali della funzione docente, sono alla base dell'identità culturale della professionalità dell'insegnante. Riappropriarsi della centralità della pedagogia e della didattica viva e mostrare che è altrettanto scientifica della psicologia clinica mi sembra un modo anche per ridare dignità agli insegnanti e far sì che non vivano un enorme complesso d'inferiorità nel rapporto con altre figure professionali. Inoltre la comunità scientifica e i diversi ricercatori nell'ambito pedagogico e psicopedagogico devono accompagnare il mondo della scuola e gli insegnanti in un lavoro di elaborazione delle proprie esperienze. Per esempio, una grande ricerca -azione partecipata che coinvolga direttamente la scuola e gli insegnanti, ma anche gli alunni e i genitori, sui temi della gestione dei gruppi classe, degli apprendimenti, della valutazione non solo delle performance, ma anche dei processi d'insegnamento/apprendimento, sulle pratiche didattiche e i progetti pedagogici nelle scuole. Una ricerca accompagnata a livello locale dalle diverse strutture universitarie territoriali, che possa durare un anno e più, per arrivare ad una grande conferenza nazionale, con delle proposte per elaborare linee guide e un nuovo progetto per la scuola del futuro. Non come oggi, provvedimenti calati dall'alto –- e spesso vissuti male dalla scuola --, ma un processo partecipativo, che parta dal basso e valorizzi esperienze, competenze e riflessioni propositive\footcite{Boscaino2013}.
