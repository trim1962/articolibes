\chapter{Esempi}
\label{cha:Esempi}
\minitoc
\mtcskip                                % put some skip here
\minilof                                % a minilof
\mtcskip                                % put some skip here
\minilot
\opt{prima}{
\section{Frazioni numeriche}
\label{sec:Frazioninumeriche}
\begin{table}[H]
	\caption{Trovare la somma di due frazioni con denominatore diverso}
	\label{tab:Trovaresommaduefrazionidenominatorediverso1}
\begin{enumerate}
	\item Prerequisiti 
\begin{itemize}
	\item frazioni
	\item mcm
	 \item precedenza operazioni
 \item somma fra frazioni
\end{itemize}
  \item Scopo: Determinare la somma fra due frazioni irriducibili
  \item Testo: Dato $\dfrac{7}{3}+\dfrac{5}{4}$  determinarne la somma.
  \item Svolgimento: 
  \begin{enumerate}
  \item calcolo il $mcm$ fra $3$ e $4$ $mcm(3,4)=12$
	\item applico la formula per la somma di due frazioni con denominatore diverso\nobs\vref{tab:prodottifrazioni} \[\dfrac{(12\div 3)\cdot 7+(12\div 4)\cdot 5}{12}\]
	\item Per la regola delle precedenze\nobs\vref{Tab:precedenze} devo fare prima la divisone $12\div 3$ e poi la divisione $12\div 4$
	\item la somma diviene \[\dfrac{4\cdot 7+3\cdot 5}{12}\]
	\item Per la regola delle precedenze\nobs\vref{Tab:precedenze} prima devo fare la moltiplicazione $4\cdot 7$ e poi la moltiplicazione $3\cdot 5$
	\item la somma diviene \[\dfrac{28+15}{12}=\dfrac{43}{12}\]
\end{enumerate}
  \end{enumerate}
\end{table}
%2
\begin{table}[H]
	\caption{Trovare la somma di due frazioni con denominatore diverso}
	\label{tab:Trovaresommaduefrazionidenominatorediverso2}
\begin{enumerate}
	\item Prerequisiti 
\begin{itemize}
	\item frazioni
	\item mcm
	 \item precedenza operazioni
 \item somma fra frazioni
\end{itemize}
  \item Scopo: Determinare la somma fra due frazioni irriducibili
  \item Testo: Dato $\dfrac{3}{5}+\dfrac{7}{8}$  determinarne la somma.
  \item Svolgimento: 
  \begin{enumerate}
  \item calcolo il $mcm$ fra $5$ e $8$ $mcm(5,8)=40$
	\item applico la formula per la somma di due frazioni con denominatore diverso\nobs\vref{tab:prodottifrazioni} \[\dfrac{(40\div 5)\cdot \cdots+(\cdots\div 8)\cdot 5}{12}\]
	\item Per la regola delle precedenze\nobs\vref{Tab:precedenze} devo fare prima la divisone $40\div 5$ e poi la divisione $\cdots\div 8$
	\item la somma diviene \[\dfrac{8\cdot 3+5\cdot \cdots}{40}\]
	\item Per la regola delle precedenze\nobs\vref{Tab:precedenze} prima devo fare la moltiplicazione $8\cdot 3$ e poi la moltiplicazione $5\cdot\cdots$
	\item la somma diviene \[\dfrac{24+\cdots}{40}=\dfrac{59}{40}\]
 \end{enumerate}
  \end{enumerate}
\end{table}
\section{Polinomi}
\label{sec:Polinomi}
\begin{equation}
\underbrace{\underbrace{[\overbrace{(a-2b)(a-2b+c)}^{\text{1}}+\overbrace{(3a-2b)(a+b)}^{\text{2}}]}_{\text{3}}(a+2b)}_{\text{4}}
\label{es:esercizio1}
\end{equation}
Per risolvere l'esercizio\nobs\vref{es:esercizio1} inizio a controllare la tavola delle precedenze\nobs\vref{tab:precedenzaparentesi}. Vi sono cinque parentesi tonde. All'interno di esse non vi sono operazioni da eseguire passo alla parentesi quadre. 
Osservo che fra le parentesi quadre vi sono due moltiplicazioni $(1)$ e $(2)$ e una somma $(3)$ mentre fuori vi è la moltiplicazione $(4)$.
Controllo la tabella delle precedenze per le operazioni\nobs\vref{tab:precedenzaoperazioni} devo fare prima le moltiplicazioni  $(1)$ e $(2)$ poi la somma $(3)$.
Inizio con $(1)$ 
\renewcommand\arraystretch{2}
\[
\begin{tabular}{C|C|C|C}
%
\bm{a}&+a^2&-2ab&+ac\\
\hline
\bm{-2b}&-2ab&+4b^2&-2bc\\
\hline
&\bm{a}&\bm{-2b}&\bm{c}\\
%
\end{tabular}
=a^2\overline{-2ab}+ac\overline{-2ab}+4b^2-bc=
\]
sommo i monomi simili e ottengo la soluzione di $(1)$ 
\[=a^2-3ab+ac+2b^2-bc\]
Continuo con $(2)$  
\[
\begin{tabular}{C|C|C}
%
\bm{3a}&+3a^2&3ab\\
\hline
\bm{-2b}&-2ab&-2b^2\\
\hline
&\bm{a}&\bm{b}\\
%
\end{tabular}
=
3a^2+\overline{3ab}-\overline{2ab}-2b^2=
\]
sommo i monomi simili e ottengo la soluzione di $(2)$ 
 \[=3a^2+ab-2b^2\]
Per cui $(3)$ diviene
\[ [\underbrace{a^2-3ab+ac+\cancel{2b^2}-bc}_{\text{1}}+\underbrace{3a^2+ab\cancel{-2b^2}}_{\text{2}}]\]
\[ [\underbrace{4a^2-2ab+ac-bc}_{\text{3}}]\]
Resta da fare la moltiplicazione $(4)$ 
\[\underbrace{[4a^2-2ab+ac-bc](a+2b)}_{\text{4}}\]
\[
\begin{tabular}{C|C|C|C|C}
%
\bm{a}&4a^3&-2a^2b&a^2c&-abc\\
\hline
\bm{2b}&8a^2b&-4ab^2&2abc&-2b^2c\\
\hline
&\bm{4a^2}&\bm{-2ab}&\bm{ac}&\bm{-bc}\\
%
\end{tabular}
=4a^3-2a^2b+a^2c-abc+8a^2b-4ab^2+2abc-2b^2c=
\]
Sommando i monomi simili
\[=4a^3+a^2c+6a^2b-4ab^2+abc-2b^2c\]
\renewcommand\arraystretch{1}
} %FINE PRIMA
\opt{secondo}{
	\section{Circuiti e reti}
	\label{sec:CircuitieReti}
	\begin{table}[H]
		\caption{In un circuito con due resistenze $R_1$ e $R_2$ in parallelo, trovare la formula che da $R_2$ note $R_1$ e $R_{eq}$}
		\label{tab:Trovarediffangoli}
		\begin{enumerate}
			\item Prerequisiti 
			\begin{itemize}
				\item mcm
				\item Equazioni di primo grado
				\item Resistenze in parallelo
				\item $\dfrac{1}{R_{eq}}=\dfrac{1}{R_1}+\dfrac{1}{R_2}+\cdots+\dfrac{1}{R_n}$			
			\end{itemize}
			\item Scopo: Determinare una resistenza note l'altra e la resistenza equivalente in un circuito in parallelo
			\item Testo: Determinare $R_1$ noti $R_2$ e $R_{eq}$
			\item Svolgimento: Si usa la formula che da la resistenza equivalente in parallelo.
			\begin{enumerate}
				\item $\dfrac{1}{R_{eq}}=\dfrac{1}{R_1}+\dfrac{1}{R_2}+\cdots+\dfrac{1}{R_n}$
				\item $\dfrac{1}{R_{eq}}=\dfrac{1}{R_1}+\dfrac{1}{R_2}$
				\item trovo mcm fra $R_1$, $R_2$ e $R_{eq}$
				\item $\dfrac{R_1\cdot R_2=R_{eq}\cdot (R_1+R_2)}{R_1\cdot R_2\cdot R_{eq}}$
				\item $R_1\cdot R_2=R_{eq}\cdot (R_1+R_2)$
				\item $R_1\cdot R_2=R_{eq}\cdot R_1+R_{eq}\cdot R_2$
				\item $R_1\cdot R_2-R_{eq}\cdot R_1=R_{eq}\cdot R_2$
				\item $R_1\cdot (R_2-R_{eq})=R_{eq}\cdot R_2$
				\item $R_1=\dfrac{R_{eq}\cdot R_2}{(R_2-R_{eq})}$
			\end{enumerate}
		\end{enumerate}
	\end{table}
} % % % % % % % % % % % %FINE SECONDO
\opt{terzo}{
\section{Goniometria}
\label{sec:EsempiGoniometria}
\begin{table}[H]
	\caption{Trovare seno coseno tangente cotangente noti seno o coseno.}
	\label{tab:trovaresencosnoti}
\begin{enumerate}
	\item Prerequisiti 
\begin{itemize}
	\item I radicali
	\item Circonferenza goniometrica\index{Circonferenza goniometrica}
	\item Seno\index{Funzione!Seno}, Coseno\index{Funzione!Coseno}, Tangente\index{Funzione!Tangente}, Cotangente\index{Funzione!Cotangente}
  \begin {align*}
	\cos\alpha=\pm\sqrt{1-\sen^2\alpha}\\
	\sen\alpha=\pm\sqrt{1-\cos^2\alpha}\\
	\tg\alpha=\dfrac{\sen\alpha}{\cos\alpha}\\
	\cotg\alpha=\dfrac{1}{\tg\alpha}
	\end{align*}
\end{itemize}
  \item Scopo: Determinare le funzioni goniometriche dato il valore del seno o il coseno di un angolo.
  \item Testo: Dato $\sen\alpha=\dfrac{3}{5}$ con l'angolo $\alpha$ tale che $\ang{90}<\alpha<\ang{180}$ determinare: coseno, tangente e cotangente di $\alpha$
  \item Svolgimento: Si inizia con il determinare il coseno di un angolo successivamente la tangente e per finire la cotangente, quantità che possiamo conoscere noti seno e coseno. 
  \begin{enumerate}
	\item coseno: Dato che il coseno di un angolo è un numero relativo bisogna definire un segno ed un modulo
	\begin{enumerate}
	\item segno: Per valori dell'angolo  $\ang{90}<\alpha<\ang{180}$, secondo quadrante, il coseno è negativo. 
	\item modulo: $\cos\alpha=-\sqrt{1-\sen^2\alpha}=-\sqrt{1-\left(\dfrac{3}{5}\right)^2}=-\sqrt{1-\dfrac{9}{25}}=-\sqrt{\dfrac{25-9}{25}}=-\sqrt{\dfrac{16}{25}}=-\dfrac{4}{5}$
\end{enumerate}
	\item tangente:$\tg\alpha=\dfrac{\sen\alpha}{\cos\alpha}=\dfrac{\dfrac{3}{5}}{-\dfrac{4}{5}}=\dfrac{3}{5}\cdot\left(-\dfrac{5}{4} \right)=-\dfrac{3}{4}$
  \item cotangente:$\cotg\alpha=\dfrac{1}{\tg\alpha}=\dfrac{1}{-\dfrac{3}{4}}=-\dfrac{4}{3}$
\end{enumerate}
 \end{enumerate}
\end{table}
%\clearpage
\begin{table}[H]
	\caption{Trovare seno coseno nota la tangente}
	\label{tab:sommadifangoli}
	\begin{enumerate}
		\item Prerequisiti 
		\begin{itemize}
			\item I radicali
			\item Circonferenza goniometrica
			\item Seno, Coseno, Tangente, Cotangente
			\begin {align*}
			\cos\alpha=\pm\dfrac{1}{\sqrt{1+\tg^2\alpha}}\\
			\sen\alpha=\pm\dfrac{\tg\alpha}{\sqrt{1+\tg^2\alpha}}\\
			\sen\alpha=\cos\alpha\cdot\tg\alpha\\
			\tg\alpha=\dfrac{1}{\cotg\alpha}
		\end{align*}
	\end{itemize}
	\item Scopo: Determinare le funzioni goniometriche dato il valore della tangente di un angolo.
	\item Testo: Dato $\cotg\alpha=\dfrac{2}{5}$ con l'angolo $\alpha$ tale che $\ang{180}<\alpha<\ang{270}$ determinare: coseno e seno di $\alpha$
	\item Svolgimento: Nell'esercizio non è nota la tangente dell'angolo quindi inizio a trovare la tangente dell'angolo, poi si passa al coseno infine al seno di un angolo.
	\begin{enumerate}
		\item tangente: $\tg\alpha=\dfrac{1}{\cotg\alpha}=\dfrac{1}{\dfrac{2}{5}}=\dfrac{5}{2}$
		\item coseno: Dato che il coseno di un angolo è un numero relativo bisogna definire un segno ed un modulo.
		\begin{enumerate}
			\item segno: Per valori dell'angolo  $\ang{180}<\alpha<\ang{270}$, terzo quadrante, il coseno è negativo. 
			\item modulo:
			\begin{align*}
			\cos\alpha&=-\dfrac{1}{\sqrt{1+\tg^2\alpha}}
			=-\dfrac{1}{\sqrt{1+\left(\dfrac{5}{2}\right)^2}}=-\dfrac{1}{\sqrt{1+\dfrac{25}{4}}}
			=-\dfrac{1}{\sqrt{\dfrac{4+25}{4}}}=-\dfrac{1}{\sqrt{\dfrac{29}{4}}}\\
			&=-\dfrac{1}{\dfrac{\sqrt{29}}{\sqrt{4}}}=-\dfrac{1}{\dfrac{\sqrt{29}}{2}}=-\dfrac{2}{\sqrt{29}}=-\dfrac{2\sqrt{29}}{29}
			\end{align*}
		\end{enumerate}
		\item seno: per determinare il valore del seno ho due vie: utilizzare il valore del coseno appena determinato o calcolarlo direttamente 
		\begin{enumerate}
			\item primo caso: $\sen\alpha=\cos\alpha\cdot\tg\alpha=-\dfrac{2\sqrt{29}}{29}\cdot\dfrac{5}{2}=-\dfrac{5\sqrt{29}}{29}$
			\item secondo caso: Dato che il seno è un numero relativo bisogna determinarne segno e modulo
			\begin{enumerate}
				\item segno:  Per valori dell'angolo  $\ang{180}<\alpha<\ang{270}$, terzo quadrante, il seno è negativo.
				\item modulo:
				\begin{align*}
				\sen\alpha&=-\dfrac{\tg\alpha}{\sqrt{1+\tg^2\alpha}}
				=\dfrac{\dfrac{5}{2}}{\sqrt{1+\left(\dfrac{5}{2}\right)^2}}
				=-\dfrac{\dfrac{5}{2}}{\sqrt{1+\dfrac{25}{4}}}
				=-\dfrac{\dfrac{5}{2}}{\sqrt{\dfrac{4+25}{4}}}
				=-\dfrac{\dfrac{5}{2}}{\sqrt{\dfrac{29}{4}}}
				=-\dfrac{\dfrac{5}{2}}{\dfrac{\sqrt{29}}{\sqrt{4}}}\\
				&=-\dfrac{\dfrac{5}{2}}{\dfrac{\sqrt{29}}{2}}
				=-\dfrac{5}{2}\dfrac{2}{\sqrt{29}}
				=-\dfrac{5\sqrt{29}}{29}
				\end{align*} %$\sen\alpha=-\dfrac{\tg\alpha}{\sqrt{1+\tg^2\alpha}}=\dfrac{\dfrac{5}{2}}{\sqrt{1+\left(\dfrac{5}{2}\right\}}}=-\dfrac{\dfrac{5}{2}}{\sqrt{1+\dfrac{25}{4}}}=-\dfrac{\dfrac{5}{2}}{\sqrt{\dfrac{4+25}{4}}}=-\dfrac{\dfrac{5}{2}}{\sqrt{\dfrac{29}{4}}}=-\dfrac{\dfrac{5}{2}}{\dfrac{\sqrt{29}}{\sqrt{4}}}=-\dfrac{\dfrac{5}{2}}{\dfrac{\sqrt{29}}{2}}=-\dfrac{5}{2}\dfrac{2}{\sqrt{29}}=-\dfrac{5\sqrt{29}}{29}$
			\end{enumerate}
		\end{enumerate}
	\end{enumerate}
\end{enumerate}
\end{table}
\begin{table}[H]
	\caption{Trovare il seno e il coseno di un angolo come somma di angoli}
	\label{tab:Trovaresenocosenonotatangenteangolo}
	\begin{enumerate}
		\item Prerequisiti 
		\begin{itemize}
			\item I radicali
			\item Circonferenza goniometrica
			\item Seno, Coseno, Tangente, Cotangente
			\item Tabella angoli noti
			\begin {align*}
%			\cos(\alpha-\beta)=	{}&\cos\alpha\cos\beta+\sen\alpha\sen\beta \\ 
			\cos(\alpha+\beta)=	{}&\cos\alpha\cos\beta-\sen\alpha\sen\beta \\ 
%			\sen\left(\alpha-\beta\right)={} &\sen\alpha\cos\beta-\cos\alpha\sen\beta\\
			\sen\left(\alpha+\beta\right)={}&\sen\alpha\cos\beta+\cos\alpha\sen\beta
		\end{align*}
	\end{itemize}
	\item Scopo: Determinare le funzioni goniometriche di una angolo somma di angoli
	\item Testo: Determinare il valore di $\cos(\ang{75})$ e di $\sen(\ang{75})$
	\item Svolgimento: Si usano le formule di somma  del seno e del coseno.
	\begin{enumerate}
		\item $\ang{70}=\ang{45}+\ang{30}$
		\item coseno:
		\begin{enumerate}
			\item $\cos(\alpha+\beta)=\cos\alpha\cos\beta-\sen\alpha\sen\beta$
			\item $\cos(\ang{45}+\ang{30})=\cos\ang{45}\cos\ang{30}-\sen\ang{45}\sen\ang{30}$
			\item $=\dfrac{\sqrt{2}}{2}\dfrac{\sqrt{3}}{2}-\dfrac{\sqrt{2}}{2}\dfrac{1}{2}$
			\item $=\dfrac{\sqrt{6}}{4}-\dfrac{\sqrt{2}}{4}=\dfrac{\sqrt{6}-\sqrt{2}}{4}$
		\end{enumerate}
		\item seno:
		\begin{enumerate}
			\item $\sen\left(\alpha+\beta\right)=\sen\alpha\cos\beta+\cos\alpha\sen\beta$
			\item $\sen\left(\ang{45}+\ang{30}\right)=\sen\ang{45} \cos\ang{30}+\cos\ang{45}\sen\ang{30} $
			\item $\dfrac{\sqrt{2}}{2}\dfrac{\sqrt{3}}{2}+\dfrac{\sqrt{2}}{2}$
			\item $=\dfrac{\sqrt{6}}{4}+\dfrac{\sqrt{2}}{4}=\dfrac{\sqrt{6}+\sqrt{2}}{4}$
		\end{enumerate}
	 \end{enumerate}
\end{enumerate}
\end{table}
\begin{table}[H]
	\caption{Trovare il seno e il coseno di un angolo come differenza di angoli}
	\label{tab:Trovarediffangoli1}
	\begin{enumerate}
		\item Prerequisiti 
		\begin{itemize}
			\item I radicali
			\item Circonferenza goniometrica
			\item Seno, Coseno, Tangente, Cotangente
			\item Tabella angoli noti
			\begin {align*}
			\cos(\alpha-\beta)=	{}&\cos\alpha\cos\beta+\sen\alpha\sen\beta \\ 
			%\cos(\alpha+\beta)=	{}&\cos\alpha\cos\beta-\sen\alpha\sen\beta \\ 
			\sen\left(\alpha-\beta\right)={} &\sen\alpha\cos\beta-\cos\alpha\sen\beta\\
			%\sen\left(\alpha+\beta\right)={}&\sen\alpha\cos\beta+\cos\alpha\sen\beta
		\end{align*}
	\end{itemize}
	\item Scopo: Determinare le funzioni goniometriche di una angolo somma di angoli
	\item Testo: Determinare il valore di $\cos(\ang{15})$ e di $\sen(\ang{15})$
	\item Svolgimento: Si usano le formule di somma  del seno e del coseno.
	\begin{enumerate}
		\item $\ang{10}=\ang{45}-\ang{30}$
		\item coseno:
		\begin{enumerate}
			\item $\cos(\alpha-\beta)=\cos\alpha\cos\beta+\sen\alpha\sen\beta$
			\item $\cos(\ang{45}-\ang{30})=\cos\ang{45}\cos\ang{30}+\sen\ang{45}\sen\ang{30}$
			\item $=\dfrac{\sqrt{2}}{2}\dfrac{\sqrt{3}}{2}+\dfrac{\sqrt{2}}{2}\dfrac{1}{2}$
			\item $=\dfrac{\sqrt{6}}{4}+\dfrac{\sqrt{2}}{4}=\dfrac{\sqrt{6}+\sqrt{2}}{4}$
		\end{enumerate}
		\item seno:
		\begin{enumerate}
			\item $\sen\left(\alpha-\beta\right)=\sen\alpha\cos\beta-\cos\alpha\sen\beta$
			\item $\sen\left(\ang{45}-\ang{30}\right)=\sen\ang{45} \cos\ang{30}-\cos\ang{45}\sen\ang{30} $
			\item $\dfrac{\sqrt{2}}{2}\dfrac{\sqrt{3}}{2}-\dfrac{\sqrt{2}}{2}$
			\item $=\dfrac{\sqrt{6}}{4}-\dfrac{\sqrt{2}}{4}=\dfrac{\sqrt{6}-\sqrt{2}}{4}$
		\end{enumerate}
	\end{enumerate}
\end{enumerate}
\end{table}
} %FINE TERZO
