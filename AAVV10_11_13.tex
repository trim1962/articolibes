\author{AA.VV}
\title{Mozione finale approvata al nono Convegno Internazionale\xheadbreak La Qualità dell'integrazione scolastica e sociale}
\phantomsection
\label{cha:aavv101113}
%\begin{abstract}
%	abstract
%\end{abstract}
\maketitle
\datapub{10 novembre 2013}
\section*{NONOSTANTE TUTTO\dots}
I 3.000 insegnanti, dirigenti scolastici, educatori, psicologi, logopedisti, operatori socio sanitari e associazioni presenti al 9 Convegno Internazionale “La Qualità dell'integrazione scolastica e sociale”, per quanto sembri strano, non sono affatto travolti da quella
“crisi” che sembra provocare smarrimento di senso. Non siamo in crisi.
Nonostante tutto, continuiamo a essere convinti che in questi momenti serva il pensiero
e non il lamento. Nonostante tutto, infatti, crediamo che abbiano ancora senso inconfutabile l'idea e la pratica che solo una scuola capace di accogliere tutti e di pensare al
futuro per il loro ingresso nel mondo degli adulti, abbia titolo a chiamarsi scuola. Questo è ancora una volta necessario, ma soprattutto possibile, nonostante dieci anni di
smantellamento dei valori fondativi della scuola per tutti e per ciascuno.
Nonostante la svalutazione sociale del bene educativo, nonostante la povertà economica in cui si sono ridotte le nostre scuole, nonostante il mito della selezione e della
competizione, nonostante lo smantellamento del sistema pubblico dei servizi socio-
sanitari.
Per questo è “normale” per noi ricordarci e ricordare a tutti che i nostri alunni disabili
sono “una parte della scuola e non una scuola a parte”.
Ogni volta che ci incontriamo abbiamo anche l'occasione di mettere a fuoco i temi caldi
della fase attuale, e ci facciamo carico sempre di elaborare proposte, come quelle che
seguono, senza nessuna pretesa di esaustività.
\section*{DARE FIDUCIA ALLA SCUOLA}
Il futuro del Paese è legato alla cultura e alla civiltà che sa produrre. Dare quindi, o meglio ridare, fiducia alla scuola non è uno slogan, ma un'esigenza strategica. La fiducia
riparte dal riconoscimento del ruolo sociale degli insegnanti e dal loro pregevole impegno.\'{E} il momento di fare un nuovo patto tra educatori e società. Dare fiducia significa,
per esempio, considerare la formazione dei docenti come obiettivo strategico per la
qualità. Formazione che deve quindi essere continua, obbligatoria, verificata, fondata
sulla ricerca e la creatività professionale, senza dogmatismi e pratiche centralizzate. Per
questo si plaude all'approvazione dell'art. 16 del Decreto Scuola, che ha ufficialmente
riconosciuto questo principio. Fiducia nella scuola significa anche andare oltre una diffusa conflittualità per valorizzare la responsabilità di tutti i soggetti dell'educazione, in
una logica di comunità. Significa rinnovare continuamente un patto di alleanza con le
famiglie anche per immaginare insieme soluzioni per un futuro di autentica integrazione
sociale.
\section*{RISPETTARE LA NORMALITÀ DELL'AUTONOMIA}
In questo decennio l'autonomia della scuola è stata soffocata da politiche autoritarie
e centraliste. È indispensabile che in ogni atto della governance del sistema formativo
siano rispettate e valorizzate la flessibilità, la territorialità orizzontale, la creatività professionale come leve della qualità. Ciò deve realizzarsi sul versante sia organizzativo,
ad esempio in riferimento all'organico funzionale e di rete territoriale, sia dell'autonomia didattica, per una flessibilità dei curricoli, degli insegnamenti, delle didattiche, come
pratica effettivamente inclusiva, superando la tradizione trasmissiva, monodirezionale,
per una pedagogia dell'eterogeneità, che offra a tutti non le stesse cose nello stesso
momento, ma le cose giuste per tutti e ciascuno. La nostra migliore tradizione educati-
va vede nella normalità della vita, nella scuola come nella società, il luogo dell'inclusione, dove le relazioni diventano comunità di coeducazione per tutti.
\section*{DAL CONFLITTO ALLA MEDIAZIONE}
Si sta diffondendo in maniera preoccupante la pratica di gestione dei conflitti in sedi
extrascolastiche. Ciò avviene sia nell'ambito giuridico, sia in quello clinico. La scuola
non può essere ridotta a una disputa sui bizantinismi delle normative, né alla logica
delle controparti. Altrettanto va contrastato il rischio di una diffusa medicalizzazione che
riduce la persona a sintomo e si contrappone all'educativo con la cultura della terapia.
È necessario invece costruire, con nuove leggi, un sistema e dei luoghi di mediazione
e conciliazione che mettano al centro decisioni connotate da un valore pedagogico e
relazionale, che sono per noi la vera giustizia possibile. A tal fi ne si auspica che in uno
dei decreti legislativi previsti dalla legge di delega, approvata il 7 novembre scorso dal
Governo, siano previste pratiche di mediazione a partire dai tentativi preventivi di conciliazione, da rendere obbligatori prima della via giurisdizionale.
\section*{PER UNA VALUTAZIONE INCLUSIVA}
In merito alla contraddittoria azione dell'INVALSI, che ha prodotto l’ esclusione di
200.000 alunni perché considerati “inadeguati” a scale internazionali comparative, pro-
poniamo maggiori pratiche valutative centrate sui processi, sui contesti, sulle relazioni,
pensate in chiave formativa (e non competitiva) per migliorare la qualità della didattica
inclusiva.
\section*{LA QUESTIONE DEI BES: PER UNA INCLUSIONE VERA}
Il nostro Convegno ha registrato, come ovvio, un vivace dibattito sulle recenti direttive
ministeriali riguardanti la cosiddetta questione BES, rilevando dialetticamente sia le potenzialità, sia le riserve critiche su un tema di grande delicatezza sociale e pedagogica.
Ci unisce comunque la comune passione inclusiva. Ben venga dunque una rifl essione
che allarghi questa passione a tutti gli alunni, considerando per ognuno potenzialità e
rischi, ostacoli e facilitatori. Ma lo sguardo fondativo delle pratiche inclusive non può
che restare pedagogico. Va affrontato costruttivamente il rischio di costruire nuove
classificazioni che separino e producano un abbassamento delle attese per i ragazzi;
vogliamo invece promuovere la resilienza, valorizzare i talenti individuali e le potenzialità
della comunità educativa. Per questo è necessario che l'ICF, come modello bio-psico-
sociale in chiave educativa, diventi il principale strumento di individuazione dei percorsi
inclusivi per tutti e insieme la base per la gestione delle risorse economiche, del personale e degli strumenti. L'inclusione avrà successo solo se si rispetterà e valorizzerà
il protagonismo pedagogico della scuola, evitandole burocratiche discussioni di forma
e senza appesantimenti giuridici. Ricordando che la scuola ha un cospicuo patrimonio
inclusivo, che va oggi rilanciato e rispettato come la vera base di partenza per promuovere nuove e migliori pratiche inclusive per tutti e per ciascuno.
\section*{PER NUOVE RISORSE}
Non sembri rituale questo richiamo: la scuola non ha bisogno ma diritto a nuovi investimenti. Qui noi non ci lagniamo, siamo semplicemente critici per la condizione di povertà in cui si è ridotto il bene più prezioso di una società: l'educazione dei suoi fi gli. Riconosciamo che nel recente Decreto Scuola vi sono, fortunatamente, segnali di inversione di
tendenza. Diciamo semplicemente: ministra Carrozza, continui così, individuando con
coraggio politiche di investimento che sappiano scegliere, che puntino all'inclusione
come priorità per la qualità, che sappiano anche superare incrostazioni corporative e
privilegi. In particolare apprezziamo l'aumento in organico di diritto dei posti di sostegno, ma ricordiamo la necessità di rivedere la gestione contrattuale della mobilità dei
docenti (di tutti gli insegnanti, non solo di quelli di sostegno), in una logica di organico
funzionale di reti di scuole, per garantire continuità agli alunni.
Per migliorare, quindi, è indispensabile che:
Parlamento e Governo avviino e attuino la normativa sulla formazione iniziale di tutti i
futuri docenti curricolari sulle didattiche inclusive, con non meno di 30 CFU sia per il 1
sia per il 2 ciclo d'istruzione;
sia concretamente attuata la formazione obbligatoria in servizio su tali didattiche, sia introdotto l'obbligo contrattuale di ore di programmazione settimanale comune tra i
docenti coinvolti nei medesimi gruppi di alunni, oltre l'orario delle lezioni;
sia riaffermato l'impegno obbligatorio dei docenti curricolari nella presa in carico del
progetto inclusivo degli alunni con disabilità, per evitare l'esclusiva delega all'insegnante di sostegno;
sia rispettato il tetto massimo di 20 alunni per classe in presenza di alunni con disabilità,
come previsto dall'art. 5, comma 2, del DPR 81/2009\footcite{DPR_81_2009}.
Nonostante tutto crediamo che una scuola diversa sia possibile, non solo necessaria.
Perché continuiamo a credere nell'insegnamento di Don Milani: “Ho imparato che il
problema degli altri è uguale al mio. Sortirne insieme è la politica. Sortirne da soli è l'avarizia”\footcite{AAVV2013}.
