\author{Alain Goussot}
\title{Bisogni speciali: postilla storica e tendenze in atto}
\phantomsection
\label{cha:Goussot010813}
%\epigraph{\hspace*{20pt}}{Alain Goussot}
\maketitle
\datapub{1 Agosto 2013}
Voglio ringraziare Salvatore Nocera\pageref{cha:nocera300713} per il suo intervento che mi permette di precisare meglio alcuni aspetti della mia riflessione sui cosiddetti \citi{bisogni educativi speciali}:
\section*{I \cit{bisogni educativi speciali} non esistono, i \cit{bisogni speciali} si.}
Condivido con lui l'importanza di pensare a percorsi variegati e personalizzati che tengano conto delle differenze e delle specificità degli alunni (profilo di apprendimento, capacità, caratteristiche cognitive, socio-affettive, culturali e motorie); personalmente mi riferisco ad un filone della storia della pedagogia che mette al centro la persona dell'educano e che fa dell'esperienza di apprendimento una esperienza di crescita in grado di rispondere alla particolarità di ogni alunno. Ma ritengo, partendo proprio dall'esperienza e dalla pratica d'insegnamento ed educativa, che la logica del riconoscimento delle differenze che si basa sull'eguaglianza nell'accesso all'istruzione, alle conoscenze e ai saperi è una cosa diversa rispetto alla tendenza attuale della logica differenzialistica che categorizza accentuando le diseguaglianze tra alunni, spesso con un processo di eticchettamento non valorizzante. Non voglio qui riprendere l'argomentazione che ho sviluppato nel mio contributo precedente ma voglio sottolineare che la mia critica e le mie perplessità sulle tendenza ultime delle politiche scolastiche non riguardano la tutela del riconoscimento delle differenze nel processo di apprendimento ma la catalogazione che viene fatta della popolazione scolastica con il rischio di vedersi sviluppare \citi{percorsi predestinati}, separazioni e diseguaglianze reali. Quindi alunni di serie A e alunni di serie B, sezioni di serie A e sezioni di serie B, scuole di serie A e scuole di serie B. Aggiungo anche che mentre è congruente parlare di bisogni speciali per alunni con deficit (quindi la necessità di mediazioni, mediatori, percorsi indiretti per favorire gli apprendimenti tenendo conto delle particolarità), mi sembra più problematico parlare di \citi{bisogni educativi speciali}: come dice giustamente Nocera un alunno non vedente ha dei bisogni speciali e di attenzioni particolari sul piano pedagogico e didattico ma i suoi bisogni educativi, pure tramite un apprendimento mediato, sono quelli di tutti gli altri (bisogno di sentirsi valorizzato, di potere sviluppare il proprio potenziale di vita, bisogno di sentirsi ascoltato, amato e riconosciuto\dots). Quindi esistono bisogni speciali che richiedono risposte particolari per potere accedere come tutti alle stesse opportunità ma non esistono \citi{bisogni educativi speciali} che sono invece eguali per tutti. Credo che non sia un semplice gioco linguistico ma una questione di sostanza.
\section*{Piccola postilla storica}

La mia ricostruzione delle tendenze riguardo all'aspetto legislativo è stata sommaria ma voglio fare notare che sarebbe importante riflettere sulle tendenze in atto dal 1977; essendo anche storico è un aspetto che m'interessa e che secondo me farebbe comprendere molte cose. La stessa legislazione è sempre in qualche modo, talvolta in modo contraddittorio, il riflesso dell'evoluzione e dei cambiamenti sociali, culturali e politici che avvengono nella società. La società non è rimasta ferma dal 1977, momento dell'approvazione della legge 517\footcite{Legge_517_77} sull'integrazione scolastica degli alunni con disabilità. Quella stessa legge è stata anche il prodotto di una stagione culturale e socio-politica particolare. Quello che m'interessa mettere in evidenza è come si sono evolute le tendenze rispetto alle questioni dell'integrazione e/o dell'inclusione (come viene chiamata oggi). Grosso modo, dal mio punto di vista si può dire che la scuola italiana, assieme alla società italiana, ha vissuto un processo emancipativo dal 1969 al 1992\footcite{Legge_104_92} (legge quadro); è il periodo di un grande fermento (sopratutto negli anni 70 e 80) sociale, politico e culturale: processi di deistituzionalizzazione (vedi legge Basaglia, L180), politiche globali di Welfare con il riconoscimento dei diritti umani e sociali fondamentali (diritto all'istruzione per tutti, compreso i disabili, diritto al lavoro, alla cura, ad un alloggio decente, diritto alla protezione sociale), il riconoscimento delle differenze (politiche di genere, parità). In quei anni si sviluppano anche le esperienze delle cooperative sociali che gestiscono servizi educativi: processo integrato dei servizi scolastici, sociali e sanitari. E' il periodo della diffusione delle esperienze educative e pedagogiche che mettono in discussione il vecchio modo di concepire l'istruzione: vedi l'esperienza di Barbiana con Don Lorenzo Milani, ma anche lo sviluppo del movimento di cooperazione educativa (che fa riferimento a Célestin Freinet), le esperienze di Bruno Ciari, Mario Lodi, Albino Bernardini con bambini della scuola elementare; esperienze che fanno della classe un laboratorio interattivo dove una didattica viva che si basa sui principi di cooperazione e inclusione si attiva. Quell'attività pedagogica veicola nella traduzione concreta della pratica d'insegnamento i concetti di eguaglianza, giustizia, istruzione per tutti e riconoscimento delle differenze. L'esperienza fatta con gli alunni disabili vede svilupparsi una didattica e pedagogia speciale che diventa didattica e pedagogia per tutti (i riferimenti metodologici che si diffondono sono Maria Montessori, Ovide Decroly, Lev Vygotskij che dimostrano come si possono trasferire le metodologie elaborate con alunni con deficit a tutti gli alunni); lo sguardo è quello pedagogico di puntare sulle potenzialità. Ma sappiamo anche da ricerche recenti (Treelle\footcite{treelle} e Fondazione Agnelli, per quanto discutibili su molti versanti) sullo stato dell'integrazione nella scuola e lo stato complessivo di quest'ultima, che vi sono tuttavia diverse criticità, storture, disfunzionalità e problematiche (la collaborazione tra insegnante curriculare e insegnante di sostegno, la preparazione pedagogica e didattica del corpo docente nel suo complesso, la relazione scuola-famiglia, la gestione delle risorse e la progettualità, le classi sovraffollate\dots). Sappiamo anche che tanti insegnanti in tante scuole hanno prodotto esperienze innovative e hanno saputo rispondere alle domande poste dai loro alunni nelle classi.
\section*{Il cambio di paradigma dal 1995: la de-emancipazione della scuola}

Dal 1995 (in modo sfasato rispetto a quello che succedeva in altri paesi europei dove erano già arrivate le politiche neo liberiste) inizia quello che chiamo un processo de-emancipativo: politiche neoliberiste di smantellamento del Welfare, quindi di smantellamento dei diritti sociali e dei diritti umani fondamentali come il diritto all'istruzione, alla cura, al lavoro e alla dignità. La persona come soggetto di diritti perde centralità ed esiste solo come soggetto di bisogni e di cura; come valore di scambio sul mercato, la frammentazione e la lacerazione dei legami sociali, la crisi dei modelli educativi e dei luoghi tradizionali di questi (famiglia e scuola), l'arrivo massiccio degli immigrati in quella fase che modifica la composizione etnico-culturale del paese, l'importanza attribuita alla logica costo-beneficio (la scuola stessa finisce per adottare il linguaggio dei mercati finanziari- si parla di debito e credito, l'autonomia scolastica non è progettuale ma economica, quindi la scuola stessa diventa azienda e i dirigenti scolastici sono spesso manager alla ricerca perenne di fondi\dots, il corpo docente viene precarizzato), la diffusione massiccia delle nuove tecnologie (con il pro e il contro): tutte queste tendenze incidono fortemente sul modo di concepire il ruolo docente, ma modifica anche nel profondo la concezione stessa dell'integrazione e della gestione degli apprendimenti. Negli anni '90 con la trasformazione della popolazione scolastica che diventa multiculturale (con la presenza dei figli d'immigrati) si fanno anche delle esperienze di mediazione culturale e pedagogica, si formano anche dei mediatori interculturali, ma queste esperienze purtroppo rimangono spesso senza seguito. Progressivamente cambia lo sguardo: si passa da un paradigma pedagogico ad un paradigma clinico-terapeutico che esprime anche una medicalizzazione strisciante delle difficoltà di apprendimento. La pedagogia viene marginalizzata e la didattica diventa didatticismo (procedura tecnica); la clinica e la psicologia, per non dire la neuropsicologia diventano dominanti. Si cambia anche vocabolario: si parla di \citi{comportamenti problema}, di disturbi specifici dell'apprendimento e di disturbi dell'apprendimento (cose che esistono ma che vengono guardate con l'occhio clinico e non quello pedagogico); questo si riflette anche nell'evoluzione terminologica delle diverse direttive, in particolare quella sui \glslink{dsaa}{DSA} e quella ultima sui BES. (Interessante leggere i testi del britanico Frank Furedi che parlata del dominio del paradigma terapeutico anche nella scuola e del francese Bernard Stigler che osserva come il curare ha sostituito il prendersi cura di, in fondo l'I care di Don Lorenzo Milani). Si passa dalla logica del riconoscimento delle differenze (basata sul principio di eguaglianza e sui diritti) alla logica differenzialistica (basata sulla predeterminazione, la medicalizzazione, l'etnicizzazione e la diseguaglianza); gli insegnanti rischiano anche di essere colonizzati dallo sguardo diagnostico-clinico: per andare a caccia di 'comportamenti problema' identificando difficoltà di apprendimento e problema. Sono domande serie che pongono anche pedagogisti europei come Philippe Meirieu (pedagogia: il dovere di resistere), Jean Houssaye (autorité et éducation), Jean Pierre Pourtois, H.Desmett e Bruno Humbeeck (Les ressources de la résilience) e Charles Gardou (la société inclusive). Meirieu recentemente si chiede se occorre diagnosticare o educare? L'osservazione pedagogica non è quella diagnostica-clinica; cosa fa l'insegnante di fronte all'alunno che resiste al suo insegnamento, quale ruolo svolge la relazione educativa, quale funzione ha il gruppo classe e la scuola come comunità educante. In tutto questo aggiungiamo la situazione di tagli continui e di riforme-controriforme continue (spesso contraddittorie) che deve subire il mondo della scuola, la precarizzazione della funzione docente, il suo indebolimento, il non coinvolgimento reale degli insegnanti nell'analisi, la progettazione e le decisioni. Poi in tutto questo che voce hanno gli alunni e quale ruolo hanno le famiglie? Il gruppo di Pourtois (Belgio) mette l'accento sull'importanza di costruire dei processi co-educativi; l'alleanza educativa scuola-famiglia-territorio: che ne è realmente? Nella relazione scuola-famiglia domina la logica burocratica-amministrativo o quella democratica- partecipativa?

Non vè dubbio che la pratica concreta d'insegnamento produce possibilità di crescita e apprendimento là dove sembrava impossibile (vi sono tanti esempi passati e recenti che lo dimostrano): nella pratica didattica e pedagogica l'insegnante attento che sa ascoltare, osservare e comprendere, tramite la sua disciplina o la sua azione crea delle situazioni dinamiche che possono motivare gli alunni. La gestione e il clima del gruppo classe, l'attenzione alle caratteristiche di ognuno, all'individualità e al riconoscimento delle differenze, la capacità di sintonizzarsi con i vissuti e le storie degli alunni che si trovano nella sua classe, il dialogo con le famiglie, il delicato equilibrio tra istruzione e educazione che oggi viene spesso spezzato accentuando il primo elemento con le tecniche di trasmissione, le pratiche di mediazione pedagogica per facilitare l'espressione delle potenzialità di tutti facendo della classe un laboratorio interattivo, accogliente e uno spazio dove è possibile imparare scoprendo e sperimentando se stesso: ecco alcuni aspetti fondamentali che fanno dell'operatore pedagogico e della scuola un promotore di autonomie e di competenze. Qui si pone per quanto riguarda l'Università ma anche la scuola nel suo complesso la questione della formazione/preparazione dell'insegnante, del sistema di valori che sta alla base del suo agire, del progetto globale che deve investire tutta la comunità come responsabile della formazione delle future generazioni\footcite{Goussot2013a}.   