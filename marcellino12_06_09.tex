\begin{abstract}
	Un'assistenza specialistica ad personam che dev'essere fornita al singolo studente con disabilità – in aggiunta all'assistente igienico-personale, all'insegnante di sostegno e agli insegnanti curricolari – per sopperire ai problemi di autonomia e/o comunicazione sussistenti nello studente stesso. Si tratta di un aspetto peculiare della garanzia di integrazione – da non confondere mai con l'assistente igienico-personale – che esaminiamo approfonditamente in questa ampia analisi
\end{abstract}
\author{Francesco Marcellino}
\title{L'assistente all'autonomia e alla comunicazione: adempimenti e funzioni}
\label{cha:marcellino120609}
\maketitle
Affrontare in generale il tema dell'integrazione scolastica dello studente con disabilità è compito arduo. Lo diviene ancor di più dissertare su un aspetto peculiare della garanzia di integrazione, ovvero, sull'assistenza all'autonomia e alla comunicazione.
Innanzitutto un \cit{avviso ai lettori}: l'assistente all'autonomia e alla comunicazione è ruolo e funzione diversa da quella dell'assistente igienico-personale. Si registrano, infatti, sul tema, molte confusioni e fraintendimenti che qui, invece, si vogliono evitare.

L'assistenza igienico-personale (o cosiddetta \cit{assistenza di base}) dev'essere fornita tendenzialmente a tutti gli studenti con disabilità da parte dei collaboratori scolastici (ex bidelli), così come previsto dal Contratto Collettivo Nazionale di Lavoro Comparto Scuola del 16 maggio 2003\footcite{ccnl_2003}, nonché dalla Nota del Ministero protocollo n. 3390\footcite{Nota3390_2001} del 30 novembre 2001.
L'assistenza consiste – come afferma la contrattazione collettiva – \caporali{nel prestare ausilio materiale agli alunni portatori di handicap nell'accesso dalle aree esterne alle strutture scolastiche, all'interno e nell'uscita da esse, nonché nell'uso dei servizi igienici e nella cura dell'igiene personale}. Per svolgere questa mansione, i collaboratori scolastici hanno diritto a frequentare un corso di formazione e a ricevere un premio incentivante in busta. Dal canto suo, è il Dirigente Scolastico che deve assicurare il diritto all'assistenza (si veda ancora la citata Nota Ministeriale 3390/01\footcite{Nota3390_2001}).
Cosa accade nell'ipotesi in cui, concretamente, l'assistenza di base non venga garantita allo studente disabile? La famiglia deve diffidare con lettera il Dirigente Scolastico a garantire il servizio ovvero ad attivare anche l'azione disciplinare nei confronti dei dipendenti inadempienti. Occorre inoltre riferire che – sebbene la contrattazione collettiva di settore espressamente preveda quanto sopra riferito – si sono registrati dei ritardi nella formazione del personale scolastico. Cosi, all'esclusivo fine di non far mancare agli studenti con disabilità l'assistenza di base, si è previsto (ad esempio in Sicilia) che i Comuni, in sostituzione dell'ente scolastico, potessero contribuire a fornire il personale. Quest'ultimo, spesso, viene offerto agli istituti scolastici mediante aggiudicazione di enti privati. Tant'è che, con Legge della Regione Sicilia n. 15 del 5 novembre 2004, articolo 22, si è statuita la competenza dei Comuni singoli o associati e delle Province Regionali \caporali{ad erogare in aggiunta al servizio di trasporto dal domicilio alle strutture scolastiche, il servizio di assistenza igienico personale ed altri servizi specialistici come già disposta dalle ll.rr. N. 68/81, N. 16/86 e 6/00} (si veda in tal senso la Circolare n. 3 del 7 marzo 2005, prodotta dall'Assessorato della Regione Sicilia alla Famiglia).
L'erogazione del suddetto servizio da parte degli Enti Pubblici locali è da ritenersi tuttavia del tutto eccezionale e provvisoria, solo fin quando, quindi, gli organi scolastici non godranno di proprio personale formato. Se ed allorquando il personale scolastico sarà stato formato, non si dovrà più ricorrere al sostegno e alla fornitura del suddetto servizio mediante gli Enti Pubblici. E questi ultimi dovrebbero vigilare su ciò, anche per il considerevole risparmio di cui potrebbero godere. Il personale scolastico formato, infatti, non potrà più esimersi dallo svolgere le mansioni (pena un procedimento disciplinare che dovrà attivare il dirigente scolastico), determinando in tal modo una maggiore efficienza di sistema e un risparmio di risorse per gli Enti Pubblici locali.

\section*{L'assistente all'autonomia e alla comunicazione nella legislazione vigente}
Chiarito quanto sopra, affrontiamo il tema dell'assistente all'autonomia e alla comunicazione. Esso è previsto dai seguenti atti normativi:
- Articoli 42-45 del Decreto del Presidente della Repubblica (DPR) 24 luglio 1977, n. 616\footcite{DPR_616_1977}.
- Articolo 13, comma 3 della Legge 104/92\footcite{Legge_104_92}.

L'articolo 42 del DPR 616/77\footcite{DPR_616_1977}, intitolato Assistenza  scolastica, afferma: \caporali{Le  funzioni  amministrative relative alla materia dell'”assistenza scolastica” concernono tutte le strutture, i servizi e le  attività  destinate  a  facilitare mediante erogazioni e provvidenze in denaro o  mediante  servizi individuali o collettivi, a favore degli alunni  di  istituzioni scolastiche pubbliche o private, anche se adulti, l'assolvimento dell'obbligo  scolastico  nonché,  per  gli  studenti  capaci  e meritevoli ancorché privi di mezzi, la prosecuzione degli studi.
Le funzioni suddette concernono fra l'altro: gli interventi di assistenza medico-psichica; l'assistenza ai minorati psico-fisici; l'erogazione gratuita dei libri di testo agli alunni delle scuole elementari}.
L'articolo 13, comma 3 della Legge 104/92\footcite{Legge_104_92} afferma invece:  \caporali{Nelle scuole di ogni ordine e grado, fermo restando, ai sensi del decreto del Presidente della Repubblica 24 luglio 1977, n. 616\footcite{DPR_616_1977}, e successive modificazioni, l'obbligo per gli enti locali di fornire l'assistenza per l'autonomia e la comunicazione personale degli alunni con handicap fisici o sensoriali, sono garantite attività di sostegno mediante l'assegnazione di docenti specializzati}.

La figura, quindi, nasce dal riferimento del secondo comma dell'articolo 42 del DPR 616/77\footcite{DPR_616_1977} (\caporali{assistenza ai minorati psico-fisici}), nonché da questo ultimo articolo 13, comma 3, come riferito (\caporali{l'obbligo per gli enti locali di fornire l'assistenza per l'autonomia e la comunicazione personale degli alunni con handicap fisici o sensoriali}).
L'Assistente all'autonomia e alla comunicazione è quindi un'assistenza specialistica ad personam (è infatti definito anche \cit{assistente ad personam}) che dev'essere fornito al singolo studente con disabilità - in aggiunta all'assistente igienico-personale, all'insegnante di sostegno e agli insegnanti curricolari - per sopperire ai problemi di autonomia e/o comunicazione sussistenti nello studente.

La tradizionale applicazione di questo istituto ha avuto quali principali destinatari gli studenti con disabilità di comunicazione (udito e parola). Ma una corretta lettura del dettato normativo ha consentito la giusta diffusione dell'assistente all'autonomia e alla comunicazione anche ad altre tipologie di disabilità.

Già l'articolo 42 citato, infatti, riaffermare l'assistenza ai minorati psico-fisici, determina un ampio bacino di utenza, ma nonostante ciò – almeno in un primo momento – vi erano delle perplessità interpretative, visto che l'articolo 13, comma 3 della Legge 104/92 si riferiva invece ad \caporali{alunni con handicap fisici o sensoriali}.

L'interpretazione più corretta – anche alla luce della visione complessiva dell'integrazione dell'alunno disabile nell'ambiente scolastico e a garanzia del concreto ed effettivo diritto all'istruzione - si ritiene che debba fondarsi sulla necessità di garantire l'assistenza specialistica ad personam a tutti gli studenti con disabilità fisica, psichica o sensoriale, la cui gravità o limitazione di autonomia, determini l'inevitabile necessità di assistenza per un regolare apprendimento delle nozioni scolastiche (orientato in questo senso è chiaramente anche l'Accordo emanato dalla Conferenza Stato-Regioni il 20 marzo 2008\footcite{ra_39_2008}).

Pertanto è evidente che all'assistente per l'autonomia e comunicazione competano funzioni specifiche che differenziano questa figura dall'insegnante di sostegno, con cui deve cooperare in sinergia, secondo gli obiettivi del PEI (Piano Educativo Individualizzato).

L'assistente all'autonomia e alla comunicazione è quindi un operatore che media la comunicazione e l'autonomia dello studente disabile con le persone che interagiscono con lui nell'ambiente scolastico e ciò può compiersi anche mediante strategie e ausili necessari per garantire un'interazione efficace.

La procedura di assegnazione dell'assistente specialistico dev'essere frutto dell'azione sinergica dei diversi organi chiamati a garantire l'integrazione scolastica dell'alunno disabile. Innanzitutto nella certificazione della USL e nella diagnosi funzionale occorre che venga riconosciuta (o meno) la necessità di questa figura di assistenza. Sarà quindi lo stesso Gruppo Multidisciplinare (in cui siedono Scuola, USL ecc.), ovvero il Gruppo di Lavoro operativo di cui all'articolo 15, comma  1 della Legge 104/92\footcite{Legge_104_92}, a completare le necessità e le modalità; il Dirigente Scolastico, invece, dovrà farsi portavoce presso l'Ente Pubblico locale, richiedendo per tempo di fornire l'assistente specializzato all'alunno nella figura professionale individuata dalla USL.
La competenza a fornire il servizio è dei Comuni per le scuole elementari e medie, della Provincia per le scuole superiori (articolo 139 del Decreto Legislativo 112/98\footcite{DL_112_1998}). Ma a tal proposito, procediamo con ordine.

Come sappiamo dall'articolo 3, comma 1 del DPR del 24 febbraio 1994\footcite{DPR1994}, per diagnosi funzionale \caporali{si intende la descrizione analitica della compromissione funzionale dello stato psicofisico dell'alunno in situazione di handicap, al momento in cui accede alla struttura sanitaria per conseguire gli interventi previsti dagli articoli 12 e 13 della legge n. 104 del 1992}.

Posto che la diagnosi funzionale stessa è \caporali{finalizzata al recupero del soggetto portatore di handicap}, essa \caporali{deve tenere particolarmente conto delle potenzialità registrabili in ordine ai seguenti aspetti:
	
- cognitivo, esaminato nelle componenti: livello di sviluppo raggiunto e capacità di integrazione delle competenze;

- affettivo-relazionale, esaminato nelle componenti: livello di autostima e rapporto con gli altri;

- linguistico, esaminato nelle componenti: comprensione, produzione e linguaggi alternativi;

- sensoriale, esaminato nella componente: tipo e grado di deficit con particolare riferimento alla vista, all'udito e al tatto;

- motorio-prassico, esaminato nelle componenti: motricità globale e motricità fine;

- neuropsicologico, esaminato nelle componenti: memoria, attenzione e organizzazione spazio temporale;

- autonomia personale e sociale }.

Tutto ciò può dunque già evidenziare la situazione psicofisica che conduce a rendere necessario l'assistente all'autonomia e alla comunicazione.

Sappiamo poi che il profilo dinamico funzionale \caporali{descrive in modo analitico i possibili livelli di risposta dell'alunno in situazione di handicap riferiti alle relazioni in atto e a quelle programmabili}. Questo, quindi, dovrebbe riferirci il mutamento di risposta e di livelli raggiungibili dall'alunno con o senza assistenza specialistica. Motivo per cui, evidenzia la necessità e la qualità/quantità di essa.

Così operando, ci troveremmo di fronte a un Piano Educativo Individualizzato dalla struttura rispettosa di quello che realmente prevede la legge, ovvero – citando sempre dal DPR del 24 febbraio 1994\footcite{DPR1994} (articolo 5) – \caporali{il documento nel quale vengono descritti gli interventi integrati ed equilibrati tra di loro, predisposti per l'alunno in situazione di handicap, in un determinato periodo di tempo, ai fini della realizzazione del diritto all'educazione e all'istruzione, di cui ai primi quattro commi dell'art. 12 della legge n. 104 del 1992;  [...] che tiene presenti i progetti didattico-educativi, riabilitativi e di socializzazione individualizzati, nonché le forme di integrazione tra attività scolastiche ed extrascolastiche, di cui alla lettera a), comma 1, dell'art. 13 della legge n. 104 del 1992\footcite{Legge_104_92}}. E, quindi, \caporali{Nella definizione del P.E.I., i soggetti di cui al precedente comma 2, propongono, ciascuno in base alla propria esperienza pedagogica, medico-scientifica e di contatto e sulla base dei dati derivanti dalla diagnosi funzionale e dal profilo dinamico funzionale, di cui ai precedenti articoli 3 e 4, gli interventi finalizzati alla piena realizzazione del diritto all'educazione, all'istruzione ed integrazione scolastica dell'alunno in situazione di handicap. Detti interventi propositivi vengono, successivamente, integrati tra di loro, in modo da giungere alla redazione conclusiva di un piano educativo che sia correlato alle disabilità dell'alunno stesso, alle sue conseguenti difficoltà e alle potenzialità dell'alunno comunque disponibili}».

Insomma, un iter amministrativo tutto volto a garantire la certificazione delle \caporali{effettive esigenze rilevate} dell'alunno con disabilità, così come recitano le ultime Leggi Finanziarie che disciplinano le modalità di assegnazione degli insegnanti di sostegno.

Qualora, quindi, dovesse esserci disomogeneità (mancato rispetto) tra la situazione di diritto certificata dagli enti competenti e la situazione di fatto vissuta in classe dall'alunno (numero non sufficiente di ore di sostegno, mancanza di assistente igienico-personale o mancanza di assistente per l'autonomia e la comunicazione), si può ragionevolmente richiedere l'immediato adempimento, nonché proporre azione giudiziaria a tutela dei diritti e degli interessi dello studente con disabilità.
Il principio di cui all'articolo 34 della Carta Costituzionale – \caporali{La scuola è aperta a tutti} - non significa banalmente che la scuola è obbligata ad accogliere tutti, ma più efficacemente ad accoglierli e fornire loro l'istruzione, l'educazione e la socializzazione adeguata e proporzionata non solo alle condizioni psicofisiche, ma alla \cit{dignità} dello studente quale essere umano e portatore di diritti e di doveri.

*Avvocato (fmarcellino@videobank.it). Il presente contributo è parte di una relazione tenuta in occasione del Convegno organizzato dal GLIP (Gruppo di Lavoro Interistituzionale Provinciale) dell'Ufficio Scolastico Provinciale di Siracusa sull'integrazione scolastica degli alunni con disabilità.

12 giugno 2009

© Riproduzione riservata