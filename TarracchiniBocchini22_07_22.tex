\author{Ermanno Tarracchini  Valeria Bocchini}
\title{I bisogni umani di crescita ed apprendimento non sono speciali}
\phantomsection
\label{cha:TarracchiniBocchini }
%\epigraph{\hspace*{20pt}}{Ermanno Tarracchini  Valeria Bocchini}
\maketitle
\datapub{22 Luglio 2013}
\section*{La questione terminologica}
E’ stato scritto che \cit{il concetto di \glslink{besa}{BES} non ha alcun valore clinico, ma \cit{politico} e dunque dovrebbe agire nei contesti delle politiche di riconoscimento dei diritti e di allocazione delle risorse. L'eventuale sua utilità dovrà essere in questi contesti, in cui recentemente è apparso con evidenza anche a livello nazionale con la Direttiva di Dicembre e la Circolare di Marzo\mancatesto.} (Ianes\pageref{cha:ianes1})

Siamo d'accordo, ma non possiamo non rilevare una contraddizione in termini quando leggiamo (sempre in Ianes\pageref{cha:ianes1})

\cit{Nel merito di questi due provvedimenti, ritengo comunque che essi siano passi avanti verso una scuola più inclusiva, anche se il concetto di \glslink{besa}{BES} è ancora prevalentemente centrato sulle patologie e non sul funzionamento umano \glslink{icfa}{ICF} e quello di inclusione, di conseguenza, è ancora visto come estensione ad alcuni alunni (con \glslink{besa}{BES}) di azioni individuali di personalizzazione-individualizzazione (peraltro necessarie) piuttosto che come strutturazione diffusa di Didattiche inclusive}.

Ancora non  possiamo non essere d'accordo sui principi generali riportati e sull'intento dichiarato di una \cit{strutturazione diffusa di Didattiche Inclusive}  ma simile intento è tradito dalle parole \cit{alunni con \glslink{besa}{BES}}. Quando, un qualsivoglia provvedimento permette  l'impiego di una tale terminologia  (che sia \cit{I miei alunni con \glslink{dsaa}{DSA}} o \cit{i miei alunni con Deficit di Attenzione ed Iperattività} \cit{i miei alunni con \glslink{besa}{BES}} o \cit{con \glslink{fila}{FIL}}\dots) anziché chiamare per nome e cognome ogni singolo studente,  nasconde l'ottica della ghettizzazione. Proprio per questo, esprimiamo un dubbio, un ragionevole dubbio come ricerca della verità, a proposito  dell'ennesima etichettatura \cit{Bisogni Educativi Speciali}. Noi insegnanti, educatori e pedagogisti dobbiamo lavorare  per una scuola migliore e di qualità per tutti, e non creare categorie e ghetti, funzionali alla logica della medicalizzazione - cioè del profitto - più o meno mascherata come estensione dei diritti da una categoria di persone ad un'altra.
\section*{Per un movimento pedagogico-educativo}

Dobbiamo riprenderci l'educazione, per dare ad ognuno quello di cui abbisogna secondo i suoi mezzi – cioè tenendo conto e rispettando la sua personale biografia - e non \cit{privilegiare} soltanto coloro che pagano lo stigma di una segnalazione per  un \cit{disturbo} o per  un  \cit{bisogno educativo-apprenditivo} non soddisfatto dal precedente iter scolastico.

Siamo per una ripresa del ruolo guida della pedagogia nella scuola e nella società, per riprenderci la delega offerta agli  esperti  della \cit{psiche}  ed imposta dal sopravvento della cultura \cit{terapeutica}, intesa in senso clinico - e non in quello, etico e pedagogico, del prendersi cura del prossimo - importata dal mondo anglosassone a partire dalla metà del secolo scorso, sotto forma di massiccia prassi medicalizzante al servizio del profitto ricavato dall'invenzione di disturbi e malattie  alla moda, come \cit{il mito del bambino iperattivo.} Occorre una risposta ai bisogni educativi personali di tutti: contro lo  stigma, le etichette e le categorie dei disturbi misti o dei bisogni speciali (anche  perché dallo \cit{speciale} al \cit{disturbo} il cammino potrebbe essere  breve e scivoloso). 
\section*{Scuola: educazione e istruzione}

Premettiamo che siamo, o siamo stati, insegnanti per il  sostegno da 25-30, docenti universitari a contratto nei corsi SSIS dell'Università di Modena e Reggio Emilia, e che ora condividiamo la preoccupazione per l'applicazione, nelle variegate realtà scolastiche italiane, della Circolare n.8/13\footcite{cm8_2013} sui cosiddetti \glslink{besa}{BES} e, soprattutto, per il business della formazione da attuare sui docenti, ambito che ha già scatenato e scatenerà feroci appetiti di monopolio universitario-accademico o psicologico-clinico e iniziative orientate all'approccio \cit{patologico} e \cit{dispensativo}, anziché promuovere l'auto-formazione dei docenti in termini di ricerca e di sperimentazione.

Ci chiediamo: come fare per impedire l'ulteriore perniciosa infiltrazione dell'ottica dei disturbi nella formazione dei docenti attraverso  i \glslink{besa}{BES}? Come impedire che la formazione sui \glslink{besa}{BES} diventi tout court un potenziamento ed un'ulteriore esasperazione della de-formazione sui “disturbi” che in tutti questi anni ha colonizzato le menti dei docenti, imperversando nel mondo della scuola attraverso l'esasperante ossessione diagnostica dei cosiddetti \glslink{dsaa}{DSA}, costringendo gli insegnanti nei termini di legge (L.170/10)\footcite{legge170} a dispensare gli alunni dall'apprendimento e, al contempo,  a dispensare se stessi dall'insegnamento?  

Eppure tutti sappiamo che la scuola non può essere dispensata dall'educazione e dall'istruzione come sta succedendo in questi anni: si legittimano insegnanti che non devono più preoccuparsi dell'apprendimento da parte degli studenti, che sono dispensati dall'insegnamento, dal  trovare strategie pedagogiche alternative per raggiungere la mente dei loro studenti (se c'è riuscita Anna Sullivan con Helen Keller, una bimba sordo-cieca\dots). Quale futuro per loro e per i nostri figli-studenti? L'identificazione e la riduzione della loro persona ad un loro presunto disturbo o bisogno speciale, oppure la valorizzazione di biografie e personalità in continua evoluzione le quali, se non verrà loro impedito, sapranno prima o poi trovare una strada?

\cit{No, tu non sei una r scritta con brutta calligrafia... un'inversione di sillaba... una lettura stentata}, ripetiamo ai nostri studenti perché non restino inchiodati alla parte eventualmente mancante del loro cammino di conoscenza ed apprendimento, ad un'etichetta  che vorrebbe ridurre la loro personalità ad un errore di scrittura, lettura, calcolo o - perché no? - ad un \cit{bisogno speciale}.
\section*{Pensami adulto}

Allora pensiamo una scuola che li immagini adulti, una scuola che li sostenga senza stigmi lungo il loro percorso di crescita di cui non conosciamo ancora gli esiti, ma che possiamo già intravedere o immaginare alla luce di nuovi stili educativi: un nuovo clima nella vita di classe e nel rapporto docente-discente,  una scuola più lenta, la \cit{pedagogia della lentezza} contro l'ossessione del tutto e subito, del saper scrivere e leggere perfettamente dopo tre mesi di primaria. Ritmi più lenti e diversificati per tutti, ossia senza la fretta e le ossessioni valutative e diagnostiche  delle categorie più o meno medicalizzanti. La pedagogia é la disciplina, nata con l'umanità stessa,  dell'accompagnamento e della valorizzazione delle potenzialità personali di ognuno,  della fiducia nel loro ampio e positivo dispiegamento, nel rispetto dei tempi personali e del soddisfacimento dei bisogni di crescita dei piccoli della specie umana.
\section*{BES e DSA: stigma, esperti e misure dispensative}

Ma cosa c'entrano i cosiddetti \glslink{dsaa}{DSA} con i cosiddetti \glslink{besa}{BES}? C'entrano perché, secondo le intenzioni della circolare sui \glslink{besa}{BES}, li accomunerebbe la - per noi deleteria - ottica della dispensa dall'apprendimento della L.170\footcite{legge170}.

Certamente è condivisibile la ricerca di una maggiore \cit{inclusività} (Ianes) e non possiamo che concordare con l'idea di restituire le responsabilità educative a noi insegnanti (Canevaro\footcite{Canevaro2013a}), di valorizzare le nostre competenze pedagogiche, didattiche ed educative (sempre che tale valorizzazione non si traduca solamente in uno sterile aumento del carico burocratico). Tutto ciò, inoltre, sarebbe condivisibile se andasse nella direzione di interrompere la perniciosa abitudine, consolidatasi negli ultimi decenni della fine del secolo scorso, della delega a presunti esperti e specialisti dei disturbi (pensiamo, appunto, ai cosiddetti \glslink{dsaa}{DSA}). Ma chi ne garantirà l'applicazione in questa nuova direzione e non, piuttosto, un allargamento nell'ottica del disturbo, con ulteriore delega agli psicologi per farci dettare (assurdamente) le linee didattico-educative a favore degli studenti che più hanno bisogno della nostra attenzione?

Purtroppo, nefasti presagi di quale piega prenderà l'applicazione della circolare sui \glslink{besa}{BES} li si vedono già dalle proposte di formazione che circolano: ci vengono i brividi quando leggiamo formazione a tappeto dei curricolari: formazione su che cosa? Su una commistione di informazioni clinico - ezio - patologiche improponibile, poiché non sorrette da alcuna seria, e non solo millantata, evidenza scientifica, e di istruzioni di tecniche compensative e dispensative copia conforme dei corsi e dei master sui \glslink{dsaa}{DSA}?! Vogliamo cambiare etichette, più o meno medicalizzanti, per non cambiare la sostanza, che resta sempre il profitto della formazione e dell'indotto editoriale e della produzione di strumenti/ausili (l'incredibile e geniale scoperta del computer, della calcolatrice, della tavola pitagorica, dei formulari, delle mappe e delle sintesi vocali! Strumenti davvero specialistici che, se gli esperti non ce li avessero consigliati, la scuola non avrebbe mai introdotto nella didattica!) Vogliamo davvero perseverare in quest'ottica della dispensa che annichilisce la figura dell'insegnante-ricercatore? No, finalmente, devono essere i docenti di provata esperienza e capacità professionale, sia curricolari che di sostegno, a prendere il timone della barca!
\section*{Riprendiamoci la lentezza e la pedagogia}

Chiederemo, però, aiuto soprattutto a dei \cit{veri} esperti: esperti di vita, della loro vita e di quella dei loro figli-studenti, ossia ai genitori, perché ci presentino i loro figli, ci raccontino le qualità dei loro figli. Qualità che noi a scuola non facciamo nemmeno in tempo a scorgere e a prendere in considerazione, presi come siamo dalla fretta, dai voti, dalle verifiche, dal registro e dal programma. La rapidità, la velocità, la frenesia consumistica del mercato ha colonizzato anche la scuola, per eccellenza luogo dei tempi lunghi e della lentezza. Occorre riprenderci il senso della lentezza, come ci indica lucidamente Gianfranco Zavalloni in Pedagogia della lumaca. Occorre opporsi. Ad esempio a forme come questa: la \cit{diagnosi precoce} della cosiddetta dislessia attraverso un test di velocità di lettura che le maestre, spesso \cit{istruite} da psicologi estranei e ignari rispetto al quotidiano ambiente umano e pedagogico che si instaura in una classe, devono somministrare ai loro bambini nella primaria. Quanti secondi impieghi a leggere questa parola e questa non-parola e questa frase? Più di 5 secondi? Ah no, troppi! Tempo scaduto: hai bisogno di una segnalazione scolastica! Un ossimoro, un orrore etico e pedagogico imposto dall'industria della medicalizzazione.

Contro questo modo di far scuola, riprendiamoci la pedagogia dei grandi del Novecento: la pedagogia della mano come organo dell'intelligenza e del linguaggio di Maria Montessori, dei lavori manuali e dei laboratori di Freinet, la pedagogia delle evocazioni di de La Garanderie, la pedagogia dell'aiuto reciproco, della scrittura collettiva e della solidarietà di Don Milani: quella dell'uscirne insieme altrimenti è l'avarizia a prevalere. Ricominciamo a tessere i fili di un'altra storia, la storia dell'amore, dell'attenzione pedagogica e scientifica per la cura e la crescita degli esseri umani.
\section*{L'aiuto della comunità adulta}

Per non subire l'imposizione del \cit{tutto e subito}, dovremo chiedere, come dicevamo, ai genitori di attivarsi per diventare protagonisti nell'educazione dei loro figli. Chiediamo ai genitori di farci formazione sul come collaborare con loro per aiutare i loro figli in difficoltà. Avremo bisogno delle competenze educative dei genitori che, più di qualunque altro, conoscono i propri figli: le loro competenze sono state ingiustamente ignorate ed escluse dalla scuola.

In più, se ne avremo bisogno, potremmo avvalerci della consulenza pedagogica di pedagogisti esperti in problematiche infantili e adolescenziali lontani dall'ottica del disturbo. Nell'ottica della dialettica del biologico e del sociale, chiederemo aiuto anche ai medici, ai pediatri, ai neurologi, ai sociologi, ai logopedisti, ai fisiatri, ai fisioterapisti ecc\dots, cioè agli esperti e ai tecnici di patologie organico-funzionali, di problematiche biologiche o sociali e non a presunti esperti di presunti disturbi. Presunti disturbi che sono, in realtà, sintomi di un possibile disagio provocato da una società e da una scuola malate. La società della fretta, del tutto e subito, non é compatibile con i tempi e i ritmi della crescita dei nostri piccoli, occorre rallentare, saper aspettare, fermarsi se necessario. Una società frenetica, una scuola dell'ossessione valutativa e diagnostica, sono compatibili, esclusivamente con la logica della competizione, cioè del profitto.

Chiederemo aiuto anche agli artisti, ai musicisti, agli attori, ai cantanti, agli sportivi, agli artigiani, ai nonni, alla solidarietà intergenerazionale al fine di attivare altri percorsi espressivi e di dispiegamento delle potenzialità dei più piccoli. Questa volta la via dobbiamo tracciarla noi operatori della scuola, \cit{esperti} di didattica, di pedagogia e di inclusione. La formazione di base ce la faremo noi, tra di noi, con la messa in comune delle intelligenze, delle esperienze ormai più che trentennali di integrazione e di tutto quanto è scaturito dalla creatività ed intelligenza pedagogia e scientifica degli insegnanti italiani. E vi assicuriamo che è tanta. Ecco, se le linee applicative andranno in questa direzione allora, superando l'ambiguità terminologica e concettuale dei \glslink{besa}{BES}, si potrà collaborare per un cambiamento radicale della scuola, ma dovranno essere rispettate e valorizzate le nostre esperienze.
\section*{La questione epistemologica}

Oltre che per compiacere la moda e l'abuso di sigle ed etichette, perché riunire i deficit certificati dalla L.104 in un unico contenitore insieme a tutte quelle espressioni di diversità dei cammini di conoscenza e di apprendimento riguardanti bambini e ragazzi che non hanno deficit o problematiche organiche scientificamente dimostrabili? E' anche per questo che ci preoccupa il \cit{cappello} dei \glslink{besa}{BES}. Ci preoccupa perché ingloba pericolosamente, a ben vedere, tutto quanto può creare disagio alla scuola: dai veri deficit allo svantaggio socio-culturale, economico e linguistico. Simile propensione, come già detto, potrebbe permettere a qualche insegnante di esasperare le derive medicalizzanti utilizzando come prevalente (se non esclusivo) strumento di intervento didattico quello delle misure dispensative.

Il grave pericolo a nostro avviso è che i docenti (ma anche molti pedagogisti) già addestrati dagli psicologi ad essere \cit{educatori da riporto} - l'espressione è forte, ma rende la realtà di quello che sempre più sta avvenendo - cioè deformati da massicci ed invasivi corsi e corsetti all'insegna del riconoscimento esasperato del deficit, procedano con la medesima logica ed inglobino altri studenti nell'ottica del disturbo con stigli ed etichette, anziché farli uscire dal recinto della patologia o prevenirne la patologizzazione. Occorre interrompere il flusso medicalizzante, altrimenti, l'ottica del disturbo ingloberà anche i cosiddetti Bisogni Speciali. Tanto più che psicologi alla disperata ricerca di lavoro si improvvisano educatori e con il consenso del potere politico-amministrativo invadono sempre più l'ambito educativo-pedagogico a loro estraneo.

Allora troviamo non soltanto un'altra terminologia, ma soprattutto un'altra via, perché non si tratta di una questione semplicemente etimologica, bensì soprattutto epistemologica: quali fondamenti etici, pedagogici e scientifici stanno alla base della categoria e terminologia \cit{Bisogni Educativi Speciali}\footcite{Tarracchini2013}?

NOTA

Ermanno Tarracchini e Valeria Bocchini sono docenti specializzati per il sostegno ed ex docenti SSIS per il sostegno.