\author{Giuseppe Felaco}
\title{Tutti avrebbero dei vantaggi}
\label{cha:felaco260413}
\begin{abstract}
Rimettendo al centro dell’integrazione scolastica gli insegnanti curricolari, ci sarebbero - secondo Giuseppe Felaco -- vantaggi per gli alunni, con disabilità e non, sia per gli insegnanti stessi. \caporali{Diventa quindi indispensabile –- scrive Felaco –- un cambiamento di direzione che permetta di rafforzare questo fondamento}
\end{abstract}
%\epigraph{Rimettendo al centro dell’integrazione scolastica gli insegnanti curricolari, ci sarebbero - secondo Giuseppe Felaco - vantaggi per gli alunni, con disabilità e non, sia per gli insegnanti stessi. «Diventa quindi indispensabile – scrive Felaco – un cambiamento di direzione che permetta di rafforzare questo fondamento»}{Giuseppe Felaco}
\maketitle
Sono sempre più convinto che a scuola le figure che ruotano intorno ai nostri ragazzi siano troppe e si può facilmente osservare che ciò genera solo confusione e la “perdita di vista” da parte di chi dovrebbe assumersi la presa in carico.

In realtà la figura fondamentale - per realizzare l'integrazione scolastica degli alunni con disabilità – è l'insegnante curricolare, che dovrebbe diventare il fondamento sul quale porre le basi per raggiungere gli obiettivi che tutti ci prefiggiamo.

A questo punto, quindi, diventa indispensabile un cambiamento di direzione che permetta di rafforzare questo fondamento. Infatti, cambiare ora realmente le cose – e in profondità – sarebbe decisivo per i nostri ragazzi, perché è ora che ne hanno bisogno. Siamo ancora in tempo per rigenerare questo tessuto gravemente malato. I nostri soli nemici sono la rassegnazione, la frammentazione, la sfiducia, lo scetticismo e la cieca difesa del proprio tornaconto. Non ce ne sono altri.

Bisogna operare una scelta che funga da spartiacque tra la cattiva scuola e quella buona, che dia visibilità a tanti insegnanti curricolari preparati, che aspettano solo di poter liberamente dimostrare le loro capacità professionali.

La chiave di volta sta dunque nel convertire i fondi che servono per pagare gli stipendi agli insegnanti di sostegno in incentivi mensili da assegnare agli insegnanti curricolari, che hanno una classe dove sia presente un alunno con disabilità. Nel caso poi si dovessero presentare delle difficoltà o l'insegnante curricolare ritenesse di non farcela a gestire la classe, egli potrebbe richiedere l'aiuto dell'insegnante di sostegno, da remunerare attingendo dal surplus assegnatogli in precedenza. A questo punto entrerebbero in gioco gli unici strumenti idonei: i compagni, tanti compagni, che diventerebbero i mezzi con cui gli insegnanti potrebbero realizzare la vera integrazione.
Tutti avrebbero dei vantaggi.

Gli alunni svilupperebbero la disponibilità, la sensibilità, la volontà, il carattere, il senso pratico, la capacità di comunicare e quella di organizzarsi. Ciò servirebbe anche per accrescere l'autostima e ridistribuire le posizioni del valore in classe degli alunni.

Gli alunni “speciali” avrebbero la luce negli occhi e la risata gioiosa di chi, rifiutato dal gruppo, verrebbe finalmente accettato, insieme alla ritrovata motivazione per studiare e ai sorrisi pieni di gratitudine, agli abbracci e ai “grazie” di genitori riconoscenti, per studenti “perduti” che invece decidono di continuare e di farcela.

Gli insegnanti, infine, percepirebbero finalmente uno stipendio adeguato al lavoro svolto e alla loro professionalità. Oltre alla soddisfazione interiore di aver “fatto la differenza”, di aver realizzato qualcosa che conta veramente, di aver lasciato un segno indelebile per il futuro, per così tante persone, per così tanto tempo.\footcite{Felaco2013}


Genitore.
26 aprile 2013
© Riproduzione riservata
 
 
