\author{Salvatore Nocera}
\title{INVALSI: perché nessun alunno con disabilità \cit{fa media}?}
\label{nocera150513}
\begin{abstract}
Non è infatti \cit{strano}, se non discriminatorio, escludere dalla media statistica delle prove INVALSI –- in corso di svolgimento nelle varie scuole –- i risultati di tutti gli alunni con disabilità, compresi quelli con disabilità motoria alle gambe, con minorazioni visive o audiolesioni, ovvero studenti che le possono svolgere in identiche condizioni dei compagni? Un chiarimento si imporrebbe, da parte del Ministero
\end{abstract}
%\epigraph{Non è infatti “strano”, se non discriminatorio, escludere dalla media statistica delle prove INVALSI – in corso di svolgimento nelle varie scuole – i risultati di tutti gli alunni con disabilità, compresi quelli con disabilità motoria alle gambe, con minorazioni visive o audiolesioni, ovvero studenti che le possono svolgere in identiche condizioni dei compagni? Un chiarimento si imporrebbe, da parte del Ministero}{Salvatore Nocera}
\maketitle
Nel 2007 era stata la Legge 176/07 a introdurre nella scuola le prove standardizzate INVALSI (Istituto Nazionale per la Valutazione del Sistema Educativo di Istruzione e di Formazione), nell'ipotesi di ottenere una valutazione oggettiva dei risultati e quindi non dipendente dalla soggettività di giudizio dei singoli docenti.
Senza qui entrare nel merito della bontà dei contenuti delle prove INVALSI svolte negli anni scorsi anni e della criticità di questo strumento “sedicente oggettivo”, ci si limiterà qui a sintetizzare i punti salienti della Nota dell'INVALSI del 23 aprile 2013, relativamente alla somministrazione delle prove stesse agli alunni con Bisogni Educativi Speciali, per quest'anno scolastico.

In premessa quella Nota stabilisce innanzitutto che «l'esito delle prove non entra a far parte della valutazione dei singoli alunni» e che «i risultati ottenuti da tutti gli alunni certificati con disabilità o diagnosticati con DSA [disturbi specifici di apprendimento, N.d.R.] non entreranno a far parte della media statistica dei risultati delle prove ».
Viene quindi stabilito che per gli alunni con disabilità intellettiva, la scuola, tramite il proprio Dirigente Scolastico, decida fra tre possibilità: non farli partecipare per nulla alle prove; farli partecipare con l'insegnante per le attività di sostegno o l'assistente in un'aula separata, per non disturbare e per non inficiare le modalità di somministrazione ai compagni; farli partecipare in classe con i compagni, ma senza assistenza e/o lettura ad alta voce delle prove.
E ancora, per gli alunni con minorazione visiva è consentito l'uso di prove trascritte in braille o in formato elettronico con sintesi vocale in cuffia o in un'altra aula, mentre per quelli diagnosticati con \glslink{dsaa}{DSA}, sono consentiti gli strumenti dispensativi e compensativi, sempre con la possibilità della scelta tra l'aula dei compagni o un'altra.
Per tutti gli alunni di questi gruppi, infine, sono permessi tempi di somministrazione più lunghi (massimo 30 minuti).

La novità maggiore di quest'anno è costituita poi dalla previsione degli alunni con altri BES (Bisogni Educativi Speciali), ai sensi della recente Direttiva Ministeriale del 27 dicembre 2012\footcite{dir27Dic2012} (Strumenti d'intervento per alunni con Bisogni Educativi Speciali e organizzazione territoriale per l'inclusione scolastica) e della successiva e conseguente Circolare 8/13\footcite{cm8_2013} del 6 marzo scorso. Per loro – nel corrente anno – non è prevista alcuna norma particolare e quindi vengono trattati come tutti gli alunni senza BES: «Si precisa che gli allievi afferenti all'“Area dello svantaggio socioeconomico, linguistico e culturale” (in base alla definizione della C.M. n. 8/13) NON sono dispensati dallo svolgimento ordinario delle prove INVALSI. Tali allievi devono svolgere regolarmente le prove senza alcuna variazione né dei tempi, né delle modalità di svolgimento delle stesse». [...] «Per il presente anno scolastico», infine, essi «NON devono essere segnalati come BES» (aggiornamento del 29 aprile scorso alla Nota INVALSI del 23 aprile precedente).

Un'osservazione conclusiva si impone: non è “strano” – se non addirittura discriminatorio – stabilire che tutti gli alunni con disabilità debbano essere esclusi dalla media statistica dei risultati delle prove? Un alunno con disabilità motoria alle gambe, ad esempio, non è nelle identiche condizioni dei compagni non disabili? Così come non si capisce perché gli alunni minorati della vista e resi autonomi grazie alle nuove tecnologie non possano rientrare nell'analisi statistica dei risultati. Senza naturalmente trascurare – a maggior ragione, anzi – l'incomprensibile esclusione degli alunni audiolesi che devono svolgere le prove scritte, come tutti gli altri.
Su tale aspetto, quindi, si renderebbe opportuno un chiarimento da parte del Ministero, dal momento che la Nota esaminata non proviene da esso, ma dell'INVALSI.

Vicepresidente nazionale della \glslink{fisha}{FISH}  (Federazione Italiana per il Superamento dell'Handicap). Responsabile del Settore Legale dell'Osservatorio Scolastico dell'AIPD (Associazione Italiana Persone Down). Il presente testo è il riadattamento di una scheda apparsa anche nel sito dell’AIPD.

15 maggio 2013
Ultimo aggiornamento: 16 maggio 2013 10:24