%C
\newglossaryentry{ctsg}{name={CTS},  
	description={Centri territoriali di Supporto, rete pubblica di Centri per gli ausili. Tale rete, distribuita uniformemente su tutto il territorio italiano, offre consulenze e formazione a insegnanti, genitori e alunni sul tema delle tecnologie applicate a favore degli alunni disabili.Vedi:\glslink{ntdg}{NTD} }}
%I
\newglossaryentry{icfg}{name={ICF},  
	description={Classificazione Internazionale del Funzionamento, della Disabilità e della Salute, definita dall'Organizzazione Mondiale della Sanità nel 2001}} 
%\newglossaryentry{idcg}{name={IDC},  
%	description={Classificazione internazionale delle malattie e dei problemi correlati, stilata dall'Organizzazione mondiale della sanità (OMS-WHO)}}
\newglossaryentry{icdg}{name={ICD},  
	description={ICD (dall'inglese International Classification of Diseases; in particolare, International Statistical Classification of Diseases, Injuries and Causes of Death) è la classificazione internazionale delle malattie e dei problemi correlati, stilata dall'Organizzazione mondiale della sanità}}
%F
\newglossaryentry{filg}{name={FIL},  
	description={Condizione clinica caratterizzata da un funzionamento cognitivo borderline (una zona di confine tra normalità e ritardo mentale): il Quoziente Intellettivo, misurato con reattivi di livello standardizzati, oscilla tra 71 e 84}}
%G

\newglossaryentry{glipg}{name={GLIP},  
	description={Hanno compiti di consulenza e proposta
		al Dirigente scolastico regionale, di consulenza alle singole scuole, di collaborazione con gli enti
		locali e le unità sanitarie locali per la conclusione e la verifica dell'esecuzione degli accordi di
		programma per l'impostazione e l'attuazione dei piani educativi individualizzati, nonché per
		qualsiasi altra attività inerente all'integrazione degli alunni in difficoltà di apprendimento” (legge
		104/92\footcite{Legge_104_92}, art. 15)}}
\newglossaryentry{glirg}{name={GLIR},  
	description={Attivano ogni possibile iniziativa finalizzata alla stipula di Accordi di
		programma regionali per il coordinamento, l’ ottimizzazione e l'uso delle
		risorse, riconducendo le iniziative regionali ad un quadro unitario
		compatibile con i programmi nazionali d'istruzione e formazione e con
		quelli socio - sanitari\footcite{LineGuida2009}}}
%L
\newglossaryentry{l2g}{name={L2},  
	description={L2 si intende in linguistica e in glottodidattica qualsiasi lingua che venga appresa in un secondo momento rispetto alla madrelingua o lingua materna, a sua volta chiamata in gergo tecnico L1}}
%N
\newglossaryentry{ntdg}{name={NTD},  
	description={Il Progetto interministeriale “Nuove Tecnologie e Disabilità”, cofinanziato dal Dipartimento per l'Innovazione Tecnologica della Presidenza del Consiglio dei Ministri e dal Ministero della Pubblica Istruzione, è articolato in sette azioni, indipendenti ma coordinate, che hanno l’obiettivo di integrare la didattica speciale con le risorse delle nuove tecnologie, al fine di sostenere l'apprendimento e l’inclusione nella scuola degli alunni in situazione di disabilità}
		}
%P
\newglossaryentry{pdpg}{name={PDP},  
	description={Piano Didattico Personalizzato per gli studenti con disturbi specifici dell'apprendimento}}
\newglossaryentry{peig}{name={PEI},  
	description={Piano Educativo Individualizzato per gli studenti con disabilità}}
%S
\newglossaryentry{sidig}{name={SIDI},  
	description={Il SIDI è un sistema centralizzato, caratterizzato da un'unica interfaccia, accessibile ovunque via web, che offre alle scuole le funzionalità necessarie allo svolgimento delle operazioni gestionali, amministrative e contabili. Il portale è il punto di accesso unico a tutti i servizi, per utenti istituzionali ed esterni, e offre servizi differenziati in funzione delle autorizzazioni e del profilo dell'utente. In questo modo i dati sono patrimonio delle singole scuole, ma vanno anche ad alimentare una base dati unica, gestita a livello centrale, integrata e completa delle informazioni di tutte le scuole, statali e non statali.}}




