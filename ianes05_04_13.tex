\author{Dario Ianes}
\title{Bisogni Educativi Speciali su base ICF: un passo verso la scuola inclusiva}
\phantomsection
\label{cha:scataglini170613}
\begin{abstract}
In questo stimolo alla riflessione e alla discussione vorrei sostenere una tesi che ritengo, e non
da oggi, particolarmente importante per lo sviluppo delle qualità inclusive della Scuola italiana
(Ianes, 2005a, 2005b; Ianes e Macchia, 2008). Credo che leggere le situazioni di alcuni alunni
attraverso il concetto di Bisogno Educativo Speciale (BES), fondato su base ICF, possa far fare alla
nostra Scuola un significativo passo in avanti verso la piena inclusione.
\end{abstract}
\maketitle
\datapub{5 Aprile 2013}
\section*{Alcune considerazioni preliminari}
\begin{enumerate}
	\item Il concetto di "bisogno" ha anche connotazioni negative nella nostra lingua e, per qualche
	aspetto, anche in alcune teorie psicologiche. Credo che questa "negatività" possa condizionare
	alcune attuali posizioni critiche nei confronti del concetto di BES, ma questo effetto alone
	improprio va superato. Esaminando infatti alcune teorie psicologiche che si sono occupate di
	bisogni (Maslow, Murray, Lewin) e posizioni filosofiche come quella di Heidegger, credo si possa
	pensare al concetto di bisogno non tanto come ad una mancanza, privazione o deficienza, in sé
	negativa, ma come ad una situazione di dipendenza (interdipendenza) della persona dai suoi
	ecosistemi, relazione che (se tutto va bene) porta alla persona che cresce alimenti positivi per il
	suo sviluppo. In altre parole, cresco bene in apprendimenti e partecipazione se questa relazione
	"gira bene" e posso trovare risposte e alimenti adeguati al mio sviluppo.
	\item I sostenitori del modello sociale della disabilità e molti studiosi dell'area dei Disability
	Studies non amano molto il concetto di "speciale", né applicato ai bisogni, né agli interventi
	educativi e didattici. Lo ritengono negativo, svalorizzante ed espressione di una costruzione sociale
	emarginante che opera per l'oppressione e la marginalizzazione di particolari gruppi minoritari di
	persone, tra cui chi ha qualche differenza rispetto ad aspettative e standard culturalmente e
	storicamente determinati, e dunque arbitrari. Sarebbe molto meglio, ad esempio secondo Vehmas,
	pensare che "ci sono solo bisogni che sono unici in ogni individuo" ( 2010, p. 92). Questo è
	sicuramente vero, ma ci sono situazioni problematiche di funzionamento che fanno diventare
	speciali i bisogni normalissimi e unici: in tali situazioni diventa più complesso fare in modo che i
	bisogni ottengano risposte adeguate. Dunque situazioni più complesse che richiedono azioni più
	complesse, "speciali". Sicuramente esistono dinamiche di oppressione e marginalizzazione, ma
	esistono anche situazioni problematiche "in sé", al di là dei meccanismi di potere, che ovviamente
	vanno smascherati e combattuti. Non credo dunque che pensare alle situazioni problematiche
	nella scuola come "speciali" inneschi meccanismi emarginanti, anche se esiste certamente il rischio del labeling e della possibile arbitrarietà dei processi decisionali che ci fanno dire che un
	particolare funzionamento individuale è "problematico" (vedi più avanti). Temo che una posizione
	critica radicale alle istituzioni formative e sociali, come quella espressa da molti esponenti dei
	Disability Studies, ci privi di possibilità operative positive nella realtà attuale della scuola. Spesso,
	come sappiamo, l'ottimo è nemico del bene.
	\item Nella nostra Scuola c'è da tempo l'integrazione degli alunni con disabilità (fatta più o meno
	bene; si veda Canevaro et al. 2007; Ianes e Canevaro, 2008), ma siamo ancora lontani
	dall'inclusione, e cioè dal riconoscere e rispondere efficacemente ai diritti di individualizzazione di
	tutti gli alunni che hanno una qualche difficoltà di funzionamento.
	
	Una Scuola che sa rispondere adeguatamente a tutte le difficoltà degli alunni e sa prevenirle,
	ove possibile, diventa poi una Scuola davvero e profondamente inclusiva per tutti gli alunni, dove si
	eliminano le barriere all'apprendimento e alla partecipazione di ognuno. Questo è il traguardo a
	cui tendere, traguardo che è ormai ben discusso anche nella letteratura scientifica internazionale
	più avanzata (Booth e Ainscow, 2008) e anche nelle posizioni degli interpreti italiani dei Disabilities
	Studies (Medeghini, D'Alessio, Marra, Vadalà e Valtellina, 2013). Nella letteratura scientifica
	internazionale il concetto di "Inclusione" si applica infatti a tutti gli alunni, come garanzia diffusa e
	stabile di poter partecipare alla vita scolastica e di raggiungere il massimo possibile in termini di
	apprendimenti e partecipazione sociale. La scuola inclusiva dovrebbe allora mettere in campo tutti
	i facilitatori possibili e rimuovere tutte le barriere all'apprendimento e alla partecipazione di tutti
	gli alunni, al di là delle varie etichette diagnostiche. Nel 2005 sostenevo invece di usare la coppia di
	concetti Inclusione-Bisogni Educativi Speciali in modo tattico (consapevole della differenza di tale
	posizione rispetto agli studi sull'Inclusive Education) per stimolare un primo allargamento della
	cultura del riconoscimento politico dei bisogni e delle relative risorse per l'individualizzazione
	anche a quegli alunni in difficoltà varie ma senza diagnosi di disabilità. Allora non esisteva ancora la
	Legge 170 del 2010 sui DSA e la Direttiva Ministeriale del 2012 sui BES.
	\item Il concetto di BES non ha alcun valore clinico, ma "politico" e dunque dovrebbe agire nei
	contesti delle politiche di riconoscimento dei diritti e di allocazione delle risorse. L'eventuale sua
	utilità dovrà essere in questi contesti, in cui recentemente è apparso con evidenza anche a livello
	nazionale con la Direttiva di Dicembre e la Circolare di Marzo.
	\item Non racchiudiamo la nostra riflessione e il dibattito nel recinto degli atti amministrativi del
	MIUR sopra citati, naturalmente abbiamo tutti considerato le varie prese di posizione e commenti,
	ma le analisi che cercheremo di fare devono essere più ampie. Nel merito di questi due
	provvedimenti, ritengo comunque che essi siano passi avanti verso una scuola più inclusiva, anche
	se il concetto di BES è ancora prevalentemente centrato sulle patologie e non sul funzionamento
	umano ICF e quello di inclusione, di conseguenza, è ancora visto come estensione ad alcuni alunni
	(con BES) di azioni individuali di personalizzazione-individualizzazione (peraltro necessarie)
	piuttosto che strutturazione diffusa di Didattiche inclusive.
\end{enumerate}
\section*{Ma qual è la reale utilità del concetto di Bisogno Educativo Speciale?}
Come si vedrà nel dettaglio nelle pagine che seguono, il concetto di Bisogno Educativo
Speciale è, a mio modo di vedere, una macrocategoria che comprende dentro di sé tutte le
possibili difficoltà educative-apprenditive degli alunni, sia le situazioni considerate
tradizionalmente come disabilità mentale, fisica, sensoriale, sia quelle di deficit in specifici
apprendimenti clinicamente significative, la dislessia, il disturbo da deficit attentivo, ad esempio, e
altre varie situazioni di problematicità psicologica, comportamentale, relazionale, apprenditiva, di
contesto socio-culturale, ecc.

Tutte queste situazioni sono diversissime tra di loro, ma nella loro clamorosa diversità c'è però
un dato che le avvicina, e che le rende, a mio avviso, sostanzialmente uguali nel loro diritto a
ricevere un'attenzione educativo-didattica sufficientemente individualizzata ed efficace: tutte
questi alunni hanno un funzionamento per qualche aspetto problematico, che rende loro più
difficile trovare una risposta adeguata ai propri bisogni.
Si obietterà che non ha senso creare una nuova macrocategoria, se esistono già le singole
categorie che la compongono. Non è cioè sufficiente parlare ad esempio, di ritardo mentale,
dislessia, disturbi della condotta, depressione, ecc.? Per ribattere questo punto entriamo nel
dettaglio del nostro ragionamento.
\section*{“Bisogno Educativo Speciale” non è una diagnosi clinica, ma una dimensione pedagogico-politica.}
Dobbiamo distinguere bene la nostra proposta che mira ad una lettura equa di tutti i bisogni
degli alunni da una modalità di riconoscimento-comprensione di una situazione problematica che
operi attraverso una diagnosi clinica di tipo nosografico ed eziologico, che rileva segni e sintomi e li
attribuisce a una serie di cause che li hanno prodotti. È ovviamente utile fare bene questo tipo di
diagnosi, ove possibile (si pensi alla dislessia, ai disturbi dello spettro autistico, ecc.) ma è un tipo
di riconoscimento che divide e distingue le difficoltà degli alunni in base alla loro causa, come fa la
nostra legislazione, la Legge 104 del 1992, e i successivi atti che regolano l'attribuzione di risorse
aggiuntive alla Scuola per far fronte alle difficoltà degli alunni, dando legittimità soltanto ai bisogni
che hanno un fondamento chiaro nella minorazione del corpo del soggetto, minorazione che deve
essere stabile o progressiva.

Una diagnosi nosografica ed eziologica è ovviamente fondamentale per progettare e realizzare
interventi riabilitativi, abilitativi, terapeutici, preventivi, epidemiologici, ecc., ma non ci aiuta a
fondare politiche di equità reale nelle nostre Scuole. È una diagnosi che frammenta, che consolida
appartenenze e categorie; abbiamo invece bisogno di un riconoscimento più ampio, e per questo
più equo, che non distingua tra bisogni di serie A, quelli evidentemente fondati su qualche
minorazione corporea e, con la Legge 170, su diagnosi cliniche, e bisogni di serie B, quelli per cui
non è chiara, o non c'è, una base corporea e/o clinica secondo i vari DSM o ICD.

Abbiamo bisogno di politiche eque di riconoscimento dei reali bisogni degli alunni, al di là
delle etichette diagnostiche. Può darsi infatti che un alunno con una situazione sociale e culturale
disastrosa abbia un funzionamento reale ben più compromesso e bisognoso di interventi (in una
Scuola davvero inclusiva) rispetto al funzionamento reale di un alunno con sindrome di Down, che
però può vantare un certissimo pedigree cromosomico. Il primo alunno non avrà, con la
legislazione e le prassi attuali, altrettanta tutela e risorse aggiuntive rispetto a quelle che spettano
al suo compagno con la sindrome di Down. E questo non è equo. Le recenti Direttiva Ministeriale
del 27 dicembre 2012 e Circolare 6 marzo 2013 sugli alunni con BES cercano di muoversi nella
direzione di un riconoscimento più equo di varie situazioni di difficoltà, ma risentono ancora
troppo di un'idea di BES legata alle varie patologie e ancora sganciata da un'antropologia
biopsicosociale del funzionamento umano fondata su ICF-OMS.

Le politiche di riconoscimento e legittimazione prioritaria dei bisogni "su base organica" sono
la logica conseguenza del dominio culturale (che diventa politico) del modello medico più
tradizionale, dove contano solamente le variabili biostrutturali. Ma se il corpo funziona bene, se
non è ammalato, si può dire che la persona ha buona salute, vive una situazione di benessere?

Secondo l'Organizzazione Mondiale della Sanità, la salute non è assenza di malattia, ma
benessere bio-psico-sociale, piena realizzazione del proprio potenziale, della propria capability
(Sen, 2011). Questo chiama fortemente in causa dimensioni sociali, culturali, economiche,
religiose, ecc. che non sono biostrutturali. Se accettiamo il dominio del modello medico
tradizionale saremo costretti a cercare sempre un'eziologia biostrutturale oppure a negare lo
status di reale difficoltà a una problematicità di funzionamento che non sia evidentemente causata
da menomazioni o danni fisici. E le situazioni problematiche di cui non si conoscono le cause?

Per una lettura e riconoscimento dei bisogni reali di un alunno, a noi interessa di più
comprendere la situazione attuale di funzionamento, per così dire «a valle» di una qualche
eziologia. Comprendere cioè l'intreccio di elementi che adesso, qui e ora, costituisce il
funzionamento di quell'alunno in quella serie di contesti.

A scuola si fanno i conti quotidianamente con i funzionamenti «a valle», con gli intrecci più
diversi di fattori personali e sociali, che nel tempo rendono molto differenti i funzionamenti anche
di persone «uguali» per alcuni aspetti biostrutturali.

Infatti, esistono forse due alunni con sindrome di Down uguali? Fino a non molti anni fa lo si
pensava, perché il dominio del modello medico tradizionale era assoluto. Ora non è più così, e
conseguentemente dobbiamo attrezzarci concettualmente e con coerenti prassi legislative e
attuative per dare piena cittadinanza e riconoscimento a ogni forma di funzionamento
problematico, a prescindere dall'origine.

Molte forze spingono attualmente in questa direzione. Ne cito solo alcune per farci riflettere.

Quante volte gli psicologi o i neuropsichiatri delle ASL hanno certificato come disabili (secondo
la Legge 104 e l'Atto di Indirizzo del febbraio 1994) alunni che invece avevano altre situazioni di
difficoltà? Per queste prassi di certificazione «indebitamente» allargate e generose si sono levate
alte le grida di allarme $\dots$ ma di chi? Di chi non vuole che si allarghi troppo la «marea» degli
insegnanti di sostegno? Di chi vuole conservare i «privilegi» delle competenze di alunni con
disabilità tradizionalmente protette? Di chi vede nel ricorso alla certificazione facile l'incapacità
della Scuola di far fronte con le proprie competenze alle difficoltà degli alunni? Di chi vede nella
certificazione facile il meccanismo chiave per ottenere posti di lavoro? Questo aumento di
certificazioni ci deve far pensare però anche al fatto che esistono davvero, e sono tante, le
situazioni di reale difficoltà, per le quali una Scuola davvero inclusiva deve organizzare adeguate
risorse per l'individualizzazione.

Un'altra forza che spinge nella direzione da noi auspicata è la diffusione forte e convinta che il
modello ICF dell'Organizzazione Mondiale della Sanità ha avuto e ha tuttora in Italia, diffusione
enormemente maggiore rispetto ad altri Paesi europei. Il modello ICF, come si vedrà nelle pagine
seguenti, è radicalmente bio-psico-sociale, ci obbliga cioè a considerare la globalità e la
complessità dei funzionamenti delle persone, e non solo gli aspetti biostrutturali. Questo è stato il
motivo per cui si è fondato proprio su ICF il nostro concetto di Bisogno Educativo Speciale (Ianes,
2005a), che assume così un significato ben diverso da quello in uso abitualmente nella letteratura
anglosassone.

Nel testo del 2005 appena citato avevo sostenuto l'importanza di usare il concetto di Bisogno
Educativo Speciale riferendomi anche alla letteratura psicopedagogica e alla normativa del Regno
Unito e in parte degli Stati Uniti, dove viene largamente utilizzato.

Credo sia importante ricordare alcuni passaggi dell'UNESCO che riguardano nello specifico la
definizione di BES. Il primo affronta subito quello che ritengo essere il punto fondamentale e cioè
l'"estensione" di questa condizione.

Il concetto di Bisogno Educativo Speciale si estende al di là di quelli che sono inclusi nelle
categorie di disabilità, per coprire quegli alunni che vanno male a scuola (failing) per una varietà di
altre ragioni che sono note nel loro impedire un progresso ottimale. (Unesco, 1997)

Come si vede, già nel 1997, l'Unesco cerca di definire il Bisogno Educativo Speciale con un
concetto più esteso di quello che veniva tradizionalmente incluso nelle categorie di disabilità.

L'Agenzia europea per lo sviluppo dell'educazione per bisogni speciali, in una delle sue analisi
delle tendenze inclusive dei sistemi scolastici di diciotto Paesi europei, nel novembre 2003,
confronta le varie prassi scolastiche e le varie normative di riferimento, attribuendo un approccio
inclusivo totale a Italia, Spagna, Grecia, Portogallo, Svezia, Islanda e Norvegia, e conclude in questo
modo:

In quasi tutti questi Paesi il concetto di Bisogno Educativo Speciale è nell'agenda. Sempre più
persone sono convinte che l'approccio medico dovrebbe essere sostituito con uno più educativo: il
focus centrale si è spostato sulle conseguenze della disabilità per l'educazione. Però, allo stesso
tempo è chiaro che questo approccio è molto complesso e i Paesi stanno discutendo delle
implicazioni pratiche di questa filosofia. (Meijer, 2003, p. 126)


Nel 2001, lo Special Educational Needs and Disability Act del Regno Unito e il relativo Code of
Practice specificano meglio i bisogni del bambino, raggruppandoli in quattro grandi aree:
comunicazione e interazione, cognizione e apprendimento, sviluppo del comportamento,
emozione e socialità, attività sensoriale e fisica. Sulla base di questa ripartizione dei bisogni
vengono meglio specificati i Bisogni Educativi Speciali:
\begin{quote}
 \mancatesto Si trovano difficoltà di apprendimento, generale e specifiche, difficoltà comportamentali,
 emozionali e sociali, difficoltà di comunicazione e di interazione, difficoltà di linguaggio, disturbo
 dello spettro autistico, difficoltà sensoriali e fisiche, minorazioni uditive, minorazioni visive,
 difficoltà fisiche e mediche. (Department for Education and Skills, 2001, p. 45)
\end{quote}
Nel Manuale del Coordinatore scolastico per i BES ( The SENCO Handbook; Cowne, 2003, p.
14) si specifica ulteriormente che il bambino con BES
\begin{quote}
	\mancatesto è un bambino che non risponde nella maniera attesa al curricolo o non riesce a
	fronteggiare il normale ambiente di classe senza aiuto aggiuntivo.
\end{quote} 

Da quella analisi di varie concettualizzazioni e testi normativi risultava chiaro che nel concetto di
Bisogno Educativo Speciale entravano tutte le varie difficoltà/disturbi dell'apprendimento, del
comportamento, e altre problematicità. Questo allargamento e questo riconoscimento ufficiale
sono ovviamente positivi rispetto alla nostra legislazione più restrittiva in senso biostrutturale, ma
non sono ancora sufficienti, perché non includono altre condizioni particolari (come, ad esempio,
l'essere migranti e non conoscere l'italiano) che invece devono essere considerate come Bisogno
Educativo Speciale, se fondiamo questo concetto sul modello base di human functioning di ICF.

In Italia ICF si è diffuso con forza nel mondo dell'educazione e della Scuola, grazie anche al
fatto che ha trovato una forte affinità con la cultura pedagogica italiana e con la sua visione
antropologica, molto sociale e legata ai contesti di vita. Non succede lo stesso in altri Paesi europei,
dove la cultura pedagogica ha seguito sviluppi diversi dalla nostra e dove ICF viene addirittura
osteggiato da chi segue una visione culturale e sociale delle difficoltà e disabilità perché ritenuto —
a torto — troppo «medico» (Terzi, 2008).

L'Intesa Stato-Regioni, siglata il 20 marzo 2008 sulla presa in carico globale dell'alunno con
disabilità, prevede per la prima volta a chiare lettere l'uso di ICF come modello antropologico su
cui fare la diagnosi funzionale per gli alunni con disabilità: «La Diagnosi Funzionale è redatta
secondo i criteri del modello bio-psico-sociale alla base dell'ICF dell'Organizzazione Mondiale della
Sanità» (art. 2, comma 2).

Un'altra testimonianza della forza del trend di riconoscimento dei Bisogni Educativi Speciali è
la Legge provinciale n.5 del 7 agosto 2006, «Sistema educativo di istruzione e formazione del
Trentino», una legge di riforma complessiva della Scuola di questa Provincia, che parla
esplicitamente di alunni con Bisogni Educativi Speciali: «attivare servizi e iniziative per il sostegno e
l'integrazione degli alunni con bisogni educativi speciali, derivanti da disabilità, da disturbi e da
difficoltà di apprendimento ovvero da situazioni di svantaggio determinate da particolari condizioni
sociali o ambientali» (art. 2, comma 1, lettera h).
\section*{Sempre più difficoltà ed eterogeneità degli alunni}
Sono sempre di più gli alunni che per una qualche difficoltà di «funzionamento» preoccupano
gli insegnanti e le famiglie. Sulle forme, buone e meno buone, che assume questa preoccupazione
torneremo più avanti. Occupiamoci ora più da vicino di queste varie e diversissime difficoltà.

Nelle classi si trovano molti alunni con difficoltà nell'ambito dell'apprendimento e dello
sviluppo di competenze. In questa grande categoria possiamo includere varie difficoltà: dai più
tradizionali disturbi specifici dell'apprendimento (dislessia, disgrafia, discalculia), al disturbo da
deficit attentivo con o senza iperattività, a disturbi nella comprensione del testo, alle difficoltà
visuo-spaziali, alle difficoltà motorie, alla goffaggine, alla disprassia evolutiva, ecc. Troviamo anche
gli alunni con ritardo mentale e ritardi nello sviluppo, originati dalle cause più diverse. Hanno una
difficoltà di apprendimento e di sviluppo anche alunni con difficoltà di linguaggio o disturbi
specifici nell'eloquio e nella fonazione. Ci sono anche gli alunni con disturbi dello spettro autistico,
dall'autismo più chiuso con ritardo mentale alla sindrome di Asperger o all'autismo ad alto
funzionamento. Accanto a questi alunni con aspetti patologici nell'apprendimento e nello sviluppo
troviamo anche alunni che hanno «soltanto» un apprendimento difficile, rallentato, uno scarso
rendimento scolastico.

Nelle classi troviamo anche alunni con varie difficoltà emozionali: timidezza, collera, ansia,
inibizione, depressione, ecc. Forme più complesse di difficoltà sono invece quelle riferibili alla
dimensione psichica e psicopatologica: disturbi della personalità, psicosi, disturbi
dell'attaccamento o altre condizioni psichiatriche.

Più frequenti però sono le difficoltà comportamentali e nelle relazioni: dal semplice
comportamento aggressivo fino ad atti autolesionistici, bullismo, disturbi del comportamento
alimentare, disturbi della condotta, oppositività, delinquenza, uso di droghe, ecc. La sfera delle
relazioni produce infatti molto spesso delle difficoltà nell'ambito psicoaffettivo, rivolte
prevalentemente all'interno: bambini isolati, ritirati in sé, eccessivamente dipendenti, passivi, ecc.

Gli insegnanti possono incontrare difficoltà educative e didattiche, oltre che psicologiche,
anche con alunni che hanno delle compromissioni fisiche rilevanti, traumi, esiti di incidenti,
menomazioni sensoriali, malattie croniche o acute, allergie, disturbi neurologici, paralisi cerebrali
infantili, epilessie, ecc.

L'ambito familiare degli alunni, inoltre, può creare anche notevoli disagi: pensiamo alle
situazioni delle famiglie disgregate, patologiche, trascuranti o con episodi di abuso o di
maltrattamento, agli alunni che hanno subito eventi drammatici come ad esempio lutti o
carcerazione dei familiari, o che comunque vivono alti livelli di conflitto.

Accanto a queste difficoltà, un insegnante ne conosce molte altre di origine sociale ed
economica: povertà, deprivazione culturale, difficoltà lavorative ed esistenziali, ecc. Sempre più poi
nella Scuola italiana troviamo alunni che provengono da ambiti culturali e linguistici anche molto
diversi: il caso degli alunni migranti è evidente, ma è chiara anche la difficoltà che può avere un
alunno per un percorso di precedente scolarità particolarmente difficile, per problemi relazionali o
di apprendimento con altri insegnanti.

Il mondo della Scuola è inoltre sempre più attento anche a quelle difficoltà «soft» che si
manifestano con problemi motivazionali, disturbi dell'immagine di sé e dell'identità, deficit di
autostima, insicurezza e disorientamento del progetto di vita. Credo che un insegnante esperto e
sensibile conosca bene questa multiforme e sfaccettata galassia di difficoltà, le più varie, le più
diverse, che si trovano sempre più spesso nella nostre classi, legate ognuna a una singola storia di
un singolo bambino e delle sue ecologie di vita.

Ogni insegnante sa bene, per esperienza diretta, che gli alunni che lo preoccupano per
qualche forma di difficoltà sono ben di più di quel 2-3\% di alunni certificati dall'Azienda Sanitaria
con una disabilità. Queste percentuali sono soltanto la punta di un iceberg: sotto vi sta almeno un
10-15\% di alunni con qualche altro tipo di difficoltà.

Nella percezione degli insegnanti si trova molto spesso l'impressione che questi casi
aumentino sempre di più, che le difficoltà, di vario genere, siano sempre più presenti nella nostre
classi. Credo che diversi fattori contribuiscano a questa percezione di incremento e di maggiore
diffusione. Oggettivamente alcune condizioni sono in reale aumento dal punto di vista epidemiologico: i disturbi dell'attenzione, il bullismo, le condizioni varie dello spettro autistico.

Però, accanto a questo aumento oggettivo, dobbiamo anche tener conto di altri fenomeni
concorrenti: da un lato la sempre maggiore capacità diagnostica di psicologi e neuropsichiatri, oltre
che di quelle figure professionali preziose, come i logopedisti e gli psicomotricisti, che si occupano
sempre più di apprendimento e della sua psicopatologia sulla base di modelli teorici e applicativi
tratti dalle teorie dell'apprendimento, dalla psicologia cognitiva, dalla neuropsicologia e dalle
neuroscienze, piuttosto che dalle teorie psicoanalitiche e psicodinamiche. Dall'altro si riscontra la
sempre maggiore capacità osservativa e interpretativa degli insegnanti, che riescono ad accorgersi
sempre meglio delle varie condizioni di difficoltà. Professionalmente sono sempre di più gli
insegnanti in grado di cogliere le difficoltà di apprendimento, i deficit o i disagi. Molto spesso
anche i genitori si accorgono in tempo delle difficoltà dei figli e cercano un aiuto competente.

Sono dunque molti e molto diversi gli alunni che preoccupano gli insegnanti.

In questi alunni, i bisogni educativi normali, e cioè quelli di sviluppo delle competenze, di
appartenenza sociale, di identità autonoma, di valorizzazione e di autostima, di accettazione, solo
per citarne alcuni, diventano bisogni speciali, più complessi, in quanto è più difficile ottenere una
risposta adeguata a soddisfarli. E questo per una «difficoltà di funzionamento» biopsicosociale
dell'alunno. Da qui il concetto di Bisogno Educativo Speciale su base ICF.

La percezione di difficoltà da parte degli insegnanti deve essere letta anche sullo sfondo di una
sempre crescente consapevolezza dell'eterogeneità delle classi. Gli insegnanti si rendono conto
sempre di più che le classi sono abitate, di norma, da alunni che percepiscono essere sempre più
diversi. Ne vedono la diversità nei processi di apprendimento, negli stili di pensiero, nelle
dinamiche di relazione e di attaccamento, nei vissuti familiari, sociali e culturali. Se pensiamo
semplicemente al concetto di intelligenza, vedremo con facilità come ormai non esista più alcun
insegnante che pensi a un'unica intelligenza, diversamente ripartita tra i suoi alunni solo dal punto
di vista quantitativo.

I profili degli alunni diventano sempre più ricchi di sfumature psicologiche, relazionali,
motivazionali, identitarie, anche attraverso un uso consapevole e avanzato di modalità nuove di
valutazione autentica e di portfolio (Tuffanelli, 2004; Pavone, 2006). Le varie e diverse provenienze
culturali, geografiche e linguistiche completano l'opera.

Si incrociano dunque e si enfatizzano due percezioni di differenza: una legata alle difficoltà di
un singolo alunno, l'altra alle eterogeneità del contesto classe. Questo incrocio aumenta molto
spesso l'ansia degli insegnanti, dei dirigenti e delle famiglie. In alcuni casi, questa ansia porta a una
specie di affanno, a una sensazione di non essere in grado di rispondere con buona qualità
formativa, di individualizzare in modo sufficiente, di includere realmente nella vita scolastica
dell'apprendimento e delle relazioni, con risposte formative adeguate ed efficaci, tutti questi vari
alunni con differenze e difficoltà.

Qui sta l'esigenza dell'inclusione di poter rispondere con
un'individualizzazione/personalizzazione «sufficientemente buona» prioritariamente a tutti gli
alunni con Bisogni Educativi Speciali nell'ottica che, in prospettiva, tutti gli alunni, qualunque sia la
loro situazione di funzionamento possano raggiungere il loro massimo potenziale di
apprendimento e di partecipazione.
\section*{I BES come difficoltà evolutiva di funzionamento educativo e/o apprenditivo}
Approfondiamo il concetto di Bisogno Educativo Speciale, provando a definire i criteri per una
concettualizzazione valida e utile operativamente.

Innanzitutto dovrebbe essere una concettualizzazione che abbia le caratteristiche della
sensibilità, che riesca cioè a cogliere in tempo e precocemente il maggior numero possibile di
condizioni di difficoltà dei bambini. Ovviamente la sensibilità non dovrebbe essere utilizzata con
eccessiva ampiezza, per evitare che troppi bambini vengano considerati in situazione di BES. Un
concetto troppo ampio produrrebbe troppi «falsi positivi» e di conseguenza risulterebbe dannoso.
La questione del «cosa» fa scattare la definizione di problematicità, la questione del «confine» tra
stranezza e problematicità sarà discussa ampiamente più avanti.

Accanto alle caratteristiche della sensibilità credo debbano essere definite anche la
caratteristiche della reversibilità e della temporaneità della definizione di alunno con Bisogno
Educativo Speciale. Molte situazioni che si configurano senz'altro con BES non sono affatto stabili e
cristallizzate, anzi sono soggette a forti mutamenti nel tempo, a miglioramenti e di conseguenza
alla reversibilità. Credo che la definizione di Bisogno Educativo Speciale debba portare con sé
proprio questo senso di provvisorietà e di reversibilità, non in tutti casi, ma certo maggiormente
rispetto alle etichette diagnostiche tradizionali, più rigide e più stabili. Questa reversibilità
evidentemente facilita la famiglia e l'alunno stesso ad accettare un percorso di conoscenza e di
approfondimento della difficoltà e di successivo intervento di individualizzazione. Sono infatti note
a tutti le difficoltà che vive oggi una famiglia nell'intraprendere un percorso diagnostico che ha
come unico sbocco una diagnosi clinica e magari misure di supporto segreganti e stigmatizzanti.

Un'altra caratteristica importante e positiva del concetto di BES credo sia quella del minor
impatto stigmatizzante che questa definizione ha rispetto ad altre quali disabilità, dislessia,
discalculia, disturbo da deficit attentivo con iperattività, disturbo specifico di apprendimento, ecc.
Se il concetto di Bisogno Educativo Speciale deriva da un modello globale di funzionamento
educativo e apprenditivo ed è considerato come possibilmente transitorio e reversibile, allora
l'impatto psicologico e sociale di questa valutazione e riconoscimento sarà più lieve e meno
doloroso per l'alunno e la sua famiglia.

La concettualizzazione di BES che sostengo inoltre non dovrà fare riferimento alle origini
eziologiche dei disturbi né alle classificazioni patologiche, ma dovrà partire dalla situazione
complessiva di funzionamento educativo e apprenditivo del soggetto, qualunque siano le cause
che originano una difficoltà di funzionamento.

Tale concettualizzazione di Bisogno Educativo Speciale dovrà anche fondarsi sulla necessità di
individualizzazione, personalizzazione, di educazione/didattica speciale e di inclusione.
Un tentativo di definizione originale potrebbe dunque essere il seguente:
\begin{quote}
Il Bisogno Educativo Speciale è qualsiasi difficoltà evolutiva, in ambito educativo e/o
apprenditivo, che consiste in un funzionamento problematico come risultante
dall'interrelazione reciproca tra i sette ambiti della salute secondo il modello ICF
dell'Organizzazione Mondiale della Sanità. Il funzionamento è problematico per l'alunno,
in termini di danno, ostacolo o stigma sociale, indipendentemente dall'eziologia, e necessita di educazione/didattica speciale individualizzata.
\end{quote}
Esaminiamo nel dettaglio le singole componenti di questa definizione.

Un Bisogno Educativo Speciale è una difficoltà che si deve manifestare in età evolutiva, e cioè
entro i primi 18 anni di vita del soggetto.

Questa difficoltà si manifesta negli ambiti di vita dell'educazione e/o dell'apprendimento
scolastico/istruzione. Può coinvolgere, a vario livello, le relazioni educative, formali e/o informali,
lo sviluppo di competenze e di comportamenti adattivi, gli apprendimenti scolastici e di vita
quotidiana, lo sviluppo di attività personali e di partecipazione ai vari ruoli sociali. Anche un lieve
difetto fisico, che non incide affatto sulla funzionalità cognitiva e apprenditiva, può causare
difficoltà psicologiche e timore di visibilità sociale, limitando così la partecipazione del bambino a
varie occasioni educative e sociali, e così via.

Certamente si apprende per tutto l'arco della vita, ma i primi 18 anni sono sicuramente più
collegati al concetto di educazione e di istruzione formale. Per questo si può parlare correttamente
di Bisogno Educativo Speciale soltanto entro l'età evolutiva, anche se, ovviamente, esistono tanti
disturbi, a insorgenza nell'età adulta, che compromettono la sfera dell'apprendimento e della
partecipazione sociale e attività della persona.

Una componente della definizione è il concetto di funzionamento globale del soggetto, ovvero
di salute bio-psico-sociale della persona come buon funzionamento, frutto dell'interconnessione
dei vari ambiti, come sono stati definiti nel 2002 dal modello ICF dell'Organizzazione Mondiale
della Sanità e nella sua revisione per bambini e adolescenti del 2007. È proprio del funzionamento
globale del soggetto, della sua salute, globalmente e sistemicamente intesa, che dobbiamo
occuparci, che dobbiamo conoscere a fondo in tutte le sue varie interconnessioni, a prescindere
dalle varie eziologie che possono aver danneggiato singoli aspetti del funzionamento. Il modello
ICF ci fornisce un'ottima base concettuale per costruire una griglia di conoscenza del
funzionamento educativo e apprenditivo del soggetto, come vedremo tra poco.

Un'altra componente della definizione di BES riguarda la necessità di tracciare il confine tra
una deviazione di funzionamento problematica per il contesto familiare e/o per gli insegnanti, ad
esempio, ma non per l'alunno, e invece una deviazione di funzionamento realmente problematica
anche per l'alunno che la manifesta, oppure ancora una deviazione niente affatto problematica per
il contesto relazionale, ma problematica per l'alunno.

Potrebbe infatti accadere che le persone attorno al bambino vivano un problema di
funzionamento educativo-apprenditivo, ma questo problema sia esclusivamente loro e non del
bambino stesso (aspettative troppo rigide e convenzionali sui tempi di apprendimento della
lettura): in un caso del genere tale deviazione di funzionamento non dovrebbe essere «corretta» in
alcun modo. In quei casi invece in cui la difficoltà di funzionamento danneggia direttamente il
bambino oltre che gli altri, lo ostacola direttamente o lo stigmatizza, è evidente la necessità di
prendersene cura, anche se, in qualche caso, tale difficoltà non è vissuta come particolarmente
problematica e preoccupante dagli altri (ad esempio, un bambino passivo, timido, chiuso in sé,
poco intraprendente può essere vissuto come un bambino «tranquillo e riposante»).

In questa definizione è evidente una continuità tra Bisogno Educativo Speciale e normalità, un
continuum tra «normalità» e «problematicità», dove il punto di passaggio rischia di essere
arbitrario, se non vengono definiti dei criteri il più possibile oggettivi a tutela del soggetto,
dell'alunno. Più avanti saranno discussi più dettagliatamente i tre criteri di danno, ostacolo e
stigma sociale, che propongo come base il più possibile “oggettiva” della decisione di
problematicità.
\section*{I BES sulla base del modello ICF}
Come risulta evidente, in questa idea di Bisogno Educativo Speciale è centrale il concetto di
funzionamento educativo-apprenditivo. In questa dimensione del funzionamento, peraltro
totalizzante l'esperienza evolutiva di ogni bambino, si attribuisce un'enfasi prevalente al concetto di
apprendimento (nei più disparati ambiti) frutto dell'intreccio tra le varie spinte evolutive
endogene, per maturazione biologica programmata geneticamente, e le mediazioni educative degli
ambienti. Nei contesti delle varie forme di educazione, formale e informale, il bambino cresce
apprendendo, sviluppando competenze negli ambiti più diversi: cognitivo, linguistico,
interpersonale, motorio, valoriale, autoriflessivo, ecc. Il bambino funziona bene dal punto di vista
evolutivo se riesce a intrecciare positivamente le spinte biologiche alla crescita con le varie forme
di apprendimento, date dall'esperienza e dal contatto con le relazioni umane e gli ambienti fisici.

L'educazione media questo intreccio, nelle sue molteplici azioni quotidiane, fornendo stimoli,
guida, accompagnamento, feedback, significati, obiettivi e gratificazioni, modelli, ecc. e il bambino
funziona bene dal punto di vista educativo se integra questi messaggi con la sua spontanea
iniziativa e con le spinte biologiche.

Il funzionamento educativo è dunque intrecciato tra biologia, esperienze di ambienti e
relazioni e attività e iniziative del soggetto.

Per comprendere meglio questo intreccio e leggerlo nella mescolanza delle sue componenti
abbiamo bisogno di una cornice forte che orienti l'analisi, una cornice concettuale e antropologica
condivisa dalle varie ottiche e culture professionali. L'ICF dell'Organizzazione Mondiale della Sanità
(2002; 2007) è il modello concettuale che serve a questa lettura.

Rispetto al concetto di BES, credo sia molto appropriato proporre la struttura concettuale
dell'ICF, perché questo approccio parla di salute e di funzionamento globale, non di disabilità o di
varie patologie. Secondo l'Organizzazione Mondiale della Sanità, infatti, una situazione, e cioè il
funzionamento di una persona, va letto e compreso profondamente in modo globale, sistemico e
complesso. Credo dunque che questo modello sia utile per una lettura globale dei Bisogni Educativi
Speciali in un'ottica di salute e di funzionamento, come frutto di relazioni tra vari ambiti interni ed
esterni al bambino.

Come si desume dallo schema ICF/OMS, la situazione di salute di una persona, nel nostro caso
il suo funzionamento educativo-apprenditivo, è la risultante globale delle reciproche influenze tra i
sette fattori considerati.

Condizioni fisiche e fattori contestuali stanno agli estremi superiori e inferiori del modello: la
dotazione biologica da un lato e dall'altro l'ambiente in cui il bambino cresce, dove accanto ai
fattori esterni, come le relazioni, le culture, gli ambienti fisici, ecc. egli incontra anche fattori
contestuali personali, e cioè le dimensioni psicologiche che fanno da sfondo interno alle sue azioni,
per esempio, autostima, identità, motivazioni, ecc. Questi contesti potranno essere dei mediatori
facilitanti o delle barriere.

Nella grande dialettica fra queste due enormi classi di forze, biologiche e contestuali, si trova il
corpo del bambino, come concretamente si sta sviluppando dal punto di vista strutturale e come si
stanno sviluppando le varie funzioni, da quelle mentali a quelle motorie e di altro genere.

Il corpo del bambino agisce poi nei contesti sviluppando reali capacità e attività personali, e
partecipa socialmente ai vari ruoli, familiari, comunitari, scolastici, ecc.

Quando i vari fattori interagiscono in modo positivo, il bambino crescerà sano e funzionerà
bene dal punto di vista educativo-apprenditivo, altrimenti il suo funzionamento sarà difficoltoso,
ostacolato, disabilitato, ammalato, con Bisogni Educativi Speciali, ecc.

La comprensione il più possibile profonda e completa del funzionamento educativo-apprenditivo di
un bambino sarà possibile soltanto se riusciremo a cogliere le singole dimensioni ma soprattutto se
riusciremo a integrarle in una visione complessa e completa. Si tratta di vedere non le singole stelle
(le singole capacità, performance o fattori contestuali, ecc.), ma la costellazione che dà significato e
senso a una figura, a una serie di relazioni di interconnessione (Ianes e Biasioli, 2005).

L'alunno potrà avere una difficoltà di funzionamento, e cioè un Bisogno Educativo Speciale,
originata dalle infinite combinazioni possibili tra i 7 ambiti di funzionamento.

Una difficoltà di funzionamento potrà originarsi da condizioni fisiche problematiche: malattie
varie, acute o croniche, fragilità, allergie o intolleranze alimentari, patrimoni cromosomici
particolari, lesioni, traumi, malformazioni, disturbi del ciclo del sonno-veglia, disturbi del
metabolismo, della crescita, ecc. In questi casi il funzionamento globale è minacciato da un input
biologicamente significativo, che irrompe sulla scena e può condizionare in maniera drammatica
l'apprendimento e l'educazione. Spesso problemi in questo ambito portano a problemi anche
nell'ambito successivo, quello delle strutture corporee.

L'ambito delle strutture corporee può originare a sua volta difficoltà di funzionamento
educativo e apprenditivo, pensiamo al ruolo della malformazioni o mancanza di arti, organi o parti
di essi, come ad esempio strutture cerebrali o strutture necessarie per la fonazione o la
locomozione. È evidente come l'iniziativa e l'attività personale del bambino saranno più o meno
profondamente danneggiati da deficit strutturali nel corpo. Le strutture del corpo però devono
funzionare a livelli sempre più evoluti, e infatti il terzo elemento di funzionamento è definito
dall'Organizzazione Mondiale della Sanità come «funzioni corporee»: il bambino può avere una
difficoltà di funzionamento educativo-apprenditivo originata da deficit funzionali, come deficit
visivi, motori, aprassie, afasie, deficit sensomotori, deficit nell'attenzione, nella memoria, nella
regolazione dell'attivazione (arousal), ecc. È evidente che i deficit funzionali più legati al difficoltoso
funzionamento educativo-apprenditivo sono quelli delle funzioni cerebrali e mentali, sia globali
che specifiche.

Con il suo corpo, in strutture e funzioni, il bambino apprende attività personali e partecipa
socialmente. Un'altra possibile fonte di difficile funzionamento educativo, e di conseguenza di
Bisogno Educativo Speciale, è infatti una ridotta dotazione di attività personali. Il bambino può
avere deficit di capacità e/o performance. Nel caso delle «capacità» il bambino agisce in modo
virtualmente «puro», senza cioè risentire degli effetti facilitanti o barrieranti dei vari fattori
contestuali ambientali e personali. Nel caso delle «performance» il bambino agisce attraverso
l'effetto facilitante/barrierante del vari fattori contestuali. Una comprensione approfondita del
funzionamento del bambino dovrà tener conto — e mettere in relazione — le sue capacità, le sue
performance e i ruoli facilitanti/barrieranti dei fattori contestuali in vari ambiti di attività:
apprendimento, di applicazione delle conoscenze, di pianificazione delle sue azioni, di linguaggio e
di comunicazione, di autoregolazione metacognitiva, di interazione, di autonomia personale e
sociale, di cura del proprio luogo di vita. Questa è una situazione molto nota all'insegnante:
l'alunno non sa fare bene le cose che sarebbe importante facesse per sviluppare patrimoni sempre
più ampi di competenze. Un bambino con scarse attività personali sa fare meno cose, o le fa in
forme deficitarie, anche se può essere perfettamente integro dal punto di vista strutturale e
funzionale corporeo. Può avere infatti dei deficit di capacità e di performance dovuti a scarse
esperienze di apprendimento, di stimolazione oppure all'influenza negativa (nel caso delle
performance) dei fattori contestuali.

Un'ulteriore fonte di funzionamento educativo-apprenditivo difficoltoso è la partecipazione
sociale. Secondo l'Organizzazione Mondiale della Sanità una persona “funziona bene” se partecipa
socialmente, se riveste ruoli di vita sociale in modo integrato e attivo; dunque non è sufficiente
avere un corpo integro e funzionante, presentare anche molte attività personali, bisogna anche
partecipare socialmente. In questo ambito possono generarsi (o co-generarsi) difficoltà specifiche
che diventano Bisogno Educativo Speciale: difficoltà nello svolgere i ruoli previsti dall'essere
alunno, essere compagno di classe ed essere utente di servizi rivolti all'infanzia, culturali, sportivi,
sociali. Il bambino che venisse ostacolato nella partecipazione, emarginato o allontanano, isolato,
rifiutato, vivrebbe un elemento significativamente determinante per lo sviluppo di un Bisogno
Educativo Speciale.

Dalle due classi di fattori contestuali, ambientali e personali, si possono originare varie
combinazioni di BES. Un bambino può infatti vivere fattori contestuali ambientali molto difficili:
una famiglia problematica, un contesto culturale e linguistico diverso, una situazione socio-
economica difficile, subire atteggiamenti ostili, indifferenza o rifiuto, può subire scarsità di servizi,
poche risorse educative sanitarie, incontrare barriere architettoniche, ecc.

Anche nei fattori contestuali personali si possono originare cause o concause di Bisogno
Educativo Speciale: scarsa autostima, reazioni emozionali eccessive, scarsa motivazione, stili
attributivi distorti, ecc.

Nella nostra proposta basata su ICF, tutti i fattori ambientali vanno ovviamente considerati,
tanto più quelli socioeconomici. Se non li valutiamo bene e approfonditamente nel loro influsso
positivo/negativo sul funzionamento, la nostra analisi non sarà compiutamente bio-psico-sociale,
perché ci mancherà un pezzo importante di «sociale». Questa non completezza della versione
originale di ICF può, almeno in parte, spiegare come per gran parte degli studiosi della disabilità da
un punto di vita sociale/culturale ICF sia tuttora visto come un modello troppo bio-medico e poco
«sociale».

In uno qualsiasi di questi sette ambiti si può generare dunque una causa o concausa di
Bisogno Educativo Speciale, che interagisce continuamente in maniera sistemica con gli altri
elementi, che potranno essere favorevoli o avversi.
Attraverso queste interazioni complesse si produce giorno dopo giorno il funzionamento
educativo-apprenditivo del bambino. Ovviamente, il peso dei singoli ambiti varierà da bambino a
bambino, anche all'interno della stessa condizione biologica originaria e della stessa condizione contestuale ambientale, due bambini figli di migranti senegalesi non sono affatto uguali, ad
esempio.

In questo modo, il modello ICF ci aiuta a definire le diverse situazioni di BES degli alunni:
alcune di esse saranno caratterizzate da problemi biologici, corporei e di attività personali, altre
principalmente da problemi contestuali ambientali, di attività personali e di partecipazione, altre
primariamente da fattori contestuali ambientali, altre principalmente da difficoltà di
partecipazione sociale, discriminazione, ostilità e così via, in un intreccio potenzialmente infinito di
interazioni.
\subsection*{La soglia tra funzionamento normale e problematico}
Se abbiamo messo il concetto di funzionamento globale di un alunno alla base di quello di
Bisogno Educativo Speciale, nasce evidente il problema del dove porre la soglia tra funzionamento
«normale» e funzionamento «problematico». Ne avevamo già precedentemente accennato, ma
ora la questione va affrontata più nel dettaglio. Potremmo ipotizzare un continuum di
funzionamento sul quale si deve formulare un giudizio, a un certo punto, di disfunzionalità e di
problematicità per il soggetto.

Evidentemente l'insegnante, l'educatore e il genitore «sentono» attraverso il loro disagio una
problematicità di apprendimento e di sviluppo nel bambino, ma questo loro disagio non è affatto
sufficiente per giudicare realmente problematico il funzionamento educativo-apprenditivo del
bambino. Questo loro disagio educativo è il primo motore, la prima energia che li mette in moto
per prendersi cura dello sviluppo del bambino, però può essere eccessiva e di conseguenza essi
possono giudicare problematica una situazione di sviluppo, ritardata o differente, che in realtà non
è problematica per l'alunno. Potrebbe essere una preoccupazione più per se stessi, per la propria
tranquillità che non per il benessere e lo sviluppo del bambino e di conseguenza creerebbe falsi
positivi in nome dell'ansia dell'insegnante o del genitore.

In realtà, la valutazione del Bisogno Educativo Speciale deve difendere il bambino da un
eccesso di preoccupazione che diventa iperprotezione e limitazione in nome del proprio benessere
di insegnante e genitore, ma anche da una scarsa preoccupazione, da un non cogliere in anticipo
possibili fonti di difficoltà

L'insegnante e il genitore «sufficientemente buoni» colgono in tempo, si accorgono
prestissimo che qualcosa non va, che il funzionamento del bambino e dell'alunno in qualche modo
sono negativamente condizionati. Ma allora come passare da una sensazione soggettiva di disagio
a una valutazione il più possibile oggettiva che quello stato di funzionamento in quel particolare
momento è effettivamente problematico per il bambino? Per fare questo passaggio dovremmo
avere alcuni criteri il più possibile oggettivi per decidere.

Il primo criterio può essere quello del danno, effettivamente vissuto dall'alunno e prodotto su
altri, alunni o adulti, rispetto alla sua integrità attuale fisica, psicologica o relazionale. Una
situazione di funzionamento è realmente problematica per un bambino se lo danneggia
direttamente o danneggia altri: si pensi a disturbi del comportamento gravi, all'autolesionismo, a
disturbi emozionali gravi, a gravi deficit di attività personali, a situazioni di grandi rifiuti o
allontanamento del gruppo. In questi casi si può osservare un danno diretto al bambino o ad altri che lo circondano. Se questo accade è evidente che la situazione è realmente problematica e non è
affatto un falso positivo. Ne conseguirà un obbligo deontologico a intervenire e una legittimazione
forte, in nome del benessere del bambino, ad agire urgentemente.

Ma ci può essere una situazione in cui il danno non è attualmente osservabile in maniera
chiara e allora si potrebbe assumere il criterio dell'ostacolo: un funzionamento problematico è tale
realmente per quel bambino se lo ostacola nel suo sviluppo futuro, se cioè lo condizionerà nei
futuri apprendimenti cognitivi, sociali, relazionali ed emotivi. In questi casi la situazione difficile
non riesce a danneggiare oggi direttamente il bambino o altri, ma lo pone e li pone in situazione di
svantaggio per ulteriori successivi sviluppi. Si pensi alle difficoltà di linguaggio ma anche ai disturbi
dell'apprendimento lievi o alle difficoltà emotive o comportamentali. Anche con questo secondo
criterio potremmo dunque decidere che una situazione di funzionamento è realmente
problematica per quel bambino, oltre che per gli adulti, e che di conseguenza dovremo intervenire
per aiutarlo nello sviluppo.

Potremmo però incontrare una situazione in cui non sia dimostrabile un danno o un ostacolo
al bambino o ad altri da parte del suo scarso funzionamento educativo-apprenditivo: in queste
situazioni di stranezze e di bizzarrie dovremmo analizzare la situazione rispetto a un terzo criterio
che potremmo definire dello stigma sociale. Con questo criterio ci si chiede se oggettivamente il
bambino, attraverso il suo scarso funzionamento educativo-apprenditivo, stia peggiorando la sua
immagine sociale, stia costruendosi ulteriori processi di stigmatizzazione sociale, soprattutto se
appartiene a qualche categoria socialmente debole. Come adulti, insegnanti e genitori, credo
abbiamo il dovere etico di tutelare e di migliorare se possibile l'immagine dei nostri alunni e dei
nostri figli. Anche perché un'immagine sociale negativa evidentemente diventerà poi ostacolo e
successivamente danno al loro sviluppo. E`evidente però che questo terzo criterio è il più difficile
ed esiste realmente il rischio che chi valuta, soprattutto se solo –- ma questo non dovrebbe mai
accadere --, subisca la pressione sociale conformista.

Con questi tre criteri potremo dunque decidere se la preoccupazione che viviamo nei
confronti dell'apprendimento e dello sviluppo dei nostri alunni o figli è realmente fondata, ha
realmente identificato un Bisogno Educativo Speciale su cui dobbiamo assolutamente intervenire
in senso pedagogico, psicologico e didattico, oltre che naturalmente fisico e biologico, se
necessario. In questo caso l'intervento risponde a un preciso obbligo deontologico.

In conclusione, risulta abbastanza chiaro che questa idea di Bisogno Educativo Speciale
fondata sul funzionamento globale della persona, come definito dall'Organizzazione Mondiale
della Sanità nel modello ICF, porta a un superamento delle categorie diagnostiche tradizionali nella
fase del riconoscimento "politico" di una situazione problematica a motivo della quale l'alunno ha
diritto a un intervento individualizzato e inclusivo.

Ciò non significa ovviamente ignorare o rifiutare le diagnosi cliniche nosografiche ed
eziologiche, che hanno un profondo significato per gli aspetti conoscitivi legati alla terapia, alla
prevenzione, ecc. Nel nostro caso cerchiamo un modo globale, per così dire a valle della diagnosi,
più largo, più comprensivo e più rispondente a quella che è una reale situazione di BES e di
difficoltà. In questo modello di Bisogno Educativo Speciale entrano anche alunni che non
potrebbero essere diagnosticati con alcuna delle condizioni patologiche tradizionali, ma che hanno
talvolta enormi Bisogni Educativi Speciali che vanno riconosciuti in tempo, esattamente, anche se
sfuggono ai sistemi tradizionali di classificazione, e a cui va data una risposta inclusiva efficace\footcite{Ianes2013}.

Booth T. e Ainscow M. (2008) L`index per l`inclusione scolastica, Trento, Erickson

Canevaro A. (a cura di) (2007) L`integrazione scolastica degli alunni disabili, Trento,
Erickson

Cowne E. (2003) The SENCO Handbook : working within a wholo-school approach,
Cambridge, Fulton

Department of Education and Skills (2001) Special Educational Skills. Code of practice. Londra,
DES

Ianes D. ( 2005a) Bisogni Educativi Speciali e inclusione. Trento, Erickson

Ianes D. (2005b) Bisogni Educativi Speciali e inclusione. Software gestionale. Trento, Erickson

Ianes D. e Biasioli U. (2005) L'ICF come strumento di classificazione, descrizione e
comprensione delle competenze, in “L`Integrazione scolastica e sociale”, vol. 4, n. 5, pp. 391-422

Ianes D. e Canevaro A. (2008) Facciamo il punto su...l`integrazione scolastica. Trento, Erickson

Ianes D. e Macchia V. (2008) La didattica per i Bisogni Educativi Speciali. Trento, Erickson

Meijer C. (2003) Special education across Europe. Middelfart, European Agency for
development in Special Needs Education

Medeghini R. et al. (2013) Disability Studies Trento, Erickson
OMS (2007) ICF-CY, Trento, Erickson

Sen A. (2011) L`idea di giustizia, Milano, Mondadori

Terzi L. (2008) Justice and Equality in Education, London, Continuum

UNESCO (1997) International standard classification of education (ISCED) Parigi

Vehmas S. Special needs: a philosophical analysis in “International Journal of Inclusive
Education”, Vol. 14, n. 1, February 2010, 87-96
