\author{Salvatore Nocera}
\title{Perché quei ricorsi offuscano la vera cultura inclusiva}
\phantomsection
\label{cha:salvatorenocera210114}
\begin{abstract}
Prende spunto, Salvatore Nocera, da un recente intervento apparso sulle pagine del nostro giornale, per ribadire le ragioni che, a suo parere, rendono rischiosi – ai fini della vera cultura inclusiva degli alunni con disabilità – i ricorsi basati sulla non discriminazione e mirati a ottenere un maggior numero di ore di sostegno
\end{abstract}
\maketitle
\datapub{21 gennaio 2014}
Ho letto con molta attenzione lo scritto dell'amico avvocato Gaetano De Luca del Servizio Legale LEDHA (Lega per i Diritti delle Persone con Disabilità), pubblicato nei giorni scorsi da « Superando.it», con il titolo Tutela antidiscriminatoria e inclusione scolastica, in particolare riguardo alle contestazioni mosse da chi scrive ai ricorsi per non discriminazione, al fine di ottenere un maggior numero di ore di sostegno didattico per l'inclusione degli alunni con disabilità.
Debbo però confessare di essere piuttosto tenace nel ribadire i miei timori, anche dopo la lettura di quello scritto. De Luca, infatti, rintuzza le critiche all'utilizzo del ricorso per non discriminazione, principalmente sulla base di due controdeduzioni, la prima delle quali è che se non si facessero ricorsi, gli alunni con poche ore di sostegno rimarrebbero discriminati, in quanto il docente per il sostegno è una figura professionale immancabile per l'inclusione scolastica.

A tal proposito devo ribadire di non avere mai affermato che non si debbano promuovere ricorsi ai TAR [Tribunali Amministrativi Regionali, N.d.R.], per ottenere il numero di ore di sostegno secondo le «effettive esigenze» dei singoli alunni, come stabilisce l'articolo 1, comma 605, lettera b della Legge 296/06\footcite{Legge_296_2006}, né che si possa fare a meno del sostegno. Ho solo affermato – e ripeto – che utilizzare il ricorso per discriminazione può produrre dei danni alla cultura dell'inclusione, come cercherò di chiarire qui di seguito.

Tutte le decisioni dei Tribunali Civili sui ricorsi per discriminazione hanno sempre affermato che la discriminazione stessa consiste nel fatto che gli alunni con disabilità hanno una riduzione del numero di ore di sostegno a differenza dei compagni non disabili i quali tali riduzioni non subiscono. In ciò consiste appunto la discriminazione, che viene superata solo assegnando il massimo del sostegno agli alunni con disabilità, possibilmente per tutta la durata dell'orario scolastico, come avviene per le ore curricolari per i compagni non disabili. In altre parole, si crea un parallelo tra il numero delle ore curricolari per i compagni e quello delle ore di sostegno per gli alunni con disabilità.

Ebbene, è qui che, a mio avviso, si annida il rischio dell'attacco –- sia pure involontario -– alla cultura e alla prassi inclusiva italiana. Queste ultime, infatti, nei primi decenni, ovvero dal 1971 al 1994 (data di attuazione della Legge 104/92\footcite{Legge_104_92}), hanno poggiato sul principio della presa in carico del progetto inclusivo da parte dei docenti curricolari, sostenuti dal collega per il sostegno; ciò ha fatto sì che gli alunni con disabilità fossero ritenuti dai docenti curricolari come alunni a pieno titolo, alla pari dei compagni, mentre in seguito –- a causa della progressiva mancata formazione inclusiva dei docenti curricolari e del crescente sovraffollamento delle classi –- si è determinata un'inarrestabile delega del progetto inclusivo ai soli docenti per il sostegno.

Ora, lo strumento tecnico-giuridico del ricorso per discriminazione favorisce tale delega, senza dare nulla di più di quanto non possa dare agli alunni con disabilità il tradizionale ricorso al TAR per violazione delle norme, che garantisce il massimo delle ore di sostegno e il tetto massimo di venti alunni per classe, prescritto dal DPR 81/09\footcite{DPR_81_2009}.

A mio sommesso avviso, pertanto, il ricorso per discriminazione è certamente utilissimo in altre circostanze, come ad esempio quando la scuola pretende che, in occasione della gita scolastica, le spese dell'accompagnatore dell'alunno con disabilità siano a carico della sua famiglia. Qui sì, infatti, si è di fronte a una palese discriminazione, con quelle spese che devono essere senz'altro a carico della scuola, senza che l'eventuale Sentenza di accoglimento della richiesta avanzata dalla famiglia danneggi in alcun modo l'immagine e la cultura dell'integrazione, anzi rafforzandone i postulati.

Invece –- e mi permetto di insistere -– le Sentenze di discriminazione per le ore di sostegno, fin qui pronunciate, condannano sì la discriminazione, ma offuscano la limpidezza dei principi in materia di inclusione, perché contrappongono o mettono in parallelo le ore curricolari per i compagni a quelle di sostegno per gli alunni con disabilità.

Mi si scusi per la testardaggine, ma fino a quando non mi si dimostrerà che le motivazioni delle Sentenze sulla discriminazione per le ore di sostegno non riguardano un confronto tra le ore curricolari dei compagni e quelle di sostegno degli alunni con disabilità, non riuscirò a superare le mie perplessità.

Spero per altro che le nota del collega De Luca e questa mia contribuiscano a riaccendere l'attenzione dell'opinione pubblica sull'inclusione scolastica degli alunni con disabilità, che rischia oggi un vero e proprio oscuramento, a causa sia dei gravissimi problemi dovuti alla situazione economica, con il taglio ai servizi degli Enti Locali e delle Scuole, sia anche a seguito della recente normativa sui BES (Bisogni Educativi Speciali) [Direttiva Ministeriale del 27 dicembre 2012\footcite{dir27Dic2012}, Circolare Ministeriale 8/13\footcite{cm8_2013} e Nota Ministeriale n. 2563 del 22 novembre 2013\footcite{Nota_2563_2013}, N.d.R.], che ritengo stia attirando su di sé molta più attenzione rispetto a quella dovuta agli alunni con disabilità\footcite{Nocera2014}.