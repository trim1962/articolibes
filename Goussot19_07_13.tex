\author{Alain Goussot}
\title{Quale inclusione? Riflessioni critiche sui bisogni educativi speciali}
\phantomsection
\label{Goussot190713}
%\epigraph{\hspace*{20pt}}{Alain Goussot}
 \maketitle
 \datapub{19 Luglio 2013}
 Mi permetto di proporre alcune riflessioni in riferimento al dibattito in corso nel mondo della scuola e degli ambienti pedagogici sulla questione dei cosiddetti \citi{bisogni educativi speciali} che ha trovato una sua esplicita formalizzazione nei documenti del Miur di dicembre 2012 e marzo 2013. Considero la questione estremamente delicata e complessa ma anche importante poiché è il riflesso di una concezione della scuola e di una visione della gestione delle differenze in termini di apprendimento, crescita individuale e collettiva. In sostanza ne va del modello di società che vogliamo costruire formando le future generazioni e quindi della nostra idea di democrazia. Faccio rapidamente alcune considerazioni e pongo alcuni quesiti sui quali invito il mondo della scuola ma anche dell'educazione in generale a riflettere seriamente:
\section*{ I rischi della logica differenzialistica e delle stigmatizzazioni sofisticate}
 
 Ricordo che nel 1977 con la legge sull'integrazione scolastica degli alunni disabili nella scuola di tutti si superava, almeno così si pensava allora, la logica differenzialistica delle classi differenziali, delle scuole speciali e delle sezioni ghetto. Si affermava il principio dell'eguaglianza delle opportunità nell'accesso all'istruzione e all'educazione predisponendo strumenti e risorse (vedi insegnante di sostegno) per favorire lo sviluppo delle potenzialità di tutti gli alunni tramite una attività pedagogica accogliente, in grado di promuovere l'individualizzazione dei percorsi di apprendimento e l'attività di gruppo (produttrice di esperienze di socialità). Tutto andava quindi nella direzione di lottare contro l'esclusione, la marginalizzazione e la stigmatizzazione/inferiorizzazione dell'alunno disabile. Negli anni si sono sviluppate esperienze didattiche e pedagogiche ricche di innovazione ma sono anche emerse molti limiti e tante criticità. Con una direttiva del 2010 il ministero pone la questione degli alunni con disturbi specifici dell'apprendimento (dislessia, disgrafia, disortografia, discalculia); si promuovono corsi di formazione per insegnanti (curriculari e di sostegno). Comincia a porsi una domanda: se è giusto essere attenti al fenomeno dei DSA non v'è il rischio di una identificazione rapida tra difficoltà di apprendimento e disturbi specifici? Non v'è anche il rischio di accentuare lo sguardo clinico-diagnostico a scapito dello sguardo pedagogico che dovrebbe essere quello dell'insegnante? Abbiamo anche visto gli alunni con ADHD (sindrome da deficit di attenzione e iperattività); anche qui una nozione e categoria ambigua e molto discussa: cosa vuol dire? Chi sono? Quale attenzione pedagogica da parte dell'insegnante (una volta lo psicopedagogista francese Henri Wallon parlava di \citi{bambino turbolento}; si capisce che dire turbolento e dire iperattivo non è la stessa cosa, non è lo stesso sguardo; il primo colloca la questione nell’ambito educativo, il secondo in quello clinico-sintomatologico). Adesso abbiamo i BES: chi sono? In parte si riprende alcune categorie precedenti e si aggiunge: gli alunni con difficoltà di apprendimento (quale alunno non presenta difficoltà di apprendimento?), gli alunni con disagio psico-sociale (la povertà sociale è un problema?), quelli con difficoltà linguistico culturali (l'essere figlio/a d'immigrati è un problema?), gli alunni con un \citi{funzionamento intellettivo limite} (cosa vuol dire esattamente?). Insomma una ulteriore categoria insieme ambigua, generica e anche funzionale al paradigma clinico-diagnostico-terapeutico che sta colonizzando culturalmente la scuola e la società. Faccio notare che le categorie usate non sono per niente neutrali e che mentre la logica differenzialistica tende a produrre e riprodurre diseguaglianze (stigmatizzazioni sofisticate) il riconoscimento delle differenze passa tramite un'azione pedagogica basata sul principio di eguaglianza nell'accesso ai saperi e alle conoscenze. Insomma la logica differenzialistica delle categorizzazioni continue non ha nulla a che fare con il riconoscimento delle differenze.
\section*{ Quale inclusione?}
 
 Anche sulla questione dell'inclusione occorre confrontarsi e chiarire meglio di cosa stiamo parlando. Per anni si è parlato di integrazione, in particolare in riferimento all'integrazione scolastica e sociale degli alunni con disabilità (distinguendo la disabilità-prodotta da un deficit sensoriale, motorio, intellettivo dall'handicap prodotto o conseguenza socio-culturale, ostacoli generati dalla società nell'interazione con il soggetto con disabilità); si diceva che fosse importante creare delle opportunità e delle situazioni educative e formative in grado di rimuovere barriere e ostacoli. Di modificare tramite la mediazione dell'azione educativa pregiudizi e situazioni handicappanti produttrici di esclusione, autoesclusione e stigmatizzazione/interiorizzazione. Poi da alcuni anni si è cominciato a parlare d'inclusione, precisando che si voleva sottolineare che il cambiamento non poteva essere a senso unico ma reciproco (soggetto e ambiente). Troviamo queste considerazioni già nei lavori dello psicopedagogista sovietico Lev Vygotskij che parla di mediazioni: quello che oggi vengono definite con le espressioni strumenti compensativi e dispensativi (uso di tecniche, ausili e di accompagnamento e supporti). Produrre esperienze di apprendimento mediato per favorire lo sviluppo delle potenzialità di tutti gli alunni, appunto in una prospettiva d'integrazione e/o d'inclusione. Ma sorge un dubbio: se il concetto d'inclusione è strettamente connesso agli indirizzi proposti sui cosiddetti Bes si muove nella direzione del differenzialismo, allora cosa vuol dire includere? Un concetto chiave rimane quello di adattamento funzionale. Quindi si tratta di adattare, per il bene dell'alunno \citi{Bes}, di \citi{normalizzare}, di \citi{curare} di \citi{riparare}. Ma a questo punto non si rischia di riprodurre le diseguaglianze che si dichiara di volere combattere? Non si rischia di fornire una giustificazione \citi{scientifica} all'esistenza, purtroppo reale, delle sezioni ghetto nelle scuole, e, quindi, di riprodurre la logica delle classi differenziali? Nei documenti del ministero si parla della valutazione inclusività delle scuole: ma chi si occuperà di questa valutazione? Quale formazione e competenze avranno i valutatori? Quali criteri di valutazione saranno utilizzati? Non vorrei che i criteri (diffusi nei sistemi di valutazione PISA) usati (successo scolastico, abbandono e dispersione scolastica, autofinanziamento, progettualità approvate e realizzate) finissero per penalizzare ulteriormente le scuole delle periferie, le scuole povere dei quartieri emarginati, le scuole collocate nelle zone ad alta presenza di immigrati. Vorrebbe dire riprodurre e accentuare le diseguaglianze e essere in contraddizione con il detto costituzionale della Repubblica italiana. Sono quesiti posti sia sul piano della riflessione filosofica, pedagogica e sociologica da eminenti studiosi e pensatori come il tedesco Jurgen Habermas (l'inclusione dell'altro) e il francese Charles Gardou (la società inclusiva). Inoltre si pone anche la questione della relazione e del tipo di collaborazione tra insegnante curricolare e insegnante di sostegno, ma anche quella del rapporto tra scuola, famiglie e territorio: è quello che nei loro lavori recenti dei colleghi belgi come J.P.Pourtois, H.Desmett e B.Humbeeck chiamano \citi{processi co-educativi}: come si costruisce l'alleanza co-educativa tra i diversi attori della comunità? Come si può attivare e realizzare insieme dei processi di emancipazione che garantiscono la giustizia nei processi di apprendimento?
 \section*{Didattica o didatticismo? La marginalizzazione della pedagogia}
 
 La gestione del gruppo classe e l'organizzazione degli apprendimenti sono due aspetti fondamentali dell'attività docente. La tendenza va sempre di più (lo si vede nella formazione stessa del personale docente) nella direzione delle procedure didattiche, della tecnologia didattica, dell'uso degli strumenti; si sostituisce la didattica come processo vivo (che implica la relazione complessa tra docente, alunni, metodi, strumenti, comunità scolastica) con il didatticismo inteso come procedura. Interessante notare che la figura dell'alunno come soggetto significante del processo d'insegnamento/apprendimento è assente. Se è presente lo è solo come fonte di problema. Il rischio è di vedere l'insegnante diventare un operatore della diagnosi e della procedura tecnica per valutare la performance dell'alunno in termini stretti d'istruzione (come se istruzione e educazione non fossero interconnesse in modo vivo nell'esperienza in classe). La pedagogia (quindi la formazione pedagogica dell'insegnante che dovrebbe andare a caccia di risorse, capacità, potenzialità e non di \citi{comportamenti problema}) viene marginalizzata nella cultura scolastica e colonizzata dallo sguardo di una certa psicologia clinica. Non a caso i documenti ministeriali non fanno praticamente mai riferimento alla lunga e ricca esperienza delle pedagogie attive e dell'educazione nuova; ancora meno di quelle prodotte dalla pedagogia speciale.
 \section*{Quale modello organizzativo, quale politica? Logica burocratica o democratica?}
 
 Si parla di docenti esperti e preparati sui \citi{BES}, si parla di Centri territoriali per l'inclusione: ma cosa vuol dire in modo preciso? Chi saranno questi docenti esperti dei BES ? Quale formazione avranno? Quali compiti e competenze? Che fine faranno gli insegnanti specializzati o di sostegno? Vediamo in tutto questo una risposta tecnocratica-burocratica ad una questione di ordine culturale, pedagogica e sociale; di nuovo vediamo una scuola e un corpo docente deprivato del proprio protagonismo, della possibilità di partecipare all'analisi e anche all'elaborazione di proposte concrete per favorire l'effettiva eguaglianza delle opportunità per tutti gli alunni nell'accesso all'istruzione e all'educazione. V'è bisogno del contributo degli insegnanti che ogni giorno attivano delle esperienze pedagogiche e didattiche nelle loro classi, che ogni giorno affrontano la complessità e le difficoltà del mestiere dell'insegnante in una società sempre più atomizzata e individualistica. Gli alunni portano a scuola le contraddizioni che vivono nelle loro famiglia e che produce una società che fa di ognuno un consumatore-spettatore e non un soggetto responsabile consapevole del legame tra individualità e comunità, tra diritti e doveri, tra desideri personali e bene comune. Gli insegnanti vanno coinvolti non come destinatari di indagini predisposte da pool di esperti, non come mere esecutori di direttive ministeriali o di tecniche specializzate ma come attori/autori in grado di produrre senso e di fornire, tramite la loro pratica, proposte e indicazioni per un rinnovamento della nostra scuola repubblicana.
 
 Mi fermo qui. Sono solo alcuni spunti di riflessione; sono convinto che occorre rimettere al centro l'azione pedagogica e promuovere un autentico confronto dando voce agli operatori della scuola, agli insegnanti, agli educatori, ma anche agli alunni e ai genitori che spesso si trovano a dovere fare delle scelte senza capire di cosa si sta parlando. Ne va del futuro dei nostri figli, della scuola della Repubblica e anche del futuro della democrazia in questo paese\footcite{Goussot2013}.