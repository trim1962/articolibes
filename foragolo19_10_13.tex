\author{Flavio Fogarolo}
\title{I BES non si certificano!}
\label{cha:FlavioFogarolo19102013}
\maketitle
\datapub{19 ottobre 2013}

Le recenti disposizioni ministeriali sugli alunni con Bisogni Educativi Speciali (BES) hanno suscitato, come è noto, un vivace dibattito dentro e fuori la scuola, con molti pronunciamenti pienamente favorevoli ed altri critici o preoccupati. Si evidenzia da molte parti soprattutto il rischio di medicalizzare dei semplici problemi educativi e di etichettare in questo modo delle normali differenze individuali.

Il rischio è reale ma non deriva, a parer mio, dalla direttiva o circolare ministeriale e di sicuro il MIUR non ha "inventato" i BES.

Da molti anni, prima con la disabilità poi con i DSA, domina nelle scuole, unico e praticamente incontrastato, un modello di tipo clinico: i bisogni vanno "certificati", ossia riconosciuti formalmente da un'autorità sanitaria esterna alla scuola, e solo in seguito a questa procedura gli insegnanti si attivano personalizzando gli interventi.

Passare da una impostazione di questi tipo ad una pedagogico-didattica, come richiesto e previsto per i BES, non è per nulla banale e non possiamo stupirci se questa innovazione sia fonte di difficoltà, non tanto di resistenze, nelle scuole.

Si registra infatti spesso, in questi primi mesi di Applicazione delle nuove disposizioni, la propensione ad applicare anche ai BES il modello clinico, con la scuola che tende ad assumere il ruolo degli specialisti e individuare i BES in base a misurazioni oggettive, o presunte tali, una volta definita la soglia critica. In alcune scuole, ad esempio, si sottopongono tutti gli alunni a delle prove standardizzate di lettura e vengono considerati come BES tutti gli alunni che riportano punteggi inferiori ad un certo livello. È una forzatura che non è certo imputabile alla Circolare Ministeriale (CM 8/13\footcite{cm8_2013}) che anzi afferma chiaramente che l'individuazione va fatta in base a "ben fondate considerazioni psicopedagogiche e didattiche".

Ma cosa vuol dire, in pratica, applicare un modello pedagogico e non un modello clinico?

In estrema sintesi, significa considerare l'efficacia e la convenienza degli interventi proposti, non la sola entità dei bisogni e, tanto meno, il loro riconoscimento nominale.

I BES non si certificano! Non possono farlo i servizi sanitari, né in modo diretto o esplicito («il bambino/ragazzo XY è un alunno con Bisogni Educativi Speciali», fortunatamente pochi ma arrivano alle scuole anche certificati di questo tipo), né indiretto, e questa è una pratica invece molto diffusa: dopo avere indicati disturbi o difficoltà si conclude dicendo che per questo alunno la scuola deve applicare le tutele previste dalla CM n. 8 del 2013\footcite{cm8_2013}. Inserire in una diagnosi clinica una dichiarazione di questo tipo è un'inaccettabile invasione di campo: l'applicazione della circolare n. 8, ossia di fatto l'individuazione dell'alunno come BES, rientra nell'ambito della didattica, non della clinica, ed è pertanto una prerogativa esclusiva della scuola. Una prerogativa che non si basa certamente sull'arbitrio ("A scuola facciamo quello che vogliamo") ma su quella assunzione di responsabilità che è strettamente connessa all'autonomia scolastica e educativa. A fronte di un bisogno clinicamente accertato e documentato la scuola deve organizzarsi e dare una risposta ma è "responsabilmente" autonoma nel decidere cosa fare e come farlo, attenta cioè a verificare l'efficacia degli interventi attivati e a rivedere le scelte se necessario.

La certificazione di disabilità o di DSA si basa su considerazioni il più possibile oggettive ma nel momento in cui la scuola identifica un alunno con Bisogni Educativi Speciali deve considerare, e valutare, non solo i bisogni ma anche il contesto e la convenienza dell'intervento di personalizzazione proposto.

Vale la pena ricordare cosa dice al riguardo la Circolare Ministeriale n. 8 del marzo 2013: 
\begin{quote}
	\mancatesto è compito doveroso dei Consigli di classe o dei teams dei docenti nelle scuole primarie indicare in quali altri casi sia opportuna e necessaria l'adozione di una personalizzazione della didattica\mancatesto
\end{quote}

La scuola quindi non dichiara gli alunni BES, né tanto meno li certifica, ma a individua quelli per i quali è "opportuna e necessaria" una personalizzazione formalizzata, ossia un PDP.

Pertanto il PDP non è una conseguenza di questo riconoscimento come per la disabilità e i DSA ("Questo alunno è BES quindi la scuola deve predisporre un PDP") ma parte integrante dell'identificazione della situazione di bisogno("Questo alunno è BES perché secondo la scuola ha bisogno di un PDP").


Certamente non tutti gli alunni che hanno qualche difficoltà rientrano tra i BES e non per tutti quelli che hanno bisogno di una qualche forma di personalizzazione deve essere predisposto un PDP.

La scuola ha tanti modi, strumenti e procedure per adattare la didattica ai bisogni individuali, molti dei quali assai più semplici e informali ma in certi casi ugualmente efficaci, se non di più.

Identificare un alunno come BES significa riconoscere per lui la necessità non solo di un percorso didattico diverso da quello dei compagni ma anche di una sua ufficializzazione, come assunzione formale di impegni e responsabilità da parte della scuola e, se possibile, anche della famiglia. Ossia di un PDP, appunto.


La scuola è chiamata pertanto a decidere sull'opportunità di questa scelta, che di sicuro non dipende solo dall'entità del bisogno ma si basa sulla valutazione dell'effettiva convenienza della strategia didattica personalizzata che si intende attuare.

Dobbiamo cioè rispondere a domande di questo tipo: per questo alunno, in questa scuola, in questo momento, è veramente necessario, utile, opportuno stendere un PDP?

La valutazione di convenienza deve considerare gli aspetti positivi e negativi dell'intervento e prevedere, con ragionevole certezza, che i vantaggi saranno prevalenti.

Perché, certamente, scelte di questo tipo non hanno solo aspetti positivi! Sappiamo benissimo che la scelta di differenziare formalmente il percorso didattico di un alunno rispetto a quello dei compagni comporta spesso ricadute anche gravi nel campo dell'autostima, dell'accettazione, del rapporto con i compagni, delle tensioni familiari e altro. Sono rischi che vanno previsti, valutati, analizzati (prevedendo e attivando eventuali azioni correttive) e confrontati con i benefici previsti o attesi; ma si va avanti solo se il bilancio è nettamente positivo, almeno nelle previsioni e potenzialità.

Questo modo di procedere nell'individuazione dei BES, basato sulla stima tra vantaggi e svantaggi, comporta almeno due importanti conseguenze:
\begin{description}
	\item[--] la prima è che questa valutazione è fortemente condizionata dal contesto e quindi uno stesso alunno può essere considerato BES in una realtà scolastica e non in un'altra. È una situazione ovviamente inconcepibile per la disabilità e i DSA: un alunno dislessico, ad esempio, rimane tale anche se cambia scuola, su questo non ci sono dubbi. Ma per gli alunni con BES individuati dalla scuola non è così: un alunno può aver necessità di una personalizzazione formalizzata in una scuola mentre in un'altra può non essercene bisogno. O viceversa. Pensiamo ad esempio ad una classe a tempo pieno della scuola primaria gestita da una coppia di insegnanti che lavora assieme da molti anni, condividendo fino al minimo dettaglio quotidiano il metodo di insegnamento; di fronte ai bisogni di  un bambino in difficoltà per loro a volte un dialogo davanti alla macchina del caffé risulta efficace quanto una dettagliata personalizzazione scritta redatta in altri contesti, se non di più. Ma se l'alunno dovesse cambiare scuola, e passare ad esempio alle medie con tanti insegnanti che si incontrano solo nel consiglio di classe, può essere necessaria una personalizzazione più strutturata, forse anche attraverso un PDP.
	\item[--] la seconda conseguenza è che, almeno a grandi linee, quando identifica l'alunno come BES la scuola deve aver già chiaro il tipo di intervento che intende attuare con quello specifico alunno, a supporto delle sue difficoltà, perché solo in questo modo è possibile una consapevole valutazione di convenienza. Paradossalmente possiamo dire che gli alunni nei confronti dei quali ci si sente impotenti perché non si sa cosa fare per loro, per quanto evidenti e gravi siano i loro bisogni educativi, non possono essere considerati BES finché non si sarà grado di dire come si intende effettivamente personalizzare il loro percorso per poter valutare se esso sarà opportuno e conveniente.
\end{description}

Andranno certamente considerate anche le esigenze di personalizzazione collegate alla definizione dei livelli minimi di competenze nonché alle forme e criteri di valutazione, ma sempre considerando criteri di opportunità e convenienza. Le esigenze connesse alla valutazione, ad esempio,saranno molto più sensibili quando si avvicinano gli esami, o nel secondo ciclo di istruzione; in altri casi la valutazione ha comunque per tutti un ruolo prevalentemente educativo e alcune personalizzazioni possono essere introdotte anche senza bisogno di un adempimento formale (ossia del PDP). In ogni caso esse derivano sempre da delle scelte e dal confronto tra vantaggi e svantaggi.


In questa fase è purtroppo elevato il rischio di considerare le attenzioni agli alunni con bisogni educativi speciali come un adempimento burocratico in più, oneroso, se non vessatorio, per le scuole e, purtroppo,certe puntigliose interpretazioni delle disposizioni ministeriali (piano educativo individualizzato, piano annuale dell'inclusione, autovalutazione dell'inclusione, rilevazione dei bisogni) rafforzano questa impressione. Per questo occorre assolutamente riaffermare il valore educativo di ogni procedura.

La riscoperta attenzione verso gli alunni con Bisogni Educativi Speciali va vissuta realmente, non solo a parole, come un'opportunità per le scuole, ossia come la "possibilità", non l'obbligo, di fare alcune cose che prima sembravano impossibili, o quantomeno di dubbia legittimità, come formalizzare un percorso diverso anche per chi non ha portato a scuola documenti o certificati particolari. Adesso sappiamo ufficialmente che possiamo fare molto anche per loro, per i nostri "sans papiers" che , almeno a scuola, non devono necessariamente essere considerati cittadini di serie B. Un atto di giustizia ma anche un altro passo avanti per un'effettiva responsabilità e autonomia delle scuole\footcite{Fogarolo2013}. 


Flavio Fogarolo
19 ottobre 2013