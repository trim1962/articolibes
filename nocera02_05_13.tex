\author{Salvatore Nocera}
\title{A proposito di quelle proposte sull'inclusione }
\label{nocera020513}
%\epigraph{Sono certamente degne di essere discusse – secondo Salvatore Nocera – alcune proposte apparse in queste settimane nel nostro giornale, a proposito dell'inclusione scolastica degli alunni con disabilità. Ma è anche necessario puntualizzare quanto accade oggi nella realtà scolastica e in particolare la mancata applicazione di alcune norme ritenute fondamentali}{Salvatore Nocera}
\begin{abstract}
Sono certamente degne di essere discusse – secondo Salvatore Nocera – alcune proposte apparse in queste settimane nel nostro giornale, a proposito dell'inclusione scolastica degli alunni con disabilità. Ma è anche necessario puntualizzare quanto accade oggi nella realtà scolastica e in particolare la mancata applicazione di alcune norme ritenute fondamentali
\end{abstract}
\maketitle
 Sono due gli articoli di Giuseppe Felaco, apparsi recentemente in «Superando.it», sui quali vorrei soffermarmi. Il primo (Non più solo i genitori!\ref{cha:felaco190413}) concerne l'erogazione dell'assegno comunale di cura alle famiglie che garantissero spazi comuni ludici ai giovani con disabilità, evitando così il ricovero in strutture segreganti, il secondo, invece (Tutti avrebbero dei vantaggi)\ref{cha:felaco260413}, riguarda l'aumento stipendiale dei docenti curricolari preparati per l'inclusione degli alunni con disabilità, ottenuto con la riduzione notevole o addirittura la progressiva scomparsa dei docenti per il sostegno.

Il primo articolo percorre un sentiero assai battuto nelle politiche sociali di molti Comuni e di talune Regioni; si pensi, ad esempio, all'applicazione della Legge 162/98 in Sardegna, ma anche in altre parti d'Italia. La soluzione proposta sarebbe alternativa agli attuali Centri Socio educativi-Riabilitativi, sulla cui radicale trasformazione si è pure di recente soffermato un importante convegno promosso nelle Marche dal Gruppo Solidarietà di Moie di Maiolati (Ancona).
Più controcorrente e rivoluzionaria è la seconda proposta di abolire il docente per il sostegno, ripartendo il suo compenso tra gli insegnanti curricolari. La questione non è nuova, essendo tale idea stata già avanzata – sia pure con diverse modalità – dalla Fondazione Agnelli e dall'Associazione TreeLLLe [esattamente nel rapporto intitolato Gli alunni con disabilità nella scuola italiana: bilancio e proposte, Erickson, 2011\footcite{treellle2011alunni}, elaborato appunto dalla Fondazione Agnelli, insieme all'Associazione TreeLLLe\footcite{treellle2011alunni} e alla Caritas Italiana, N.d.R.].
Si tratta di una proposta che parte da un'esigenza indiscutibile, sempre sostenuta dal sottoscritto, per averla io stesso sperimentata, quando studiavo da minorato visivo negli Anni Cinquanta nel “profondo Sud”, e cioè il fatto che gli attori primi dell'inclusione scolastica debbano essere i docenti curricolari e i compagni di classe. È tuttavia una proposta che non precisa la necessità di due condizioni e cioè, da una parte, una preventiva e seria formazione dei docenti curricolari, dall'altra il ridotto numero di alunni per classe.
Quanto alla prima condizione, infatti, non ha ancora avuto attuazione il Decreto Ministeriale 249/10 che ha previsto la formazione iniziale sulle didattiche per l'inclusione di tutti i futuri docenti curricolari; un provvedimento, quest'ultimo, che è necessario immediatamente attuare, anche se esso manca di un serio numero di crediti formativi per i futuri docenti di scuola secondaria. E occorre anche un accordo sindacale col Ministero, per rendere obbligatorio l'aggiornamento in servizio dei docenti curricolari sulle tematiche dell'inclusione, mentre attualmente tale formazione è puramente facoltativa.
Quanto poi alla seconda condizione, esiste l'articolo 5, comma 2 del Decreto del Presidente della Repubblica (DPR) 81/09\footcite{DPR_81_2009} sul tetto massimo – «di norma» – di venti alunni nelle classi frequentate da ragazzi con disabilità; l'unica eccezione è consentita dall'articolo 4 del medesimo DPR, con l'aumento del 10\% e con il tetto, quindi, che può salire a ventidue alunni. Ebbene, questa norma non è quasi mai rispettata e bisogna ricorrere ai TAR (Tribunali Amministrativi Regionali),  per evitare classi illegittime.

Pertanto, se e quando si verificheranno queste condizioni, penso che la proposta possa essere presa in considerazione. Ed è anche da ricordare che dopo la maggiore tutela realizzata con la Legge 170/10 sui DSA (disturbi specifici di apprendimento) e dopo la recente Direttiva Ministeriale del 27 dicembre 2012 (Strumenti d'intervento per alunni con Bisogni Educativi Speciali e organizzazione territoriale per l'inclusione scolastica) e la successiva Circolare 8/13 sugli altri casi di BES (bisogni educativi speciali), si è avuto appunto un maggiore riconoscimento dei casi di difficoltà di apprendimento, e quindi il verificarsi di quelle due condizioni diventa ancor più pressante.
Ovviamente resta anche l'enorme problema di come risolvere la situazione di oltre centomila docenti per il sostegno attualmente in servizio. Non è infatti lontanamente pensabile che essi vengano licenziati; bisognerebbe quindi, a mio avviso, prevedere accordi sindacali per il mancato rinnovo di nomine con il pensionamento e il contemporaneo aumento di stipendio per i docenti curricolari. Un'operazione assai complessa, questa, che i Sindacati e il Ministero dovrebbero affrontare. Ma ci sarà realmente la volontà politica di tutti per farlo?\footcite{nocera5}

Vicepresidente nazionale della FISH (Federazione Italiana per il Superamento dell'Handicap).

2 maggio 2013
Ultimo aggiornamento: 2 maggio 2013 11:34

© Riproduzione riservata