\author{Maurizio Tiriticco}
\title{A proposito di un convegno sui BES}
\label{cha:Maurizio Tiriticco221013}
\maketitle
\datapub{22 ottobre 2013}
Il Comune di Cortona assieme al MIUR e all'Associazione Autismo Toscana ed alla web community Chiamalascuola, ha organizzato un seminario regionale per il 25 e il 26 ottobre p.v. presso il Centro Convegni S.Agostino, il cui titolo è “Officina per i Bisogni Educativi Speciali: orientamenti, pratiche e innovazioni”. E so che sono in programmazione altri convegni in materia, ma\dots

Ho espresso più volte il mio pensiero e le mie preoccupazioni su questa vicenda dei. BES! Ottima cosa preoccuparsi dei BES, pessima cosa affrontarli come sembra che ci si stia accingendo a fare! \'{E} sufficiente citare un passo di Dario Janes, esperto in materia, riportato nella locandina del convegno, che così recita: 
\begin{quote}
\'{E} la scuola che osserva i singoli ragazzi, ne legge i bisogni, li riconosce e di conseguenza mette in campo tutti i facilitatori possibili e rimuove le barriere all'apprendimento per tutti gli alunni, al di là delle etichette diagnostiche. È un discorso di equità, che consente davvero quella personalizzazione spesso rimasta sulla carta. Dall'altra parte dà maggiore responsabilità agli insegnanti curricolari, senza deleghe al sostegno.
\end{quote}
Ottime considerazioni, ma\dots emergono due preoccupazioni. Rileggo: \cit{al di là delle etichette diagnostiche} E \cit{maggiore responsabilità agli insegnanti curricolari senza deleghe al sostegno}: parole pesanti! Occorre solo riflettere su chi stabilisce dove ha inizio \cit{l'al di là}. E su che cosa significa per un insegnante curricolare non avere \cit{delega al sostegno}. Parafrasando un noto adagio latino, mi sembra che si voglia spingere il calzolaio ad andare oltre quei calzari di cui soltanto è competente!

In effetti, si vogliono attribuire all'insegnante curricolare compiti che non sono suoi, anche se è più che ovvio che l'insegnante curricolare si fa carico di tutte quelle caratteristiche che attengono ad ogni singolo alunno. Nessuno è mai eguale ad un altro e queste cose l'insegnante le sa, le mette in conto e le affronta di conseguenza! Altra cosa è chiedergli di avventurarsi su un terreno che non è il suo! Al massimo l'insegnante può segnalare a chi ne ha la competenza che l'alunno x presenta date problematiche, E solo chi è competente può esprimere valutazioni in merito, valutazioni delle quali l'insegnante curricolare “normale” (la sua professionalità è ben descritta negli articoli 26 e 27 del Contratto di Lavoro che lo riguarda) terrà in debito conto! Temo molto l'improvvisazione di professionisti incompetenti – io stesso sono incompetente in materia – su questioni che vanno oltre la loro professionalità.

Ma, ammettiamo che tutto proceda per il meglio! E, se in una classe si individuano cinque o sei BES, che si fa? Occorre progettare altrettanti percorsi individualizzati e/o personalizzati (sono due concetti ben diversi, ma che spesso i documenti che leggo vengono usati indifferentemente). In effetti, si richiede un'attività progettuale in aggiunta a quella già richiesta dalla normale progettazione educativa e didattica relativa alla classe in cui si insegna. Potremmo anche essere d'accordo su tutto ma, se si chiede un simile lavoro aggiuntivo al nostro insegnante, o meglio agli insegnanti di un normale consiglio di classe, quale sarà la retribuzione dovuta? A meno che l’”invenzione” dei BES (che finora ci sono sempre stati e… non ce ne siamo mai accorti), checché ne dicano i documenti internazionali, nel nostro Paese non sia una bella trovata per scaricare tutti i disabili sugli insegnanti curricolari.

Se vogliamo veramente garantire a tutti il “successo formativo”, individualizzato e/o personalizzato che sia – è un impegno che abbiamo assunto con il varo dell'autonomia scolastica – la strada da percorrere implica risorse su risorse! E le nozze con i fichi secchi sono quelle che sono! Mah! Chi vivrà, vedrà! Gli insegnanti non sono tutti Dario Ianes\dots

Un'ultima considerazione. Forse con la conclusione dell'obbligo di istruzione, una qualche trovata la si troverà: magari competenze \cit{personalizzate}\dots Ma, alla conclusione del secondo ciclo di istruzione che accadrà? Avremo un \cit{tecnico della gestione aziendale} a tutto tondo e uno con i BES? Avremo un diplomato in \cit{meccanica, meccatronica ed energia} a tutto tondo e uno con i BES? E con i relativi gradi di gravità? E poi, come la mettiamo con il mondo del lavoro? Concludo: la questione dei BES è troppo seria e non possiamo affrontarla e risolverla con la superficialità che emerge da ogni passo della Direttiva ministeriale!

So bene che Dario Ianes sostiene che \cit{i BES sono un passo in avanti}, e potremmo sostenerlo in tanti, ma\dots la materia è troppo importante perché possa essere affrontata come sembra. Caricare sugli insegnanti nuove e pesanti responsabilità! Sempre loro a pagare in cambio di nulla!\footcite{Tiriticco2013}


