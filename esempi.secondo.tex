
	\section{Circuiti e reti}
	\label{sec:CircuitieReti}
	\begin{table}[H]
		\caption{In un circuito con due resistenze $R_1$ e $R_2$ in parallelo, trovare la formula che da $R_2$ note $R_1$ e $R_{eq}$}
		\label{tab:Trovarediffangoli}
		\begin{enumerate}
			\item Prerequisiti 
			\begin{itemize}
				\item mcm
				\item Equazioni di primo grado
				\item Resistenze in parallelo
				\item $\dfrac{1}{R_{eq}}=\dfrac{1}{R_1}+\dfrac{1}{R_2}+\cdots+\dfrac{1}{R_n}$			
			\end{itemize}
			\item Scopo: Determinare una resistenza note l'altra e la resistenza equivalente in un circuito in parallelo
			\item Testo: Determinare $R_1$ noti $R_2$ e $R_{eq}$
			\item Svolgimento: Si usa la formula che da la resistenza equivalente in parallelo.
			\begin{enumerate}
				\item $\dfrac{1}{R_{eq}}=\dfrac{1}{R_1}+\dfrac{1}{R_2}+\cdots+\dfrac{1}{R_n}$
				\item $\dfrac{1}{R_{eq}}=\dfrac{1}{R_1}+\dfrac{1}{R_2}$
				\item trovo mcm fra $R_1$, $R_2$ e $R_{eq}$
				\item $\dfrac{R_1\cdot R_2=R_{eq}\cdot (R_1+R_2)}{R_1\cdot R_2\cdot R_{eq}}$
				\item $R_1\cdot R_2=R_{eq}\cdot (R_1+R_2)$
				\item $R_1\cdot R_2=R_{eq}\cdot R_1+R_{eq}\cdot R_2$
				\item $R_1\cdot R_2-R_{eq}\cdot R_1=R_{eq}\cdot R_2$
				\item $R_1\cdot (R_2-R_{eq})=R_{eq}\cdot R_2$
				\item $R_1=\dfrac{R_{eq}\cdot R_2}{(R_2-R_{eq})}$
			\end{enumerate}
		\end{enumerate}
	\end{table}
 % % % % % % % % % % % %FINE SECONDO
