\chapter{Formule inverse}
\label{cha:semplificazioni}
\minitoc
\mtcskip                                % put some skip here
\minilof                                % a minilof
\mtcskip                                % put some skip here
\minilot
\section[Primo caso]{Primo caso $a\cdot c=b$}
\label{sec:primocasosemp}
Problema: dato $a\cdot c=b$ trovare $c$.
\begin{align*}
a\cdot c=b&&\\
\text{Divido per $a$ entrambi i lati dell'uguaglianza, e ottengo}\\
\dfrac{a\cdot c}{a}=\dfrac{b}{a}&&\\
\text{Semplifico a sinistra e ottengo c}\\
c=\dfrac{b}{a}&&
\end{align*}
\section[Secondo caso]{Secondo caso $a=\dfrac{b}{c}$}
\label{sec:secondocasosemp}
Problema: dato $a=\dfrac{b}{c}$ trovare $c$.
\begin{align*}
a=\dfrac{b}{c}&&\\
\text{Moltiplico per $c$ entrambi i lati dell'uguaglianza ed ottengo }\\
a\cdot c=\dfrac{b}{c}\cdot c&&\\
\text{Semplifico a destra e ottengo}\\
a\cdot c=b&&\\
\text{Quindi procedo come in\nobs\ref{sec:primocasosemp} }
\end{align*}
\section[Terzo caso]{Terzo caso $a=\dfrac{b}{c+d}$}
\label{sec:terzocasosemp}
\subsection{Trovare b}
\label{sec:terzocasosemptrovareb}
Problema: dato $a=\dfrac{b}{c+d}$ trovare $b$
\begin{align*}
a=\dfrac{b}{c+d}&&\\
\text{Moltiplico per $c+d$ entrambi i lati dell'uguaglianza ed ottengo }\\
a\cdot (c+d)=\dfrac{b}{c+d}\cdot(c+d)&&\\
\text{Semplifico a destra e ottengo}\\
a\cdot (c+d)=b&&
\end{align*}
\subsection{Trovare d}
\label{sec:terzocasosemptrovared}
Problema: dato $a=\dfrac{b}{c+d}$ trovare $d$
\begin{align*}
a=\dfrac{b}{c+d}&&\\
\text{Moltiplico per $c+d$ entrambi i lati dell'uguaglianza ed ottengo }\\
a\cdot (c+d)=\dfrac{b}{c+d}\cdot(c+d)&&\\
\text{Semplifico a destra e ottengo}\\
a\cdot (c+d)=b&&\\
ac+ad=b&&\\
ad=-ac+b&&\\
\text{Divido per a}\\
d=\dfrac{-ac+b}{a}
\end{align*}
\section[Quarto caso]{Quarto caso $\dfrac{1}{a}=\dfrac{b+ c}{b\cdot c}$}
\label{sec:quartocasosemp}
\subsection{Trovare b}
\label{sec:quartocasosemptrovareb}
Problema: dato $\dfrac{1}{a}=\dfrac{b+ c}{b\cdot c}$
\begin{align*}
\dfrac{1}{a}=\dfrac{b+ c}{b\cdot c}&&\\
\text{calcolo il m.c.m }\\
\dfrac{b\cdot c=a\cdot(b+c)}{a\cdot b\cdot c}&&\\
\text{tolgo il m.c.m }\\
b\cdot c=a\cdot(b+c)&&\\
b\cdot c=a\cdot b+a\cdot c &&\\
b\cdot c-a\cdot b=a\cdot c &&\\
b\cdot (c-a)=a\cdot c &&\\
b=\dfrac{a\cdot c}{c-a}
\end{align*}
